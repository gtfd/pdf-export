
\documentclass[twocolumn]{book}
\usepackage[utf8x]{inputenc}
\usepackage[T1]{fontenc}
\usepackage{textcomp}
\usepackage{xparse}
\ExplSyntaxOn
\NewDocumentCommand{\commandline}{v}
  { \immediate \write 18 { \tl_to_str:n {#1}  }  }
\ExplSyntaxOff
\usepackage{graphicx}
\usepackage[inkscape={/usr/bin/inkscape -z -C }]{svg}
\usepackage{import}
\usepackage{pifont}
\usepackage{filecontents}
\usepackage[ngerman]{babel}
\usepackage{color} 
\usepackage{fancyhdr}
\usepackage[defaultlines=4, all]{nowidow}
\definecolor{darkblue}{rgb}{0, 0,.4}
\usepackage[colorlinks=true, urlcolor=darkblue, linkcolor=blue]{hyperref}
\usepackage[hyphens]{url}
\urlstyle{same}
\parindent0cm
\setlength{\columnsep}{30 pt}
\clubpenalty = 10000
\widowpenalty = 10000
\displaywidowpenalty = 10000
\tolerance=500
\let\mypdfximage\pdfximage
\protected\def\pdfximage{\immediate\mypdfximage}
\pagestyle{fancy}
\renewcommand{\headrulewidth}{0.4pt}
\begin{titlepage}
\title{RISM Musical Sources}
\author{\copyright \ 2017 by \ RISM}
\date{\today}
\end{titlepage}
\begin{document}
\maketitle
\thispagestyle{empty}
\twocolumn[
\begin{@twocolumnfalse}
  \vspace*{250px}
  \begin{center}
    This document is licensed under the Creative Commons Attribution - ShareAlike 3.0 License. \\
    To view a copy of this license visit: \\
    \url{http://creativecommons.org/licenses/by-sa/3.0/legalcode}.
  \end{center}
\end{@twocolumnfalse}]
\renewcommand*\contentsname{\hfill Table of content \hfill}
\tableofcontents
\thispagestyle{empty}
\newcommand\hfillplus[1]{{\unskip\nobreak\hfill\penalty50\
  \mbox{}\nobreak\hfill#1}}
\newcommand\invisiblesection[1]{%
  \refstepcounter{section}%
  \addcontentsline{toc}{section}{\protect\numberline{\thesection}#1}%
  \sectionmark{#1}}

\chapter*{\centering Catalog of musical sources}
\addcontentsline{toc}{chapter}{Catalog of musical sources}
\fancyhead{}
\fancyhead[C]{\small Répertoire International des Sources Musicales}
\setlength{\columnseprule}{0.5pt}

\newline \par \vspace{7pt} \textcolor{darkblue}{\textbf{Alday, F.  ??}}\hfillplus{1}
\newline Duets (instr.)  E|b; G; g  
\newline 2 vl
\newline \begin{itshape}Trois | Duos | Pour deux Violons [Es, G, g] | composées et dédiés | à Mr. Brunmann | par Alday | Opéra Troisième 2|e Livre de Duos\end{itshape} 
\newline \textcolor{darkblue}{\ding{\numexpr181 + 01}}  part(s)  
\newline print
\newline RISM-ID: 00000991012664
\newline Decombe  (pbl)
\newline D-Fh  Str Duo 0012
\newline \par \vspace{7pt} \textcolor{darkblue}{\textbf{Anonymus  }}\hfillplus{2}
\newline Ach Gott tu dich erbarmen  1t  
\newline org
\newline \begin{itshape}[f.23v, at left:] Ach Gott thue | Dich erbar | men.\end{itshape} 
\newline \textcolor{darkblue}{\ding{\numexpr181 + 01}}  1 parts  
\newline Manuscript copy
\newline 1.1.1  org  1t  
\begin{filecontents*}{2-1.code}
@clef:G-2
@keysig:
@timesig:3
@data:4'A/24AA/BA/4GAB/24GG/AA/4''C2'B/A4A/
\end{filecontents*}
\commandline{ verovio --spacing-non-linear=0.50 -w 1500 --spacing-system=0.5 --adjust-page-height -b 0 2-1.code }
\newline
\includesvg[width=220pt]{2-1}%

\newline RISM-ID: 455010854
\newline Alte Foliierung auf f.23v mit "24"
\newline D-Fh  Ms 001
\newline $\rightarrow$ In collection 418 (455008343)

\newline \par \vspace{7pt} \textcolor{darkblue}{\textbf{Anonymus  }}\hfillplus{3}
\newline Ach Gott und Herr wie groß und schwer  G  
\newline org
\newline \begin{itshape}[f.15v, at left:] Ach Gott und Herr | wie gros und schwer.\end{itshape} 
\newline \textcolor{darkblue}{\ding{\numexpr181 + 01}}  1 parts  
\newline Manuscript copy
\newline 1.1.1  org  G  
\begin{filecontents*}{3-1.code}
@clef:G-2
@keysig:xF
@timesig:c/
@data:2'B4''DC/2'B4BA/GG2F/A4B{8BB}/{B''D}4C2D/
\end{filecontents*}
\commandline{ verovio --spacing-non-linear=0.50 -w 1500 --spacing-system=0.5 --adjust-page-height -b 0 3-1.code }
\newline
\includesvg[width=220pt]{3-1}%

\newline RISM-ID: 455010829
\newline D-Fh  Ms 001
\newline $\rightarrow$ In collection 418 (455008343)

\newline \par \vspace{7pt} \textcolor{darkblue}{\textbf{Anonymus  }}\hfillplus{4}
\newline Ach Gott und Herr wie groß und schwer  C  
\newline V, org
\newline \begin{itshape}[without title]\end{itshape} 
\newline \textcolor{darkblue}{\ding{\numexpr181 + 01}}  1 parts  
\newline Manuscript copy
\newline 1.1.1  V  C
\newline \begin{footnotesize} Ach Gott und Herr wie groß und schwer \end{footnotesize}  
\begin{filecontents*}{4-1.code}
@clef:C-1
@keysig:
@timesig:c
@data:2''C'BAG/GAB''C/
\end{filecontents*}
\commandline{ verovio --spacing-non-linear=0.50 -w 1500 --spacing-system=0.5 --adjust-page-height -b 0 4-1.code }
\newline
\includesvg[width=220pt]{4-1}%

\newline RISM-ID: 455010990
\newline Unter den beiden Systemen eine Textstrophe
\newline Beide Stimmen weisen im Verlauf kleinere Ungenauigkeiten auf
\newline D-Fh  Ms 001
\newline $\rightarrow$ In collection 418 (455008343)

\newline \par \vspace{7pt} \textcolor{darkblue}{\textbf{Anonymus  }}\hfillplus{5}
\newline Ach Gott vom Himmel sieh darein  D  
\newline org
\newline \begin{itshape}[f.16v, at left:] Ach Gott Von | Himmel sieh | darein.\end{itshape} 
\newline \textcolor{darkblue}{\ding{\numexpr181 + 01}}  1 parts  
\newline Manuscript copy
\newline 1.1.1  org  D  
\begin{filecontents*}{5-1.code}
@clef:G-2
@keysig:xFC
@timesig:c/
@data:2'A4BAG''DD'BA(''C)'BAG''C'BA4(G)://:
\end{filecontents*}
\commandline{ verovio --spacing-non-linear=0.50 -w 1500 --spacing-system=0.5 --adjust-page-height -b 0 5-1.code }
\newline
\includesvg[width=220pt]{5-1}%

\newline RISM-ID: 455010831
\newline D-Fh  Ms 001
\newline $\rightarrow$ In collection 418 (455008343)

\newline \par \vspace{7pt} \textcolor{darkblue}{\textbf{Anonymus  }}\hfillplus{6}
\newline Ach Herr mich armen Sünder. Excerpts  F  
\newline org
\newline \begin{itshape}[f.20r, at left:] Ach Herr mich | armen Sünder\end{itshape} 
\newline \textcolor{darkblue}{\ding{\numexpr181 + 01}}  1 parts  
\newline Manuscript copy
\newline 1.1.1  org  F  
\begin{filecontents*}{6-1.code}
@clef:G-2
@keysig:bB
@timesig:0
@data:4'A''D8C4'B8A/4G(A)''EF/8F4E8E2(D)://:
\end{filecontents*}
\commandline{ verovio --spacing-non-linear=0.50 -w 1500 --spacing-system=0.5 --adjust-page-height -b 0 6-1.code }
\newline
\includesvg[width=220pt]{6-1}%

\newline RISM-ID: 455010840
\newline Das Stück ist unvollständig und durchgestrichen
\newline Alte Foliierung vorhanden: "20"
\newline D-Fh  Ms 001
\newline $\rightarrow$ In collection 418 (455008343)

\newline \par \vspace{7pt} \textcolor{darkblue}{\textbf{Anonymus  }}\hfillplus{7}
\newline Ach Herr mich armen Sünder  F  
\newline org
\newline \begin{itshape}[f.31v, at left:] Ach Herr mich | armen Sünd.\end{itshape} 
\newline \textcolor{darkblue}{\ding{\numexpr181 + 01}}  1 parts  
\newline Manuscript copy
\newline 1.1.1  org  F  
\begin{filecontents*}{7-1.code}
@clef:G-2
@keysig:bB
@timesig:0
@data:4''CD8C4'B8A4GA/''EF8F4E8F4D://:8-FEC{DE}4F
\end{filecontents*}
\commandline{ verovio --spacing-non-linear=0.50 -w 1500 --spacing-system=0.5 --adjust-page-height -b 0 7-1.code }
\newline
\includesvg[width=220pt]{7-1}%

\newline RISM-ID: 455010877
\newline Das Stück ist durchgestrichen
\newline Alte Foliierung auf f.31v mit "32"
\newline D-Fh  Ms 001
\newline $\rightarrow$ In collection 418 (455008343)

\newline \par \vspace{7pt} \textcolor{darkblue}{\textbf{Anonymus  }}\hfillplus{8}
\newline Ach Herr mich armen Sünder  C  
\newline org
\newline \begin{itshape}[f.40r, at left:] Ach Herr mich | armen Sünd\end{itshape} 
\newline \textcolor{darkblue}{\ding{\numexpr181 + 01}}  1 parts  
\newline Manuscript copy
\newline 1.1.1  org  C  
\begin{filecontents*}{8-1.code}
@clef:G-2
@keysig:
@timesig:0
@data:4'EA8G4F8E4DEB''C8C4'B8B4A://:
\end{filecontents*}
\commandline{ verovio --spacing-non-linear=0.50 -w 1500 --spacing-system=0.5 --adjust-page-height -b 0 8-1.code }
\newline
\includesvg[width=220pt]{8-1}%

\newline RISM-ID: 455010896
\newline Alte Foliierung auf den gegenüberliegenden Seiten als Aufschlagfoliierung jeweils "39"
\newline D-Fh  Ms 001
\newline $\rightarrow$ In collection 418 (455008343)

\newline \par \vspace{7pt} \textcolor{darkblue}{\textbf{Anonymus  }}\hfillplus{9}
\newline Ach Jesu Christ dich zu uns wend'  G  
\newline org
\newline \begin{itshape}[f.25v, at left:] Ach Jesu Christ | dich zu uns wend\end{itshape} 
\newline \textcolor{darkblue}{\ding{\numexpr181 + 01}}  1 parts  
\newline Manuscript copy
\newline 1.1.1  org  G  
\begin{filecontents*}{9-1.code}
@clef:G-2
@keysig:xF
@timesig:0
@data:4-'GG{8''D'B+}{BAB''C}2D4-ED{8'BA+}{AGGF}2G
\end{filecontents*}
\commandline{ verovio --spacing-non-linear=0.50 -w 1500 --spacing-system=0.5 --adjust-page-height -b 0 9-1.code }
\newline
\includesvg[width=220pt]{9-1}%

\newline RISM-ID: 455010861
\newline Alte Foliierung auf f.25v mit "26"
\newline D-Fh  Ms 001
\newline $\rightarrow$ In collection 418 (455008343)

\newline \par \vspace{7pt} \textcolor{darkblue}{\textbf{Anonymus  }}\hfillplus{10}
\newline Ach Jesu mein wie große Pein  e  
\newline org
\newline \begin{itshape}[f.59v, at left:] Ach Jesu mein, | wie große Pein\end{itshape} 
\newline \textcolor{darkblue}{\ding{\numexpr181 + 01}}  1 parts  
\newline Manuscript copy
\newline 1.1.1  org  e  
\begin{filecontents*}{10-1.code}
@clef:G-2
@keysig:xF
@timesig:c
@data:8-{'B8.6''E'G}{8FB8.6GA}{8BB}4''ExD://:
\end{filecontents*}
\commandline{ verovio --spacing-non-linear=0.50 -w 1500 --spacing-system=0.5 --adjust-page-height -b 0 10-1.code }
\newline
\includesvg[width=220pt]{10-1}%

\newline RISM-ID: 455010929
\newline Alte Foliierung auf f.59v mit "51"
\newline D-Fh  Ms 001
\newline $\rightarrow$ In collection 418 (455008343)

\newline \par \vspace{7pt} \textcolor{darkblue}{\textbf{Anonymus  }}\hfillplus{11}
\newline Ach bleib bei uns Herr Jesu Christ  1t  
\newline org
\newline \begin{itshape}[f.39v, at left:] Ach bleib beÿ | Unß H J. C.\end{itshape} 
\newline \textcolor{darkblue}{\ding{\numexpr181 + 01}}  1 parts  
\newline Manuscript copy
\newline 1.1.1  org  1t  
\begin{filecontents*}{11-1.code}
@clef:G-2
@keysig:bB
@timesig:0
@data:4'B{8AB}4''C{8DC}{C'B}4A8-{''FED}{8.C6'B8AB}x''C4D8C2(D)
\end{filecontents*}
\commandline{ verovio --spacing-non-linear=0.50 -w 1500 --spacing-system=0.5 --adjust-page-height -b 0 11-1.code }
\newline
\includesvg[width=220pt]{11-1}%

\newline RISM-ID: 455010894
\newline Alte Foliierung auf den gegenüberliegenden Seiten als Aufschlagfoliierung jeweils "39"
\newline D-Fh  Ms 001
\newline $\rightarrow$ In collection 418 (455008343)

\newline \par \vspace{7pt} \textcolor{darkblue}{\textbf{Anonymus  }}\hfillplus{12}
\newline Ach du meines Lebens Lust  a  
\newline org
\newline \begin{itshape}[f.78v, at left:] Ach du meines | Lebens Lust.\end{itshape} 
\newline \textcolor{darkblue}{\ding{\numexpr181 + 01}}  1 parts  
\newline Manuscript copy
\newline 1.1.1  org  a  
\begin{filecontents*}{12-1.code}
@clef:G-2
@keysig:
@timesig:0
@data:{8'AABB}{''CC}4'G{8AAEE}{GG}4C{8''CCDD}{EE}4'B
\end{filecontents*}
\commandline{ verovio --spacing-non-linear=0.50 -w 1500 --spacing-system=0.5 --adjust-page-height -b 0 12-1.code }
\newline
\includesvg[width=220pt]{12-1}%

\newline 1.2.1  org  
\begin{filecontents*}{12-2.code}
@clef:G-2
@keysig:
@timesig:0
@data:{6'AxFxGA}{8BB}{6''CC'BA}4G{6AAGF}{8EE}{6GGFD}{8EC}
\end{filecontents*}
\commandline{ verovio --spacing-non-linear=0.50 -w 1500 --spacing-system=0.5 --adjust-page-height -b 0 12-2.code }
\newline
\includesvg[width=220pt]{12-2}%

\newline 1.3.1  org  
\begin{filecontents*}{12-3.code}
@clef:G-2
@keysig:
@timesig:0
@data:{6'AB''C'A}{B''CD'B}{''CEDC}{'BAGE}{FGAF}{EFGC}
\end{filecontents*}
\commandline{ verovio --spacing-non-linear=0.50 -w 1500 --spacing-system=0.5 --adjust-page-height -b 0 12-3.code }
\newline
\includesvg[width=220pt]{12-3}%

\newline 1.4.1  org  
\begin{filecontents*}{12-4.code}
@clef:G-2
@keysig:
@timesig:0
@data:{8'AABB}{''CC}4'G{8AAEE}{GG}4C{8''CCDD}{EE}4'B
\end{filecontents*}
\commandline{ verovio --spacing-non-linear=0.50 -w 1500 --spacing-system=0.5 --adjust-page-height -b 0 12-4.code }
\newline
\includesvg[width=220pt]{12-4}%

\newline 1.5.1  org  
\begin{filecontents*}{12-5.code}
@clef:G-2
@keysig:
@timesig:0
@data:{3'AGAB}{''C'B''C'A}{BAB''C}{DED'B}{''CEDC}{'B''CDC}{'B''C'BA}{GxFGE}
\end{filecontents*}
\commandline{ verovio --spacing-non-linear=0.50 -w 1500 --spacing-system=0.5 --adjust-page-height -b 0 12-5.code }
\newline
\includesvg[width=220pt]{12-5}%

\newline 1.6.1  org  
\begin{filecontents*}{12-6.code}
@clef:G-2
@keysig:
@timesig:0
@data:{8'AABB}{''CC}4'G{8AAEE}{GG}4C{8''CCDD}{EE}4'B
\end{filecontents*}
\commandline{ verovio --spacing-non-linear=0.50 -w 1500 --spacing-system=0.5 --adjust-page-height -b 0 12-6.code }
\newline
\includesvg[width=220pt]{12-6}%

\newline 1.7.1  org  
\begin{filecontents*}{12-7.code}
@clef:G-2
@keysig:
@timesig:0
@data:8(6{'AB''C};3)({CDE})({'B''CD})({DE'B})({''CDE})({'AB''C})({'BAG})({GFE})
\end{filecontents*}
\commandline{ verovio --spacing-non-linear=0.50 -w 1500 --spacing-system=0.5 --adjust-page-height -b 0 12-7.code }
\newline
\includesvg[width=220pt]{12-7}%

\newline RISM-ID: 455010955
\newline Choral mit sechs Variationen, am Ende der sechsten Variation "Finis."
\newline Alte Foliierung auf f.78v mit "62"
\newline D-Fh  Ms 001
\newline $\rightarrow$ In collection 418 (455008343)

\newline \par \vspace{7pt} \textcolor{darkblue}{\textbf{Anonymus  }}\hfillplus{13}
\newline Ach mein Herr Jesu willst du mich ganz und gar verlassen. Arr  1t  
\newline V, org
\newline \begin{itshape}[f.51v, at left:] Sinfon:\end{itshape} 
\newline \textcolor{darkblue}{\ding{\numexpr181 + 01}}  1 parts  
\newline Manuscript copy
\newline 1.1.1  org  1t  
\begin{filecontents*}{13-1.code}
@clef:G-2
@keysig:bB
@timesig:c/
@data:8-''D{8.6AA8D6EF}{GAG8.6xFD}{8GA8.6BG}{AA}6-A{AA}4G6-{GGG}
\end{filecontents*}
\commandline{ verovio --spacing-non-linear=0.50 -w 1500 --spacing-system=0.5 --adjust-page-height -b 0 13-1.code }
\newline
\includesvg[width=220pt]{13-1}%

\newline 1.2.1  V  1t
\newline \begin{footnotesize} Ach mein Herr Jesu willst du mich ganz und gar verlassen \end{footnotesize}  
\begin{filecontents*}{13-2.code}
@clef:G-2
@keysig:bB
@timesig:c/
@data:4-8-'G{8.6''DC8'BA}{8.6GAxFD}{GA8B''C}{'AA}://:
\end{filecontents*}
\commandline{ verovio --spacing-non-linear=0.50 -w 1500 --spacing-system=0.5 --adjust-page-height -b 0 13-2.code }
\newline
\includesvg[width=220pt]{13-2}%

\newline RISM-ID: 455010912
\newline Unter und neben der Tabulatur insgesamt 7 Textstrophen
\newline Alte Foliierung auf f.51v mit "45"
\newline D-Fh  Ms 001
\newline $\rightarrow$ In collection 418 (455008343)

\newline \par \vspace{7pt} \textcolor{darkblue}{\textbf{Anonymus  }}\hfillplus{14}
\newline Ach mein herzliebes Jesulein  a  
\newline org
\newline \begin{itshape}[f.5v, at left:] 6. | Ach mein Hertz Liebes | Jesulein wie gern | wolt ich\end{itshape} 
\newline \textcolor{darkblue}{\ding{\numexpr181 + 01}}  1 parts  
\newline Manuscript copy
\newline 1.1.1  org  a
\newline \begin{footnotesize} [Ach mein herzliebes Jesulein] \end{footnotesize}  
\begin{filecontents*}{14-1.code}
@clef:G-2
@keysig:
@timesig:3
@data:4'AAA/24''C'B/AA/2.G/4BBB/24''DC/'BB/2.A/
\end{filecontents*}
\commandline{ verovio --spacing-non-linear=0.50 -w 1500 --spacing-system=0.5 --adjust-page-height -b 0 14-1.code }
\newline
\includesvg[width=220pt]{14-1}%

\newline RISM-ID: 455010806
\newline Sechs Textstrophen stehen unterhalb der Tabulaturauf f.5v, eine siebte auf f.6r
\newline D-Fh  Ms 001
\newline $\rightarrow$ In collection 418 (455008343)

\newline \par \vspace{7pt} \textcolor{darkblue}{\textbf{Anonymus  }}\hfillplus{15}
\newline Ach was ist doch unser Leben  a  
\newline org
\newline \begin{itshape}[f.6v, at left:] 8. | Ach was ist doch | unser Leben\end{itshape} 
\newline \textcolor{darkblue}{\ding{\numexpr181 + 01}}  1 parts  
\newline Manuscript copy
\newline 1.1.1  org  a  
\begin{filecontents*}{15-1.code}
@clef:G-2
@keysig:
@timesig:0
@data:{8'AABB}{''CC}4'G{8AAEE}{GG}4E://:
\end{filecontents*}
\commandline{ verovio --spacing-non-linear=0.50 -w 1500 --spacing-system=0.5 --adjust-page-height -b 0 15-1.code }
\newline
\includesvg[width=220pt]{15-1}%

\newline RISM-ID: 455010808
\newline D-Fh  Ms 001
\newline $\rightarrow$ In collection 418 (455008343)

\newline \par \vspace{7pt} \textcolor{darkblue}{\textbf{Anonymus  }}\hfillplus{16}
\newline Ach wie sehnlich wart' ich der Zeit. Excerpts    
\newline V, org
\newline \begin{itshape}[f.123v, heading:] Ach wie sehnl: wart ich d' Zeit\end{itshape} 
\newline \textcolor{darkblue}{\ding{\numexpr181 + 01}}  1 parts  
\newline Manuscript copy
\newline 1.1.1  V
\newline \begin{footnotesize} Ach wie sehnlich wart' ich der Zeit \end{footnotesize}  
\begin{filecontents*}{16-1.code}
@clef:C-1
@keysig:
@timesig:0
@data:1'G''DDC'AB''CD/D'B''D1.C2C1'B://:
\end{filecontents*}
\commandline{ verovio --spacing-non-linear=0.50 -w 1500 --spacing-system=0.5 --adjust-page-height -b 0 16-1.code }
\newline
\includesvg[width=220pt]{16-1}%

\newline RISM-ID: 455011038
\newline Nur die obere Melodiestimme und eine Textzeile ist ausgefüllt, eine Baßsystem ist leer
\newline Alte Paginierung auf f.123v vorhanden: "80"
\newline D-Fh  Ms 001
\newline $\rightarrow$ In collection 418 (455008343)

\newline \par \vspace{7pt} \textcolor{darkblue}{\textbf{Anonymus  }}\hfillplus{17}
\newline Ade du süße Welt. Arr  C  
\newline V, org
\newline \begin{itshape}[f.105r, between the systems:] Ade du süße Welt\end{itshape} 
\newline \textcolor{darkblue}{\ding{\numexpr181 + 01}}  1 parts  
\newline Manuscript copy
\newline 1.1.1  V  C
\newline \begin{footnotesize} Ade du süße Welt \end{footnotesize}  
\begin{filecontents*}{17-1.code}
@clef:C-1
@keysig:
@timesig:c/
@data:1'E2GFDD1C/G2''C'BAA1G/G2FAAxGAA/
\end{filecontents*}
\commandline{ verovio --spacing-non-linear=0.50 -w 1500 --spacing-system=0.5 --adjust-page-height -b 0 17-1.code }
\newline
\includesvg[width=220pt]{17-1}%

\newline RISM-ID: 455011000
\newline Zwischen den beiden Systemen nur der Textbeginn wie angegeben
\newline Alte Zählung auf f.105r vorhanden: "76"
\newline D-Fh  Ms 001
\newline $\rightarrow$ In collection 418 (455008343)

\newline \par \vspace{7pt} \textcolor{darkblue}{\textbf{Anonymus  }}\hfillplus{18}
\newline Airs. Arr  G  
\newline pf
\newline \begin{itshape}[f.105r, between the systems:] Aria\end{itshape} 
\newline \textcolor{darkblue}{\ding{\numexpr181 + 01}}  1 parts  
\newline Manuscript copy
\newline 1.1.1  pf  G  
\begin{filecontents*}{18-1.code}
@clef:C-1
@keysig:xF
@timesig:c
@data:8-''D{CD}{8.'B6B}8A{3''DEF}/{8.6GF}
\end{filecontents*}
\commandline{ verovio --spacing-non-linear=0.50 -w 1500 --spacing-system=0.5 --adjust-page-height -b 0 18-1.code }
\newline
\includesvg[width=220pt]{18-1}%

\newline RISM-ID: 455010999
\newline Unter den beiden Systemen 3 Textstrophen
\newline Alte Zählung auf f.105r vorhanden: "76"
\newline D-Fh  Ms 001
\newline $\rightarrow$ In collection 418 (455008343)

\newline \par \vspace{7pt} \textcolor{darkblue}{\textbf{Anonymus  }}\hfillplus{19}
\newline Airs  B|b  
\newline org
\newline \begin{itshape}[f.115r, heading:] Aria ex B\end{itshape} 
\newline \textcolor{darkblue}{\ding{\numexpr181 + 01}}  1 parts  
\newline Manuscript copy
\newline 1.1.1  org  B|b  
\begin{filecontents*}{19-1.code}
@clef:C-1
@keysig:bBE
@timesig:c/
@data:{6'B''C}/!{866DC'B}{''C'BA}4B8-F/!{G''C}{'AB}{6''C'BAG}{8F''F}/f
\end{filecontents*}
\commandline{ verovio --spacing-non-linear=0.50 -w 1500 --spacing-system=0.5 --adjust-page-height -b 0 19-1.code }
\newline
\includesvg[width=220pt]{19-1}%

\newline RISM-ID: 455011019
\newline Ohne alte Paginierung
\newline D-Fh  Ms 001
\newline $\rightarrow$ In collection 418 (455008343)

\newline \par \vspace{7pt} \textcolor{darkblue}{\textbf{Anonymus  }}\hfillplus{20}
\newline Airs  e  
\newline org
\newline \begin{itshape}[f.115v, heading:] Aria ex E mol.\end{itshape} 
\newline \textcolor{darkblue}{\ding{\numexpr181 + 01}}  1 parts  
\newline Manuscript copy
\newline 1.1.1  org  e  
\begin{filecontents*}{20-1.code}
@clef:C-1
@keysig:xF
@timesig:c/
@data:4'E{8GA}4BE/''E{8FE}4xD'B/E{8B''C}4.D8E/{'F6GA}{8.A6G}2G://:
\end{filecontents*}
\commandline{ verovio --spacing-non-linear=0.50 -w 1500 --spacing-system=0.5 --adjust-page-height -b 0 20-1.code }
\newline
\includesvg[width=220pt]{20-1}%

\newline 1.2.1  org  e  
\begin{filecontents*}{20-2.code}
@clef:C-1
@keysig:xF
@timesig:3/4
@data:4'B''E'B/,B'E,B/'B''C'B/,B'C,B/'ABA/,ABA/
\end{filecontents*}
\commandline{ verovio --spacing-non-linear=0.50 -w 1500 --spacing-system=0.5 --adjust-page-height -b 0 20-2.code }
\newline
\includesvg[width=220pt]{20-2}%

\newline RISM-ID: 455011020
\newline Ohne alte Paginierung
\newline D-Fh  Ms 001
\newline $\rightarrow$ In collection 418 (455008343)

\newline \par \vspace{7pt} \textcolor{darkblue}{\textbf{Anonymus  }}\hfillplus{21}
\newline Airs  D  
\newline org
\newline \begin{itshape}[f.116r, heading:] Aria\end{itshape} 
\newline \textcolor{darkblue}{\ding{\numexpr181 + 01}}  1 parts  
\newline Manuscript copy
\newline 1.1.1  org  D  
\begin{filecontents*}{21-1.code}
@clef:C-1
@keysig:xFC
@timesig:c/
@data:4'AA''DD/{8EFGA}4F{8EF}/4GFE{8DE}/4FF2E://:
\end{filecontents*}
\commandline{ verovio --spacing-non-linear=0.50 -w 1500 --spacing-system=0.5 --adjust-page-height -b 0 21-1.code }
\newline
\includesvg[width=220pt]{21-1}%

\newline RISM-ID: 455011022
\newline Ohne alte Paginierung
\newline D-Fh  Ms 001
\newline $\rightarrow$ In collection 418 (455008343)

\newline \par \vspace{7pt} \textcolor{darkblue}{\textbf{Anonymus  }}\hfillplus{22}
\newline Airs  d  
\newline org
\newline \begin{itshape}[f.116v, heading:] Aria\end{itshape} 
\newline \textcolor{darkblue}{\ding{\numexpr181 + 01}}  1 parts  
\newline Manuscript copy
\newline 1.1.1  org  d  
\begin{filecontents*}{22-1.code}
@clef:C-1
@keysig:bB
@timesig:c/
@data:4'D{8.6AG}{FE}4D/''D{8.6ED}{xC'B}4A/{8''DC}{C'B}{BA}{AG}/{F6GA}{8.E6Dt}2D://:
\end{filecontents*}
\commandline{ verovio --spacing-non-linear=0.50 -w 1500 --spacing-system=0.5 --adjust-page-height -b 0 22-1.code }
\newline
\includesvg[width=220pt]{22-1}%

\newline RISM-ID: 455011023
\newline Ohne alte Paginierung
\newline D-Fh  Ms 001
\newline $\rightarrow$ In collection 418 (455008343)

\newline \par \vspace{7pt} \textcolor{darkblue}{\textbf{Anonymus  }}\hfillplus{23}
\newline Airs  E|b  
\newline org
\newline \begin{itshape}[f.117v, heading:] Aria ex Es\end{itshape} 
\newline \textcolor{darkblue}{\ding{\numexpr181 + 01}}  1 parts  
\newline Manuscript copy
\newline 1.1.1  org  E|b  
\begin{filecontents*}{23-1.code}
@clef:C-1
@keysig:bBEA
@timesig:2/4
@data:{8'EF}{GA}/888.6{BE}{EB}/{''C'E}{E''C}/{'BE}{EB}/4''CDt/E{8'EF}/4G{8.F6Et}/2E://:
\end{filecontents*}
\commandline{ verovio --spacing-non-linear=0.50 -w 1500 --spacing-system=0.5 --adjust-page-height -b 0 23-1.code }
\newline
\includesvg[width=220pt]{23-1}%

\newline RISM-ID: 455011027
\newline Ohne alte Paginierung
\newline D-Fh  Ms 001
\newline $\rightarrow$ In collection 418 (455008343)

\newline \par \vspace{7pt} \textcolor{darkblue}{\textbf{Anonymus  }}\hfillplus{24}
\newline Airs  G  
\newline org
\newline \begin{itshape}[f.118v, heading:] Aria.\end{itshape} 
\newline \textcolor{darkblue}{\ding{\numexpr181 + 01}}  1 parts  
\newline Manuscript copy
\newline 1.1.1  org  G  
\begin{filecontents*}{24-1.code}
@clef:C-1
@keysig:xF
@timesig:3/4
@data:4''D4.'G{6AB}/4.F8A4D/''DDE/{8EDDCC'B}/4AAB/4.x''C{6'B''C}4D/'GGF/F{8ED}4E/
\end{filecontents*}
\commandline{ verovio --spacing-non-linear=0.50 -w 1500 --spacing-system=0.5 --adjust-page-height -b 0 24-1.code }
\newline
\includesvg[width=220pt]{24-1}%

\newline RISM-ID: 455011029
\newline Ohne alte Paginierung
\newline D-Fh  Ms 001
\newline $\rightarrow$ In collection 418 (455008343)

\newline \par \vspace{7pt} \textcolor{darkblue}{\textbf{Anonymus  }}\hfillplus{25}
\newline Alle Menschen müssen sterben  C  
\newline org
\newline \begin{itshape}[f.66v, at left:] Alle Menschen | müßen sterben\end{itshape} 
\newline \textcolor{darkblue}{\ding{\numexpr181 + 01}}  1 parts  
\newline Manuscript copy
\newline 1.1.1  org  C  
\begin{filecontents*}{25-1.code}
@clef:G-2
@keysig:
@timesig:0
@data:{8''CC'GA}{8.6GF8EC}{8GGFE}{8.D6C}4C://:
\end{filecontents*}
\commandline{ verovio --spacing-non-linear=0.50 -w 1500 --spacing-system=0.5 --adjust-page-height -b 0 25-1.code }
\newline
\includesvg[width=220pt]{25-1}%

\newline RISM-ID: 455010940
\newline Alte Foliierung auf f.66v mit "57"
\newline D-Fh  Ms 001
\newline $\rightarrow$ In collection 418 (455008343)

\newline \par \vspace{7pt} \textcolor{darkblue}{\textbf{Anonymus  }}\hfillplus{26}
\newline Allein Gott in der Höh sei Ehr  F  
\newline org
\newline \begin{itshape}[f.26v, at left:] Allein Gott | in der Höh | seÿ Ehr.\end{itshape} 
\newline \textcolor{darkblue}{\ding{\numexpr181 + 01}}  1 parts  
\newline Manuscript copy
\newline 1.1.1  org  F  
\begin{filecontents*}{26-1.code}
@clef:G-2
@keysig:bB
@timesig:3/4
@data:4'F/24AB/''C'B/AG/AA/AG/4BAG/FDE/2F://:
\end{filecontents*}
\commandline{ verovio --spacing-non-linear=0.50 -w 1500 --spacing-system=0.5 --adjust-page-height -b 0 26-1.code }
\newline
\includesvg[width=220pt]{26-1}%

\newline RISM-ID: 455010862
\newline Alte Foliierung auf f.26v mit "27"
\newline D-Fh  Ms 001
\newline $\rightarrow$ In collection 418 (455008343)

\newline \par \vspace{7pt} \textcolor{darkblue}{\textbf{Anonymus  }}\hfillplus{27}
\newline Allein Gott in der Höh sei Ehr  G  
\newline org
\newline \begin{itshape}[f.40v, at left:] Allein Gott in | Der Höh seÿ Ehr\end{itshape} 
\newline \textcolor{darkblue}{\ding{\numexpr181 + 01}}  1 parts  
\newline Manuscript copy
\newline 1.1.1  org  G  
\begin{filecontents*}{27-1.code}
@clef:G-2
@keysig:xF
@timesig:3/4
@data:4'G/24B''C/DC/'BA/BB/BA/4B''C'B/AGE/2.F/G://:
\end{filecontents*}
\commandline{ verovio --spacing-non-linear=0.50 -w 1500 --spacing-system=0.5 --adjust-page-height -b 0 27-1.code }
\newline
\includesvg[width=220pt]{27-1}%

\newline RISM-ID: 455010898
\newline Alte Foliierung auf f.40v mit "40"
\newline D-Fh  Ms 001
\newline $\rightarrow$ In collection 418 (455008343)

\newline \par \vspace{7pt} \textcolor{darkblue}{\textbf{Anonymus  }}\hfillplus{28}
\newline Allein Gott in der Höh sei Ehr  G  
\newline org
\newline \begin{itshape}[f.106v, heading:] Allein Gott in der Höh sey Ehr:\end{itshape} 
\newline \textcolor{darkblue}{\ding{\numexpr181 + 01}}  1 parts  
\newline Manuscript copy
\newline 1.1.1  org  G  
\begin{filecontents*}{28-1.code}
@clef:F-4
@keysig:xF
@timesig:3/2
@data:4,GA2B%C-1 'C/4GA2B''C/12DC/'BA/B-/
\end{filecontents*}
\commandline{ verovio --spacing-non-linear=0.50 -w 1500 --spacing-system=0.5 --adjust-page-height -b 0 28-1.code }
\newline
\includesvg[width=220pt]{28-1}%

\newline RISM-ID: 455011002
\newline Alte Paginierung vorhanden, f.106v: "79" und f.107r: "80"
\newline D-Fh  Ms 001
\newline $\rightarrow$ In collection 418 (455008343)

\newline \par \vspace{7pt} \textcolor{darkblue}{\textbf{Anonymus  }}\hfillplus{29}
\newline Allein zu dir Herr Jesu Christ  C  
\newline org
\newline \begin{itshape}[f.16r, at left:] Allein zu dir | Herr Jesu Christ.\end{itshape} 
\newline \textcolor{darkblue}{\ding{\numexpr181 + 01}}  1 parts  
\newline Manuscript copy
\newline 1.1.1  org  C  
\begin{filecontents*}{29-1.code}
@clef:G-2
@keysig:
@timesig:0
@data:2''C4'GA{8.6''CD8ED+}{DCC'B}2''C/E{8DC'BG}{8.6AB8''CD+}{DC}4'B2A://:
\end{filecontents*}
\commandline{ verovio --spacing-non-linear=0.50 -w 1500 --spacing-system=0.5 --adjust-page-height -b 0 29-1.code }
\newline
\includesvg[width=220pt]{29-1}%

\newline RISM-ID: 455010830
\newline Alte Foliierung vorhanden: "16"
\newline D-Fh  Ms 001
\newline $\rightarrow$ In collection 418 (455008343)

\newline \par \vspace{7pt} \textcolor{darkblue}{\textbf{Anonymus  }}\hfillplus{30}
\newline Allemandes  C  
\newline pf
\newline \begin{itshape}[f.57v, at left:] Allamand\end{itshape} 
\newline \textcolor{darkblue}{\ding{\numexpr181 + 01}}  1 parts  
\newline Manuscript copy
\newline 1.1.1  org  C  
\begin{filecontents*}{30-1.code}
@clef:G-2
@keysig:
@timesig:0
@data:4'G{8.G6G}{AB''C'A}{8GF}{6EEFG}{AB''C'B}4A2(G)://:
\end{filecontents*}
\commandline{ verovio --spacing-non-linear=0.50 -w 1500 --spacing-system=0.5 --adjust-page-height -b 0 30-1.code }
\newline
\includesvg[width=220pt]{30-1}%

\newline RISM-ID: 455010923
\newline Ohne alte Foliierung
\newline D-Fh  Ms 001
\newline $\rightarrow$ In collection 418 (455008343)

\newline \par \vspace{7pt} \textcolor{darkblue}{\textbf{Anonymus  }}\hfillplus{31}
\newline Allemandes  d  
\newline pf
\newline \begin{itshape}[f.57v, at left:] Allamand. | W. C. B.\end{itshape} 
\newline \textcolor{darkblue}{\ding{\numexpr181 + 01}}  1 parts  
\newline Manuscript copy
\newline 1.1.1  pf  d  
\begin{filecontents*}{31-1.code}
@clef:G-2
@keysig:bB
@timesig:0
@data:6'A{''DEFG}{8A6EG}{FFGA}{BA8G}{6F'AB''xC}{D'A''CD}{8.'B6A}4A://:
\end{filecontents*}
\commandline{ verovio --spacing-non-linear=0.50 -w 1500 --spacing-system=0.5 --adjust-page-height -b 0 31-1.code }
\newline
\includesvg[width=220pt]{31-1}%

\newline RISM-ID: 455010924
\newline Ohne alte Foliierung
\newline D-Fh  Ms 001
\newline $\rightarrow$ In collection 418 (455008343)

\newline \par \vspace{7pt} \textcolor{darkblue}{\textbf{Anonymus  }}\hfillplus{32}
\newline An Wasserflüssen Babylon  F  
\newline org
\newline \begin{itshape}[f.33v, at left:] An Waßer Flüssen | Babÿlon.\end{itshape} 
\newline \textcolor{darkblue}{\ding{\numexpr181 + 01}}  1 parts  
\newline Manuscript copy
\newline 1.1.1  org  F  
\begin{filecontents*}{32-1.code}
@clef:G-2
@keysig:bB
@timesig:0
@data:2''C4DC2'A''C4'BB2A/G4AB4.''C8'B4ABAG2(F)://:
\end{filecontents*}
\commandline{ verovio --spacing-non-linear=0.50 -w 1500 --spacing-system=0.5 --adjust-page-height -b 0 32-1.code }
\newline
\includesvg[width=220pt]{32-1}%

\newline RISM-ID: 455010883
\newline Alte Foliierung auf f.33v mit "34"
\newline D-Fh  Ms 001
\newline $\rightarrow$ In collection 418 (455008343)

\newline \par \vspace{7pt} \textcolor{darkblue}{\textbf{Anonymus  }}\hfillplus{33}
\newline An die Geliebte  A  
\newline V, mandora
\newline \begin{itshape}[heading, f.45v:] An die oder den geliebten\end{itshape} 
\newline \textcolor{darkblue}{\ding{\numexpr181 + 01}}  1 parts  
\newline Manuscript copy
\newline Zink, Joseph Michael
\newline 1.1.1  V  A
\newline \begin{footnotesize} Ist denn Liebe ein Verbrechen \end{footnotesize}  
\begin{filecontents*}{33-1.code}
@clef:G-2
@keysig:xFCG
@timesig:c/
@data:4'EE/AA{8A6BA}8GA/4''C'B-8''D'B/{AG}{FE}4.''E8D/qD2C4-8'EA/4''CC{8C6DC}8'BA/4GF8-AGF/
\end{filecontents*}
\commandline{ verovio --spacing-non-linear=0.50 -w 1500 --spacing-system=0.5 --adjust-page-height -b 0 33-1.code }
\newline
\includesvg[width=220pt]{33-1}%

\newline RISM-ID: 455011121
\newline Drei Strophen
\newline Der Schreiber ist der selbe wie der 70 Divertimenti in der gleichen Handschrift
\newline D-Fh  Ms 002
\newline $\rightarrow$ In collection 419 (455008345)

\newline \par \vspace{7pt} \textcolor{darkblue}{\textbf{Anonymus  }}\hfillplus{34}
\newline Ännchen von Tharau  A  
\newline V, pf
\newline \begin{itshape}[heading, f.60v:] Anchen von Tharau.\end{itshape} 
\newline \textcolor{darkblue}{\ding{\numexpr181 + 01}}  1 parts  
\newline Manuscript copy
\newline 1.1.1  V  A
\newline \begin{footnotesize} Ännchen von Tharau ist's die mir gefällt \end{footnotesize}  
\begin{filecontents*}{34-1.code}
@clef:G-2
@keysig:xFCG
@timesig:3/4
@data:4.'E8F4E/EAA/4.B8''C4'B/2A4-/GGG/4.B8A4G/FGF/2E4-/
\end{filecontents*}
\commandline{ verovio --spacing-non-linear=0.50 -w 1500 --spacing-system=0.5 --adjust-page-height -b 0 34-1.code }
\newline
\includesvg[width=220pt]{34-1}%

\newline RISM-ID: 455010794
\newline Drei Strophen
\newline D-Fh  Ms 002
\newline $\rightarrow$ In collection 419 (455008345)

\newline \par \vspace{7pt} \textcolor{darkblue}{\textbf{Anonymus  }}\hfillplus{35}
\newline Auf meinen lieben Gott  1t  
\newline org
\newline \begin{itshape}[f.30v, at left:] Auff meinen | lieben | Gott.\end{itshape} 
\newline \textcolor{darkblue}{\ding{\numexpr181 + 01}}  1 parts  
\newline Manuscript copy
\newline 1.1.1  org  1t  
\begin{filecontents*}{35-1.code}
@clef:G-2
@keysig:
@timesig:0
@data:2'D4DAAGA/AAGFFE/EFGAAAxGA/
\end{filecontents*}
\commandline{ verovio --spacing-non-linear=0.50 -w 1500 --spacing-system=0.5 --adjust-page-height -b 0 35-1.code }
\newline
\includesvg[width=220pt]{35-1}%

\newline RISM-ID: 455010876
\newline Alte Foliierung auf f.30v mit "31"
\newline D-Fh  Ms 001
\newline $\rightarrow$ In collection 418 (455008343)

\newline \par \vspace{7pt} \textcolor{darkblue}{\textbf{Anonymus  }}\hfillplus{36}
\newline Aus tiefer Not schrei ich zu dir  a  
\newline org
\newline \begin{itshape}[f.17r, at left:] Auß tieffer Noth | schreÿ ich zu | dir.\end{itshape} 
\newline \textcolor{darkblue}{\ding{\numexpr181 + 01}}  1 parts  
\newline Manuscript copy
\newline 1.1.1  org  a  
\begin{filecontents*}{36-1.code}
@clef:G-2
@keysig:
@timesig:0
@data:2'B4EB''C'BGA(B)/B''CDC'AGFE://:
\end{filecontents*}
\commandline{ verovio --spacing-non-linear=0.50 -w 1500 --spacing-system=0.5 --adjust-page-height -b 0 36-1.code }
\newline
\includesvg[width=220pt]{36-1}%

\newline RISM-ID: 455010833
\newline Alte Foliierung vorhanden: "17"
\newline D-Fh  Ms 001
\newline $\rightarrow$ In collection 418 (455008343)

\newline \par \vspace{7pt} \textcolor{darkblue}{\textbf{Anonymus  }}\hfillplus{37}
\newline Ballets  C  
\newline pf
\newline \begin{itshape}[f.57v, at left:] Ballet.\end{itshape} 
\newline \textcolor{darkblue}{\ding{\numexpr181 + 01}}  1 parts  
\newline Manuscript copy
\newline 1.1.1  pf  C  
\begin{filecontents*}{37-1.code}
@clef:G-2
@keysig:
@timesig:0
@data:{6''GGEF}4G{6FFDE}4F{6EEDC}{6'BG''CD}{8E6.3DC}4C://:
\end{filecontents*}
\commandline{ verovio --spacing-non-linear=0.50 -w 1500 --spacing-system=0.5 --adjust-page-height -b 0 37-1.code }
\newline
\includesvg[width=220pt]{37-1}%

\newline RISM-ID: 455010925
\newline Ohne alte Foliierung
\newline D-Fh  Ms 001
\newline $\rightarrow$ In collection 418 (455008343)

\newline \par \vspace{7pt} \textcolor{darkblue}{\textbf{Anonymus  }}\hfillplus{38}
\newline Ballets  G  
\newline pf
\newline \begin{itshape}[f.58v, at left:] Ballet.\end{itshape} 
\newline \textcolor{darkblue}{\ding{\numexpr181 + 01}}  1 parts  
\newline Manuscript copy
\newline 1.1.1  pf  G  
\begin{filecontents*}{38-1.code}
@clef:G-2
@keysig:xF
@timesig:0
@data:{8''D6EF8GG}{6'AB''CD}{8'BG}{6''CDEF}{8GG}{6'EFGA}4B://:
\end{filecontents*}
\commandline{ verovio --spacing-non-linear=0.50 -w 1500 --spacing-system=0.5 --adjust-page-height -b 0 38-1.code }
\newline
\includesvg[width=220pt]{38-1}%

\newline RISM-ID: 455010927
\newline Ohne alte Foliierung
\newline D-Fh  Ms 001
\newline $\rightarrow$ In collection 418 (455008343)

\newline \par \vspace{7pt} \textcolor{darkblue}{\textbf{Anonymus  }}\hfillplus{39}
\newline Ballets  G  
\newline pf
\newline \begin{itshape}[f.80v, at left:] Ballet\end{itshape} 
\newline \textcolor{darkblue}{\ding{\numexpr181 + 01}}  1 parts  
\newline Manuscript copy
\newline 1.1.1  pf  G  
\begin{filecontents*}{39-1.code}
@clef:G-2
@keysig:xF
@timesig:0
@data:8.6{'GG}{FD}{6.3GGAA}{8.6BB}{''CC}{6'B''CDE}{xC'A''DD}4D://:
\end{filecontents*}
\commandline{ verovio --spacing-non-linear=0.50 -w 1500 --spacing-system=0.5 --adjust-page-height -b 0 39-1.code }
\newline
\includesvg[width=220pt]{39-1}%

\newline RISM-ID: 455010957
\newline Ohne alte Foliierung
\newline D-Fh  Ms 001
\newline $\rightarrow$ In collection 418 (455008343)

\newline \par \vspace{7pt} \textcolor{darkblue}{\textbf{Anonymus  }}\hfillplus{40}
\newline Ballets  G  
\newline pf
\newline \begin{itshape}[f.85v, at left:] Ballet.\end{itshape} 
\newline \textcolor{darkblue}{\ding{\numexpr181 + 01}}  1 parts  
\newline Manuscript copy
\newline 1.1.1  pf  G  
\begin{filecontents*}{40-1.code}
@clef:G-2
@keysig:xF
@timesig:0
@data:8'G{8.G6A}{B''C'AB}4G8-{6B''C}{8D6ED}{xCD'B''C}4D
\end{filecontents*}
\commandline{ verovio --spacing-non-linear=0.50 -w 1500 --spacing-system=0.5 --adjust-page-height -b 0 40-1.code }
\newline
\includesvg[width=220pt]{40-1}%

\newline RISM-ID: 455010966
\newline Ohne alte Foliierung
\newline D-Fh  Ms 001
\newline $\rightarrow$ In collection 418 (455008343)

\newline \par \vspace{7pt} \textcolor{darkblue}{\textbf{Anonymus  }}\hfillplus{41}
\newline Battaglias. Arr  F  
\newline org
\newline \begin{itshape}[f.81v, at left:] Bataglia\end{itshape} 
\newline \textcolor{darkblue}{\ding{\numexpr181 + 01}}  1 parts  
\newline Manuscript copy
\newline 1.1.1  org  F  
\begin{filecontents*}{41-1.code}
@clef:G-2
@keysig:bB
@timesig:0
@data:{8''F6FF}8F6-C{DEFF}{EFGE}{8F6FF}4F://:
\end{filecontents*}
\commandline{ verovio --spacing-non-linear=0.50 -w 1500 --spacing-system=0.5 --adjust-page-height -b 0 41-1.code }
\newline
\includesvg[width=220pt]{41-1}%

\newline RISM-ID: 455010961
\newline Ohne alte Foliierung
\newline D-Fh  Ms 001
\newline $\rightarrow$ In collection 418 (455008343)

\newline \par \vspace{7pt} \textcolor{darkblue}{\textbf{Anonymus  }}\hfillplus{42}
\newline Betrübtes Herz sei wohlgemut  G  
\newline org
\newline \begin{itshape}[f.71v, at left:] Betrübtes Herz | seÿ Wohlgemuth\end{itshape} 
\newline \textcolor{darkblue}{\ding{\numexpr181 + 01}}  1 parts  
\newline Manuscript copy
\newline 1.1.1  org  G  
\begin{filecontents*}{42-1.code}
@clef:G-2
@keysig:xF
@timesig:0
@data:4'D{8DD}4E{8GF+}{FF}4G8-{B''DC}{'B''E8.'A6G}2G://:
\end{filecontents*}
\commandline{ verovio --spacing-non-linear=0.50 -w 1500 --spacing-system=0.5 --adjust-page-height -b 0 42-1.code }
\newline
\includesvg[width=220pt]{42-1}%

\newline RISM-ID: 455010946
\newline Alte Foliierung auf f.71v mit "60"
\newline D-Fh  Ms 001
\newline $\rightarrow$ In collection 418 (455008343)

\newline \par \vspace{7pt} \textcolor{darkblue}{\textbf{Anonymus  }}\hfillplus{43}
\newline Betrügliche Welt. Excerpts. Arr  e  
\newline V, org
\newline \begin{itshape}[without title]\end{itshape} 
\newline \textcolor{darkblue}{\ding{\numexpr181 + 01}}  1 parts  
\newline Manuscript copy
\newline 1.1.1  V  e
\newline \begin{footnotesize} Betrügliche Welt \end{footnotesize}  
\begin{filecontents*}{43-1.code}
@clef:C-1
@keysig:xF
@timesig:3
@data:12'EEFxD1E/12AABG1A/
\end{filecontents*}
\commandline{ verovio --spacing-non-linear=0.50 -w 1500 --spacing-system=0.5 --adjust-page-height -b 0 43-1.code }
\newline
\includesvg[width=220pt]{43-1}%

\newline RISM-ID: 455011037
\newline Nur die obere Melodiestimme und eine Textzeile ist ausgefüllt, eine Baßsystem ist leer
\newline Alte Paginierung auf f.123v vorhanden: "80"
\newline D-Fh  Ms 001
\newline $\rightarrow$ In collection 418 (455008343)

\newline \par \vspace{7pt} \textcolor{darkblue}{\textbf{Anonymus  }}\hfillplus{44}
\newline Bourrées  E  
\newline org
\newline \begin{itshape}[f.117v, heading:] Bouree ex E dur.\end{itshape} 
\newline \textcolor{darkblue}{\ding{\numexpr181 + 01}}  1 parts  
\newline Manuscript copy
\newline 1.1.1  org  E  
\begin{filecontents*}{44-1.code}
@clef:C-1
@keysig:xFCGD
@timesig:c/
@data:{8''ED}/4C{8'B''C}{'A''C'BA}/4G{8FG}{EGFE}/4DE{8FGAB}/
\end{filecontents*}
\commandline{ verovio --spacing-non-linear=0.50 -w 1500 --spacing-system=0.5 --adjust-page-height -b 0 44-1.code }
\newline
\includesvg[width=220pt]{44-1}%

\newline RISM-ID: 455011026
\newline Ohne alte Paginierung
\newline D-Fh  Ms 001
\newline $\rightarrow$ In collection 418 (455008343)

\newline \par \vspace{7pt} \textcolor{darkblue}{\textbf{Anonymus  }}\hfillplus{45}
\newline Chorale arrangements  d  
\newline org
\newline \begin{itshape}[without title]\end{itshape} 
\newline \textcolor{darkblue}{\ding{\numexpr181 + 01}}  1 parts  
\newline Manuscript copy
\newline 1.1.1  org  d  
\begin{filecontents*}{45-1.code}
@clef:G-2
@keysig:bB
@timesig:0
@data:4'DEFGFEDxC2D/4''DDEDE{8C'A}4''DxC2D/
\end{filecontents*}
\commandline{ verovio --spacing-non-linear=0.50 -w 1500 --spacing-system=0.5 --adjust-page-height -b 0 45-1.code }
\newline
\includesvg[width=220pt]{45-1}%

\newline RISM-ID: 455010834
\newline D-Fh  Ms 001
\newline $\rightarrow$ In collection 418 (455008343)

\newline \par \vspace{7pt} \textcolor{darkblue}{\textbf{Anonymus  }}\hfillplus{46}
\newline Christ der du bist der helle Tag  1t  
\newline org
\newline \begin{itshape}[f.32v, at left:] Christ der du | bist der hel= | le Tag.\end{itshape} 
\newline \textcolor{darkblue}{\ding{\numexpr181 + 01}}  1 parts  
\newline Manuscript copy
\newline 1.1.1  org  1t  
\begin{filecontents*}{46-1.code}
@clef:G-2
@keysig:bB
@timesig:0
@data:2'G4GABGB''C1(D)2D4DD2F4DD+D{8C'B}4''CC2('B)
\end{filecontents*}
\commandline{ verovio --spacing-non-linear=0.50 -w 1500 --spacing-system=0.5 --adjust-page-height -b 0 46-1.code }
\newline
\includesvg[width=220pt]{46-1}%

\newline RISM-ID: 455010880
\newline Alte Foliierung auf f.32v mit "33"
\newline D-Fh  Ms 001
\newline $\rightarrow$ In collection 418 (455008343)

\newline \par \vspace{7pt} \textcolor{darkblue}{\textbf{Anonymus  }}\hfillplus{47}
\newline Christ ist erstanden. Excerpts  1t  
\newline org
\newline \begin{itshape}[without title]\end{itshape} 
\newline \textcolor{darkblue}{\ding{\numexpr181 + 01}}  1 parts  
\newline Manuscript copy
\newline 1.1.1  org  1t
\newline \begin{footnotesize} [Christ ist erstanden] \end{footnotesize}  
\begin{filecontents*}{47-1.code}
@clef:C-1
@keysig:
@timesig:c/
@data:1'A/2xGA/''CD/1('A)/A/2FG/AF/EF/1(D)/
\end{filecontents*}
\commandline{ verovio --spacing-non-linear=0.50 -w 1500 --spacing-system=0.5 --adjust-page-height -b 0 47-1.code }
\newline
\includesvg[width=220pt]{47-1}%

\newline RISM-ID: 455011016
\newline Nur die obere Melodiestimme und eine Textzeile ist ausgefüllt, eine Baßsystem ist leer
\newline D-Fh  Ms 001
\newline $\rightarrow$ In collection 418 (455008343)

\newline \par \vspace{7pt} \textcolor{darkblue}{\textbf{Anonymus  }}\hfillplus{48}
\newline Christ lag in Todesbanden    
\newline org
\newline \begin{itshape}[f.105v, heading:] Christ Lag in Todes Banden.\end{itshape} 
\newline \textcolor{darkblue}{\ding{\numexpr181 + 01}}  1 parts  
\newline Manuscript copy
\newline 1.1.1  org  
\begin{filecontents*}{48-1.code}
@clef:F-4
@keysig:
@timesig:c
@data:1,A/2G4AB/%G-2 1'A/2G4.A8B/2''CD/C'B/1A/
\end{filecontents*}
\commandline{ verovio --spacing-non-linear=0.50 -w 1500 --spacing-system=0.5 --adjust-page-height -b 0 48-1.code }
\newline
\includesvg[width=220pt]{48-1}%

\newline RISM-ID: 455011001
\newline Alte Paginierung vorhanden, f.105v: "77" und f.106r: "78"
\newline D-Fh  Ms 001
\newline $\rightarrow$ In collection 418 (455008343)

\newline \par \vspace{7pt} \textcolor{darkblue}{\textbf{Anonymus  }}\hfillplus{49}
\newline Christe der du bist Tag und Licht  1t  
\newline org
\newline \begin{itshape}[f.31v, at left:] Christe der du | bist Tag und | Licht.\end{itshape} 
\newline \textcolor{darkblue}{\ding{\numexpr181 + 01}}  1 parts  
\newline Manuscript copy
\newline 1.1.1  org  1t  
\begin{filecontents*}{49-1.code}
@clef:G-2
@keysig:bB
@timesig:0
@data:2'G4GBGF4GABAGBBBBFGBA2G(A)
\end{filecontents*}
\commandline{ verovio --spacing-non-linear=0.50 -w 1500 --spacing-system=0.5 --adjust-page-height -b 0 49-1.code }
\newline
\includesvg[width=220pt]{49-1}%

\newline RISM-ID: 455010878
\newline Alte Foliierung auf f.31v mit "32"
\newline D-Fh  Ms 001
\newline $\rightarrow$ In collection 418 (455008343)

\newline \par \vspace{7pt} \textcolor{darkblue}{\textbf{Anonymus  }}\hfillplus{50}
\newline Christus der uns selig macht    
\newline org
\newline \begin{itshape}[f.29v, at left:] Christus der | uns seelig | macht\end{itshape} 
\newline \textcolor{darkblue}{\ding{\numexpr181 + 01}}  1 parts  
\newline Manuscript copy
\newline 1.1.1  org  
\begin{filecontents*}{50-1.code}
@clef:G-2
@keysig:
@timesig:0
@data:4''EEEDDC2('B)/4AxGAB''DEDC2('B)/
\end{filecontents*}
\commandline{ verovio --spacing-non-linear=0.50 -w 1500 --spacing-system=0.5 --adjust-page-height -b 0 50-1.code }
\newline
\includesvg[width=220pt]{50-1}%

\newline RISM-ID: 455010871
\newline Alte Foliierung auf den gegenüberliegenden Seiten als Aufschlagfoliierung jeweils "30"
\newline D-Fh  Ms 001
\newline $\rightarrow$ In collection 418 (455008343)

\newline \par \vspace{7pt} \textcolor{darkblue}{\textbf{Anonymus  }}\hfillplus{51}
\newline Christus ist mein Leben  F  
\newline org
\newline \begin{itshape}[f.14v, at left:] Christus ist mein | Leben\end{itshape} 
\newline \textcolor{darkblue}{\ding{\numexpr181 + 01}}  1 parts  
\newline Manuscript copy
\newline 1.1.1  org  F  
\begin{filecontents*}{51-1.code}
@clef:G-2
@keysig:bB
@timesig:0
@data:2'F4AGAB2''C4'A''DC'BAG2A
\end{filecontents*}
\commandline{ verovio --spacing-non-linear=0.50 -w 1500 --spacing-system=0.5 --adjust-page-height -b 0 51-1.code }
\newline
\includesvg[width=220pt]{51-1}%

\newline RISM-ID: 455010826
\newline D-Fh  Ms 001
\newline $\rightarrow$ In collection 418 (455008343)

\newline \par \vspace{7pt} \textcolor{darkblue}{\textbf{Anonymus  }}\hfillplus{52}
\newline Courantes  D  
\newline pf
\newline \begin{itshape}[f.10r, at left:] Courant\end{itshape} 
\newline \textcolor{darkblue}{\ding{\numexpr181 + 01}}  1 parts  
\newline Manuscript copy
\newline 1.1.1  pf  D  
\begin{filecontents*}{52-1.code}
@clef:G-2
@keysig:xFC
@timesig:3
@data:4--{6''DEFG}/4.A8B4A/AAxG/4.A{8GFE}/4DEF/4.G8A4B/B2A/2.G/
\end{filecontents*}
\commandline{ verovio --spacing-non-linear=0.50 -w 1500 --spacing-system=0.5 --adjust-page-height -b 0 52-1.code }
\newline
\includesvg[width=220pt]{52-1}%

\newline RISM-ID: 455010816
\newline Die letzten beiden Takte der Courante stehen f.9v unten
\newline Alte Foliierung vorhanden: "10"
\newline D-Fh  Ms 001
\newline $\rightarrow$ In collection 418 (455008343)

\newline \par \vspace{7pt} \textcolor{darkblue}{\textbf{Anonymus  }}\hfillplus{53}
\newline Courantes  D  
\newline pf
\newline \begin{itshape}[f.10v, at left:] 16. | Courant\end{itshape} 
\newline \textcolor{darkblue}{\ding{\numexpr181 + 01}}  1 parts  
\newline Manuscript copy
\newline 1.1.1  pf  D  
\begin{filecontents*}{53-1.code}
@clef:G-2
@keysig:xFC
@timesig:3
@data:4--''A/4.A{8GFE}/4F2E/2.D/4DEF/4.E{8DC'B}/4''C2'B/
\end{filecontents*}
\commandline{ verovio --spacing-non-linear=0.50 -w 1500 --spacing-system=0.5 --adjust-page-height -b 0 53-1.code }
\newline
\includesvg[width=220pt]{53-1}%

\newline RISM-ID: 455010817
\newline D-Fh  Ms 001
\newline $\rightarrow$ In collection 418 (455008343)

\newline \par \vspace{7pt} \textcolor{darkblue}{\textbf{Anonymus  }}\hfillplus{54}
\newline Courantes  C  
\newline pf
\newline \begin{itshape}[f.58v, at left:] Courant.\end{itshape} 
\newline \textcolor{darkblue}{\ding{\numexpr181 + 01}}  1 parts  
\newline Manuscript copy
\newline 1.1.1  pf  C  
\begin{filecontents*}{54-1.code}
@clef:G-2
@keysig:
@timesig:3
@data:4''CCG/G{8FGAG}/4FFF/2E4D/4EDC/'BG''D/2D4E/2D4-://:
\end{filecontents*}
\commandline{ verovio --spacing-non-linear=0.50 -w 1500 --spacing-system=0.5 --adjust-page-height -b 0 54-1.code }
\newline
\includesvg[width=220pt]{54-1}%

\newline RISM-ID: 455010926
\newline Ohne alte Foliierung
\newline D-Fh  Ms 001
\newline $\rightarrow$ In collection 418 (455008343)

\newline \par \vspace{7pt} \textcolor{darkblue}{\textbf{Anonymus  }}\hfillplus{55}
\newline Da Jesus an dem Kreuze stund    
\newline org
\newline \begin{itshape}[f.31v, at left:] Da Jesus an | Xr stund.\end{itshape} 
\newline \textcolor{darkblue}{\ding{\numexpr181 + 01}}  1 parts  
\newline Manuscript copy
\newline 1.1.1  org  
\begin{filecontents*}{55-1.code}
@clef:G-2
@keysig:
@timesig:0
@data:2'B4''CC'B''CDC2'B/4x''CD'B''C'AAFE/
\end{filecontents*}
\commandline{ verovio --spacing-non-linear=0.50 -w 1500 --spacing-system=0.5 --adjust-page-height -b 0 55-1.code }
\newline
\includesvg[width=220pt]{55-1}%

\newline RISM-ID: 455010879
\newline Alte Foliierung auf f.31v mit "32"
\newline D-Fh  Ms 001
\newline $\rightarrow$ In collection 418 (455008343)

\newline \par \vspace{7pt} \textcolor{darkblue}{\textbf{Anonymus  }}\hfillplus{56}
\newline Da Jesus an dem Kreuze stund  e  
\newline org
\newline \begin{itshape}[f.50v, at left:] Da Jesus an | dem Kreütze stund\end{itshape} 
\newline \textcolor{darkblue}{\ding{\numexpr181 + 01}}  1 parts  
\newline Manuscript copy
\newline 1.1.1  org  e  
\begin{filecontents*}{56-1.code}
@clef:G-2
@keysig:xF
@timesig:0
@data:2'B4''C'BAB''DC2'B/4-''CD'B''C'AAF2E
\end{filecontents*}
\commandline{ verovio --spacing-non-linear=0.50 -w 1500 --spacing-system=0.5 --adjust-page-height -b 0 56-1.code }
\newline
\includesvg[width=220pt]{56-1}%

\newline RISM-ID: 455010911
\newline Ohne alte Foliierung
\newline D-Fh  Ms 001
\newline $\rightarrow$ In collection 418 (455008343)

\newline \par \vspace{7pt} \textcolor{darkblue}{\textbf{Anonymus  }}\hfillplus{57}
\newline Da kam ich auf den Boden ei ei ei  G  
\newline V, mandora
\newline \begin{itshape}[without title]\end{itshape} 
\newline \textcolor{darkblue}{\ding{\numexpr181 + 01}}  1 parts  
\newline Manuscript copy
\newline Zink, Joseph Michael
\newline 1.1.1  V  G
\newline \begin{footnotesize} Da kam ich auf den Boden ei ei ei \end{footnotesize}  
\begin{filecontents*}{57-1.code}
@clef:G-2
@keysig:xF
@timesig:2/4
@data:8'D/GGGG/4GG/A''D/'G8-D/4GG/GG/A''D/'G8-G/AGA''C/
\end{filecontents*}
\commandline{ verovio --spacing-non-linear=0.50 -w 1500 --spacing-system=0.5 --adjust-page-height -b 0 57-1.code }
\newline
\includesvg[width=220pt]{57-1}%

\newline RISM-ID: 455010798
\newline Sechs Strophen
\newline Der Schreiber ist der selbe wie der 70 Divertimenti in der gleichen Handschrift
\newline D-Fh  Ms 002
\newline $\rightarrow$ In collection 419 (455008345)

\newline \par \vspace{7pt} \textcolor{darkblue}{\textbf{Anonymus  }}\hfillplus{58}
\newline Dances  d  
\newline pf
\newline \begin{itshape}[f.8v, at left:] 13. | Tantz, | [f.9r, at left:] Proportion\end{itshape} 
\newline \textcolor{darkblue}{\ding{\numexpr181 + 01}}  1 parts  
\newline Manuscript copy
\newline 1.1.1  pf  d  
\begin{filecontents*}{58-1.code}
@clef:G-2
@keysig:bB
@timesig:0
@data:4-8-''A{BGGF}4.G8A{FFFG}4.A8A{GFFE}4.F://8F{FF}{6FGAG}
\end{filecontents*}
\commandline{ verovio --spacing-non-linear=0.50 -w 1500 --spacing-system=0.5 --adjust-page-height -b 0 58-1.code }
\newline
\includesvg[width=220pt]{58-1}%

\newline 1.2.1  pf  d  
\begin{filecontents*}{58-2.code}
@clef:G-2
@keysig:bB
@timesig:3
@data:4--''A/BGF/2G4A/F{8DEFG}/4.'A{8nB''C'A}/{''GAFG}4E/2F://4F/4.F{8GAG}/
\end{filecontents*}
\commandline{ verovio --spacing-non-linear=0.50 -w 1500 --spacing-system=0.5 --adjust-page-height -b 0 58-2.code }
\newline
\includesvg[width=220pt]{58-2}%

\newline RISM-ID: 455010813
\newline Die "Proportion" ist eine Variante des vorhergehenden Tanzes im Dreiertakt
\newline Alte Foliierung vorhanden: "9"
\newline D-Fh  Ms 001
\newline $\rightarrow$ In collection 418 (455008343)

\newline \par \vspace{7pt} \textcolor{darkblue}{\textbf{Anonymus  }}\hfillplus{59}
\newline Dank sagen wir alle Gott unserm Herrn. Arr  1t  
\newline V, org
\newline \begin{itshape}[without title]\end{itshape} 
\newline \textcolor{darkblue}{\ding{\numexpr181 + 01}}  1 parts  
\newline Manuscript copy
\newline 1.1.1  V  1t
\newline \begin{footnotesize} Dank sagen wir alle Gott unserm Herrn \end{footnotesize}  
\begin{filecontents*}{59-1.code}
@clef:C-1
@keysig:
@timesig:c
@data:2'F4GG2AAG/A''CC4'AB2''C1C/2CCDC'G''C'B''C'A1G/
\end{filecontents*}
\commandline{ verovio --spacing-non-linear=0.50 -w 1500 --spacing-system=0.5 --adjust-page-height -b 0 59-1.code }
\newline
\includesvg[width=220pt]{59-1}%

\newline RISM-ID: 455010994
\newline Zwischen den beiden Systemen eine Textstrophe
\newline Alte Zählung auf f.103v vorhanden: "73"
\newline D-Fh  Ms 001
\newline $\rightarrow$ In collection 418 (455008343)

\newline \par \vspace{7pt} \textcolor{darkblue}{\textbf{Anonymus  }}\hfillplus{60}
\newline Das Hindu-Mädchen  F  
\newline V, pf
\newline \begin{itshape}[heading, f.63r:] Das Hindumaedchen.\end{itshape} 
\newline \textcolor{darkblue}{\ding{\numexpr181 + 01}}  1 parts  
\newline Manuscript copy
\newline 1.1.1  pf  F  
\begin{filecontents*}{60-1.code}
@clef:G-2
@keysig:bB
@timesig:c
@data:4''F{8.6CF}4.A8F/4EDC'G/4.8FE''CE/4F'A2(F)/
\end{filecontents*}
\commandline{ verovio --spacing-non-linear=0.50 -w 1500 --spacing-system=0.5 --adjust-page-height -b 0 60-1.code }
\newline
\includesvg[width=220pt]{60-1}%

\newline 1.1.2  V  F
\newline \begin{footnotesize} Reich mit des Orients Segen beladen \end{footnotesize}  
\begin{filecontents*}{60-2.code}
@clef:G-2
@keysig:bB
@timesig:c
@data:4'F{8.6AB}{''CC}4D/C{8.6C'A}4FC/B{8.6''CC}4.'A8''C/C'BBA4Aqq6BAr4G/4G8.6GnB{BB}''C'G/
\end{filecontents*}
\commandline{ verovio --spacing-non-linear=0.50 -w 1500 --spacing-system=0.5 --adjust-page-height -b 0 60-2.code }
\newline
\includesvg[width=220pt]{60-2}%

\newline RISM-ID: 455010792
\newline Vier Strophen
\newline D-Fh  Ms 002
\newline $\rightarrow$ In collection 419 (455008345)

\newline \par \vspace{7pt} \textcolor{darkblue}{\textbf{Anonymus  }}\hfillplus{61}
\newline Das Jesulein soll doch mein Trost. Arr    
\newline V, org
\newline \begin{itshape}[without title]\end{itshape} 
\newline \textcolor{darkblue}{\ding{\numexpr181 + 01}}  1 parts  
\newline Manuscript copy
\newline 1.1.1  V
\newline \begin{footnotesize} Das Jesulein soll doch mein Trost \end{footnotesize}  
\begin{filecontents*}{61-1.code}
@clef:C-1
@keysig:
@timesig:c/
@data:1'G/2GG/xFG/ED/1(D)/D/2EF/GB/4AG2A/1(G)://:
\end{filecontents*}
\commandline{ verovio --spacing-non-linear=0.50 -w 1500 --spacing-system=0.5 --adjust-page-height -b 0 61-1.code }
\newline
\includesvg[width=220pt]{61-1}%

\newline RISM-ID: 455010982
\newline Unter den beiden Systemen 3 Textstrophen
\newline Ohne alte Foliierung
\newline D-Fh  Ms 001
\newline $\rightarrow$ In collection 418 (455008343)

\newline \par \vspace{7pt} \textcolor{darkblue}{\textbf{Anonymus  }}\hfillplus{62}
\newline Das Wörtchen Du  C  
\newline V, mandora
\newline \begin{itshape}[heading, f.57v:] Das Wörtchen du\end{itshape} 
\newline \textcolor{darkblue}{\ding{\numexpr181 + 01}}  1 parts  
\newline Manuscript copy
\newline Zink, Joseph Michael
\newline 1.1.1  V  C
\newline \begin{footnotesize} Wie kommt es daß in Liebessachen das Wörtchen Du so süße klingt \end{footnotesize}  
\begin{filecontents*}{62-1.code}
@clef:G-2
@keysig:
@timesig:c
@data:8'GAB/!{''CD}{ED}{C'B}{''C'A}/2G8E''C'BA/!4.G8G4G8FD/2C8-GAB/f
\end{filecontents*}
\commandline{ verovio --spacing-non-linear=0.50 -w 1500 --spacing-system=0.5 --adjust-page-height -b 0 62-1.code }
\newline
\includesvg[width=220pt]{62-1}%

\newline RISM-ID: 455010797
\newline Vier Strophen
\newline Der Schreiber ist der selbe wie der 70 Divertimenti in der gleichen Handschrift
\newline D-Fh  Ms 002
\newline $\rightarrow$ In collection 419 (455008345)

\newline \par \vspace{7pt} \textcolor{darkblue}{\textbf{Anonymus  }}\hfillplus{63}
\newline Der Bräut'gam wird bald rufen    
\newline org
\newline \begin{itshape}[f.56v, at left:] Der Bräutga | wird bald\end{itshape} 
\newline \textcolor{darkblue}{\ding{\numexpr181 + 01}}  1 parts  
\newline Manuscript copy
\newline 1.1.1  org  
\begin{filecontents*}{63-1.code}
@clef:G-2
@keysig:
@timesig:3
@data:4'GAB/24''C'B/AB/4''DED/24C'B/A-/4GAB/24''C'B/AB/
\end{filecontents*}
\commandline{ verovio --spacing-non-linear=0.50 -w 1500 --spacing-system=0.5 --adjust-page-height -b 0 63-1.code }
\newline
\includesvg[width=220pt]{63-1}%

\newline RISM-ID: 455010920
\newline Alte Foliierung auf f.56v mit "50"
\newline D-Fh  Ms 001
\newline $\rightarrow$ In collection 418 (455008343)

\newline \par \vspace{7pt} \textcolor{darkblue}{\textbf{Anonymus  }}\hfillplus{64}
\newline Der Herr ist mein getreuer Hirt  1t  
\newline org
\newline \begin{itshape}[f.30v, at left:] Der Herr ist | mein getreuer | Hirt.\end{itshape} 
\newline \textcolor{darkblue}{\ding{\numexpr181 + 01}}  1 parts  
\newline Manuscript copy
\newline 1.1.1  org  1t  
\begin{filecontents*}{64-1.code}
@clef:G-2
@keysig:bB
@timesig:0
@data:2'G4GDGA{8B''C}4'AG/GB''CDC{8'B''C}4'AG://:
\end{filecontents*}
\commandline{ verovio --spacing-non-linear=0.50 -w 1500 --spacing-system=0.5 --adjust-page-height -b 0 64-1.code }
\newline
\includesvg[width=220pt]{64-1}%

\newline RISM-ID: 455010874
\newline Alte Foliierung auf f.30v mit "31"
\newline D-Fh  Ms 001
\newline $\rightarrow$ In collection 418 (455008343)

\newline \par \vspace{7pt} \textcolor{darkblue}{\textbf{Anonymus  }}\hfillplus{65}
\newline Der Mann  D  
\newline V, mandora
\newline \begin{itshape}[heading, f.58v:] Der Mann.\end{itshape} 
\newline \textcolor{darkblue}{\ding{\numexpr181 + 01}}  1 parts  
\newline Manuscript copy
\newline 1.1.1  V  D
\newline \begin{footnotesize} Wohl dem Manne dessen Herz \end{footnotesize}  
\begin{filecontents*}{65-1.code}
@clef:G-2
@keysig:xFC
@timesig:c
@data:8'ABA/4AA''F8D'B/B4A8--GFE/4DFA''C/D-8-'ABA/4AA{8''FC}{D'B}/2A8-GFE/
\end{filecontents*}
\commandline{ verovio --spacing-non-linear=0.50 -w 1500 --spacing-system=0.5 --adjust-page-height -b 0 65-1.code }
\newline
\includesvg[width=220pt]{65-1}%

\newline RISM-ID: 455010796
\newline Vier Strophen
\newline Gleim, Johann Wilhelm Ludwig  (lyr)
\newline D-Fh  Ms 002
\newline $\rightarrow$ In collection 419 (455008345)

\newline \par \vspace{7pt} \textcolor{darkblue}{\textbf{Anonymus  }}\hfillplus{66}
\newline Der Tag der ist so freudenreich  F  
\newline org
\newline \begin{itshape}[f.25r, at left:] Der Tag | der ist so | freuden | reich.\end{itshape} 
\newline \textcolor{darkblue}{\ding{\numexpr181 + 01}}  1 parts  
\newline Manuscript copy
\newline 1.1.1  org  F  
\begin{filecontents*}{66-1.code}
@clef:G-2
@keysig:bB
@timesig:0
@data:2'F4FFGA{8BABG}4F8-A4GFDEFG2(F)://:
\end{filecontents*}
\commandline{ verovio --spacing-non-linear=0.50 -w 1500 --spacing-system=0.5 --adjust-page-height -b 0 66-1.code }
\newline
\includesvg[width=220pt]{66-1}%

\newline RISM-ID: 455010858
\newline Alte Foliierung auf f.24v mit "25"
\newline D-Fh  Ms 001
\newline $\rightarrow$ In collection 418 (455008343)

\newline \par \vspace{7pt} \textcolor{darkblue}{\textbf{Anonymus  }}\hfillplus{67}
\newline Des heil'gen Geistes reiche Gnad'    
\newline org
\newline \begin{itshape}[f.5r, at left:] 5. | Des heilgen Geistes | reiche Gnad.\end{itshape} 
\newline \textcolor{darkblue}{\ding{\numexpr181 + 01}}  1 parts  
\newline Manuscript copy
\newline 1.1.1  org  
\begin{filecontents*}{67-1.code}
@clef:G-2
@keysig:
@timesig:3
@data:4'AAB/24AA/A''D/C-/4'AA''C/24CD/CC/'A-/
\end{filecontents*}
\commandline{ verovio --spacing-non-linear=0.50 -w 1500 --spacing-system=0.5 --adjust-page-height -b 0 67-1.code }
\newline
\includesvg[width=220pt]{67-1}%

\newline RISM-ID: 455010805
\newline Alte Foliierung vorhanden: "5"
\newline D-Fh  Ms 001
\newline $\rightarrow$ In collection 418 (455008343)

\newline \par \vspace{7pt} \textcolor{darkblue}{\textbf{Anonymus  }}\hfillplus{68}
\newline Des heil'gen Geistes reiche Gnad'. Excerpts    
\newline org
\newline \begin{itshape}[f.5r, between the systems:] Deß Heil: Geistes reiche Gnad.\end{itshape} 
\newline \textcolor{darkblue}{\ding{\numexpr181 + 01}}  1 parts  
\newline Manuscript copy
\newline 1.1.1  org  
\begin{filecontents*}{68-1.code}
@clef:C-1
@keysig:
@timesig:0
@data:1'AAA''C2'B1AAGFE/GGFGAFEF/
\end{filecontents*}
\commandline{ verovio --spacing-non-linear=0.50 -w 1500 --spacing-system=0.5 --adjust-page-height -b 0 68-1.code }
\newline
\includesvg[width=220pt]{68-1}%

\newline RISM-ID: 455010978
\newline Am unteren Rand des Blattes zwei Systeme, wovon nur das obere mit einem Schlüssel und der Melodie versehen ist
\newline Alte Foliierung vorhanden: "5"
\newline D-Fh  Ms 001
\newline $\rightarrow$ In collection 418 (455008343)

\newline \par \vspace{7pt} \textcolor{darkblue}{\textbf{Anonymus  }}\hfillplus{69}
\newline Die Klage  A  
\newline V, mandora
\newline \begin{itshape}[heading, f.38v:] Die Klage\end{itshape} 
\newline \textcolor{darkblue}{\ding{\numexpr181 + 01}}  1 parts  
\newline Manuscript copy
\newline 1.1.1  V  A
\newline \begin{footnotesize} Ach ich liebte war so glücklich \end{footnotesize}  
\begin{filecontents*}{69-1.code}
@clef:G-2
@keysig:xFCG
@timesig:3/4
@data:24''C'B/AG/''DC/4x'ABB/2''C{8CE}/4D'FB/24A''C/'B-/
\end{filecontents*}
\commandline{ verovio --spacing-non-linear=0.50 -w 1500 --spacing-system=0.5 --adjust-page-height -b 0 69-1.code }
\newline
\includesvg[width=220pt]{69-1}%

\newline RISM-ID: 455011113
\newline Sechs Strophen
\newline Bretzner, Christoph Friedrich  (lyr)
\newline D-Fh  Ms 002
\newline $\rightarrow$ In collection 419 (455008345)

\newline \par \vspace{7pt} \textcolor{darkblue}{\textbf{Anonymus  }}\hfillplus{70}
\newline Die betrunkene Welt  A  
\newline V, guit
\newline \begin{itshape}[heading, f.61v:] Die betrunkene Welt\end{itshape} 
\newline \textcolor{darkblue}{\ding{\numexpr181 + 01}}  1 parts  
\newline Manuscript copy
\newline 1.1.1  V  A
\newline \begin{footnotesize} Komm ich soeben zum Wirtshaus heraus \end{footnotesize}  
\begin{filecontents*}{70-1.code}
@clef:G-2
@keysig:xFCG
@timesig:3/4
@data:4'A{8AG}{AB}/4''C'AE/
\end{filecontents*}
\commandline{ verovio --spacing-non-linear=0.50 -w 1500 --spacing-system=0.5 --adjust-page-height -b 0 70-1.code }
\newline
\includesvg[width=220pt]{70-1}%

\newline RISM-ID: 455010793
\newline Sechs Strophen
\newline D-Fh  Ms 002
\newline $\rightarrow$ In collection 419 (455008345)

\newline \par \vspace{7pt} \textcolor{darkblue}{\textbf{Anonymus  }}\hfillplus{71}
\newline Dort oben auf der Alm  G  
\newline V, mandora
\newline \begin{itshape}[without title]\end{itshape} 
\newline \textcolor{darkblue}{\ding{\numexpr181 + 01}}  1 parts  
\newline Manuscript copy
\newline 1.1.1  V  G
\newline \begin{footnotesize} Dort oben auf der Alm \end{footnotesize}  
\begin{filecontents*}{71-1.code}
@clef:G-2
@keysig:xF
@timesig:3/4
@data:{8'B''C}/4.D8'B''ED/4.D{6C'A}4F/4.''D{6ED}{8C'A}/
\end{filecontents*}
\commandline{ verovio --spacing-non-linear=0.50 -w 1500 --spacing-system=0.5 --adjust-page-height -b 0 71-1.code }
\newline
\includesvg[width=220pt]{71-1}%

\newline RISM-ID: 455011117
\newline Vier Strophen
\newline D-Fh  Ms 002
\newline $\rightarrow$ In collection 419 (455008345)

\newline \par \vspace{7pt} \textcolor{darkblue}{\textbf{Anonymus  }}\hfillplus{72}
\newline Du Friedefürst Herr Jesu Christ  G  
\newline org
\newline \begin{itshape}[f.11r, at left:] 17. | Du Friedefürst | H. Jesu Christ\end{itshape} 
\newline \textcolor{darkblue}{\ding{\numexpr181 + 01}}  1 parts  
\newline Manuscript copy
\newline 1.1.1  org  G  
\begin{filecontents*}{72-1.code}
@clef:G-2
@keysig:xF
@timesig:0
@data:2'B{633GAB}{AB''C}{6'B''C}{633DED}{6C'A''DE}4'B-''DC'BAA2B://:
\end{filecontents*}
\commandline{ verovio --spacing-non-linear=0.50 -w 1500 --spacing-system=0.5 --adjust-page-height -b 0 72-1.code }
\newline
\includesvg[width=220pt]{72-1}%

\newline RISM-ID: 455010818
\newline Alte Foliierung vorhanden: "11"
\newline D-Fh  Ms 001
\newline $\rightarrow$ In collection 418 (455008343)

\newline \par \vspace{7pt} \textcolor{darkblue}{\textbf{Anonymus  }}\hfillplus{73}
\newline Du bist aller Dinge Schöne meine Freundin. Arr  G  
\newline V (4)
\newline \begin{itshape}[f.72v, at left:] Du bist aller | Dinge schöne\end{itshape} 
\newline \textcolor{darkblue}{\ding{\numexpr181 + 01}}  short score (tabulature): f.72v-74r  
\newline Manuscript copy
\newline 1.1.1  T  G
\newline \begin{footnotesize} Du bist aller Dinge Schöne meine Freundin \end{footnotesize}  
\begin{filecontents*}{73-1.code}
@clef:C-4
@keysig:xF
@timesig:0
@data:!{8'DDCC}{,BB}4A!fG-
\end{filecontents*}
\commandline{ verovio --spacing-non-linear=0.50 -w 1500 --spacing-system=0.5 --adjust-page-height -b 0 73-1.code }
\newline
\includesvg[width=220pt]{73-1}%

\newline RISM-ID: 455010948
\newline In den unteren beiden Tabulaturstimmen ist der Gesangstext eingetragen
\newline Alte Foliierung auf f.72v mit "61"
\newline D-Fh  Ms 001
\newline $\rightarrow$ In collection 418 (455008343)

\newline \par \vspace{7pt} \textcolor{darkblue}{\textbf{Anonymus  }}\hfillplus{74}
\newline Du bist der rechte David Herr  C  
\newline org
\newline \begin{itshape}[f.69v, at left:] Du bist der | rechte\end{itshape} 
\newline \textcolor{darkblue}{\ding{\numexpr181 + 01}}  1 parts  
\newline Manuscript copy
\newline 1.1.1  org  C  
\begin{filecontents*}{74-1.code}
@clef:G-2
@keysig:
@timesig:c/
@data:{8''E6EE8GE}{FE}4D
\end{filecontents*}
\commandline{ verovio --spacing-non-linear=0.50 -w 1500 --spacing-system=0.5 --adjust-page-height -b 0 74-1.code }
\newline
\includesvg[width=220pt]{74-1}%

\newline RISM-ID: 455010943
\newline Alte Foliierung auf f.69v mit "59"
\newline D-Fh  Ms 001
\newline $\rightarrow$ In collection 418 (455008343)

\newline \par \vspace{7pt} \textcolor{darkblue}{\textbf{Anonymus  }}\hfillplus{75}
\newline Durch Adams Fall ist ganz verderbt  a  
\newline org
\newline \begin{itshape}[f.21v, at left:] Durch Adams | fall ist ganz | V derbt.\end{itshape} 
\newline \textcolor{darkblue}{\ding{\numexpr181 + 01}}  1 parts  
\newline Manuscript copy
\newline 1.1.1  org  a  
\begin{filecontents*}{75-1.code}
@clef:G-2
@keysig:
@timesig:0
@data:2'A4AAGAFE2D/A4''CD'AB2''C'A://:
\end{filecontents*}
\commandline{ verovio --spacing-non-linear=0.50 -w 1500 --spacing-system=0.5 --adjust-page-height -b 0 75-1.code }
\newline
\includesvg[width=220pt]{75-1}%

\newline RISM-ID: 455010845
\newline Alte Foliierung auf den gegenüberliegenden Seiten als Aufschlagfoliierung jeweils "22"
\newline D-Fh  Ms 001
\newline $\rightarrow$ In collection 418 (455008343)

\newline \par \vspace{7pt} \textcolor{darkblue}{\textbf{Anonymus  }}\hfillplus{76}
\newline Ein' feste Burg ist unser Gott  C  
\newline org
\newline \begin{itshape}[f.24r, at left:] Ein feste Burgk | ist unser | Gott\end{itshape} 
\newline \textcolor{darkblue}{\ding{\numexpr181 + 01}}  1 parts  
\newline Manuscript copy
\newline 1.1.1  org  C  
\begin{filecontents*}{76-1.code}
@clef:G-2
@keysig:
@timesig:0
@data:4''CCC'GA''C'BA2(G)''C4'BA2G4AF+FE2D(C)://:
\end{filecontents*}
\commandline{ verovio --spacing-non-linear=0.50 -w 1500 --spacing-system=0.5 --adjust-page-height -b 0 76-1.code }
\newline
\includesvg[width=220pt]{76-1}%

\newline RISM-ID: 455010855
\newline Alte Foliierung auf f.23v mit "24"
\newline D-Fh  Ms 001
\newline $\rightarrow$ In collection 418 (455008343)

\newline \par \vspace{7pt} \textcolor{darkblue}{\textbf{Anonymus  }}\hfillplus{77}
\newline Einen guten Kampf hab' ich  C  
\newline org
\newline \begin{itshape}[f.6r, at left:] 7. | Einen Guten Kampff | hab ich.\end{itshape} 
\newline \textcolor{darkblue}{\ding{\numexpr181 + 01}}  1 parts  
\newline Manuscript copy
\newline 1.1.1  org  C  
\begin{filecontents*}{77-1.code}
@clef:G-2
@keysig:
@timesig:0
@data:8.6''EE{8DD}{CC}4'B{8AGA''C}4'B''C://:
\end{filecontents*}
\commandline{ verovio --spacing-non-linear=0.50 -w 1500 --spacing-system=0.5 --adjust-page-height -b 0 77-1.code }
\newline
\includesvg[width=220pt]{77-1}%

\newline RISM-ID: 455010807
\newline Alte Foliierung vorhanden: "6"
\newline D-Fh  Ms 001
\newline $\rightarrow$ In collection 418 (455008343)

\newline \par \vspace{7pt} \textcolor{darkblue}{\textbf{Anonymus  }}\hfillplus{78}
\newline Erbarm dich mein o Herre Gott  e  
\newline org
\newline \begin{itshape}[f.17r, at left:] Erbarm dich mein | O Herre Gott.\end{itshape} 
\newline \textcolor{darkblue}{\ding{\numexpr181 + 01}}  1 parts  
\newline Manuscript copy
\newline 1.1.1  org  e  
\begin{filecontents*}{78-1.code}
@clef:G-2
@keysig:xF
@timesig:0
@data:2'E4GGAB''C'BAG''C'B''C'AGF(E)://:
\end{filecontents*}
\commandline{ verovio --spacing-non-linear=0.50 -w 1500 --spacing-system=0.5 --adjust-page-height -b 0 78-1.code }
\newline
\includesvg[width=220pt]{78-1}%

\newline RISM-ID: 455010832
\newline Alte Foliierung vorhanden: "17"
\newline D-Fh  Ms 001
\newline $\rightarrow$ In collection 418 (455008343)

\newline \par \vspace{7pt} \textcolor{darkblue}{\textbf{Anonymus  }}\hfillplus{79}
\newline Erhalt uns Herr bei deinem Wort  1t  
\newline org
\newline \begin{itshape}[f.20r, at left:] Erhalt uns Herr | beÿ deinen wort.\end{itshape} 
\newline \textcolor{darkblue}{\ding{\numexpr181 + 01}}  1 parts  
\newline Manuscript copy
\newline 1.1.1  org  1t  
\begin{filecontents*}{79-1.code}
@clef:G-2
@keysig:bB
@timesig:0
@data:2'G4BGxFGBA1(G)4B''CCD'B''CC(D)/
\end{filecontents*}
\commandline{ verovio --spacing-non-linear=0.50 -w 1500 --spacing-system=0.5 --adjust-page-height -b 0 79-1.code }
\newline
\includesvg[width=220pt]{79-1}%

\newline RISM-ID: 455010843
\newline Alte Foliierungen vorhanden: "20" und "21"
\newline D-Fh  Ms 001
\newline $\rightarrow$ In collection 418 (455008343)

\newline \par \vspace{7pt} \textcolor{darkblue}{\textbf{Anonymus  }}\hfillplus{80}
\newline Ermuntert euch ihr müden Seelen  G  
\newline org
\newline \begin{itshape}[f.62r, at left:] Ermundert | euch ihr mü= | den See= | len\end{itshape} 
\newline \textcolor{darkblue}{\ding{\numexpr181 + 01}}  1 parts  
\newline Manuscript copy
\newline 1.1.1  org  G  
\begin{filecontents*}{80-1.code}
@clef:G-2
@keysig:xF
@timesig:3
@data:4''DDD/{8DEDC}4'B/AGG/2F4E/GAB/
\end{filecontents*}
\commandline{ verovio --spacing-non-linear=0.50 -w 1500 --spacing-system=0.5 --adjust-page-height -b 0 80-1.code }
\newline
\includesvg[width=220pt]{80-1}%

\newline RISM-ID: 455010933
\newline Alte Foliierung auf den gegenüberliegenden Seiten als Aufschlagfoliierung jeweils "53"
\newline D-Fh  Ms 001
\newline $\rightarrow$ In collection 418 (455008343)

\newline \par \vspace{7pt} \textcolor{darkblue}{\textbf{Anonymus  }}\hfillplus{81}
\newline Ermuntre dich mein schwacher Geist  G  
\newline V, org
\newline \begin{itshape}[without title]\end{itshape} 
\newline \textcolor{darkblue}{\ding{\numexpr181 + 01}}  1 parts  
\newline Manuscript copy
\newline 1.1.1  V  G
\newline \begin{footnotesize} Ermuntre dich mein schwacher Geist \end{footnotesize}  
\begin{filecontents*}{81-1.code}
@clef:C-1
@keysig:xF
@timesig:3
@data:2''DDD/12CC/'BB/1.A/2''DDD/12C'B/1.A/G://:
\end{filecontents*}
\commandline{ verovio --spacing-non-linear=0.50 -w 1500 --spacing-system=0.5 --adjust-page-height -b 0 81-1.code }
\newline
\includesvg[width=220pt]{81-1}%

\newline RISM-ID: 455010985
\newline Unter den beiden Systemen 4 Textstrophen aufgeführt mit der Strophenzählung 1, 10, 11 und 12
\newline Ohne alte Foliierung
\newline D-Fh  Ms 001
\newline $\rightarrow$ In collection 418 (455008343)

\newline \par \vspace{7pt} \textcolor{darkblue}{\textbf{Anonymus  }}\hfillplus{82}
\newline Erschienen ist der herrlich' Tag  1t  
\newline org
\newline \begin{itshape}[f.32v, at left:] Erschienen ist | der herlig tag | Ostern\end{itshape} 
\newline \textcolor{darkblue}{\ding{\numexpr181 + 01}}  1 parts  
\newline Manuscript copy
\newline 1.1.1  org  1t  
\begin{filecontents*}{82-1.code}
@clef:G-2
@keysig:
@timesig:3/4
@data:4'DDD/24AB/''C'A/GG/B''C/D'A/''C'B/AA/
\end{filecontents*}
\commandline{ verovio --spacing-non-linear=0.50 -w 1500 --spacing-system=0.5 --adjust-page-height -b 0 82-1.code }
\newline
\includesvg[width=220pt]{82-1}%

\newline RISM-ID: 455010881
\newline Alte Foliierung auf f.32v mit "33"
\newline D-Fh  Ms 001
\newline $\rightarrow$ In collection 418 (455008343)

\newline \par \vspace{7pt} \textcolor{darkblue}{\textbf{Anonymus  }}\hfillplus{83}
\newline Erschienen ist der herrlich' Tag    
\newline org
\newline \begin{itshape}[f.108r, heading:] Erschienen ist der herrlich Tag.\end{itshape} 
\newline \textcolor{darkblue}{\ding{\numexpr181 + 01}}  1 parts  
\newline Manuscript copy
\newline 1.1.1  org  
\begin{filecontents*}{83-1.code}
@clef:C-1
@keysig:xF
@timesig:3/2
@data:=2/2'DDD/1A2B/2.''C4'B2A/12GA/Bx''C/1D4'AB/1''C2'B/1.A/
\end{filecontents*}
\commandline{ verovio --spacing-non-linear=0.50 -w 1500 --spacing-system=0.5 --adjust-page-height -b 0 83-1.code }
\newline
\includesvg[width=220pt]{83-1}%

\newline RISM-ID: 455011004
\newline Alte Paginierung vorhanden, f.108r: "82"
\newline D-Fh  Ms 001
\newline $\rightarrow$ In collection 418 (455008343)

\newline \par \vspace{7pt} \textcolor{darkblue}{\textbf{Anonymus  }}\hfillplus{84}
\newline Es ist das Heil uns kommen her  F  
\newline org
\newline \begin{itshape}[f.28v, at left:] Es ist das | Heÿl uns | kommen | her.\end{itshape} 
\newline \textcolor{darkblue}{\ding{\numexpr181 + 01}}  1 parts  
\newline Manuscript copy
\newline 1.1.1  org  F  
\begin{filecontents*}{84-1.code}
@clef:G-2
@keysig:bB
@timesig:0
@data:2''C4CCCxDDC'B/''C'AFAB''CDC://:
\end{filecontents*}
\commandline{ verovio --spacing-non-linear=0.50 -w 1500 --spacing-system=0.5 --adjust-page-height -b 0 84-1.code }
\newline
\includesvg[width=220pt]{84-1}%

\newline RISM-ID: 455010870
\newline Alte Foliierung auf f.28v mit "29"
\newline D-Fh  Ms 001
\newline $\rightarrow$ In collection 418 (455008343)

\newline \par \vspace{7pt} \textcolor{darkblue}{\textbf{Anonymus  }}\hfillplus{85}
\newline Es ist gewisslich an der Zeit  G  
\newline org
\newline \begin{itshape}[f.11v, at left:] Fuga super | Es ist gewißlich | an der Zeit\end{itshape} 
\newline \textcolor{darkblue}{\ding{\numexpr181 + 01}}  1 parts  
\newline Manuscript copy
\newline 1.1.1  org  G  
\begin{filecontents*}{85-1.code}
@clef:F-4
@keysig:xF
@timesig:0
@data:4,G{8GB}{AGAA}4B{6GGB'xC}{8DECD}
\end{filecontents*}
\commandline{ verovio --spacing-non-linear=0.50 -w 1500 --spacing-system=0.5 --adjust-page-height -b 0 85-1.code }
\newline
\includesvg[width=220pt]{85-1}%

\newline RISM-ID: 455010819
\newline Alte Foliierung vorhanden: "12"
\newline D-Fh  Ms 001
\newline $\rightarrow$ In collection 418 (455008343)

\newline \par \vspace{7pt} \textcolor{darkblue}{\textbf{Anonymus  }}\hfillplus{86}
\newline Es ist gewisslich an der Zeit  C  
\newline org
\newline \begin{itshape}[f.22v, at left:] Es ist gewißlich | ander Zeit\end{itshape} 
\newline \textcolor{darkblue}{\ding{\numexpr181 + 01}}  1 parts  
\newline Manuscript copy
\newline 1.1.1  org  C  
\begin{filecontents*}{86-1.code}
@clef:G-2
@keysig:
@timesig:0
@data:2''C4CEDCDD2E/C4EDFE2DC://:
\end{filecontents*}
\commandline{ verovio --spacing-non-linear=0.50 -w 1500 --spacing-system=0.5 --adjust-page-height -b 0 86-1.code }
\newline
\includesvg[width=220pt]{86-1}%

\newline RISM-ID: 455010850
\newline Alte Foliierung auf den gegenüberliegenden Seiten als Aufschlagfoliierung jeweils "23"
\newline D-Fh  Ms 001
\newline $\rightarrow$ In collection 418 (455008343)

\newline \par \vspace{7pt} \textcolor{darkblue}{\textbf{Anonymus  }}\hfillplus{87}
\newline Es ist gewisslich an der Zeit  G  
\newline org
\newline \begin{itshape}[f.22v, at left:] Aliud.\end{itshape} 
\newline \textcolor{darkblue}{\ding{\numexpr181 + 01}}  1 parts  
\newline Manuscript copy
\newline 1.1.1  org  G  
\begin{filecontents*}{87-1.code}
@clef:G-2
@keysig:xF
@timesig:0
@data:2'G4GBAGAA2B/G4BA''C'B2AG://:
\end{filecontents*}
\commandline{ verovio --spacing-non-linear=0.50 -w 1500 --spacing-system=0.5 --adjust-page-height -b 0 87-1.code }
\newline
\includesvg[width=220pt]{87-1}%

\newline RISM-ID: 455010851
\newline Alte Foliierung auf den gegenüberliegenden Seiten als Aufschlagfoliierung jeweils "23"
\newline D-Fh  Ms 001
\newline $\rightarrow$ In collection 418 (455008343)

\newline \par \vspace{7pt} \textcolor{darkblue}{\textbf{Anonymus  }}\hfillplus{88}
\newline Es spricht der Unweisen Mund wohl  C  
\newline org
\newline \begin{itshape}[f.20v, at left:] Es spricht der | Unweisen mund | wohl\end{itshape} 
\newline \textcolor{darkblue}{\ding{\numexpr181 + 01}}  1 parts  
\newline Manuscript copy
\newline 1.1.1  org  C  
\begin{filecontents*}{88-1.code}
@clef:G-2
@keysig:
@timesig:0
@data:2''C4C'BG''CDE1(C)4'G''CDEF{8DC}4DC://:
\end{filecontents*}
\commandline{ verovio --spacing-non-linear=0.50 -w 1500 --spacing-system=0.5 --adjust-page-height -b 0 88-1.code }
\newline
\includesvg[width=220pt]{88-1}%

\newline RISM-ID: 455010841
\newline Alte Foliierungen vorhanden: "20" und "21"
\newline D-Fh  Ms 001
\newline $\rightarrow$ In collection 418 (455008343)

\newline \par \vspace{7pt} \textcolor{darkblue}{\textbf{Anonymus  }}\hfillplus{89}
\newline Es spricht der Unweisen Mund wohl    
\newline org
\newline \begin{itshape}[f.43r, at left:] Es spricht | der Unw.\end{itshape} 
\newline \textcolor{darkblue}{\ding{\numexpr181 + 01}}  1 parts  
\newline Manuscript copy
\newline 1.1.1  org  
\begin{filecontents*}{89-1.code}
@clef:G-2
@keysig:
@timesig:c/
@data:2'G4GFDGAB2G/4DGAB''C{8'AG}4AG://:
\end{filecontents*}
\commandline{ verovio --spacing-non-linear=0.50 -w 1500 --spacing-system=0.5 --adjust-page-height -b 0 89-1.code }
\newline
\includesvg[width=220pt]{89-1}%

\newline RISM-ID: 455010904
\newline Alte Foliierung auf den gegenüberliegenden Seiten als Aufschlagfoliierung jeweils "42"
\newline D-Fh  Ms 001
\newline $\rightarrow$ In collection 418 (455008343)

\newline \par \vspace{7pt} \textcolor{darkblue}{\textbf{Anonymus  }}\hfillplus{90}
\newline Es wollt' und Gott genädig sein  d  
\newline org
\newline \begin{itshape}[f.27v, at left:] Es wollt uns\end{itshape} 
\newline \textcolor{darkblue}{\ding{\numexpr181 + 01}}  1 parts  
\newline Manuscript copy
\newline 1.1.1  org  d  
\begin{filecontents*}{90-1.code}
@clef:G-2
@keysig:bB
@timesig:0
@data:2'A4BAGA2''CD4.C8'B2A/
\end{filecontents*}
\commandline{ verovio --spacing-non-linear=0.50 -w 1500 --spacing-system=0.5 --adjust-page-height -b 0 90-1.code }
\newline
\includesvg[width=220pt]{90-1}%

\newline RISM-ID: 455010866
\newline Alte Foliierung auf den gegenüberliegenden Seiten als Aufschlagfoliierung jeweils "28"
\newline D-Fh  Ms 001
\newline $\rightarrow$ In collection 418 (455008343)

\newline \par \vspace{7pt} \textcolor{darkblue}{\textbf{Anonymus  }}\hfillplus{91}
\newline Fanfares. Arr  C  
\newline i
\newline \begin{itshape}[f.76r, at left:] Tromp. | H. Heinrichs | Leibst.\end{itshape} 
\newline \textcolor{darkblue}{\ding{\numexpr181 + 01}}  1 parts  
\newline Manuscript copy
\newline 1.1.1  org  C  
\begin{filecontents*}{91-1.code}
@clef:G-2
@keysig:
@timesig:0
@data:{8''C6CC8CC}4CC{8.E6D}{CDEF}2G://:
\end{filecontents*}
\commandline{ verovio --spacing-non-linear=0.50 -w 1500 --spacing-system=0.5 --adjust-page-height -b 0 91-1.code }
\newline
\includesvg[width=220pt]{91-1}%

\newline RISM-ID: 455010953
\newline Zweistimmiges Fanfarenartiges Stück, dessen Besetzung Trompeten sein könnten, im Zusammenhang mit vorliegender Handschrift einem Tasteninstrument zugeordnet werden
\newline Ohne alte Foliierung
\newline D-Fh  Ms 001
\newline $\rightarrow$ In collection 418 (455008343)

\newline \par \vspace{7pt} \textcolor{darkblue}{\textbf{Anonymus  }}\hfillplus{92}
\newline Frohlockt  C  
\newline org
\newline \begin{itshape}[f.67v, at left:] frolockt\end{itshape} 
\newline \textcolor{darkblue}{\ding{\numexpr181 + 01}}  1 parts  
\newline Manuscript copy
\newline 1.1.1  org  C  
\begin{filecontents*}{92-1.code}
@clef:G-2
@keysig:
@timesig:c/
@data:{8''E6EG8ED}{CC}{E6EG8ED}{C6CG8ED}{CC}-E4FDExF2G://:
\end{filecontents*}
\commandline{ verovio --spacing-non-linear=0.50 -w 1500 --spacing-system=0.5 --adjust-page-height -b 0 92-1.code }
\newline
\includesvg[width=220pt]{92-1}%

\newline RISM-ID: 455010814
\newline Ohne alte Foliierung
\newline D-Fh  Ms 001
\newline $\rightarrow$ In collection 418 (455008343)

\newline \par \vspace{7pt} \textcolor{darkblue}{\textbf{Anonymus  }}\hfillplus{93}
\newline Fugues  G  
\newline org
\newline \begin{itshape}[f.70v, at left:] Fuga | i. c. g.\end{itshape} 
\newline \textcolor{darkblue}{\ding{\numexpr181 + 01}}  1 parts  
\newline Manuscript copy
\newline 1.1.1  org  G  
\begin{filecontents*}{93-1.code}
@clef:G-2
@keysig:xF
@timesig:0
@data:{8''D6DD}{'BGB''C}4D{8EE}6-{EDC}4D6-{DC'B}{8A6AA}4B
\end{filecontents*}
\commandline{ verovio --spacing-non-linear=0.50 -w 1500 --spacing-system=0.5 --adjust-page-height -b 0 93-1.code }
\newline
\includesvg[width=220pt]{93-1}%

\newline RISM-ID: 455010944
\newline Ohne alte Foliierung
\newline D-Fh  Ms 001
\newline $\rightarrow$ In collection 418 (455008343)

\newline \par \vspace{7pt} \textcolor{darkblue}{\textbf{Anonymus  }}\hfillplus{94}
\newline Fugues  C  
\newline org
\newline \begin{itshape}[f.76v, at left:] Fuga.\end{itshape} 
\newline \textcolor{darkblue}{\ding{\numexpr181 + 01}}  1 parts  
\newline Manuscript copy
\newline 1.1.1  org  C  
\begin{filecontents*}{94-1.code}
@clef:G-2
@keysig:
@timesig:0
@data:4'G{8GG}{E6CD}{ECEF}4G{8GG}{FDEF}{8G6FE}4D2E
\end{filecontents*}
\commandline{ verovio --spacing-non-linear=0.50 -w 1500 --spacing-system=0.5 --adjust-page-height -b 0 94-1.code }
\newline
\includesvg[width=220pt]{94-1}%

\newline RISM-ID: 455010954
\newline Ohne alte Foliierung
\newline D-Fh  Ms 001
\newline $\rightarrow$ In collection 418 (455008343)

\newline \par \vspace{7pt} \textcolor{darkblue}{\textbf{Anonymus  }}\hfillplus{95}
\newline Fugues  C  
\newline org
\newline \begin{itshape}[f.79v, at left:] Fuga. | [under the system:] Georg Michael Oster.\end{itshape} 
\newline \textcolor{darkblue}{\ding{\numexpr181 + 01}}  1 parts  
\newline Manuscript copy
\newline 1.1.1  org  C  
\begin{filecontents*}{95-1.code}
@clef:G-2
@keysig:
@timesig:0
@data:4'G{8EA}{6GFED}{8CF}2E{6FE8D}{6EDEF}{8GE}!4D{8ED}!f{6GFED}{8CF}
\end{filecontents*}
\commandline{ verovio --spacing-non-linear=0.50 -w 1500 --spacing-system=0.5 --adjust-page-height -b 0 95-1.code }
\newline
\includesvg[width=220pt]{95-1}%

\newline RISM-ID: 455010956
\newline Ohne alte Foliierung
\newline Oster, Georg Michael  (oth)
\newline D-Fh  Ms 001
\newline $\rightarrow$ In collection 418 (455008343)

\newline \par \vspace{7pt} \textcolor{darkblue}{\textbf{Anonymus  }}\hfillplus{96}
\newline Fugues  3t  
\newline org
\newline \begin{itshape}[f.81v, at left:] Fuga | 3. Toni.\end{itshape} 
\newline \textcolor{darkblue}{\ding{\numexpr181 + 01}}  1 parts  
\newline Manuscript copy
\newline 1.1.1  org  3t  
\begin{filecontents*}{96-1.code}
@clef:G-2
@keysig:
@timesig:0
@data:{6''E'BB''C}{D'AAB}4''C'B{6A''D}8D{8.6CC}4'BAGxF{6E,BB'C}{D,AAB}8'C{6DC}4,B2(x'C)
\end{filecontents*}
\commandline{ verovio --spacing-non-linear=0.50 -w 1500 --spacing-system=0.5 --adjust-page-height -b 0 96-1.code }
\newline
\includesvg[width=220pt]{96-1}%

\newline RISM-ID: 455010959
\newline Ohne alte Foliierung
\newline D-Fh  Ms 001
\newline $\rightarrow$ In collection 418 (455008343)

\newline \par \vspace{7pt} \textcolor{darkblue}{\textbf{Anonymus  }}\hfillplus{97}
\newline Fugues  C  
\newline org
\newline \begin{itshape}[f.84v, at left:] Fuga. | a 4 V.\end{itshape} 
\newline \textcolor{darkblue}{\ding{\numexpr181 + 01}}  1 parts  
\newline Manuscript copy
\newline 1.1.1  org  C  
\begin{filecontents*}{97-1.code}
@clef:G-2
@keysig:
@timesig:0
@data:{8'G6EF8GA}{GFED}{AG6AB8''C}{'BA}4G
\end{filecontents*}
\commandline{ verovio --spacing-non-linear=0.50 -w 1500 --spacing-system=0.5 --adjust-page-height -b 0 97-1.code }
\newline
\includesvg[width=220pt]{97-1}%

\newline RISM-ID: 455010964
\newline Ohne alte Foliierung
\newline D-Fh  Ms 001
\newline $\rightarrow$ In collection 418 (455008343)

\newline \par \vspace{7pt} \textcolor{darkblue}{\textbf{Anonymus  }}\hfillplus{98}
\newline Fugues  G  
\newline org
\newline \begin{itshape}[f.86r, at left:] Fuga.\end{itshape} 
\newline \textcolor{darkblue}{\ding{\numexpr181 + 01}}  1 parts  
\newline Manuscript copy
\newline 1.1.1  org  G  
\begin{filecontents*}{98-1.code}
@clef:G-2
@keysig:xF
@timesig:0
@data:{8''D6'GA}{BGB''C}2D{8C6DC}{8'BA}2BA
\end{filecontents*}
\commandline{ verovio --spacing-non-linear=0.50 -w 1500 --spacing-system=0.5 --adjust-page-height -b 0 98-1.code }
\newline
\includesvg[width=220pt]{98-1}%

\newline RISM-ID: 455010965
\newline Ohne alte Foliierung
\newline D-Fh  Ms 001
\newline $\rightarrow$ In collection 418 (455008343)

\newline \par \vspace{7pt} \textcolor{darkblue}{\textbf{Anonymus  }}\hfillplus{99}
\newline Fugues  G  
\newline org
\newline \begin{itshape}[f.88v, at left:] Fuga | ex G.\end{itshape} 
\newline \textcolor{darkblue}{\ding{\numexpr181 + 01}}  1 parts  
\newline Manuscript copy
\newline 1.1.1  org  G  
\begin{filecontents*}{99-1.code}
@clef:G-2
@keysig:xF
@timesig:0
@data:4'D,B{8'CxC}4D8-nC4C4,B
\end{filecontents*}
\commandline{ verovio --spacing-non-linear=0.50 -w 1500 --spacing-system=0.5 --adjust-page-height -b 0 99-1.code }
\newline
\includesvg[width=220pt]{99-1}%

\newline RISM-ID: 455010969
\newline Ohne alte Foliierung
\newline D-Fh  Ms 001
\newline $\rightarrow$ In collection 418 (455008343)

\newline \par \vspace{7pt} \textcolor{darkblue}{\textbf{Anonymus  }}\hfillplus{100}
\newline Galliards  C  
\newline pf
\newline \begin{itshape}[f.8r, at left:] 12. | Galliard\end{itshape} 
\newline \textcolor{darkblue}{\ding{\numexpr181 + 01}}  1 parts  
\newline Manuscript copy
\newline 1.1.1  pf  C  
\begin{filecontents*}{100-1.code}
@clef:G-2
@keysig:
@timesig:3
@data:!{8''CDEEEE}/4.E{8DExF}/4G4.G8xF/2.G/!f
\end{filecontents*}
\commandline{ verovio --spacing-non-linear=0.50 -w 1500 --spacing-system=0.5 --adjust-page-height -b 0 100-1.code }
\newline
\includesvg[width=220pt]{100-1}%

\newline RISM-ID: 455010812
\newline Alte Foliierung vorhanden: "8"
\newline D-Fh  Ms 001
\newline $\rightarrow$ In collection 418 (455008343)

\newline \par \vspace{7pt} \textcolor{darkblue}{\textbf{Anonymus  }}\hfillplus{101}
\newline Gelobet seist du Jesu Christ    
\newline org
\newline \begin{itshape}[f.24v, at left:] Gelobet seÿstu | Jesu Christ.\end{itshape} 
\newline \textcolor{darkblue}{\ding{\numexpr181 + 01}}  1 parts  
\newline Manuscript copy
\newline 1.1.1  org  
\begin{filecontents*}{101-1.code}
@clef:G-2
@keysig:
@timesig:0
@data:2'G4GGAG''CD(2C)4'B''D'BA2(G)4BAB''D'AGE(D)
\end{filecontents*}
\commandline{ verovio --spacing-non-linear=0.50 -w 1500 --spacing-system=0.5 --adjust-page-height -b 0 101-1.code }
\newline
\includesvg[width=220pt]{101-1}%

\newline RISM-ID: 455010856
\newline Alte Foliierung auf f.24v mit "25"
\newline D-Fh  Ms 001
\newline $\rightarrow$ In collection 418 (455008343)

\newline \par \vspace{7pt} \textcolor{darkblue}{\textbf{Anonymus  }}\hfillplus{102}
\newline Gewonnen der Satanas lieget. Arr    
\newline V, org
\newline \begin{itshape}[f.120v, heading:] Aria\end{itshape} 
\newline \textcolor{darkblue}{\ding{\numexpr181 + 01}}  1 parts  
\newline Manuscript copy
\newline 1.1.1  V with org
\newline \begin{footnotesize} Gewonnen der Satanas lieget \end{footnotesize}  
\begin{filecontents*}{102-1.code}
@clef:C-1
@keysig:
@timesig:6/4
@data:4--''D/D'B''DD'B''D/4.'A8G4xFGGG/AAAB{8Bx''C}4D/4.D8D4xCDD{8DC}/
\end{filecontents*}
\commandline{ verovio --spacing-non-linear=0.50 -w 1500 --spacing-system=0.5 --adjust-page-height -b 0 102-1.code }
\newline
\includesvg[width=220pt]{102-1}%

\newline RISM-ID: 455011031
\newline Zwischen den beiden Systemen 5 Textstrophen
\newline Alte Paginierung auf f.120v vorhanden: "75"
\newline D-Fh  Ms 001
\newline $\rightarrow$ In collection 418 (455008343)

\newline \par \vspace{7pt} \textcolor{darkblue}{\textbf{Anonymus  }}\hfillplus{103}
\newline Gigues  g  
\newline pf
\newline \begin{itshape}[f.92v, under the system:] Gigue ex G Mol\end{itshape} 
\newline \textcolor{darkblue}{\ding{\numexpr181 + 01}}  1 parts  
\newline Manuscript copy
\newline 1.1.1  pf  g  
\begin{filecontents*}{103-1.code}
@clef:C-1
@keysig:bBE
@timesig:6/4
@data:{8''ED}/4'B''C'B2A{8''ED}/4'B''C'B2A4''D/'BD''D'BD''D/i/
\end{filecontents*}
\commandline{ verovio --spacing-non-linear=0.50 -w 1500 --spacing-system=0.5 --adjust-page-height -b 0 103-1.code }
\newline
\includesvg[width=220pt]{103-1}%

\newline RISM-ID: 455010977
\newline Ohne alte Foliierung
\newline D-Fh  Ms 001
\newline $\rightarrow$ In collection 418 (455008343)

\newline \par \vspace{7pt} \textcolor{darkblue}{\textbf{Anonymus  }}\hfillplus{104}
\newline Gott Vater der du deine Sonn'  a  
\newline org
\newline \begin{itshape}[f.113v, heading:] Gott Vater der du deine Sonn φ\end{itshape} 
\newline \textcolor{darkblue}{\ding{\numexpr181 + 01}}  1 parts  
\newline Manuscript copy
\newline 1.1.1  org  a  
\begin{filecontents*}{104-1.code}
@clef:C-1
@keysig:
@timesig:c
@data:2'A4ExF/2xG4.A8B/4''C'B2(A)/A4B''C/D'B''C'B/2(A)A/
\end{filecontents*}
\commandline{ verovio --spacing-non-linear=0.50 -w 1500 --spacing-system=0.5 --adjust-page-height -b 0 104-1.code }
\newline
\includesvg[width=220pt]{104-1}%

\newline RISM-ID: 455011015
\newline Alte Paginierung vorhanden, f.113v: "93"
\newline D-Fh  Ms 001
\newline $\rightarrow$ In collection 418 (455008343)

\newline \par \vspace{7pt} \textcolor{darkblue}{\textbf{Anonymus  }}\hfillplus{105}
\newline Gott der Vater wohn uns bei  F  
\newline org
\newline \begin{itshape}[f.21v, at left:] Gott der Vatter | Wohn uns beÿ.\end{itshape} 
\newline \textcolor{darkblue}{\ding{\numexpr181 + 01}}  1 parts  
\newline Manuscript copy
\newline 1.1.1  org  F  
\begin{filecontents*}{105-1.code}
@clef:G-2
@keysig:bB
@timesig:0
@data:4''CCDEFF2(E)/4DFC'ABAG(F)://:
\end{filecontents*}
\commandline{ verovio --spacing-non-linear=0.50 -w 1500 --spacing-system=0.5 --adjust-page-height -b 0 105-1.code }
\newline
\includesvg[width=220pt]{105-1}%

\newline RISM-ID: 455010847
\newline Alte Foliierung auf den gegenüberliegenden Seiten als Aufschlagfoliierung jeweils "22"
\newline D-Fh  Ms 001
\newline $\rightarrow$ In collection 418 (455008343)

\newline \par \vspace{7pt} \textcolor{darkblue}{\textbf{Anonymus  }}\hfillplus{106}
\newline Gott der Vater wohn uns bei  C  
\newline org
\newline \begin{itshape}[f.41v, at left:] Gott der | Vatter wohn\end{itshape} 
\newline \textcolor{darkblue}{\ding{\numexpr181 + 01}}  1 parts  
\newline Manuscript copy
\newline 1.1.1  org  C  
\begin{filecontents*}{106-1.code}
@clef:G-2
@keysig:
@timesig:0
@data:4'GGAB''CC'B/A''C'GEFED2C://:
\end{filecontents*}
\commandline{ verovio --spacing-non-linear=0.50 -w 1500 --spacing-system=0.5 --adjust-page-height -b 0 106-1.code }
\newline
\includesvg[width=220pt]{106-1}%

\newline RISM-ID: 455010901
\newline Alte Foliierung auf f.41v mit "41"
\newline D-Fh  Ms 001
\newline $\rightarrow$ In collection 418 (455008343)

\newline \par \vspace{7pt} \textcolor{darkblue}{\textbf{Anonymus  }}\hfillplus{107}
\newline Gott geb der Braut und Bräutigam. Arr  C  
\newline V, org
\newline \begin{itshape}[without title]\end{itshape} 
\newline \textcolor{darkblue}{\ding{\numexpr181 + 01}}  1 parts  
\newline Manuscript copy
\newline 1.1.1  V  C
\newline \begin{footnotesize} Gott geb der Braut und Bräutigam \end{footnotesize}  
\begin{filecontents*}{107-1.code}
@clef:G-2
@keysig:
@timesig:3
@data:4''CCC/2C4D/{8EE}4E-/CD'B/{8''C'B''CD}4E/DC'B/2''C4-/
\end{filecontents*}
\commandline{ verovio --spacing-non-linear=0.50 -w 1500 --spacing-system=0.5 --adjust-page-height -b 0 107-1.code }
\newline
\includesvg[width=220pt]{107-1}%

\newline RISM-ID: 455010909
\newline Unter den Tabulaturzeilen drei Strophen Gesangstext
\newline Alte Foliierung auf f.48v mit "45"
\newline D-Fh  Ms 001
\newline $\rightarrow$ In collection 418 (455008343)

\newline \par \vspace{7pt} \textcolor{darkblue}{\textbf{Anonymus  }}\hfillplus{108}
\newline Gott hat das Evangelium  a  
\newline org
\newline \begin{itshape}[f.22v, at left:] Gott hatt daß | Evangelium\end{itshape} 
\newline \textcolor{darkblue}{\ding{\numexpr181 + 01}}  1 parts  
\newline Manuscript copy
\newline 1.1.1  org  a  
\begin{filecontents*}{108-1.code}
@clef:G-2
@keysig:
@timesig:0
@data:2'A4AA''C'AGA2(F)://:4ABAGFAFD://:
\end{filecontents*}
\commandline{ verovio --spacing-non-linear=0.50 -w 1500 --spacing-system=0.5 --adjust-page-height -b 0 108-1.code }
\newline
\includesvg[width=220pt]{108-1}%

\newline RISM-ID: 455010849
\newline Alte Foliierung auf den gegenüberliegenden Seiten als Aufschlagfoliierung jeweils "23"
\newline D-Fh  Ms 001
\newline $\rightarrow$ In collection 418 (455008343)

\newline \par \vspace{7pt} \textcolor{darkblue}{\textbf{Anonymus  }}\hfillplus{109}
\newline Gott ist mein Heil mein' Hilf' und Trost  G  
\newline org
\newline \begin{itshape}[f.35v, at left:] Gott ist mein | Heÿl mein Hülff | mein Trost\end{itshape} 
\newline \textcolor{darkblue}{\ding{\numexpr181 + 01}}  1 parts  
\newline Manuscript copy
\newline 1.1.1  org  G  
\begin{filecontents*}{109-1.code}
@clef:G-2
@keysig:xF
@timesig:0
@data:4-'DGAB''C8D4C8'B4''C8-'AB4''CD'B{8AG}4AG://:
\end{filecontents*}
\commandline{ verovio --spacing-non-linear=0.50 -w 1500 --spacing-system=0.5 --adjust-page-height -b 0 109-1.code }
\newline
\includesvg[width=220pt]{109-1}%

\newline RISM-ID: 455010887
\newline Alte Foliierung auf f.35v mit "36"
\newline D-Fh  Ms 001
\newline $\rightarrow$ In collection 418 (455008343)

\newline \par \vspace{7pt} \textcolor{darkblue}{\textbf{Anonymus  }}\hfillplus{110}
\newline Gott lobet noch  G  
\newline org
\newline \begin{itshape}[f.113r, heading:] Gott lobet noch!\end{itshape} 
\newline \textcolor{darkblue}{\ding{\numexpr181 + 01}}  1 parts  
\newline Manuscript copy
\newline 1.1.1  org  G  
\begin{filecontents*}{110-1.code}
@clef:C-1
@keysig:xF
@timesig:c
@data:2''D/'BA/1(G)/2''D'B/''ED/CD/1('B)://:
\end{filecontents*}
\commandline{ verovio --spacing-non-linear=0.50 -w 1500 --spacing-system=0.5 --adjust-page-height -b 0 110-1.code }
\newline
\includesvg[width=220pt]{110-1}%

\newline RISM-ID: 455011014
\newline Alte Paginierung vorhanden, f.113r: "92"
\newline D-Fh  Ms 001
\newline $\rightarrow$ In collection 418 (455008343)

\newline \par \vspace{7pt} \textcolor{darkblue}{\textbf{Anonymus  }}\hfillplus{111}
\newline Gott sei gelobet und gebenedeiet    
\newline org
\newline \begin{itshape}[f.34v, at left:] Gott seÿ gelobet | U. gebenedeÿet.\end{itshape} 
\newline \textcolor{darkblue}{\ding{\numexpr181 + 01}}  1 parts  
\newline Manuscript copy
\newline 1.1.1  org  
\begin{filecontents*}{111-1.code}
@clef:G-2
@keysig:
@timesig:0
@data:4'G{8GG}4AG{8.6''CDC'B}4A(G)A''C'GFEDCC://:
\end{filecontents*}
\commandline{ verovio --spacing-non-linear=0.50 -w 1500 --spacing-system=0.5 --adjust-page-height -b 0 111-1.code }
\newline
\includesvg[width=220pt]{111-1}%

\newline RISM-ID: 455010885
\newline Alte Foliierung auf den gegenüberliegenden Seiten als Aufschlagfoliierung jeweils "35"
\newline D-Fh  Ms 001
\newline $\rightarrow$ In collection 418 (455008343)

\newline \par \vspace{7pt} \textcolor{darkblue}{\textbf{Anonymus  }}\hfillplus{112}
\newline Gott sei gelobet und gebenedeiet  G  
\newline V, org
\newline \begin{itshape}[f.96r, at left:] ex G.\end{itshape} 
\newline \textcolor{darkblue}{\ding{\numexpr181 + 01}}  1 parts  
\newline Manuscript copy
\newline 1.1.1  V  G
\newline \begin{footnotesize} Gott sei gelobet und gebenedeiet \end{footnotesize}  
\begin{filecontents*}{112-1.code}
@clef:C-1
@keysig:xF
@timesig:c
@data:2'G4GG/2AG/''CD/C'B/1At/(G)/
\end{filecontents*}
\commandline{ verovio --spacing-non-linear=0.50 -w 1500 --spacing-system=0.5 --adjust-page-height -b 0 112-1.code }
\newline
\includesvg[width=220pt]{112-1}%

\newline RISM-ID: 455010984
\newline Unter den beiden Systemen 3 Textstrophen
\newline Alte Paginierung auf f.98v vorhanden: "67"
\newline D-Fh  Ms 001
\newline $\rightarrow$ In collection 418 (455008343)

\newline \par \vspace{7pt} \textcolor{darkblue}{\textbf{Anonymus  }}\hfillplus{113}
\newline Gottes Sohn ist kommen  F  
\newline org
\newline \begin{itshape}[f.21v, at left:] Gottes Sohn | ist kom̅en.\end{itshape} 
\newline \textcolor{darkblue}{\ding{\numexpr181 + 01}}  1 parts  
\newline Manuscript copy
\newline 1.1.1  org  F  
\begin{filecontents*}{113-1.code}
@clef:G-2
@keysig:bB
@timesig:0
@data:4'FFAB''CD2C4DEFC'AB2''C
\end{filecontents*}
\commandline{ verovio --spacing-non-linear=0.50 -w 1500 --spacing-system=0.5 --adjust-page-height -b 0 113-1.code }
\newline
\includesvg[width=220pt]{113-1}%

\newline RISM-ID: 455010848
\newline Alte Foliierung auf den gegenüberliegenden Seiten als Aufschlagfoliierung jeweils "22"
\newline D-Fh  Ms 001
\newline $\rightarrow$ In collection 418 (455008343)

\newline \par \vspace{7pt} \textcolor{darkblue}{\textbf{Anonymus  }}\hfillplus{114}
\newline Helft mir Gottes Güte preisen  a  
\newline org
\newline \begin{itshape}[f.40v, at left:] Helfft mir Gotts | Güte preisen.\end{itshape} 
\newline \textcolor{darkblue}{\ding{\numexpr181 + 01}}  1 parts  
\newline Manuscript copy
\newline 1.1.1  org  a  
\begin{filecontents*}{114-1.code}
@clef:G-2
@keysig:
@timesig:0
@data:4-'AAB''C'A2B4GGAABB2E://:
\end{filecontents*}
\commandline{ verovio --spacing-non-linear=0.50 -w 1500 --spacing-system=0.5 --adjust-page-height -b 0 114-1.code }
\newline
\includesvg[width=220pt]{114-1}%

\newline RISM-ID: 455010897
\newline Alte Foliierung auf f.40v mit "40"
\newline D-Fh  Ms 001
\newline $\rightarrow$ In collection 418 (455008343)

\newline \par \vspace{7pt} \textcolor{darkblue}{\textbf{Anonymus  }}\hfillplus{115}
\newline Herr Gott dich loben alle wir  B|b  
\newline org
\newline \begin{itshape}[f.14v, at the tail of the page:] Aliud Melod: ex b. Herr Gott dich loben | alle wir\end{itshape} 
\newline \textcolor{darkblue}{\ding{\numexpr181 + 01}}  1 parts  
\newline Manuscript copy
\newline 1.1.1  org  B|b  
\begin{filecontents*}{115-1.code}
@clef:C-1
@keysig:bBE
@timesig:3/2
@data:2--'B/12BA/GF/B''C/(D)D/DD/C'B/''ED/(C)'B/
\end{filecontents*}
\commandline{ verovio --spacing-non-linear=0.50 -w 1500 --spacing-system=0.5 --adjust-page-height -b 0 115-1.code }
\newline
\includesvg[width=220pt]{115-1}%

\newline RISM-ID: 455010827
\newline Die einstimmige Alternativfassung steht im C-Schlüssel und ist von anderer Hand geschrieben als die übrigen Einträge
\newline D-Fh  Ms 001
\newline $\rightarrow$ In collection 418 (455008343)

\newline \par \vspace{7pt} \textcolor{darkblue}{\textbf{Anonymus  }}\hfillplus{116}
\newline Herr Gott dich loben alle wir  G  
\newline org
\newline \begin{itshape}[f.15r, at left:] Herr Gott dich | Loben alle | wir\end{itshape} 
\newline \textcolor{darkblue}{\ding{\numexpr181 + 01}}  1 parts  
\newline Manuscript copy
\newline 1.1.1  org  G  
\begin{filecontents*}{116-1.code}
@clef:G-2
@keysig:xF
@timesig:0
@data:4-'G4.G8F4GA{8FDEE}4D/DGAG''C{8'BGAA}4G/GG{8FE+}{ED}4GAB/
\end{filecontents*}
\commandline{ verovio --spacing-non-linear=0.50 -w 1500 --spacing-system=0.5 --adjust-page-height -b 0 116-1.code }
\newline
\includesvg[width=220pt]{116-1}%

\newline RISM-ID: 455010828
\newline Alte Foliierung vorhanden: "15"
\newline D-Fh  Ms 001
\newline $\rightarrow$ In collection 418 (455008343)

\newline \par \vspace{7pt} \textcolor{darkblue}{\textbf{Anonymus  }}\hfillplus{117}
\newline Herr Gott dich loben wir  3t  
\newline org
\newline \begin{itshape}[f.36v, at left:] Herr Gott dich | loben wir.\end{itshape} 
\newline \textcolor{darkblue}{\ding{\numexpr181 + 01}}  1 parts  
\newline Manuscript copy
\newline 1.1.1  org  3t  
\begin{filecontents*}{117-1.code}
@clef:G-2
@keysig:
@timesig:0
@data:2'EGA''C'AA
\end{filecontents*}
\commandline{ verovio --spacing-non-linear=0.50 -w 1500 --spacing-system=0.5 --adjust-page-height -b 0 117-1.code }
\newline
\includesvg[width=220pt]{117-1}%

\newline RISM-ID: 455010890
\newline Das Stück ist als alternatim-Gesang in einstimmige und vierstimmige Abschnitte aufgeteilt
\newline Alte Foliierung auf f.36v mit "37"
\newline D-Fh  Ms 001
\newline $\rightarrow$ In collection 418 (455008343)

\newline \par \vspace{7pt} \textcolor{darkblue}{\textbf{Anonymus  }}\hfillplus{118}
\newline Herr Gott dich loben wir. Arr    
\newline V, org
\newline \begin{itshape}[without title]\end{itshape} 
\newline \textcolor{darkblue}{\ding{\numexpr181 + 01}}  1 parts  
\newline Manuscript copy
\newline 1.1.1  org
\newline \begin{footnotesize} Herr Gott dich loben wir \end{footnotesize}  
\begin{filecontents*}{118-1.code}
@clef:C-1
@keysig:
@timesig:c
@data:1'EGA''C'A(A)/''C2C'BAA1(G)/''CC'BABA(A)/
\end{filecontents*}
\commandline{ verovio --spacing-non-linear=0.50 -w 1500 --spacing-system=0.5 --adjust-page-height -b 0 118-1.code }
\newline
\includesvg[width=220pt]{118-1}%

\newline RISM-ID: 455011030
\newline Zwischen den beiden Systemen eine Textstrophe auf f.119r, f.119v 5 Textstrophen
\newline Alte Paginierung auf f.119v vorhanden: "74"
\newline D-Fh  Ms 001
\newline $\rightarrow$ In collection 418 (455008343)

\newline \par \vspace{7pt} \textcolor{darkblue}{\textbf{Anonymus  }}\hfillplus{119}
\newline Herr Gott du bist von Ewigkeit  G  
\newline org
\newline \begin{itshape}[f.71v, at left:] Herr Gott du | bist von Ewig= | keit\end{itshape} 
\newline \textcolor{darkblue}{\ding{\numexpr181 + 01}}  1 parts  
\newline Manuscript copy
\newline 1.1.1  org  G  
\begin{filecontents*}{119-1.code}
@clef:G-2
@keysig:xF
@timesig:0
@data:4-8-''D{'BGAB}{GG}4F8F4F8E{DG}4A4.B://:
\end{filecontents*}
\commandline{ verovio --spacing-non-linear=0.50 -w 1500 --spacing-system=0.5 --adjust-page-height -b 0 119-1.code }
\newline
\includesvg[width=220pt]{119-1}%

\newline RISM-ID: 455010947
\newline Alte Foliierung auf f.71v mit "60"
\newline D-Fh  Ms 001
\newline $\rightarrow$ In collection 418 (455008343)

\newline \par \vspace{7pt} \textcolor{darkblue}{\textbf{Anonymus  }}\hfillplus{120}
\newline Herr Gott nun schleuß den Himmel auf  a  
\newline org
\newline \begin{itshape}[f.41v, at left:] Herr Gott | nun schleuß\end{itshape} 
\newline \textcolor{darkblue}{\ding{\numexpr181 + 01}}  1 parts  
\newline Manuscript copy
\newline 1.1.1  org  a  
\begin{filecontents*}{120-1.code}
@clef:G-2
@keysig:
@timesig:0
@data:4'AxG8A4B8''C{'AA}4B8''C4'G8''CE4E8xD2E://:
\end{filecontents*}
\commandline{ verovio --spacing-non-linear=0.50 -w 1500 --spacing-system=0.5 --adjust-page-height -b 0 120-1.code }
\newline
\includesvg[width=220pt]{120-1}%

\newline RISM-ID: 455010900
\newline Alte Foliierung auf f.41v mit "41"
\newline D-Fh  Ms 001
\newline $\rightarrow$ In collection 418 (455008343)

\newline \par \vspace{7pt} \textcolor{darkblue}{\textbf{Anonymus  }}\hfillplus{121}
\newline Herr Jesu Christ wahr' Mensch und Gott    
\newline org
\newline \begin{itshape}[f.29v, at left:] HERR JESU CHRIST | wahr Mensch und Gott\end{itshape} 
\newline \textcolor{darkblue}{\ding{\numexpr181 + 01}}  1 parts  
\newline Manuscript copy
\newline 1.1.1  org  
\begin{filecontents*}{121-1.code}
@clef:G-2
@keysig:
@timesig:0
@data:2'E4EF2GE4FG2(E)/E4GA2''C'B4AA2(E)/
\end{filecontents*}
\commandline{ verovio --spacing-non-linear=0.50 -w 1500 --spacing-system=0.5 --adjust-page-height -b 0 121-1.code }
\newline
\includesvg[width=220pt]{121-1}%

\newline RISM-ID: 455010872
\newline Alte Foliierung auf den gegenüberliegenden Seiten als Aufschlagfoliierung jeweils "30"
\newline D-Fh  Ms 001
\newline $\rightarrow$ In collection 418 (455008343)

\newline \par \vspace{7pt} \textcolor{darkblue}{\textbf{Anonymus  }}\hfillplus{122}
\newline Herr Jesu Lebenssonne. Excerpts. Arr    
\newline V, org
\newline \begin{itshape}[f.123v, between the systems:] H Jesu lebens Sonne.\end{itshape} 
\newline \textcolor{darkblue}{\ding{\numexpr181 + 01}}  1 parts  
\newline Manuscript copy
\newline 1.1.1  V  D
\newline \begin{footnotesize} Herr Jesu Lebenssonne \end{footnotesize}  
\begin{filecontents*}{122-1.code}
@clef:C-1
@keysig:[xFC]
@timesig:c
@data:1'A2BAAGFD/1''D2CC'BB1A://:
\end{filecontents*}
\commandline{ verovio --spacing-non-linear=0.50 -w 1500 --spacing-system=0.5 --adjust-page-height -b 0 122-1.code }
\newline
\includesvg[width=220pt]{122-1}%

\newline RISM-ID: 455011039
\newline Zwischen den Systemen nur der Textbeginn wie angegeben
\newline Alte Paginierung auf f.123v vorhanden: "80"
\newline D-Fh  Ms 001
\newline $\rightarrow$ In collection 418 (455008343)

\newline \par \vspace{7pt} \textcolor{darkblue}{\textbf{Anonymus  }}\hfillplus{123}
\newline Herr hader' mit den Had'rern mein    
\newline org
\newline \begin{itshape}[f.71v, at left:] Herr, hader mit | den Hadrern mein\end{itshape} 
\newline \textcolor{darkblue}{\ding{\numexpr181 + 01}}  1 parts  
\newline Manuscript copy
\newline 1.1.1  org  
\begin{filecontents*}{123-1.code}
@clef:G-2
@keysig:
@timesig:c/
@data:4'B''C8'B4A8B8.G6A4F/8G4G8A{B''D}4'A2(G)://:
\end{filecontents*}
\commandline{ verovio --spacing-non-linear=0.50 -w 1500 --spacing-system=0.5 --adjust-page-height -b 0 123-1.code }
\newline
\includesvg[width=220pt]{123-1}%

\newline RISM-ID: 455010945
\newline Alte Foliierung auf f.71v mit "60"
\newline D-Fh  Ms 001
\newline $\rightarrow$ In collection 418 (455008343)

\newline \par \vspace{7pt} \textcolor{darkblue}{\textbf{Anonymus  }}\hfillplus{124}
\newline Herr vergib' uns unsre Sünde  G  
\newline org
\newline \begin{itshape}[f.13v, at left:] HE Vgib | uns unsre | Sünde\end{itshape} 
\newline \textcolor{darkblue}{\ding{\numexpr181 + 01}}  1 parts  
\newline Manuscript copy
\newline 1.1.1  org  G  
\begin{filecontents*}{124-1.code}
@clef:G-2
@keysig:xF
@timesig:0
@data:4'GGAB''C'B2A(G)4A''C'GGFED(C)
\end{filecontents*}
\commandline{ verovio --spacing-non-linear=0.50 -w 1500 --spacing-system=0.5 --adjust-page-height -b 0 124-1.code }
\newline
\includesvg[width=220pt]{124-1}%

\newline RISM-ID: 455010824
\newline D-Fh  Ms 001
\newline $\rightarrow$ In collection 418 (455008343)

\newline \par \vspace{7pt} \textcolor{darkblue}{\textbf{Anonymus  }}\hfillplus{125}
\newline Herr von uns nimme deinen Zorn und Grimme. Arr  G  
\newline V, org
\newline \begin{itshape}[without title]\end{itshape} 
\newline \textcolor{darkblue}{\ding{\numexpr181 + 01}}  1 parts  
\newline Manuscript copy
\newline 1.1.1  V  G
\newline \begin{footnotesize} Herr von uns nimme deinen Zorn und Grimme \end{footnotesize}  
\begin{filecontents*}{125-1.code}
@clef:C-1
@keysig:xF
@timesig:c
@data:2'GFGAG/AABA1G2F/1GF2FG1AB/2GFE4FG2A1G/
\end{filecontents*}
\commandline{ verovio --spacing-non-linear=0.50 -w 1500 --spacing-system=0.5 --adjust-page-height -b 0 125-1.code }
\newline
\includesvg[width=220pt]{125-1}%

\newline RISM-ID: 455011032
\newline Zwischen den Systemen eine Textstrophe
\newline Alte Paginierung auf f.121r vorhanden: "76"
\newline D-Fh  Ms 001
\newline $\rightarrow$ In collection 418 (455008343)

\newline \par \vspace{7pt} \textcolor{darkblue}{\textbf{Anonymus  }}\hfillplus{126}
\newline Hilf Gott daß mir's gelinge  1t  
\newline org
\newline \begin{itshape}[f.35v, at left:] Hilf Gott das | mirs gelinge\end{itshape} 
\newline \textcolor{darkblue}{\ding{\numexpr181 + 01}}  1 parts  
\newline Manuscript copy
\newline 1.1.1  org  1t  
\begin{filecontents*}{126-1.code}
@clef:G-2
@keysig:bB
@timesig:c/
@data:2'G4AGFD2FG/GB4''D2C4'BAG://:
\end{filecontents*}
\commandline{ verovio --spacing-non-linear=0.50 -w 1500 --spacing-system=0.5 --adjust-page-height -b 0 126-1.code }
\newline
\includesvg[width=220pt]{126-1}%

\newline RISM-ID: 455010889
\newline Alte Foliierung auf f.35v mit "36"
\newline D-Fh  Ms 001
\newline $\rightarrow$ In collection 418 (455008343)

\newline \par \vspace{7pt} \textcolor{darkblue}{\textbf{Anonymus  }}\hfillplus{127}
\newline Hilf Jesu daß wir allzumal  C  
\newline V, org
\newline \begin{itshape}[without title]\end{itshape} 
\newline \textcolor{darkblue}{\ding{\numexpr181 + 01}}  1 parts  
\newline Manuscript copy
\newline 1.1.1  V  C
\newline \begin{footnotesize} Hilf Jesu daß wir allzumal \end{footnotesize}  
\begin{filecontents*}{127-1.code}
@clef:C-1
@keysig:
@timesig:c/
@data:8''C/4EC'BG/AB4.''(C)8'G/4AGFE/4.D8C2C://:
\end{filecontents*}
\commandline{ verovio --spacing-non-linear=0.50 -w 1500 --spacing-system=0.5 --adjust-page-height -b 0 127-1.code }
\newline
\includesvg[width=220pt]{127-1}%

\newline RISM-ID: 455010998
\newline Zwischen den beiden Systemen 3 Textstrophen
\newline Alte Zählung auf f.104v vorhanden: "75"
\newline D-Fh  Ms 001
\newline $\rightarrow$ In collection 418 (455008343)

\newline \par \vspace{7pt} \textcolor{darkblue}{\textbf{Anonymus  }}\hfillplus{128}
\newline Ich hab' mein' Sach' Gott heimgestellt  1t  
\newline org
\newline \begin{itshape}[f.28v, at left:] Ich hab mein | Sach Gott heim | gestellt\end{itshape} 
\newline \textcolor{darkblue}{\ding{\numexpr181 + 01}}  1 parts  
\newline Manuscript copy
\newline 1.1.1  org  1t  
\begin{filecontents*}{128-1.code}
@clef:G-2
@keysig:
@timesig:0
@data:4'GB8B4A8''D4C'BA8-{AAA}4B''C{8'AA}4G
\end{filecontents*}
\commandline{ verovio --spacing-non-linear=0.50 -w 1500 --spacing-system=0.5 --adjust-page-height -b 0 128-1.code }
\newline
\includesvg[width=220pt]{128-1}%

\newline RISM-ID: 455010869
\newline Alte Foliierung auf f.28v mit "29"
\newline D-Fh  Ms 001
\newline $\rightarrow$ In collection 418 (455008343)

\newline \par \vspace{7pt} \textcolor{darkblue}{\textbf{Anonymus  }}\hfillplus{129}
\newline Ich steh' an deiner Krippen hier. Arr  d  
\newline V, org
\newline \begin{itshape}[f.100r, at the end:] 1713.\end{itshape} 
\newline \textcolor{darkblue}{\ding{\numexpr181 + 01}}  1713 (1713)  1 parts  
\newline Manuscript copy
\newline 1.1.1  V  d
\newline \begin{footnotesize} Ich steh' an deiner Krippen hier \end{footnotesize}  
\begin{filecontents*}{129-1.code}
@clef:C-1
@keysig:bB
@timesig:c
@data:2'D4FA/DD4.D8xC/1(D)/2F4G''C/'A{8BA}2G/1(F)://:
\end{filecontents*}
\commandline{ verovio --spacing-non-linear=0.50 -w 1500 --spacing-system=0.5 --adjust-page-height -b 0 129-1.code }
\newline
\includesvg[width=220pt]{129-1}%

\newline RISM-ID: 455010986
\newline Unter den beiden Systemen 15 Textstrophen durch Zählung vorgesehen, jedoh nur 8 Textstrophen vorhanden
\newline Alte Foliierung auf f.99v vorhanden: "68"
\newline D-Fh  Ms 001
\newline $\rightarrow$ In collection 418 (455008343)

\newline \par \vspace{7pt} \textcolor{darkblue}{\textbf{Anonymus  }}\hfillplus{130}
\newline Ich weiß dass mein Erlöser lebt  3t  
\newline org
\newline \begin{itshape}[f.112v, heading:] Ich weiß daß mein Erloser lebt φ\end{itshape} 
\newline \textcolor{darkblue}{\ding{\numexpr181 + 01}}  1 parts  
\newline Manuscript copy
\newline 1.1.1  org  3t  
\begin{filecontents*}{130-1.code}
@clef:C-1
@keysig:
@timesig:c/
@data:2'AA4G2F4GAB2''C4'AABA2GG/A4GGGF2E1E://:
\end{filecontents*}
\commandline{ verovio --spacing-non-linear=0.50 -w 1500 --spacing-system=0.5 --adjust-page-height -b 0 130-1.code }
\newline
\includesvg[width=220pt]{130-1}%

\newline RISM-ID: 455011013
\newline Alte Paginierung vorhanden, f.112v: "91"
\newline D-Fh  Ms 001
\newline $\rightarrow$ In collection 418 (455008343)

\newline \par \vspace{7pt} \textcolor{darkblue}{\textbf{Anonymus  }}\hfillplus{131}
\newline Ihr hohen Berg' ihr lehret mich. Arr  D  
\newline V, org
\newline \begin{itshape}[without title]\end{itshape} 
\newline \textcolor{darkblue}{\ding{\numexpr181 + 01}}  1 parts  
\newline Manuscript copy
\newline 1.1.1  V  D
\newline \begin{footnotesize} Ihr hohen Berg ihr lehret mich \end{footnotesize}  
\begin{filecontents*}{131-1.code}
@clef:C-1
@keysig:xFC
@timesig:c
@data:1'D/2AB/AG/FE/1(D)/2A4AA/2BA/B''C/1(D)/
\end{filecontents*}
\commandline{ verovio --spacing-non-linear=0.50 -w 1500 --spacing-system=0.5 --adjust-page-height -b 0 131-1.code }
\newline
\includesvg[width=220pt]{131-1}%

\newline RISM-ID: 455010997
\newline Zwischen den beiden Systemen nur der Textbeginn wie angegeben
\newline Alte Zählung auf f.104r vorhanden: "74"
\newline D-Fh  Ms 001
\newline $\rightarrow$ In collection 418 (455008343)

\newline \par \vspace{7pt} \textcolor{darkblue}{\textbf{Anonymus  }}\hfillplus{132}
\newline In dich hab' ich gehoffet Herr  F  
\newline org
\newline \begin{itshape}[f.19v, at left:] In dich hab ich | gehoffet Herr\end{itshape} 
\newline \textcolor{darkblue}{\ding{\numexpr181 + 01}}  1 parts  
\newline Manuscript copy
\newline 1.1.1  org  F  
\begin{filecontents*}{132-1.code}
@clef:G-2
@keysig:bB
@timesig:0
@data:2'F4F''CC{8'BA}4GABG2(F)/4FGABG''C+CDD(C)/
\end{filecontents*}
\commandline{ verovio --spacing-non-linear=0.50 -w 1500 --spacing-system=0.5 --adjust-page-height -b 0 132-1.code }
\newline
\includesvg[width=220pt]{132-1}%

\newline RISM-ID: 455010839
\newline Alte Foliierung vorhanden: "19"
\newline D-Fh  Ms 001
\newline $\rightarrow$ In collection 418 (455008343)

\newline \par \vspace{7pt} \textcolor{darkblue}{\textbf{Anonymus  }}\hfillplus{133}
\newline In dich hab' ich gehoffet Herr. Excerpts. Arr    
\newline V
\newline \begin{itshape}[without title]\end{itshape} 
\newline \textcolor{darkblue}{\ding{\numexpr181 + 01}}  1 parts  
\newline Manuscript copy
\newline 1.1.1  V
\newline \begin{footnotesize} In dich hab' ich gehoffet Herr \end{footnotesize}  
\begin{filecontents*}{133-1.code}
@clef:C-1
@keysig:
@timesig:c
@data:1'D2DAEGF4DxC1(D)/D2AB''C'BAxG(1A)/
\end{filecontents*}
\commandline{ verovio --spacing-non-linear=0.50 -w 1500 --spacing-system=0.5 --adjust-page-height -b 0 133-1.code }
\newline
\includesvg[width=220pt]{133-1}%

\newline RISM-ID: 455010993
\newline Nur Melodie und eine Textstrophe ausgeführt
\newline Alte Zählung auf f.103v vorhanden: "73"
\newline D-Fh  Ms 001
\newline $\rightarrow$ In collection 418 (455008343)

\newline \par \vspace{7pt} \textcolor{darkblue}{\textbf{Anonymus  }}\hfillplus{134}
\newline In dich hab' ich gehoffet Herr  F  
\newline org
\newline \begin{itshape}[f.108v, heading:] In dich hab ich gehoffet Herr.\end{itshape} 
\newline \textcolor{darkblue}{\ding{\numexpr181 + 01}}  1 parts  
\newline Manuscript copy
\newline 1.1.1  org  F  
\begin{filecontents*}{134-1.code}
@clef:C-1
@keysig:bB
@timesig:c
@data:=1/2-4-'F/2F''C/2.C{8'BA}/2GA/BG/2F4-F/2GA/BA/
\end{filecontents*}
\commandline{ verovio --spacing-non-linear=0.50 -w 1500 --spacing-system=0.5 --adjust-page-height -b 0 134-1.code }
\newline
\includesvg[width=220pt]{134-1}%

\newline RISM-ID: 455011005
\newline Alte Paginierung vorhanden, f.108v: "83"
\newline D-Fh  Ms 001
\newline $\rightarrow$ In collection 418 (455008343)

\newline \par \vspace{7pt} \textcolor{darkblue}{\textbf{Anonymus  }}\hfillplus{135}
\newline Jesu der du meine Seele  1t  
\newline org
\newline \begin{itshape}[f.53v, at left:] Jesu der | du meine | Seele.\end{itshape} 
\newline \textcolor{darkblue}{\ding{\numexpr181 + 01}}  1 parts  
\newline Manuscript copy
\newline 1.1.1  org  1t  
\begin{filecontents*}{135-1.code}
@clef:G-2
@keysig:bB
@timesig:0
@data:{8''DD'AB}{''C'BAG}{GBAG}{xFG}4A://:
\end{filecontents*}
\commandline{ verovio --spacing-non-linear=0.50 -w 1500 --spacing-system=0.5 --adjust-page-height -b 0 135-1.code }
\newline
\includesvg[width=220pt]{135-1}%

\newline RISM-ID: 455010917
\newline Alte Foliierung auf f.53v mit "47"
\newline D-Fh  Ms 001
\newline $\rightarrow$ In collection 418 (455008343)

\newline \par \vspace{7pt} \textcolor{darkblue}{\textbf{Anonymus  }}\hfillplus{136}
\newline Jesu der du meine Seele  1t  
\newline org
\newline \begin{itshape}[f.66v, at left:] .\end{itshape} 
\newline \textcolor{darkblue}{\ding{\numexpr181 + 01}}  1 parts  
\newline Manuscript copy
\newline 1.1.1  org  
\begin{filecontents*}{136-1.code}
@clef:G-2
@keysig:
@timesig:0
@data:{8''CC'BG}{AB''CC}/{8.6EEFF}{EE}4D://:
\end{filecontents*}
\commandline{ verovio --spacing-non-linear=0.50 -w 1500 --spacing-system=0.5 --adjust-page-height -b 0 136-1.code }
\newline
\includesvg[width=220pt]{136-1}%

\newline RISM-ID: 455010941
\newline Alte Foliierung auf f.66v mit "57"
\newline D-Fh  Ms 001
\newline $\rightarrow$ In collection 418 (455008343)

\newline \par \vspace{7pt} \textcolor{darkblue}{\textbf{Anonymus  }}\hfillplus{137}
\newline Jesu du mein liebstes Leben. Arr  d  
\newline V, org
\newline \begin{itshape}[f.96v, at left:] ex d.\end{itshape} 
\newline \textcolor{darkblue}{\ding{\numexpr181 + 01}}  1 parts  
\newline Manuscript copy
\newline 1.1.1  V  d
\newline \begin{footnotesize} Jesu du mein liebstes Leben \end{footnotesize}  
\begin{filecontents*}{137-1.code}
@clef:C-1
@keysig:bB
@timesig:c
@data:4'DEFA/GFE(D)/{8AB}4''C{8'FG}4A/4.G8F2(F)/4DE{8FG}4A/GFE(D)/
\end{filecontents*}
\commandline{ verovio --spacing-non-linear=0.50 -w 1500 --spacing-system=0.5 --adjust-page-height -b 0 137-1.code }
\newline
\includesvg[width=220pt]{137-1}%

\newline RISM-ID: 455010980
\newline Unter den beiden Systemen 5 Textstrophen
\newline Alte Foliierung auf f.96v vorhanden: "65"
\newline D-Fh  Ms 001
\newline $\rightarrow$ In collection 418 (455008343)

\newline \par \vspace{7pt} \textcolor{darkblue}{\textbf{Anonymus  }}\hfillplus{138}
\newline Jesu komm doch selbst zu mir    
\newline org
\newline \begin{itshape}[f.110r, at the tail of the page:] Jesu kom doch selbst zu mir φ\end{itshape} 
\newline \textcolor{darkblue}{\ding{\numexpr181 + 01}}  1 parts  
\newline Manuscript copy
\newline 1.1.1  org  
\begin{filecontents*}{138-1.code}
@clef:C-1
@keysig:
@timesig:c/
@data:2''DDC'B2.A4G1G/2A''D'B''D2.D4xC1D/
\end{filecontents*}
\commandline{ verovio --spacing-non-linear=0.50 -w 1500 --spacing-system=0.5 --adjust-page-height -b 0 138-1.code }
\newline
\includesvg[width=220pt]{138-1}%

\newline RISM-ID: 455011007
\newline Alte Paginierung vorhanden, f.110v: "86"
\newline D-Fh  Ms 001
\newline $\rightarrow$ In collection 418 (455008343)

\newline \par \vspace{7pt} \textcolor{darkblue}{\textbf{Anonymus  }}\hfillplus{139}
\newline Jesu wo soll ich dich finden  G  
\newline org
\newline \begin{itshape}[f.55v, heading:] Continuo. | [at left:] Ritornello a. 5.\end{itshape} 
\newline \textcolor{darkblue}{\ding{\numexpr181 + 01}}  1 parts  
\newline Manuscript copy
\newline 1.1.1  org  G  
\begin{filecontents*}{139-1.code}
@clef:F-4
@keysig:
@timesig:c
@data:4,GGCD2.,,G4,G2D4,,AG2A,DD,,G1,C4D,,G2,D1(,,G)//
\end{filecontents*}
\commandline{ verovio --spacing-non-linear=0.50 -w 1500 --spacing-system=0.5 --adjust-page-height -b 0 139-1.code }
\newline
\includesvg[width=220pt]{139-1}%

\newline 1.2.1  V  G
\newline \begin{footnotesize} Jesu wo soll ich dich finden \end{footnotesize}  
\begin{filecontents*}{139-2.code}
@clef:G-2
@keysig:
@timesig:c
@data:{6'GxFGA}{BGFD}{EFGG}{6.3GF}8G
\end{filecontents*}
\commandline{ verovio --spacing-non-linear=0.50 -w 1500 --spacing-system=0.5 --adjust-page-height -b 0 139-2.code }
\newline
\includesvg[width=220pt]{139-2}%

\newline RISM-ID: 455010919
\newline Zunächst erklingt ein Ritornell, von der allerdings nur der bc notiert ist, unter dem Notentext des folgenden Chorale arrangements sind 4 Textstrophen aufgeführt
\newline Alte Foliierung auf f.55v mit "49"
\newline D-Fh  Ms 001
\newline $\rightarrow$ In collection 418 (455008343)

\newline \par \vspace{7pt} \textcolor{darkblue}{\textbf{Anonymus  }}\hfillplus{140}
\newline Jesulein du bist mein    
\newline org
\newline \begin{itshape}[f.39v, at left:] Jesulein du | bist mein.\end{itshape} 
\newline \textcolor{darkblue}{\ding{\numexpr181 + 01}}  1 parts  
\newline Manuscript copy
\newline 1.1.1  org  
\begin{filecontents*}{140-1.code}
@clef:G-2
@keysig:
@timesig:0
@data:8.''D6C4C{8C'A}4G{8AF}4E2D://:
\end{filecontents*}
\commandline{ verovio --spacing-non-linear=0.50 -w 1500 --spacing-system=0.5 --adjust-page-height -b 0 140-1.code }
\newline
\includesvg[width=220pt]{140-1}%

\newline RISM-ID: 455010895
\newline Alte Foliierung auf den gegenüberliegenden Seiten als Aufschlagfoliierung jeweils "39"
\newline D-Fh  Ms 001
\newline $\rightarrow$ In collection 418 (455008343)

\newline \par \vspace{7pt} \textcolor{darkblue}{\textbf{Anonymus  }}\hfillplus{141}
\newline Jesus Christus unser Heiland  a  
\newline org
\newline \begin{itshape}[f.82v, at left:] Jesus Christus | unser Heÿland. | contra Punct.\end{itshape} 
\newline \textcolor{darkblue}{\ding{\numexpr181 + 01}}  1 parts  
\newline Manuscript copy
\newline 1.1.1  org  a  
\begin{filecontents*}{141-1.code}
@clef:G-2
@keysig:
@timesig:0
@data:2'AA4GABAGxF2E4GA2B4AxG2A
\end{filecontents*}
\commandline{ verovio --spacing-non-linear=0.50 -w 1500 --spacing-system=0.5 --adjust-page-height -b 0 141-1.code }
\newline
\includesvg[width=220pt]{141-1}%

\newline RISM-ID: 455010962
\newline Alte Foliierung auf f.82v mit "63"
\newline D-Fh  Ms 001
\newline $\rightarrow$ In collection 418 (455008343)

\newline \par \vspace{7pt} \textcolor{darkblue}{\textbf{Anonymus  }}\hfillplus{142}
\newline Jesus Christus unser Heiland der von uns  1t  
\newline org
\newline \begin{itshape}[f.35r, at left:] Jesus Christs | Unßer Heÿland | der von.\end{itshape} 
\newline \textcolor{darkblue}{\ding{\numexpr181 + 01}}  1 parts  
\newline Manuscript copy
\newline 1.1.1  org  1t  
\begin{filecontents*}{142-1.code}
@clef:G-2
@keysig:
@timesig:0
@data:2'DA4AGA4DFFFE2D/4FFFD2FG4AGFE2(D)/
\end{filecontents*}
\commandline{ verovio --spacing-non-linear=0.50 -w 1500 --spacing-system=0.5 --adjust-page-height -b 0 142-1.code }
\newline
\includesvg[width=220pt]{142-1}%

\newline RISM-ID: 455010886
\newline Alte Foliierung auf den gegenüberliegenden Seiten als Aufschlagfoliierung jeweils "35"
\newline D-Fh  Ms 001
\newline $\rightarrow$ In collection 418 (455008343)

\newline \par \vspace{7pt} \textcolor{darkblue}{\textbf{Anonymus  }}\hfillplus{143}
\newline Keinen hat Gott verlassen  F  
\newline org
\newline \begin{itshape}[f.65v, at left:] Keinen hat | G. verlaßen\end{itshape} 
\newline \textcolor{darkblue}{\ding{\numexpr181 + 01}}  1 parts  
\newline Manuscript copy
\newline 1.1.1  org  F  
\begin{filecontents*}{143-1.code}
@clef:G-2
@keysig:bB
@timesig:3
@data:4--''C/2C4D/DCC+/C2'B/''C4-/'AGA/24GA/B''C+/4C'GG/2.(F)://:
\end{filecontents*}
\commandline{ verovio --spacing-non-linear=0.50 -w 1500 --spacing-system=0.5 --adjust-page-height -b 0 143-1.code }
\newline
\includesvg[width=220pt]{143-1}%

\newline RISM-ID: 455010937
\newline Alte Foliierung auf den gegenüberliegenden Seiten als Aufschlagfoliierung jeweils "56"
\newline D-Fh  Ms 001
\newline $\rightarrow$ In collection 418 (455008343)

\newline \par \vspace{7pt} \textcolor{darkblue}{\textbf{Anonymus  }}\hfillplus{144}
\newline Keyboard pieces  D  
\newline org
\newline \begin{itshape}[f.1r, at left:] Ex D.\end{itshape} 
\newline \textcolor{darkblue}{\ding{\numexpr181 + 01}}  1 parts  
\newline Manuscript copy
\newline 1.1.1  org  D  
\begin{filecontents*}{144-1.code}
@clef:G-2
@keysig:xFC
@timesig:c
@data:2'A+A/{6D3EF}{GAB''C}4D{6D'AB''C}{8D6CD}/{D'AB''C}{8D6CD}4'A{8A6GA}/
\end{filecontents*}
\commandline{ verovio --spacing-non-linear=0.50 -w 1500 --spacing-system=0.5 --adjust-page-height -b 0 144-1.code }
\newline
\includesvg[width=220pt]{144-1}%

\newline RISM-ID: 455010801
\newline Alte Foliierung vorhanden: "1"
\newline D-Fh  Ms 001
\newline $\rightarrow$ In collection 418 (455008343)

\newline \par \vspace{7pt} \textcolor{darkblue}{\textbf{Anonymus  }}\hfillplus{145}
\newline Keyboard pieces  F  
\newline org
\newline \begin{itshape}[without title]\end{itshape} 
\newline \textcolor{darkblue}{\ding{\numexpr181 + 01}}  1 parts  
\newline Manuscript copy
\newline 1.1.1  org  F  
\begin{filecontents*}{145-1.code}
@clef:G-2
@keysig:bB
@timesig:0
@data:{8''DD'A''C}{'BB}4A2-4-8-''F{EDCD}{C'BA''F}4ED4.8CD4C'B4.8AA2G
\end{filecontents*}
\commandline{ verovio --spacing-non-linear=0.50 -w 1500 --spacing-system=0.5 --adjust-page-height -b 0 145-1.code }
\newline
\includesvg[width=220pt]{145-1}%

\newline RISM-ID: 455010913
\newline Alte Foliierung auf den gegenüberliegenden Seiten als Aufschlagfoliierung jeweils "46"
\newline D-Fh  Ms 001
\newline $\rightarrow$ In collection 418 (455008343)

\newline \par \vspace{7pt} \textcolor{darkblue}{\textbf{Anonymus  }}\hfillplus{146}
\newline Komm Gott Schöpfer heiliger Geist  8t  
\newline org
\newline \begin{itshape}[f.38v, at left:] Kom Gott | Schöpffer heÿli= | ger Geist\end{itshape} 
\newline \textcolor{darkblue}{\ding{\numexpr181 + 01}}  1 parts  
\newline Manuscript copy
\newline 1.1.1  org  8t  
\begin{filecontents*}{146-1.code}
@clef:G-2
@keysig:
@timesig:0
@data:2'G4AGxFG''CD2C/C4'GA''CDEE2D/
\end{filecontents*}
\commandline{ verovio --spacing-non-linear=0.50 -w 1500 --spacing-system=0.5 --adjust-page-height -b 0 146-1.code }
\newline
\includesvg[width=220pt]{146-1}%

\newline RISM-ID: 455010892
\newline Alte Foliierung auf den gegenüberliegenden Seiten als Aufschlagfoliierung jeweils "38"
\newline D-Fh  Ms 001
\newline $\rightarrow$ In collection 418 (455008343)

\newline \par \vspace{7pt} \textcolor{darkblue}{\textbf{Anonymus  }}\hfillplus{147}
\newline Komm Heiliger Geist Herre Gott. Excerpts. Arr  G  
\newline V
\newline \begin{itshape}[without title]\end{itshape} 
\newline \textcolor{darkblue}{\ding{\numexpr181 + 01}}  1 parts  
\newline Manuscript copy
\newline 1.1.1  V  G
\newline \begin{footnotesize} Komm Heiliger Geist Herre Gott \end{footnotesize}  
\begin{filecontents*}{147-1.code}
@clef:C-1
@keysig:xF
@timesig:c
@data:2''D/E4DC/'B''C2D/1('A)/2B''xC/1(D)/
\end{filecontents*}
\commandline{ verovio --spacing-non-linear=0.50 -w 1500 --spacing-system=0.5 --adjust-page-height -b 0 147-1.code }
\newline
\includesvg[width=220pt]{147-1}%

\newline RISM-ID: 455010991
\newline Nur die Melodiestimme und eine Strophe sind notiert
\newline Alte Zählung auf f.102v vorhanden: "71"
\newline D-Fh  Ms 001
\newline $\rightarrow$ In collection 418 (455008343)

\newline \par \vspace{7pt} \textcolor{darkblue}{\textbf{Anonymus  }}\hfillplus{148}
\newline Kommt her zu mir  1t  
\newline org
\newline \begin{itshape}[f.19v, at left:] Kompt her | zu mir\end{itshape} 
\newline \textcolor{darkblue}{\ding{\numexpr181 + 01}}  1 parts  
\newline Manuscript copy
\newline 1.1.1  org  1t  
\begin{filecontents*}{148-1.code}
@clef:G-2
@keysig:bB
@timesig:c/
@data:2'G4GG/2.''D4xC/D'BAG/BA2B/4''DC'BA/
\end{filecontents*}
\commandline{ verovio --spacing-non-linear=0.50 -w 1500 --spacing-system=0.5 --adjust-page-height -b 0 148-1.code }
\newline
\includesvg[width=220pt]{148-1}%

\newline RISM-ID: 455010838
\newline Alte Foliierung vorhanden: "19"
\newline D-Fh  Ms 001
\newline $\rightarrow$ In collection 418 (455008343)

\newline \par \vspace{7pt} \textcolor{darkblue}{\textbf{Anonymus  }}\hfillplus{149}
\newline Kyrie Gott Vater in Ewigkeit  G  
\newline org
\newline \begin{itshape}[f.18v, at left:] Kÿrie Gott | Vatter.\end{itshape} 
\newline \textcolor{darkblue}{\ding{\numexpr181 + 01}}  1 parts  
\newline Manuscript copy
\newline 1.1.1  org  G  
\begin{filecontents*}{149-1.code}
@clef:G-2
@keysig:xF
@timesig:0
@data:2'GABB/A''C4CC2'BAG/AAGF4EE2D/
\end{filecontents*}
\commandline{ verovio --spacing-non-linear=0.50 -w 1500 --spacing-system=0.5 --adjust-page-height -b 0 149-1.code }
\newline
\includesvg[width=220pt]{149-1}%

\newline RISM-ID: 455010836
\newline D-Fh  Ms 001
\newline $\rightarrow$ In collection 418 (455008343)

\newline \par \vspace{7pt} \textcolor{darkblue}{\textbf{Anonymus  }}\hfillplus{150}
\newline Lebt jemand so wie ich  F  
\newline org
\newline \begin{itshape}[f.112r, between the systems:] Lebt jemand so wie ich.\end{itshape} 
\newline \textcolor{darkblue}{\ding{\numexpr181 + 01}}  1 parts  
\newline Manuscript copy
\newline 1.1.1  org  F  
\begin{filecontents*}{150-1.code}
@clef:C-1
@keysig:bB
@timesig:c/
@data:4-''C/4.8CD'BA/2(A)4-''C/'FG4.A8G/2G4-''C/DEFG/
\end{filecontents*}
\commandline{ verovio --spacing-non-linear=0.50 -w 1500 --spacing-system=0.5 --adjust-page-height -b 0 150-1.code }
\newline
\includesvg[width=220pt]{150-1}%

\newline RISM-ID: 455011012
\newline Alte Paginierung vorhanden, f.111v: "89"
\newline D-Fh  Ms 001
\newline $\rightarrow$ In collection 418 (455008343)

\newline \par \vspace{7pt} \textcolor{darkblue}{\textbf{Anonymus  }}\hfillplus{151}
\newline Liebeserklärung  C  
\newline V, mandora
\newline \begin{itshape}[heading, f.46v:] Liebes Erklärung.\end{itshape} 
\newline \textcolor{darkblue}{\ding{\numexpr181 + 01}}  1 parts  
\newline Manuscript copy
\newline Zink, Joseph Michael
\newline 1.1.1  V  C
\newline \begin{footnotesize} Sie ging zum Sonntagstanze \end{footnotesize}  
\begin{filecontents*}{151-1.code}
@clef:G-2
@keysig:
@timesig:6/8
@data:8'G/4''C8Cq6C{8'BA}B/4.''C48'GG/''EC{8'GA}B/48''C--E/EEED/4.C{8C'B}A/
\end{filecontents*}
\commandline{ verovio --spacing-non-linear=0.50 -w 1500 --spacing-system=0.5 --adjust-page-height -b 0 151-1.code }
\newline
\includesvg[width=220pt]{151-1}%

\newline RISM-ID: 455011122
\newline Sieben Strophen
\newline Der Schreiber ist der selbe wie der 70 Divertimenti in der gleichen Handschrift
\newline D-Fh  Ms 002
\newline $\rightarrow$ In collection 419 (455008345)

\newline \par \vspace{7pt} \textcolor{darkblue}{\textbf{Anonymus  }}\hfillplus{152}
\newline Lob sei dem allmächtigen Gott  1t  
\newline org
\newline \begin{itshape}[f.23r, at left:] Lob seÿ den All | Mächtigen | Gott.\end{itshape} 
\newline \textcolor{darkblue}{\ding{\numexpr181 + 01}}  1 parts  
\newline Manuscript copy
\newline 1.1.1  org  1t  
\begin{filecontents*}{152-1.code}
@clef:G-2
@keysig:
@timesig:0
@data:2'A4AA''C'AGF2(E)4EFEDAG2F/4''CCC'A''DC
\end{filecontents*}
\commandline{ verovio --spacing-non-linear=0.50 -w 1500 --spacing-system=0.5 --adjust-page-height -b 0 152-1.code }
\newline
\includesvg[width=220pt]{152-1}%

\newline RISM-ID: 455010853
\newline Das Stück stand ursprünglich in G-dorisch wurde jedoch in Ober- und Unterstimme mit D-dorisch überschrieben
\newline Alte Foliierung auf den gegenüberliegenden Seiten als Aufschlagfoliierung jeweils "23"
\newline D-Fh  Ms 001
\newline $\rightarrow$ In collection 418 (455008343)

\newline \par \vspace{7pt} \textcolor{darkblue}{\textbf{Anonymus  }}\hfillplus{153}
\newline Lob sei dem allmächtigen Gott    
\newline org
\newline \begin{itshape}[f.23r, at right:] Seque alia | melodia\end{itshape} 
\newline \textcolor{darkblue}{\ding{\numexpr181 + 01}}  1 parts  
\newline Manuscript copy
\newline 1.1.1  org  
\begin{filecontents*}{153-1.code}
@clef:G-2
@keysig:
@timesig:0
@data:2'G4GGFGBA2GG4AB''C'AB''C2DD4DC''D'B''C'A2G
\end{filecontents*}
\commandline{ verovio --spacing-non-linear=0.50 -w 1500 --spacing-system=0.5 --adjust-page-height -b 0 153-1.code }
\newline
\includesvg[width=220pt]{153-1}%

\newline RISM-ID: 455010907
\newline Alte Foliierung auf den gegenüberliegenden Seiten als Aufschlagfoliierung jeweils "23"
\newline D-Fh  Ms 001
\newline $\rightarrow$ In collection 418 (455008343)

\newline \par \vspace{7pt} \textcolor{darkblue}{\textbf{Anonymus  }}\hfillplus{154}
\newline Lobet den Herren alle  a  
\newline org
\newline \begin{itshape}[f.65r, at left:] â 4. | Lobet den | Herren alle\end{itshape} 
\newline \textcolor{darkblue}{\ding{\numexpr181 + 01}}  1 parts  
\newline Manuscript copy
\newline 1.1.1  org  a  
\begin{filecontents*}{154-1.code}
@clef:G-2
@keysig:
@timesig:c
@data:4''C{8'BA}{GF}4E{8E''C8.C6D}{8C'A}4A{8A''DCD}4'B{8B''C}{'AA}4B
\end{filecontents*}
\commandline{ verovio --spacing-non-linear=0.50 -w 1500 --spacing-system=0.5 --adjust-page-height -b 0 154-1.code }
\newline
\includesvg[width=220pt]{154-1}%

\newline RISM-ID: 455010935
\newline Trotz der Angabe "â 4." ist die Tabulatur nur zweistimmig
\newline Alte Foliierung auf den gegenüberliegenden Seiten als Aufschlagfoliierung jeweils "55"
\newline D-Fh  Ms 001
\newline $\rightarrow$ In collection 418 (455008343)

\newline \par \vspace{7pt} \textcolor{darkblue}{\textbf{Anonymus  }}\hfillplus{155}
\newline Macht auf die Tor der Gerechtigkeit  a  
\newline org
\newline \begin{itshape}[f.68v, at left:] Macht auff | die Thor der | â 5.\end{itshape} 
\newline \textcolor{darkblue}{\ding{\numexpr181 + 01}}  1 parts  
\newline Manuscript copy
\newline 1.1.1  org  a  
\begin{filecontents*}{155-1.code}
@clef:G-2
@keysig:
@timesig:3/2
@data:4'AAA/24B''C/D'B/2.''C/4CCE/24DD/4C2'B/2.''C://:
\end{filecontents*}
\commandline{ verovio --spacing-non-linear=0.50 -w 1500 --spacing-system=0.5 --adjust-page-height -b 0 155-1.code }
\newline
\includesvg[width=220pt]{155-1}%

\newline RISM-ID: 455010942
\newline Alte Foliierung auf f.68v mit "58"
\newline D-Fh  Ms 001
\newline $\rightarrow$ In collection 418 (455008343)

\newline \par \vspace{7pt} \textcolor{darkblue}{\textbf{Anonymus  }}\hfillplus{156}
\newline Macht auf die Tor der Gerechtigkeit. Arr  a  
\newline V, org
\newline \begin{itshape}[without title]\end{itshape} 
\newline \textcolor{darkblue}{\ding{\numexpr181 + 01}}  1 parts  
\newline Manuscript copy
\newline 1.1.1  V  a
\newline \begin{footnotesize} Macht auf die Tor der Gerechtigkeit \end{footnotesize}  
\begin{filecontents*}{156-1.code}
@clef:C-1
@keysig:
@timesig:3
@data:2'AAA/12B''C/2DDD/1.(C)/2CCE/12DC/'BB/1.(A)/
\end{filecontents*}
\commandline{ verovio --spacing-non-linear=0.50 -w 1500 --spacing-system=0.5 --adjust-page-height -b 0 156-1.code }
\newline
\includesvg[width=220pt]{156-1}%

\newline RISM-ID: 455010989
\newline Zwischen den Systemen 3 Textstrophen
\newline Alte Foliierung auf f.101v vorhanden: "70"
\newline D-Fh  Ms 001
\newline $\rightarrow$ In collection 418 (455008343)

\newline \par \vspace{7pt} \textcolor{darkblue}{\textbf{Anonymus  }}\hfillplus{157}
\newline Mädchen sind wie der Wind  A  
\newline V, mandora
\newline \begin{itshape}[without title]\end{itshape} 
\newline \textcolor{darkblue}{\ding{\numexpr181 + 01}}  1 parts  
\newline Manuscript copy
\newline 1.1.1  V  A
\newline \begin{footnotesize} Mädchen sind wie der Wind \end{footnotesize}  
\begin{filecontents*}{157-1.code}
@clef:G-2
@keysig:xFCG
@timesig:2/4
@data:8''CD4E/2-/8'B''C4D/2-/8'AABB/q8''D4C'B/8''DD4'B/
\end{filecontents*}
\commandline{ verovio --spacing-non-linear=0.50 -w 1500 --spacing-system=0.5 --adjust-page-height -b 0 157-1.code }
\newline
\includesvg[width=220pt]{157-1}%

\newline RISM-ID: 455011110
\newline Vier Strophen
\newline D-Fh  Ms 002
\newline $\rightarrow$ In collection 419 (455008345)

\newline \par \vspace{7pt} \textcolor{darkblue}{\textbf{Anonymus  }}\hfillplus{158}
\newline Mag ich Unglück nicht widerstehn  a  
\newline org
\newline \begin{itshape}[f.21v, at left:] Mag ich Unglück | nicht widstehn.\end{itshape} 
\newline \textcolor{darkblue}{\ding{\numexpr181 + 01}}  1 parts  
\newline Manuscript copy
\newline 1.1.1  org  a  
\begin{filecontents*}{158-1.code}
@clef:G-2
@keysig:
@timesig:0
@data:2'E4GGA''C{8'BA}4B2(A)/4''EDC'B/''C'BAG8E4G8A4F(E)://:
\end{filecontents*}
\commandline{ verovio --spacing-non-linear=0.50 -w 1500 --spacing-system=0.5 --adjust-page-height -b 0 158-1.code }
\newline
\includesvg[width=220pt]{158-1}%

\newline RISM-ID: 455010846
\newline Alte Foliierung auf den gegenüberliegenden Seiten als Aufschlagfoliierung jeweils "22"
\newline D-Fh  Ms 001
\newline $\rightarrow$ In collection 418 (455008343)

\newline \par \vspace{7pt} \textcolor{darkblue}{\textbf{Anonymus  }}\hfillplus{159}
\newline Masses. Excerpts    
\newline org
\newline \begin{itshape}[f.18v, at left:] Kÿrie\end{itshape} 
\newline \textcolor{darkblue}{\ding{\numexpr181 + 01}}  1 parts  
\newline Manuscript copy
\newline 1.1.1  org  
\begin{filecontents*}{159-1.code}
@clef:G-2
@keysig:
@timesig:0
@data:2'B''CDD/'B''DC'BA(G)/AAGFE(D)/
\end{filecontents*}
\commandline{ verovio --spacing-non-linear=0.50 -w 1500 --spacing-system=0.5 --adjust-page-height -b 0 159-1.code }
\newline
\includesvg[width=220pt]{159-1}%

\newline 1.2.1  org  
\begin{filecontents*}{159-2.code}
@clef:G-2
@keysig:
@timesig:0
@data:2'EDGAAGFED/DAAxGAB''C'BA(G)/
\end{filecontents*}
\commandline{ verovio --spacing-non-linear=0.50 -w 1500 --spacing-system=0.5 --adjust-page-height -b 0 159-2.code }
\newline
\includesvg[width=220pt]{159-2}%

\newline RISM-ID: 455010837
\newline Alte Foliierung vorhanden: "19"
\newline D-Fh  Ms 001
\newline $\rightarrow$ In collection 418 (455008343)

\newline \par \vspace{7pt} \textcolor{darkblue}{\textbf{Anonymus  }}\hfillplus{160}
\newline Mein Freund komme in seinen Garten. Arr  C  
\newline V, org
\newline \begin{itshape}[f.43v, at left:] Mein Freund | kom̅e.\end{itshape} 
\newline \textcolor{darkblue}{\ding{\numexpr181 + 01}}  1 parts  
\newline Manuscript copy
\newline 1.1.1  V  C
\newline \begin{footnotesize} Mein Freund komme in seinen Garten \end{footnotesize}  
\begin{filecontents*}{160-1.code}
@clef:G-2
@keysig:
@timesig:0
@data:2'G4AG2''C4-'GAG{8''CC'AB}{''CCC'B}4A{8BG}{AG}4G2G
\end{filecontents*}
\commandline{ verovio --spacing-non-linear=0.50 -w 1500 --spacing-system=0.5 --adjust-page-height -b 0 160-1.code }
\newline
\includesvg[width=220pt]{160-1}%

\newline RISM-ID: 455010905
\newline Am unteren Rand von f.43v "G. M Oster.", wobei der Schriftzug schwer entzifferbar ist
\newline Unterhalb des Notentextes verläuft der Gesangstext
\newline Alte Foliierung auf f.43v mit "43"
\newline Oster, Georg Michael  (oth)
\newline D-Fh  Ms 001
\newline $\rightarrow$ In collection 418 (455008343)

\newline \par \vspace{7pt} \textcolor{darkblue}{\textbf{Anonymus  }}\hfillplus{161}
\newline Mein Herz ruht und ist stille. Excerpts. Arr  1t  
\newline V, org
\newline \begin{itshape}[without title]\end{itshape} 
\newline \textcolor{darkblue}{\ding{\numexpr181 + 01}}  1 parts  
\newline Manuscript copy
\newline 1.1.1  V  1t
\newline \begin{footnotesize} Mein Herz ruht und ist stille \end{footnotesize}  
\begin{filecontents*}{161-1.code}
@clef:C-1
@keysig:
@timesig:3
@data:1'B2''D'GGxF1ED/G2AB''C'B1A://:
\end{filecontents*}
\commandline{ verovio --spacing-non-linear=0.50 -w 1500 --spacing-system=0.5 --adjust-page-height -b 0 161-1.code }
\newline
\includesvg[width=220pt]{161-1}%

\newline RISM-ID: 455011036
\newline Zwischen den Systemen nur der Textbeginn wie angegeben
\newline Am Ende des Stücke: "Es kann auch aus dem A.D. geschlag werd, u also eine quart höher."
\newline Alte Paginierung auf f.123r vorhanden: "79"
\newline D-Fh  Ms 001
\newline $\rightarrow$ In collection 418 (455008343)

\newline \par \vspace{7pt} \textcolor{darkblue}{\textbf{Anonymus  }}\hfillplus{162}
\newline Mein' Seel' sich freu' und lustig sei  1t  
\newline org
\newline \begin{itshape}[f.9v, at left:] 14. | Mein Seeel sich freü | und Lustig seÿ,\end{itshape} 
\newline \textcolor{darkblue}{\ding{\numexpr181 + 01}}  1 parts  
\newline Manuscript copy
\newline 1.1.1  org  1t  
\begin{filecontents*}{162-1.code}
@clef:G-2
@keysig:
@timesig:0
@data:4''D{6C3DC8'B}4A8-A{8.6B''DxCD}4D8-E{6DCD'B}{''C'A''DC}{8'BA}4B
\end{filecontents*}
\commandline{ verovio --spacing-non-linear=0.50 -w 1500 --spacing-system=0.5 --adjust-page-height -b 0 162-1.code }
\newline
\includesvg[width=220pt]{162-1}%

\newline RISM-ID: 455010815
\newline D-Fh  Ms 001
\newline $\rightarrow$ In collection 418 (455008343)

\newline \par \vspace{7pt} \textcolor{darkblue}{\textbf{Anonymus  }}\hfillplus{163}
\newline Mein' Wallfahrt ich vollendet hab'  d  
\newline org
\newline \begin{itshape}[f.45v, at left:] Mein Wallfahrt ich vollendet hab.\end{itshape} 
\newline \textcolor{darkblue}{\ding{\numexpr181 + 01}}  1 parts  
\newline Manuscript copy
\newline 1.1.1  org  d  
\begin{filecontents*}{163-1.code}
@clef:C-1
@keysig:bB
@timesig:0
@data:1'A''C'BAAGFE/AnB''C'AEFGED://:
\end{filecontents*}
\commandline{ verovio --spacing-non-linear=0.50 -w 1500 --spacing-system=0.5 --adjust-page-height -b 0 163-1.code }
\newline
\includesvg[width=220pt]{163-1}%

\newline RISM-ID: 455010908
\newline Nur die Melodiestimme im C-1-Schlüssel ist verzeichnet mit dem Textbeginn
\newline Alte Foliierung auf f.45v mit "44"
\newline D-Fh  Ms 001
\newline $\rightarrow$ In collection 418 (455008343)

\newline \par \vspace{7pt} \textcolor{darkblue}{\textbf{Anonymus  }}\hfillplus{164}
\newline Minuets  C  
\newline org
\newline \begin{itshape}[f.116v, heading:] Menuet\end{itshape} 
\newline \textcolor{darkblue}{\ding{\numexpr181 + 01}}  1 parts  
\newline Manuscript copy
\newline 1.1.1  org  C  
\begin{filecontents*}{164-1.code}
@clef:C-1
@keysig:
@timesig:3/4
@data:4''C{8CCCC}/4'B{8BBBB}/4''C{8'BAGF}/4E{8DE}4C/2G{8FE}/{8EDEC}4F/
\end{filecontents*}
\commandline{ verovio --spacing-non-linear=0.50 -w 1500 --spacing-system=0.5 --adjust-page-height -b 0 164-1.code }
\newline
\includesvg[width=220pt]{164-1}%

\newline RISM-ID: 455011024
\newline Ohne alte Paginierung
\newline D-Fh  Ms 001
\newline $\rightarrow$ In collection 418 (455008343)

\newline \par \vspace{7pt} \textcolor{darkblue}{\textbf{Anonymus  }}\hfillplus{165}
\newline Minuets  F  
\newline org
\newline \begin{itshape}[f.117r, heading:] Menuet\end{itshape} 
\newline \textcolor{darkblue}{\ding{\numexpr181 + 01}}  1 parts  
\newline Manuscript copy
\newline 1.1.1  org  F  
\begin{filecontents*}{165-1.code}
@clef:C-1
@keysig:bB
@timesig:3/4
@data:!4'FFF/F{8AGAG}/!f{FGAB}4''C/2'At4G/A{8BAGF}/{GFED}4C/
\end{filecontents*}
\commandline{ verovio --spacing-non-linear=0.50 -w 1500 --spacing-system=0.5 --adjust-page-height -b 0 165-1.code }
\newline
\includesvg[width=220pt]{165-1}%

\newline RISM-ID: 455011025
\newline Ohne alte Paginierung
\newline D-Fh  Ms 001
\newline $\rightarrow$ In collection 418 (455008343)

\newline \par \vspace{7pt} \textcolor{darkblue}{\textbf{Anonymus  }}\hfillplus{166}
\newline Minuets  G  
\newline org
\newline \begin{itshape}[f.118r, heading:] Menuet\end{itshape} 
\newline \textcolor{darkblue}{\ding{\numexpr181 + 01}}  1 parts  
\newline Manuscript copy
\newline 1.1.1  org  G  
\begin{filecontents*}{166-1.code}
@clef:C-1
@keysig:xF
@timesig:3/4
@data:4''DGF/{8GD'BG}4''E/D{8'B''D}4D/{8C'B''G'B}4A/{8DF}{AF}4D/{8DC,BA}4(B)/
\end{filecontents*}
\commandline{ verovio --spacing-non-linear=0.50 -w 1500 --spacing-system=0.5 --adjust-page-height -b 0 166-1.code }
\newline
\includesvg[width=220pt]{166-1}%

\newline RISM-ID: 455011028
\newline Ohne alte Paginierung
\newline D-Fh  Ms 001
\newline $\rightarrow$ In collection 418 (455008343)

\newline \par \vspace{7pt} \textcolor{darkblue}{\textbf{Anonymus  }}\hfillplus{167}
\newline Mir ist ein geistlich Kirchelein  a  
\newline org
\newline \begin{itshape}[f.65v, at left:] â 5. | Mir ist ein geistl | Kirchelein.\end{itshape} 
\newline \textcolor{darkblue}{\ding{\numexpr181 + 01}}  1 parts  
\newline Manuscript copy
\newline 1.1.1  org  a  
\begin{filecontents*}{167-1.code}
@clef:G-2
@keysig:
@timesig:c
@data:4'B{8''CE}{D'B''C6'BA}4B8-xG4AB{8''CDE6ED}4E8-'B4''C'B
\end{filecontents*}
\commandline{ verovio --spacing-non-linear=0.50 -w 1500 --spacing-system=0.5 --adjust-page-height -b 0 167-1.code }
\newline
\includesvg[width=220pt]{167-1}%

\newline RISM-ID: 455010936
\newline Trotz der Angabe "â 5." ist die Tabulatur nur zweistimmig
\newline Alte Foliierung auf den gegenüberliegenden Seiten als Aufschlagfoliierung jeweils "56"
\newline D-Fh  Ms 001
\newline $\rightarrow$ In collection 418 (455008343)

\newline \par \vspace{7pt} \textcolor{darkblue}{\textbf{Anonymus  }}\hfillplus{168}
\newline Mit Fried und Freud ich fahr dahin  1t  
\newline org
\newline \begin{itshape}[f.25v, at left:] Mit Fried Und | Freud.\end{itshape} 
\newline \textcolor{darkblue}{\ding{\numexpr181 + 01}}  1 parts  
\newline Manuscript copy
\newline 1.1.1  org  1t  
\begin{filecontents*}{168-1.code}
@clef:G-2
@keysig:
@timesig:0
@data:2'D4AAA''DC'B1(A)/4''C'A''C{8BA}4BA/
\end{filecontents*}
\commandline{ verovio --spacing-non-linear=0.50 -w 1500 --spacing-system=0.5 --adjust-page-height -b 0 168-1.code }
\newline
\includesvg[width=220pt]{168-1}%

\newline RISM-ID: 455010860
\newline Alte Foliierung auf f.25v mit "26"
\newline D-Fh  Ms 001
\newline $\rightarrow$ In collection 418 (455008343)

\newline \par \vspace{7pt} \textcolor{darkblue}{\textbf{Anonymus  }}\hfillplus{169}
\newline Mitten wir im Leben sind  3t  
\newline V
\newline \begin{itshape}[without title]\end{itshape} 
\newline \textcolor{darkblue}{\ding{\numexpr181 + 01}}  1 parts  
\newline \begin{small} Other parts missing\end{small} 
\newline Manuscript copy
\newline 1.1.1  V  3t
\newline \begin{footnotesize} Mitten wir im Leben sind \end{footnotesize}  
\begin{filecontents*}{169-1.code}
@clef:C-1
@keysig:
@timesig:c
@data:2'GG/AB/''CC/'B(A)/B''C/D'A/GF/1(E)://:
\end{filecontents*}
\commandline{ verovio --spacing-non-linear=0.50 -w 1500 --spacing-system=0.5 --adjust-page-height -b 0 169-1.code }
\newline
\includesvg[width=220pt]{169-1}%

\newline RISM-ID: 455011008
\newline Nur die obere Melodiestimme und drei Textstrophen sind ausgeführt, obwohl in einem zweiten System eine Baßstimme vorgesehen ist
\newline Alte Paginierung vorhanden, f.110v: "87" und f.111r: "88"
\newline D-Fh  Ms 001
\newline $\rightarrow$ In collection 418 (455008343)

\newline \par \vspace{7pt} \textcolor{darkblue}{\textbf{Anonymus  }}\hfillplus{170}
\newline Nice se più non m'ami  C  
\newline V (2), mandora
\newline \begin{itshape}[at left, f.54v:] Duettino\end{itshape} 
\newline \textcolor{darkblue}{\ding{\numexpr181 + 01}}  score: f.54v  
\newline Manuscript copy
\newline Zink, Joseph Michael
\newline 1.1.1  V  C
\newline \begin{footnotesize} Nice se più non m'ami \end{footnotesize}  
\begin{filecontents*}{170-1.code}
@clef:G-2
@keysig:
@timesig:3/4
@data:4''C8CCCC/{CD}2D/4D8C'BAG/{FE}2E/4''C8CCCb'B/{BA}2A/4G8GABG/q''D2C4-://
\end{filecontents*}
\commandline{ verovio --spacing-non-linear=0.50 -w 1500 --spacing-system=0.5 --adjust-page-height -b 0 170-1.code }
\newline
\includesvg[width=220pt]{170-1}%

\newline RISM-ID: 455010800
\newline Vier Strophen
\newline D-Fh  Ms 002
\newline $\rightarrow$ In collection 419 (455008345)

\newline \par \vspace{7pt} \textcolor{darkblue}{\textbf{Anonymus  }}\hfillplus{171}
\newline Nun Gott Lob es ist vollbracht. Arr  G  
\newline V, org
\newline \begin{itshape}[without title]\end{itshape} 
\newline \textcolor{darkblue}{\ding{\numexpr181 + 01}}  1 parts  
\newline Manuscript copy
\newline 1.1.1  V  G
\newline \begin{footnotesize} Nun Gott Lob es ist vollbracht \end{footnotesize}  
\begin{filecontents*}{171-1.code}
@clef:C-1
@keysig:xF
@timesig:c
@data:4'B''D'BA/GA2(B)/4BAGE/AG2F/1(G)/4ABAG/FE2(D)/
\end{filecontents*}
\commandline{ verovio --spacing-non-linear=0.50 -w 1500 --spacing-system=0.5 --adjust-page-height -b 0 171-1.code }
\newline
\includesvg[width=220pt]{171-1}%

\newline RISM-ID: 455010987
\newline Unter den beiden Systemen 7 Textstrophen
\newline Alte Foliierung auf f.100v vorhanden: "69"
\newline D-Fh  Ms 001
\newline $\rightarrow$ In collection 418 (455008343)

\newline \par \vspace{7pt} \textcolor{darkblue}{\textbf{Anonymus  }}\hfillplus{172}
\newline Nun bitten wir den Heiligen Geist  F  
\newline org
\newline \begin{itshape}[f.39r, at left:] Nun bitten | Wir den heili= | gen Geist.\end{itshape} 
\newline \textcolor{darkblue}{\ding{\numexpr181 + 01}}  1 parts  
\newline Manuscript copy
\newline 1.1.1  org  F  
\begin{filecontents*}{172-1.code}
@clef:G-2
@keysig:bB
@timesig:0
@data:4'F{8GG}4FDC{8DE}2(F)4A''CDC'AF{8ED}4F2F
\end{filecontents*}
\commandline{ verovio --spacing-non-linear=0.50 -w 1500 --spacing-system=0.5 --adjust-page-height -b 0 172-1.code }
\newline
\includesvg[width=220pt]{172-1}%

\newline RISM-ID: 455010893
\newline Alte Foliierung auf den gegenüberliegenden Seiten als Aufschlagfoliierung jeweils "38"
\newline D-Fh  Ms 001
\newline $\rightarrow$ In collection 418 (455008343)

\newline \par \vspace{7pt} \textcolor{darkblue}{\textbf{Anonymus  }}\hfillplus{173}
\newline Nun freut euch Gottes Kinder all  1t  
\newline org
\newline \begin{itshape}[f.38v, at left:] Nun freud euch | Gottes.\end{itshape} 
\newline \textcolor{darkblue}{\ding{\numexpr181 + 01}}  1 parts  
\newline Manuscript copy
\newline 1.1.1  org  1t  
\begin{filecontents*}{173-1.code}
@clef:G-2
@keysig:
@timesig:0
@data:2'D4DCFGAGA/GAB''C'A''DxCD/
\end{filecontents*}
\commandline{ verovio --spacing-non-linear=0.50 -w 1500 --spacing-system=0.5 --adjust-page-height -b 0 173-1.code }
\newline
\includesvg[width=220pt]{173-1}%

\newline RISM-ID: 455010891
\newline Alte Foliierung auf den gegenüberliegenden Seiten als Aufschlagfoliierung jeweils "38"
\newline D-Fh  Ms 001
\newline $\rightarrow$ In collection 418 (455008343)

\newline \par \vspace{7pt} \textcolor{darkblue}{\textbf{Anonymus  }}\hfillplus{174}
\newline Nun giebet der Höchste  F  
\newline org
\newline \begin{itshape}[heading, f.61v:] Nun giebet d. Höchste d. gnädigen Segen. | [at left:] Rittornello.\end{itshape} 
\newline \textcolor{darkblue}{\ding{\numexpr181 + 01}}  1 parts  
\newline Manuscript copy
\newline 1.1.1  org  F  
\begin{filecontents*}{174-1.code}
@clef:F-4
@keysig:bB
@timesig:0
@data:1,FCFC,,F,C,,FG,C,,B,C
\end{filecontents*}
\commandline{ verovio --spacing-non-linear=0.50 -w 1500 --spacing-system=0.5 --adjust-page-height -b 0 174-1.code }
\newline
\includesvg[width=220pt]{174-1}%

\newline 1.2.1  org  F  
\begin{filecontents*}{174-2.code}
@clef:G-2
@keysig:bB
@timesig:3
@data:4--''C/'AA''C/4.D8D4C/4.'B8B4A/GGG/24A''C/C'B/''C://:C/
\end{filecontents*}
\commandline{ verovio --spacing-non-linear=0.50 -w 1500 --spacing-system=0.5 --adjust-page-height -b 0 174-2.code }
\newline
\includesvg[width=220pt]{174-2}%

\newline RISM-ID: 455010932
\newline Zunächst erklingt die Sinfonia, von der allerdings nur der bc notiert ist, unter dem Notentext des folgenden Chorale arrangements sind 9 Textstrophen aufgeführt
\newline Alte Foliierung auf den gegenüberliegenden Seiten als Aufschlagfoliierung jeweils "53"
\newline D-Fh  Ms 001
\newline $\rightarrow$ In collection 418 (455008343)

\newline \par \vspace{7pt} \textcolor{darkblue}{\textbf{Anonymus  }}\hfillplus{175}
\newline Nun komm der Heiden Heiland  1t  
\newline org
\newline \begin{itshape}[f.23r, at left:] Nun kom der | Heÿd Heÿland\end{itshape} 
\newline \textcolor{darkblue}{\ding{\numexpr181 + 01}}  1 parts  
\newline Manuscript copy
\newline 1.1.1  org  1t  
\begin{filecontents*}{175-1.code}
@clef:G-2
@keysig:bB
@timesig:0
@data:4'GGFB{8AG}4A2G4GB''C'B''CD2('B)
\end{filecontents*}
\commandline{ verovio --spacing-non-linear=0.50 -w 1500 --spacing-system=0.5 --adjust-page-height -b 0 175-1.code }
\newline
\includesvg[width=220pt]{175-1}%

\newline RISM-ID: 455010852
\newline Alte Foliierung auf den gegenüberliegenden Seiten als Aufschlagfoliierung jeweils "23"
\newline D-Fh  Ms 001
\newline $\rightarrow$ In collection 418 (455008343)

\newline \par \vspace{7pt} \textcolor{darkblue}{\textbf{Anonymus  }}\hfillplus{176}
\newline Nun laßt uns Gott den Herren  G  
\newline org
\newline \begin{itshape}[f.107v, heading:] Nun last uns Gott den Herren\end{itshape} 
\newline \textcolor{darkblue}{\ding{\numexpr181 + 01}}  1 parts  
\newline Manuscript copy
\newline 1.1.1  org  G  
\begin{filecontents*}{176-1.code}
@clef:C-1
@keysig:xF
@timesig:6/4
@data:=1/4--'G2G4G/EFG2A4-/2G4---G/2G4A4.F8E4D/
\end{filecontents*}
\commandline{ verovio --spacing-non-linear=0.50 -w 1500 --spacing-system=0.5 --adjust-page-height -b 0 176-1.code }
\newline
\includesvg[width=220pt]{176-1}%

\newline RISM-ID: 455011003
\newline Alte Paginierung vorhanden, f.107v: "81"
\newline D-Fh  Ms 001
\newline $\rightarrow$ In collection 418 (455008343)

\newline \par \vspace{7pt} \textcolor{darkblue}{\textbf{Anonymus  }}\hfillplus{177}
\newline Nun laßt uns Gott den Herren  C  
\newline org
\newline \begin{itshape}[f.114r, heading:] Nun laßt uns Gott den H.\end{itshape} 
\newline \textcolor{darkblue}{\ding{\numexpr181 + 01}}  1 parts  
\newline Manuscript copy
\newline 1.1.1  org  C  
\begin{filecontents*}{177-1.code}
@clef:C-1
@keysig:
@timesig:3/4
@data:4''C/2C4'B/AB''C/2.D/24(C)C/CD/4.'B8A4G/2.''C/2('B)4G/
\end{filecontents*}
\commandline{ verovio --spacing-non-linear=0.50 -w 1500 --spacing-system=0.5 --adjust-page-height -b 0 177-1.code }
\newline
\includesvg[width=220pt]{177-1}%

\newline RISM-ID: 455011017
\newline Ohne alte Paginierung
\newline D-Fh  Ms 001
\newline $\rightarrow$ In collection 418 (455008343)

\newline \par \vspace{7pt} \textcolor{darkblue}{\textbf{Anonymus  }}\hfillplus{178}
\newline Nun laßt uns Gott den Herrn  G  
\newline org
\newline \begin{itshape}[f.13r, at left:] Nun last uns | Gott den Herrn\end{itshape} 
\newline \textcolor{darkblue}{\ding{\numexpr181 + 01}}  1 parts  
\newline Manuscript copy
\newline 1.1.1  org  G  
\begin{filecontents*}{178-1.code}
@clef:G-2
@keysig:xF
@timesig:0
@data:4-'GG{8FE+}{EG}4FG/GG{8AF+}{FF}4G
\end{filecontents*}
\commandline{ verovio --spacing-non-linear=0.50 -w 1500 --spacing-system=0.5 --adjust-page-height -b 0 178-1.code }
\newline
\includesvg[width=220pt]{178-1}%

\newline RISM-ID: 455010821
\newline Alte Foliierung vorhanden: "13"
\newline D-Fh  Ms 001
\newline $\rightarrow$ In collection 418 (455008343)

\newline \par \vspace{7pt} \textcolor{darkblue}{\textbf{Anonymus  }}\hfillplus{179}
\newline Nun lob mein' Seel' den Herrn  F  
\newline org
\newline \begin{itshape}[f.66v, at left:] Nun Lob mein | Seel den Herren\end{itshape} 
\newline \textcolor{darkblue}{\ding{\numexpr181 + 01}}  1 parts  
\newline Manuscript copy
\newline 1.1.1  org  F  
\begin{filecontents*}{179-1.code}
@clef:G-2
@keysig:bB
@timesig:3
@data:4--'F/24FE/DC/FG/AA/4AGA/24AG/4F2G/F://:
\end{filecontents*}
\commandline{ verovio --spacing-non-linear=0.50 -w 1500 --spacing-system=0.5 --adjust-page-height -b 0 179-1.code }
\newline
\includesvg[width=220pt]{179-1}%

\newline RISM-ID: 455010939
\newline Alte Foliierung auf f.66v mit "57"
\newline D-Fh  Ms 001
\newline $\rightarrow$ In collection 418 (455008343)

\newline \par \vspace{7pt} \textcolor{darkblue}{\textbf{Anonymus  }}\hfillplus{180}
\newline Nun sich der Tag geendet hat. Arr  a  
\newline V, org
\newline \begin{itshape}[without title]\end{itshape} 
\newline \textcolor{darkblue}{\ding{\numexpr181 + 01}}  1 parts  
\newline Manuscript copy
\newline 1.1.1  V  a
\newline \begin{footnotesize} Nun sich der Tag geendet hat \end{footnotesize}  
\begin{filecontents*}{180-1.code}
@clef:C-1
@keysig:
@timesig:c/
@data:2''E4'AB/''CCDF/2EE/4'A''DCC/2('B)''D/4C'BAA/xGA2(B)/
\end{filecontents*}
\commandline{ verovio --spacing-non-linear=0.50 -w 1500 --spacing-system=0.5 --adjust-page-height -b 0 180-1.code }
\newline
\includesvg[width=220pt]{180-1}%

\newline RISM-ID: 455010995
\newline Zwischen den beiden Systemen nur der Textbeginn wie angegeben
\newline Alte Zählung auf f.104r vorhanden: "74"
\newline D-Fh  Ms 001
\newline $\rightarrow$ In collection 418 (455008343)

\newline \par \vspace{7pt} \textcolor{darkblue}{\textbf{Anonymus  }}\hfillplus{181}
\newline O Angst und Leid o Traurigkeit  e  
\newline org
\newline \begin{itshape}[f.54v, heading:] Continuo. | [at left:] Sinfonia à 5.\end{itshape} 
\newline \textcolor{darkblue}{\ding{\numexpr181 + 01}}  1 parts  
\newline Manuscript copy
\newline 1.1.1  org  e  
\begin{filecontents*}{181-1.code}
@clef:F-4
@keysig:
@timesig:c
@data:2,EC,,BA,C,,B4GxF2EB8-{BBB}2,E8-{,,AAA}2,D
\end{filecontents*}
\commandline{ verovio --spacing-non-linear=0.50 -w 1500 --spacing-system=0.5 --adjust-page-height -b 0 181-1.code }
\newline
\includesvg[width=220pt]{181-1}%

\newline 1.2.1  org with text  e
\newline \begin{footnotesize} O Angst und Leid o Traurigkeit \end{footnotesize}  
\begin{filecontents*}{181-2.code}
@clef:G-2
@keysig:
@timesig:c
@data:8-{'B8.''E6E}{8xD'B8.6BA}4B6-{BBB}{8B6BA8.6AG}8xF-4-
\end{filecontents*}
\commandline{ verovio --spacing-non-linear=0.50 -w 1500 --spacing-system=0.5 --adjust-page-height -b 0 181-2.code }
\newline
\includesvg[width=220pt]{181-2}%

\newline RISM-ID: 455010918
\newline Zunächst erklingt die Sinfonia, von der allerdings nur der bc notiert ist, unter dem Notentext des folgenden Chorale arrangements sind 9 Textstrophen aufgeführt
\newline Alte Foliierung auf den gegenüberliegenden Seiten als Aufschlagfoliierung jeweils "48"
\newline D-Fh  Ms 001
\newline $\rightarrow$ In collection 418 (455008343)

\newline \par \vspace{7pt} \textcolor{darkblue}{\textbf{Anonymus  }}\hfillplus{182}
\newline O Ewigkeit du Donnerwort. Arr  F  
\newline V, org
\newline \begin{itshape}[without title]\end{itshape} 
\newline \textcolor{darkblue}{\ding{\numexpr181 + 01}}  1 parts  
\newline Manuscript copy
\newline 1.1.1  V  F
\newline \begin{footnotesize} O Ewigkeit du Donnerwort \end{footnotesize}  
\begin{filecontents*}{182-1.code}
@clef:C-1
@keysig:[bB]
@timesig:c
@data:!4'F8AB4''CC/!DE2(F)/f4'BA2(G)/4''C8'AF4BA/2GF/
\end{filecontents*}
\commandline{ verovio --spacing-non-linear=0.50 -w 1500 --spacing-system=0.5 --adjust-page-height -b 0 182-1.code }
\newline
\includesvg[width=220pt]{182-1}%

\newline RISM-ID: 455011033
\newline Unter den Systemen 12 Textstrophen
\newline Alte Paginierung auf f.121v vorhanden: "77" und f. 122r: "78"
\newline D-Fh  Ms 001
\newline $\rightarrow$ In collection 418 (455008343)

\newline \par \vspace{7pt} \textcolor{darkblue}{\textbf{Anonymus  }}\hfillplus{183}
\newline O Gott du höchster Gnadenhort. Arr  F  
\newline V, org
\newline \begin{itshape}[f.96r, at left:] ex F.\end{itshape} 
\newline \textcolor{darkblue}{\ding{\numexpr181 + 01}}  1 parts  
\newline Manuscript copy
\newline 1.1.1  V  F
\newline \begin{footnotesize} O Gott du höchster Gnadenhort \end{footnotesize}  
\begin{filecontents*}{183-1.code}
@clef:C-1
@keysig:bB
@timesig:3/2
@data:2'FFF/12AB/''CD/1(C)
\end{filecontents*}
\commandline{ verovio --spacing-non-linear=0.50 -w 1500 --spacing-system=0.5 --adjust-page-height -b 0 183-1.code }
\newline
\includesvg[width=220pt]{183-1}%

\newline RISM-ID: 455010979
\newline Unter den Systemen 3 Strophen
\newline Alte Foliierung auf f.96r vorhanden: "64"
\newline D-Fh  Ms 001
\newline $\rightarrow$ In collection 418 (455008343)

\newline \par \vspace{7pt} \textcolor{darkblue}{\textbf{Anonymus  }}\hfillplus{184}
\newline O Gott du höchster Gnadenhort  D  
\newline org
\newline \begin{itshape}[f.111v, heading:] O Gott du höchster φ\end{itshape} 
\newline \textcolor{darkblue}{\ding{\numexpr181 + 01}}  1 parts  
\newline Manuscript copy
\newline 1.1.1  org  D  
\begin{filecontents*}{184-1.code}
@clef:C-1
@keysig:xFC
@timesig:0
@data:1'D2DDFGAB1A/A2B''CEC4'BA2B1A/
\end{filecontents*}
\commandline{ verovio --spacing-non-linear=0.50 -w 1500 --spacing-system=0.5 --adjust-page-height -b 0 184-1.code }
\newline
\includesvg[width=220pt]{184-1}%

\newline RISM-ID: 455011009
\newline Alte Paginierung vorhanden, f.111v: "89"
\newline D-Fh  Ms 001
\newline $\rightarrow$ In collection 418 (455008343)

\newline \par \vspace{7pt} \textcolor{darkblue}{\textbf{Anonymus  }}\hfillplus{185}
\newline O Heiliger Geist o heiliger Gott  C  
\newline org
\newline \begin{itshape}[f.4v, at left:] Ô Heiliger Geist | Ô Heiliger Gott | 4.\end{itshape} 
\newline \textcolor{darkblue}{\ding{\numexpr181 + 01}}  1 parts  
\newline Manuscript copy
\newline 1.1.1  org  C  
\begin{filecontents*}{185-1.code}
@clef:G-2
@keysig:
@timesig:3
@data:4--''C/4.C8C4C/2D4'B/4.8''D'B4A/2G4-/
\end{filecontents*}
\commandline{ verovio --spacing-non-linear=0.50 -w 1500 --spacing-system=0.5 --adjust-page-height -b 0 185-1.code }
\newline
\includesvg[width=220pt]{185-1}%

\newline RISM-ID: 455010804
\newline Alte Foliierung vorhanden: "4"
\newline D-Fh  Ms 001
\newline $\rightarrow$ In collection 418 (455008343)

\newline \par \vspace{7pt} \textcolor{darkblue}{\textbf{Anonymus  }}\hfillplus{186}
\newline O Jesu du edle Gabe. Arr  G  
\newline V, org
\newline \begin{itshape}[without title]\end{itshape} 
\newline \textcolor{darkblue}{\ding{\numexpr181 + 01}}  1 parts  
\newline Manuscript copy
\newline 1.1.1  V  G
\newline \begin{footnotesize} O Jesu du edle Gabe \end{footnotesize}  
\begin{filecontents*}{186-1.code}
@clef:C-1
@keysig:[xF]
@timesig:3
@data:12'B''D'B''DC'ABB/A''D'GABBAA/
\end{filecontents*}
\commandline{ verovio --spacing-non-linear=0.50 -w 1500 --spacing-system=0.5 --adjust-page-height -b 0 186-1.code }
\newline
\includesvg[width=220pt]{186-1}%

\newline RISM-ID: 455011035
\newline Zwischen den Systemen nur der Textbeginn wie angegeben
\newline Alte Paginierung auf f.122v vorhanden: "78"
\newline D-Fh  Ms 001
\newline $\rightarrow$ In collection 418 (455008343)

\newline \par \vspace{7pt} \textcolor{darkblue}{\textbf{Anonymus  }}\hfillplus{187}
\newline O Jesu du getreuer Hirt  d  
\newline org
\newline \begin{itshape}[f.59v, at left:] O Jesu | du ge= | treuer | Hirt | a 5. v.\end{itshape} 
\newline \textcolor{darkblue}{\ding{\numexpr181 + 01}}  1 parts  
\newline Manuscript copy
\newline 1.1.1  org  d  
\begin{filecontents*}{187-1.code}
@clef:G-2
@keysig:bB
@timesig:3
@data:4''DFE/24D'A/''DD/xCC/4.D8D4C/C2C/2'A://:4A/24GF/EA/
\end{filecontents*}
\commandline{ verovio --spacing-non-linear=0.50 -w 1500 --spacing-system=0.5 --adjust-page-height -b 0 187-1.code }
\newline
\includesvg[width=220pt]{187-1}%

\newline RISM-ID: 455010930
\newline Alte Foliierung auf f.59v mit "51"
\newline D-Fh  Ms 001
\newline $\rightarrow$ In collection 418 (455008343)

\newline \par \vspace{7pt} \textcolor{darkblue}{\textbf{Anonymus  }}\hfillplus{188}
\newline O Lamm Gottes unschuldig  F  
\newline org
\newline \begin{itshape}[f.42v, at left:] O Lamb Gottes | unschuldig\end{itshape} 
\newline \textcolor{darkblue}{\ding{\numexpr181 + 01}}  1 parts  
\newline Manuscript copy
\newline 1.1.1  org  F  
\begin{filecontents*}{188-1.code}
@clef:G-2
@keysig:bB
@timesig:3
@data:6-{'CDE}42FFF''CCD2C4-42C'FGABG2F://:
\end{filecontents*}
\commandline{ verovio --spacing-non-linear=0.50 -w 1500 --spacing-system=0.5 --adjust-page-height -b 0 188-1.code }
\newline
\includesvg[width=220pt]{188-1}%

\newline RISM-ID: 455010902
\newline Alte Foliierung auf den gegenüberliegenden Seiten als Aufschlagfoliierung jeweils "42"
\newline D-Fh  Ms 001
\newline $\rightarrow$ In collection 418 (455008343)

\newline \par \vspace{7pt} \textcolor{darkblue}{\textbf{Anonymus  }}\hfillplus{189}
\newline O Lamm Gottes unschuldig. Arr  F  
\newline V, org
\newline \begin{itshape}[without title]\end{itshape} 
\newline \textcolor{darkblue}{\ding{\numexpr181 + 01}}  1 parts  
\newline Manuscript copy
\newline 1.1.1  V  F
\newline \begin{footnotesize} O Lamm Gottes unschuldig \end{footnotesize}  
\begin{filecontents*}{189-1.code}
@clef:C-1
@keysig:bB
@timesig:0
@data:1'FAB''CCD(C)/C'FGABAG(F)/
\end{filecontents*}
\commandline{ verovio --spacing-non-linear=0.50 -w 1500 --spacing-system=0.5 --adjust-page-height -b 0 189-1.code }
\newline
\includesvg[width=220pt]{189-1}%

\newline RISM-ID: 455010988
\newline Unter den beiden Systemen 3 Textstrophen durch Zählung vorgesehen, jedoch nur die erste ausgeführt
\newline Alte Foliierung auf f.100v vorhanden: "69"
\newline D-Fh  Ms 001
\newline $\rightarrow$ In collection 418 (455008343)

\newline \par \vspace{7pt} \textcolor{darkblue}{\textbf{Anonymus  }}\hfillplus{190}
\newline O Vater allmächtiger Gott  D  
\newline org
\newline \begin{itshape}[f.12v, at the end of the page:] O Vater | Allmäch | tiger G: | Christe\end{itshape} 
\newline \textcolor{darkblue}{\ding{\numexpr181 + 01}}  1 parts  
\newline Manuscript copy
\newline 1.1.1  org  G  
\begin{filecontents*}{190-1.code}
@clef:G-2
@keysig:xF
@timesig:0
@data:4'GGAB''C'B2A(G)A4''C'GGF/ED2(C)/
\end{filecontents*}
\commandline{ verovio --spacing-non-linear=0.50 -w 1500 --spacing-system=0.5 --adjust-page-height -b 0 190-1.code }
\newline
\includesvg[width=220pt]{190-1}%

\newline RISM-ID: 455010822
\newline Der Titel steht auf f.12v unten, praktisch links vor dem unteren System von f.13r
\newline Alte Foliierung vorhanden: "13"
\newline D-Fh  Ms 001
\newline $\rightarrow$ In collection 418 (455008343)

\newline \par \vspace{7pt} \textcolor{darkblue}{\textbf{Anonymus  }}\hfillplus{191}
\newline O wie selig seid ihr doch ihr Frommen. Arr  a  
\newline V, org
\newline \begin{itshape}[without title]\end{itshape} 
\newline \textcolor{darkblue}{\ding{\numexpr181 + 01}}  1 parts  
\newline Manuscript copy
\newline 1.1.1  V  a
\newline \begin{footnotesize} O wie selig seid ihr doch ihr Frommen \end{footnotesize}  
\begin{filecontents*}{191-1.code}
@clef:C-1
@keysig:
@timesig:3
@data:12''CE/C'B/2AxGA/B1''C/1.('B)/12xGB/AB/2''CDE/C1D/1.(C)/
\end{filecontents*}
\commandline{ verovio --spacing-non-linear=0.50 -w 1500 --spacing-system=0.5 --adjust-page-height -b 0 191-1.code }
\newline
\includesvg[width=220pt]{191-1}%

\newline RISM-ID: 455011034
\newline Unter den Systemen 6 Textstrophen
\newline Alte Paginierung auf f.122v vorhanden: "78"
\newline D-Fh  Ms 001
\newline $\rightarrow$ In collection 418 (455008343)

\newline \par \vspace{7pt} \textcolor{darkblue}{\textbf{Anonymus  }}\hfillplus{192}
\newline Preludes  D  
\newline org
\newline \begin{itshape}[f.74v, at left:] Praeamb: | ex Clave | D.\end{itshape} 
\newline \textcolor{darkblue}{\ding{\numexpr181 + 01}}  1 parts  
\newline Manuscript copy
\newline 1.1.1  org  D  
\begin{filecontents*}{192-1.code}
@clef:G-2
@keysig:xFC
@timesig:0
@data:4'D{8DE}{FGAB}{''CDEF}{EFGF}E4D8C2D8-{DDC}{'BB''DC}2'B''E
\end{filecontents*}
\commandline{ verovio --spacing-non-linear=0.50 -w 1500 --spacing-system=0.5 --adjust-page-height -b 0 192-1.code }
\newline
\includesvg[width=220pt]{192-1}%

\newline RISM-ID: 455010951
\newline Ohne alte Foliierung
\newline D-Fh  Ms 001
\newline $\rightarrow$ In collection 418 (455008343)

\newline \par \vspace{7pt} \textcolor{darkblue}{\textbf{Anonymus  }}\hfillplus{193}
\newline Preludes  G  
\newline org
\newline \begin{itshape}[f.75v, at left:] Prael: | ex Clave | G.\end{itshape} 
\newline \textcolor{darkblue}{\ding{\numexpr181 + 01}}  1 parts  
\newline Manuscript copy
\newline 1.1.1  org  G  
\begin{filecontents*}{193-1.code}
@clef:G-2
@keysig:xF
@timesig:0
@data:2'G4GG2A4DG4.G8F{EF}4G{8G6FE}4F{6GDEF}{GAFG}2E
\end{filecontents*}
\commandline{ verovio --spacing-non-linear=0.50 -w 1500 --spacing-system=0.5 --adjust-page-height -b 0 193-1.code }
\newline
\includesvg[width=220pt]{193-1}%

\newline RISM-ID: 455010952
\newline Ohne alte Foliierung
\newline Heinrich, Herzog  (oth)
\newline D-Fh  Ms 001
\newline $\rightarrow$ In collection 418 (455008343)

\newline \par \vspace{7pt} \textcolor{darkblue}{\textbf{Anonymus  }}\hfillplus{194}
\newline Preludes  A  
\newline org
\newline \begin{itshape}[f.81r, at left:] Praeludium | ex Clave | A\end{itshape} 
\newline \textcolor{darkblue}{\ding{\numexpr181 + 01}}  1 parts  
\newline Manuscript copy
\newline 1.1.1  org  A  
\begin{filecontents*}{194-1.code}
@clef:G-2
@keysig:xFCG
@timesig:0
@data:2'A{8AB''CD}{E'B''ED}{C'B''C'A}{''DC'BA}{GAFG}
\end{filecontents*}
\commandline{ verovio --spacing-non-linear=0.50 -w 1500 --spacing-system=0.5 --adjust-page-height -b 0 194-1.code }
\newline
\includesvg[width=220pt]{194-1}%

\newline RISM-ID: 455010958
\newline Ohne alte Foliierung
\newline D-Fh  Ms 001
\newline $\rightarrow$ In collection 418 (455008343)

\newline \par \vspace{7pt} \textcolor{darkblue}{\textbf{Anonymus  }}\hfillplus{195}
\newline Preludes  G  
\newline org
\newline \begin{itshape}[f.81v, at left:] Praeludium.\end{itshape} 
\newline \textcolor{darkblue}{\ding{\numexpr181 + 01}}  1 parts  
\newline Manuscript copy
\newline 1.1.1  org  G  
\begin{filecontents*}{195-1.code}
@clef:G-2
@keysig:xF
@timesig:0
@data:{8'GAB''C}2D'B''C'B8A4G8F2(G+)(G)//
\end{filecontents*}
\commandline{ verovio --spacing-non-linear=0.50 -w 1500 --spacing-system=0.5 --adjust-page-height -b 0 195-1.code }
\newline
\includesvg[width=220pt]{195-1}%

\newline RISM-ID: 455010960
\newline Ohne alte Foliierung
\newline D-Fh  Ms 001
\newline $\rightarrow$ In collection 418 (455008343)

\newline \par \vspace{7pt} \textcolor{darkblue}{\textbf{Anonymus  }}\hfillplus{196}
\newline Preludes  a  
\newline org
\newline \begin{itshape}[f.83v, at left:] Praeludium\end{itshape} 
\newline \textcolor{darkblue}{\ding{\numexpr181 + 01}}  1 parts  
\newline Manuscript copy
\newline 1.1.1  org  a  
\begin{filecontents*}{196-1.code}
@clef:G-2
@keysig:
@timesig:0
@data:!{8'A''EC'A}!f{8''E6EE}{8EE}{6FEGF}{8EE}4DC'BA
\end{filecontents*}
\commandline{ verovio --spacing-non-linear=0.50 -w 1500 --spacing-system=0.5 --adjust-page-height -b 0 196-1.code }
\newline
\includesvg[width=220pt]{196-1}%

\newline RISM-ID: 455010963
\newline Ohne alte Foliierung
\newline D-Fh  Ms 001
\newline $\rightarrow$ In collection 418 (455008343)

\newline \par \vspace{7pt} \textcolor{darkblue}{\textbf{Anonymus  }}\hfillplus{197}
\newline Preludes  a  
\newline org
\newline \begin{itshape}[f.87v, at left:] Praeludium | ex | A.\end{itshape} 
\newline \textcolor{darkblue}{\ding{\numexpr181 + 01}}  1 parts  
\newline Manuscript copy
\newline 1.1.1  org  a  
\begin{filecontents*}{197-1.code}
@clef:G-2
@keysig:
@timesig:0
@data:4''E{6EFDE}{CD'B''C}{8'AB}{''CD}4.E8E{6DECD}{8'B''C'BA}{xG''E}{6DECD}{8'B''C'BA}8G4A8G2A
\end{filecontents*}
\commandline{ verovio --spacing-non-linear=0.50 -w 1500 --spacing-system=0.5 --adjust-page-height -b 0 197-1.code }
\newline
\includesvg[width=220pt]{197-1}%

\newline RISM-ID: 455010968
\newline Ohne alte Foliierung
\newline D-Fh  Ms 001
\newline $\rightarrow$ In collection 418 (455008343)

\newline \par \vspace{7pt} \textcolor{darkblue}{\textbf{Anonymus  }}\hfillplus{198}
\newline Preludes  G  
\newline org
\newline \begin{itshape}[f.90v, at left:] Praeludium | ex Clave | G.\end{itshape} 
\newline \textcolor{darkblue}{\ding{\numexpr181 + 01}}  1 parts  
\newline Manuscript copy
\newline 1.1.1  org  G  
\begin{filecontents*}{198-1.code}
@clef:G-2
@keysig:xF
@timesig:0
@data:4'G{6G3FE}{DC,BA}2'G4.G8F{EF}4G2A4D{8ED}4xCDDC2D
\end{filecontents*}
\commandline{ verovio --spacing-non-linear=0.50 -w 1500 --spacing-system=0.5 --adjust-page-height -b 0 198-1.code }
\newline
\includesvg[width=220pt]{198-1}%

\newline RISM-ID: 455010973
\newline Ohne alte Foliierung
\newline D-Fh  Ms 001
\newline $\rightarrow$ In collection 418 (455008343)

\newline \par \vspace{7pt} \textcolor{darkblue}{\textbf{Anonymus  }}\hfillplus{199}
\newline Preludes  F  
\newline org
\newline \begin{itshape}[f.92v, at left:] Praeambulum. | ex Clave F.\end{itshape} 
\newline \textcolor{darkblue}{\ding{\numexpr181 + 01}}  1 parts  
\newline Manuscript copy
\newline 1.1.1  org  F  
\begin{filecontents*}{199-1.code}
@clef:G-2
@keysig:bB
@timesig:0
@data:4.'F8G{AB''C'B}{A6GA}4B+{8B6AG}4A{8BFBA}4.G8F{G6ED}4E4.E{6EF}2G
\end{filecontents*}
\commandline{ verovio --spacing-non-linear=0.50 -w 1500 --spacing-system=0.5 --adjust-page-height -b 0 199-1.code }
\newline
\includesvg[width=220pt]{199-1}%

\newline RISM-ID: 455010976
\newline Ohne alte Foliierung
\newline D-Fh  Ms 001
\newline $\rightarrow$ In collection 418 (455008343)

\newline \par \vspace{7pt} \textcolor{darkblue}{\textbf{Anonymus  }}\hfillplus{200}
\newline Sag was hilft alle Welt  F  
\newline org
\newline \begin{itshape}[f.7r, at left:] 9. | Sag was hilfft alle | Welt.\end{itshape} 
\newline \textcolor{darkblue}{\ding{\numexpr181 + 01}}  1 parts  
\newline Manuscript copy
\newline 1.1.1  org  F  
\begin{filecontents*}{200-1.code}
@clef:G-2
@keysig:bB
@timesig:0
@data:4'FG8A4B8G4A''CD{8EF+}{FD}4E
\end{filecontents*}
\commandline{ verovio --spacing-non-linear=0.50 -w 1500 --spacing-system=0.5 --adjust-page-height -b 0 200-1.code }
\newline
\includesvg[width=220pt]{200-1}%

\newline RISM-ID: 455010809
\newline Alte Foliierung dreimal auf dem Blatt vorhanden: "7"
\newline D-Fh  Ms 001
\newline $\rightarrow$ In collection 418 (455008343)

\newline \par \vspace{7pt} \textcolor{darkblue}{\textbf{Anonymus  }}\hfillplus{201}
\newline Sag was hilft alle Welt  F  
\newline org
\newline \begin{itshape}[f.42v, at left:] Sag waß | hilfft alle\end{itshape} 
\newline \textcolor{darkblue}{\ding{\numexpr181 + 01}}  1 parts  
\newline Manuscript copy
\newline 1.1.1  org  F  
\begin{filecontents*}{201-1.code}
@clef:G-2
@keysig:bB
@timesig:c/
@data:2'F4E{8DxC+}{CD}4CEF8G4A8G2F://:
\end{filecontents*}
\commandline{ verovio --spacing-non-linear=0.50 -w 1500 --spacing-system=0.5 --adjust-page-height -b 0 201-1.code }
\newline
\includesvg[width=220pt]{201-1}%

\newline RISM-ID: 455010903
\newline Alte Foliierung auf den gegenüberliegenden Seiten als Aufschlagfoliierung jeweils "42"
\newline D-Fh  Ms 001
\newline $\rightarrow$ In collection 418 (455008343)

\newline \par \vspace{7pt} \textcolor{darkblue}{\textbf{Anonymus  }}\hfillplus{202}
\newline Sag was hilft alle Welt. Arr  F  
\newline V, org
\newline \begin{itshape}[f.104r, between the systems:] Sag w hilfft alle Welt. nach dieser Melodie soll es gesungen werden\end{itshape} 
\newline \textcolor{darkblue}{\ding{\numexpr181 + 01}}  1 parts  
\newline Manuscript copy
\newline 1.1.1  V  F
\newline \begin{footnotesize} Sag was hilft alle Welt \end{footnotesize}  
\begin{filecontents*}{202-1.code}
@clef:C-1
@keysig:bB
@timesig:c/
@data:2'F4GA/BG2A+/4A''CDE/FD2C/2F4ED/C'B2A/
\end{filecontents*}
\commandline{ verovio --spacing-non-linear=0.50 -w 1500 --spacing-system=0.5 --adjust-page-height -b 0 202-1.code }
\newline
\includesvg[width=220pt]{202-1}%

\newline RISM-ID: 455010996
\newline Zwischen den beiden Systemen nur der Text wie oben im dipl. Titel angegeben
\newline Alte Zählung auf f.104r vorhanden: "74"
\newline D-Fh  Ms 001
\newline $\rightarrow$ In collection 418 (455008343)

\newline \par \vspace{7pt} \textcolor{darkblue}{\textbf{Anonymus  }}\hfillplus{203}
\newline Sarabandes. Arr  C  
\newline pf
\newline \begin{itshape}[f.7v, at left:] 10. | Saraband\end{itshape} 
\newline \textcolor{darkblue}{\ding{\numexpr181 + 01}}  1 parts  
\newline Manuscript copy
\newline 1.1.1  pf  C  
\begin{filecontents*}{203-1.code}
@clef:G-2
@keysig:
@timesig:3
@data:4''EEE/D{8ED}4'B/''G{8FEDC}/4'B{8AB}4G/''CCC/24DC://
\end{filecontents*}
\commandline{ verovio --spacing-non-linear=0.50 -w 1500 --spacing-system=0.5 --adjust-page-height -b 0 203-1.code }
\newline
\includesvg[width=220pt]{203-1}%

\newline RISM-ID: 455010810
\newline D-Fh  Ms 001
\newline $\rightarrow$ In collection 418 (455008343)

\newline \par \vspace{7pt} \textcolor{darkblue}{\textbf{Anonymus  }}\hfillplus{204}
\newline Sarabandes  G  
\newline pf
\newline \begin{itshape}[f.7v, at left:] 11. | Saraband\end{itshape} 
\newline \textcolor{darkblue}{\ding{\numexpr181 + 01}}  1 parts  
\newline Manuscript copy
\newline 1.1.1  pf  G  
\begin{filecontents*}{204-1.code}
@clef:G-2
@keysig:xF
@timesig:3
@data:4''GGD/4'B{8AB}4G/''C8D-4'G/2A4G/
\end{filecontents*}
\commandline{ verovio --spacing-non-linear=0.50 -w 1500 --spacing-system=0.5 --adjust-page-height -b 0 204-1.code }
\newline
\includesvg[width=220pt]{204-1}%

\newline RISM-ID: 455010811
\newline D-Fh  Ms 001
\newline $\rightarrow$ In collection 418 (455008343)

\newline \par \vspace{7pt} \textcolor{darkblue}{\textbf{Anonymus  }}\hfillplus{205}
\newline Sarabandes  d  
\newline pf
\newline \begin{itshape}[f.58v, at left:] Sara | band\end{itshape} 
\newline \textcolor{darkblue}{\ding{\numexpr181 + 01}}  1 parts  
\newline Manuscript copy
\newline 1.1.1  pf  d  
\begin{filecontents*}{205-1.code}
@clef:G-2
@keysig:bB
@timesig:3
@data:{8'DEFG}4A/24AA/BA/G-/4GAB/''C{8'B''CDC}/4.'B{6AB}4A://:
\end{filecontents*}
\commandline{ verovio --spacing-non-linear=0.50 -w 1500 --spacing-system=0.5 --adjust-page-height -b 0 205-1.code }
\newline
\includesvg[width=220pt]{205-1}%

\newline RISM-ID: 455010928
\newline Ohne alte Foliierung
\newline D-Fh  Ms 001
\newline $\rightarrow$ In collection 418 (455008343)

\newline \par \vspace{7pt} \textcolor{darkblue}{\textbf{Anonymus  }}\hfillplus{206}
\newline Schmücket das Fest mit Maien  A  
\newline org
\newline \begin{itshape}[without title]\end{itshape} 
\newline \textcolor{darkblue}{\ding{\numexpr181 + 01}}  1 parts  
\newline Manuscript copy
\newline 1.1.1  V  A
\newline \begin{footnotesize} Schmücket das Fest mit Maien \end{footnotesize}  
\begin{filecontents*}{206-1.code}
@clef:G-2
@keysig:xFCG
@timesig:3
@data:4''CCC/{8CDCD}4E/C2'B/24AB/''DD/CC/'B-/
\end{filecontents*}
\commandline{ verovio --spacing-non-linear=0.50 -w 1500 --spacing-system=0.5 --adjust-page-height -b 0 206-1.code }
\newline
\includesvg[width=220pt]{206-1}%

\newline RISM-ID: 455010931
\newline Unter dem Notentext sind 4 Textstrophen aufgeführt
\newline Alte Foliierung auf f.60v mit "52"
\newline D-Fh  Ms 001
\newline $\rightarrow$ In collection 418 (455008343)

\newline \par \vspace{7pt} \textcolor{darkblue}{\textbf{Anonymus  }}\hfillplus{207}
\newline Selig sind die in Christo sterben    
\newline org
\newline \begin{itshape}[f.111v, at the tail of the page:] Seelig sind die in Christo sterben φ\end{itshape} 
\newline \textcolor{darkblue}{\ding{\numexpr181 + 01}}  1 parts  
\newline Manuscript copy
\newline 1.1.1  org  
\begin{filecontents*}{207-1.code}
@clef:C-1
@keysig:
@timesig:c/
@data:2'G4ED/2CG/4AG2F/1E/2G4AB/2''C'B/4AG2xF/1(G)://:
\end{filecontents*}
\commandline{ verovio --spacing-non-linear=0.50 -w 1500 --spacing-system=0.5 --adjust-page-height -b 0 207-1.code }
\newline
\includesvg[width=220pt]{207-1}%

\newline RISM-ID: 455011010
\newline Alte Paginierung vorhanden, f.111v: "89"
\newline D-Fh  Ms 001
\newline $\rightarrow$ In collection 418 (455008343)

\newline \par \vspace{7pt} \textcolor{darkblue}{\textbf{Anonymus  }}\hfillplus{208}
\newline Singen wir aus Herzensgrund  1t  
\newline org
\newline \begin{itshape}[f.53v, at left:] Singen Wir aus | Hertzen grund.\end{itshape} 
\newline \textcolor{darkblue}{\ding{\numexpr181 + 01}}  1 parts  
\newline Manuscript copy
\newline 1.1.1  org  1t  
\begin{filecontents*}{208-1.code}
@clef:G-2
@keysig:bB
@timesig:0
@data:8.'G6A{8BA}8G4xF8G4A{8AB''C6DC}4'BA2G{8AB''C'B}{AG}4F
\end{filecontents*}
\commandline{ verovio --spacing-non-linear=0.50 -w 1500 --spacing-system=0.5 --adjust-page-height -b 0 208-1.code }
\newline
\includesvg[width=220pt]{208-1}%

\newline RISM-ID: 455010915
\newline Alte Foliierung auf f.53v mit "47"
\newline D-Fh  Ms 001
\newline $\rightarrow$ In collection 418 (455008343)

\newline \par \vspace{7pt} \textcolor{darkblue}{\textbf{Anonymus  }}\hfillplus{209}
\newline So wünsch' ich dir du Eitelkeit  1t  
\newline org
\newline \begin{itshape}[f.112r, heading:] So wünsch ich dir du Eitelkeit\end{itshape} 
\newline \textcolor{darkblue}{\ding{\numexpr181 + 01}}  1 parts  
\newline Manuscript copy
\newline 1.1.1  org  1t  
\begin{filecontents*}{209-1.code}
@clef:C-1
@keysig:bBE
@timesig:c/
@data:4-{8''CD}4E{8DC}/4'B''DC2E4-C/E{8DC}4'B{8bAG}/{FG}4FG-://:
\end{filecontents*}
\commandline{ verovio --spacing-non-linear=0.50 -w 1500 --spacing-system=0.5 --adjust-page-height -b 0 209-1.code }
\newline
\includesvg[width=220pt]{209-1}%

\newline RISM-ID: 455011011
\newline Alte Paginierung vorhanden, f.112r: "90"
\newline D-Fh  Ms 001
\newline $\rightarrow$ In collection 418 (455008343)

\newline \par \vspace{7pt} \textcolor{darkblue}{\textbf{Anonymus  }}\hfillplus{210}
\newline Sollt' es gleich bisweilen scheinen  C  
\newline org
\newline \begin{itshape}[f.33v, at left:] Solt es gleich | bisweilen | scheinen.\end{itshape} 
\newline \textcolor{darkblue}{\ding{\numexpr181 + 01}}  1 parts  
\newline Manuscript copy
\newline 1.1.1  org  C  
\begin{filecontents*}{210-1.code}
@clef:G-2
@keysig:
@timesig:0
@data:{8''C'GAE}{8.G6F8E(C)}/{8.6''C'B8A6B''C}{8.6DC8'B(G)}/
\end{filecontents*}
\commandline{ verovio --spacing-non-linear=0.50 -w 1500 --spacing-system=0.5 --adjust-page-height -b 0 210-1.code }
\newline
\includesvg[width=220pt]{210-1}%

\newline RISM-ID: 455010884
\newline Alte Foliierung auf f.33v mit "34"
\newline D-Fh  Ms 001
\newline $\rightarrow$ In collection 418 (455008343)

\newline \par \vspace{7pt} \textcolor{darkblue}{\textbf{Anonymus  }}\hfillplus{211}
\newline Sollt' es gleich bisweilen scheinen. Arr  1t  
\newline V, org
\newline \begin{itshape}[f.97v, at left:] ex d.\end{itshape} 
\newline \textcolor{darkblue}{\ding{\numexpr181 + 01}}  1 parts  
\newline Manuscript copy
\newline 1.1.1  V  1t
\newline \begin{footnotesize} Sollt' es gleich bisweilen scheinen \end{footnotesize}  
\begin{filecontents*}{211-1.code}
@clef:C-1
@keysig:
@timesig:c
@data:2'D4FG/2AA/G4AG/2F(D)/''DD/CC/'Bt4-B/2A(A)/
\end{filecontents*}
\commandline{ verovio --spacing-non-linear=0.50 -w 1500 --spacing-system=0.5 --adjust-page-height -b 0 211-1.code }
\newline
\includesvg[width=220pt]{211-1}%

\newline RISM-ID: 455010983
\newline Ziwschen den beiden Systemen 10 Textstrophen
\newline Alte Foliierung auf f.97v vorhanden: "66"
\newline D-Fh  Ms 001
\newline $\rightarrow$ In collection 418 (455008343)

\newline \par \vspace{7pt} \textcolor{darkblue}{\textbf{Anonymus  }}\hfillplus{212}
\newline Toccatas  C  
\newline org
\newline \begin{itshape}[f.49v, at left:] Toccata | ex C. | Zum Pedal\end{itshape} 
\newline \textcolor{darkblue}{\ding{\numexpr181 + 01}}  1 parts  
\newline Manuscript copy
\newline 1.1.1  org  C  
\begin{filecontents*}{212-1.code}
@clef:G-2
@keysig:
@timesig:0
@data:8''C{'BA}{6GFED}{C,GAB}{'CDEF}{GAB''C}{DEFD}{EFGF}{EDC'B}
\end{filecontents*}
\commandline{ verovio --spacing-non-linear=0.50 -w 1500 --spacing-system=0.5 --adjust-page-height -b 0 212-1.code }
\newline
\includesvg[width=220pt]{212-1}%

\newline RISM-ID: 455010910
\newline Ohne alte Foliierung
\newline D-Fh  Ms 001
\newline $\rightarrow$ In collection 418 (455008343)

\newline \par \vspace{7pt} \textcolor{darkblue}{\textbf{Anonymus  }}\hfillplus{213}
\newline Valet will ich dir geben  F  
\newline org
\newline \begin{itshape}[f.35v, at left:] Valet will ich | dir geben.\end{itshape} 
\newline \textcolor{darkblue}{\ding{\numexpr181 + 01}}  1 parts  
\newline Manuscript copy
\newline 1.1.1  org  F  
\begin{filecontents*}{213-1.code}
@clef:G-2
@keysig:bB
@timesig:0
@data:2'F4''CCDE2F4FAGFFE2(F)://:
\end{filecontents*}
\commandline{ verovio --spacing-non-linear=0.50 -w 1500 --spacing-system=0.5 --adjust-page-height -b 0 213-1.code }
\newline
\includesvg[width=220pt]{213-1}%

\newline RISM-ID: 455010888
\newline Alte Foliierung auf f.35v mit "36"
\newline D-Fh  Ms 001
\newline $\rightarrow$ In collection 418 (455008343)

\newline \par \vspace{7pt} \textcolor{darkblue}{\textbf{Anonymus  }}\hfillplus{214}
\newline Vater unser im Himmelreich  D  
\newline org
\newline \begin{itshape}[f.2v, at left:] 2. | Vater unser | im Himmel= | reich\end{itshape} 
\newline \textcolor{darkblue}{\ding{\numexpr181 + 01}}  1 parts  
\newline Manuscript copy
\newline 1.1.1  org  D  
\begin{filecontents*}{214-1.code}
@clef:G-2
@keysig:xFC
@timesig:0
@data:2'A{6.3ABAB}{6FDEF}{G''C'CG}{6.3ABGA}{FA}{3GAFG}{EFGF}{ED6E}
\end{filecontents*}
\commandline{ verovio --spacing-non-linear=0.50 -w 1500 --spacing-system=0.5 --adjust-page-height -b 0 214-1.code }
\newline
\includesvg[width=220pt]{214-1}%

\newline RISM-ID: 455010802
\newline D-Fh  Ms 001
\newline $\rightarrow$ In collection 418 (455008343)

\newline \par \vspace{7pt} \textcolor{darkblue}{\textbf{Anonymus  }}\hfillplus{215}
\newline Vater unser im Himmelreich  D  
\newline org
\newline \begin{itshape}[f.4r, at left:] 3. | Vater unser | im Him̅elreich.\end{itshape} 
\newline \textcolor{darkblue}{\ding{\numexpr181 + 01}}  1 parts  
\newline Manuscript copy
\newline 1.1.1  org  D  
\begin{filecontents*}{215-1.code}
@clef:G-2
@keysig:xFC
@timesig:0
@data:2'A{8AG}4FG{8AG}{FE}4D/AAG{8''C'BAG}4FGA/
\end{filecontents*}
\commandline{ verovio --spacing-non-linear=0.50 -w 1500 --spacing-system=0.5 --adjust-page-height -b 0 215-1.code }
\newline
\includesvg[width=220pt]{215-1}%

\newline RISM-ID: 455010803
\newline Alte Foliierung vorhanden: "4"
\newline D-Fh  Ms 001
\newline $\rightarrow$ In collection 418 (455008343)

\newline \par \vspace{7pt} \textcolor{darkblue}{\textbf{Anonymus  }}\hfillplus{216}
\newline Vom Himmel hoch da komm' ich her  C  
\newline org
\newline \begin{itshape}[f.24v, at left:] Vom Him̅el | hoch da kom̅ ich her\end{itshape} 
\newline \textcolor{darkblue}{\ding{\numexpr181 + 01}}  1 parts  
\newline Manuscript copy
\newline 1.1.1  org  C  
\begin{filecontents*}{216-1.code}
@clef:G-2
@keysig:
@timesig:0
@data:2''C4'BABGAB2(''C)4CC'GGEGF(E)
\end{filecontents*}
\commandline{ verovio --spacing-non-linear=0.50 -w 1500 --spacing-system=0.5 --adjust-page-height -b 0 216-1.code }
\newline
\includesvg[width=220pt]{216-1}%

\newline RISM-ID: 455010857
\newline Alte Foliierung auf f.24v mit "25"
\newline D-Fh  Ms 001
\newline $\rightarrow$ In collection 418 (455008343)

\newline \par \vspace{7pt} \textcolor{darkblue}{\textbf{Anonymus  }}\hfillplus{217}
\newline Von Gott will ich nicht lassen  1t  
\newline org
\newline \begin{itshape}[f.27v, at left:] Von Gott | will ich nicht | laßen.\end{itshape} 
\newline \textcolor{darkblue}{\ding{\numexpr181 + 01}}  1 parts  
\newline Manuscript copy
\newline 1.1.1  org  1t  
\begin{filecontents*}{217-1.code}
@clef:G-2
@keysig:
@timesig:0
@data:4-''DDEFD2E4xCCDDEE2'A://:
\end{filecontents*}
\commandline{ verovio --spacing-non-linear=0.50 -w 1500 --spacing-system=0.5 --adjust-page-height -b 0 217-1.code }
\newline
\includesvg[width=220pt]{217-1}%

\newline RISM-ID: 455010865
\newline Alte Foliierung auf den gegenüberliegenden Seiten als Aufschlagfoliierung jeweils "28"
\newline D-Fh  Ms 001
\newline $\rightarrow$ In collection 418 (455008343)

\newline \par \vspace{7pt} \textcolor{darkblue}{\textbf{Anonymus  }}\hfillplus{218}
\newline Von Gott will ich nicht lassen  1t  
\newline org
\newline \begin{itshape}[f.28r, at left:] Von Gott | will ich\end{itshape} 
\newline \textcolor{darkblue}{\ding{\numexpr181 + 01}}  1 parts  
\newline Manuscript copy
\newline 1.1.1  org  1t  
\begin{filecontents*}{218-1.code}
@clef:G-2
@keysig:
@timesig:0
@data:4-'AAB''C'A2B4xGGAABB2E://:
\end{filecontents*}
\commandline{ verovio --spacing-non-linear=0.50 -w 1500 --spacing-system=0.5 --adjust-page-height -b 0 218-1.code }
\newline
\includesvg[width=220pt]{218-1}%

\newline RISM-ID: 455010867
\newline Alte Foliierung auf den gegenüberliegenden Seiten als Aufschlagfoliierung jeweils "28"
\newline D-Fh  Ms 001
\newline $\rightarrow$ In collection 418 (455008343)

\newline \par \vspace{7pt} \textcolor{darkblue}{\textbf{Anonymus  }}\hfillplus{219}
\newline Von Herzen Grund ergeben. Arr  G  
\newline V, org
\newline \begin{itshape}[without title]\end{itshape} 
\newline \textcolor{darkblue}{\ding{\numexpr181 + 01}}  1 parts  
\newline Manuscript copy
\newline 1.1.1  V  G
\newline \begin{footnotesize} Von Herzen Grund ergeben \end{footnotesize}  
\begin{filecontents*}{219-1.code}
@clef:C-1
@keysig:xF
@timesig:c/
@data:1'D/2EF/GF/1E/(F)/D/2GA/B''D/C'B/1A/
\end{filecontents*}
\commandline{ verovio --spacing-non-linear=0.50 -w 1500 --spacing-system=0.5 --adjust-page-height -b 0 219-1.code }
\newline
\includesvg[width=220pt]{219-1}%

\newline RISM-ID: 455010981
\newline Unter den beiden Systemen drei Textstrophen
\newline Alte Foliierung auf f.96r vorhanden: "64"
\newline D-Fh  Ms 001
\newline $\rightarrow$ In collection 418 (455008343)

\newline \par \vspace{7pt} \textcolor{darkblue}{\textbf{Anonymus  }}\hfillplus{220}
\newline Wär' Gott nicht mit uns diese Zeit  a  
\newline org
\newline \begin{itshape}[f.20v, at left:] Wär Gott ô mit | Uns diese Zeit.\end{itshape} 
\newline \textcolor{darkblue}{\ding{\numexpr181 + 01}}  1 parts  
\newline Manuscript copy
\newline 1.1.1  org  a  
\begin{filecontents*}{220-1.code}
@clef:G-2
@keysig:
@timesig:0
@data:2'A4''CCDEDxC1(D)4DC'A''CE{8DC}4D(C)://:
\end{filecontents*}
\commandline{ verovio --spacing-non-linear=0.50 -w 1500 --spacing-system=0.5 --adjust-page-height -b 0 220-1.code }
\newline
\includesvg[width=220pt]{220-1}%

\newline RISM-ID: 455010842
\newline Alte Foliierungen vorhanden: "20" und "21"
\newline D-Fh  Ms 001
\newline $\rightarrow$ In collection 418 (455008343)

\newline \par \vspace{7pt} \textcolor{darkblue}{\textbf{Anonymus  }}\hfillplus{221}
\newline Was Gott tut das ist wohlgetan  G  
\newline org
\newline \begin{itshape}[f.21r, at the tail of the page:] Was Gott thut das ist Wohlgethan\end{itshape} 
\newline \textcolor{darkblue}{\ding{\numexpr181 + 01}}  1 parts  
\newline Manuscript copy
\newline 1.1.1  org  G  
\begin{filecontents*}{221-1.code}
@clef:G-2
@keysig:xF
@timesig:0
@data:4'DGF{8ED}{AA}4F/8G4B8A{GFEE}{DEFG}{BB}4A2G://:
\end{filecontents*}
\commandline{ verovio --spacing-non-linear=0.50 -w 1500 --spacing-system=0.5 --adjust-page-height -b 0 221-1.code }
\newline
\includesvg[width=220pt]{221-1}%

\newline RISM-ID: 455010844
\newline Alte Foliierung vorhanden: "21"
\newline D-Fh  Ms 001
\newline $\rightarrow$ In collection 418 (455008343)

\newline \par \vspace{7pt} \textcolor{darkblue}{\textbf{Anonymus  }}\hfillplus{222}
\newline Was fürchst du Feind Herodes sehr    
\newline org
\newline \begin{itshape}[f.25v, at left:] Was fürchstu Feind Herodes sehr.\end{itshape} 
\newline \textcolor{darkblue}{\ding{\numexpr181 + 01}}  1 parts  
\newline Manuscript copy
\newline 1.1.1  org  
\begin{filecontents*}{222-1.code}
@clef:G-2
@keysig:
@timesig:0
@data:2'DFF4GA2D4EFGFG2E
\end{filecontents*}
\commandline{ verovio --spacing-non-linear=0.50 -w 1500 --spacing-system=0.5 --adjust-page-height -b 0 222-1.code }
\newline
\includesvg[width=220pt]{222-1}%

\newline RISM-ID: 455010859
\newline Alte Foliierung auf f.25v mit "26"
\newline D-Fh  Ms 001
\newline $\rightarrow$ In collection 418 (455008343)

\newline \par \vspace{7pt} \textcolor{darkblue}{\textbf{Anonymus  }}\hfillplus{223}
\newline Welt packe dich  D  
\newline org
\newline \begin{itshape}[f.14r, at left:] Welt packe dich\end{itshape} 
\newline \textcolor{darkblue}{\ding{\numexpr181 + 01}}  1 parts  
\newline Manuscript copy
\newline 1.1.1  org  D  
\begin{filecontents*}{223-1.code}
@clef:G-2
@keysig:xFC
@timesig:0
@data:8-{8'A6.F3F}8F-{''D6.C3'A}8A-''C{6'BA}8B @3/4 2A4A/BB''C/DD'A/
\end{filecontents*}
\commandline{ verovio --spacing-non-linear=0.50 -w 1500 --spacing-system=0.5 --adjust-page-height -b 0 223-1.code }
\newline
\includesvg[width=220pt]{223-1}%

\newline RISM-ID: 455010825
\newline Alte Foliierung vorhanden: "14"
\newline D-Fh  Ms 001
\newline $\rightarrow$ In collection 418 (455008343)

\newline \par \vspace{7pt} \textcolor{darkblue}{\textbf{Anonymus  }}\hfillplus{224}
\newline Weltlich' Ehr' und zeitlich' Gut  F  
\newline org
\newline \begin{itshape}[f.32v, at left:] Weltlich Ehr | Undt Zeitlich | Gut.\end{itshape} 
\newline \textcolor{darkblue}{\ding{\numexpr181 + 01}}  1 parts  
\newline Manuscript copy
\newline 1.1.1  org  F  
\begin{filecontents*}{224-1.code}
@clef:G-2
@keysig:bB
@timesig:0
@data:4'A{8GF+}{FDDE}4F/FG{8AB+}{BGAB}2''C/
\end{filecontents*}
\commandline{ verovio --spacing-non-linear=0.50 -w 1500 --spacing-system=0.5 --adjust-page-height -b 0 224-1.code }
\newline
\includesvg[width=220pt]{224-1}%

\newline RISM-ID: 455010882
\newline Alte Foliierung auf f.32v mit "33"
\newline D-Fh  Ms 001
\newline $\rightarrow$ In collection 418 (455008343)

\newline \par \vspace{7pt} \textcolor{darkblue}{\textbf{Anonymus  }}\hfillplus{225}
\newline Wenn dich Unglück tut greifen an  1t  
\newline org
\newline \begin{itshape}[f.41v, at left:] Wenn | dich Unglück\end{itshape} 
\newline \textcolor{darkblue}{\ding{\numexpr181 + 01}}  1 parts  
\newline Manuscript copy
\newline 1.1.1  org  1t  
\begin{filecontents*}{225-1.code}
@clef:G-2
@keysig:
@timesig:0
@data:2'A4A''C'AGFDEEFGAB''C'ABGA
\end{filecontents*}
\commandline{ verovio --spacing-non-linear=0.50 -w 1500 --spacing-system=0.5 --adjust-page-height -b 0 225-1.code }
\newline
\includesvg[width=220pt]{225-1}%

\newline RISM-ID: 455010899
\newline Alte Foliierung auf f.41v mit "41"
\newline D-Fh  Ms 001
\newline $\rightarrow$ In collection 418 (455008343)

\newline \par \vspace{7pt} \textcolor{darkblue}{\textbf{Anonymus  }}\hfillplus{226}
\newline Wenn mein Stündlein vorhanden ist  F  
\newline org
\newline \begin{itshape}[f.28v, at left:] Wenn mein | Stündlein vor | handen ist.\end{itshape} 
\newline \textcolor{darkblue}{\ding{\numexpr181 + 01}}  1 parts  
\newline Manuscript copy
\newline 1.1.1  org  F  
\begin{filecontents*}{226-1.code}
@clef:G-2
@keysig:bB
@timesig:0
@data:2'F4CDEFGA2F/A4''CC'A''C2'BA/
\end{filecontents*}
\commandline{ verovio --spacing-non-linear=0.50 -w 1500 --spacing-system=0.5 --adjust-page-height -b 0 226-1.code }
\newline
\includesvg[width=220pt]{226-1}%

\newline RISM-ID: 455010868
\newline Alte Foliierung auf f.28v mit "29"
\newline D-Fh  Ms 001
\newline $\rightarrow$ In collection 418 (455008343)

\newline \par \vspace{7pt} \textcolor{darkblue}{\textbf{Anonymus  }}\hfillplus{227}
\newline Wer Gott das Herze giebet    
\newline org
\newline \begin{itshape}[f.66r, at left:] Wer Gott daß | [a painted heart with letter "e", means "Herze"] giebet\end{itshape} 
\newline \textcolor{darkblue}{\ding{\numexpr181 + 01}}  1 parts  
\newline Manuscript copy
\newline 1.1.1  org  
\begin{filecontents*}{227-1.code}
@clef:G-2
@keysig:
@timesig:c
@data:8-{''E8.D6E}{8.C3DE}4'B{8''ECC}{DD}4E
\end{filecontents*}
\commandline{ verovio --spacing-non-linear=0.50 -w 1500 --spacing-system=0.5 --adjust-page-height -b 0 227-1.code }
\newline
\includesvg[width=220pt]{227-1}%

\newline RISM-ID: 455010938
\newline Alte Foliierung auf den gegenüberliegenden Seiten als Aufschlagfoliierung jeweils "56"
\newline D-Fh  Ms 001
\newline $\rightarrow$ In collection 418 (455008343)

\newline \par \vspace{7pt} \textcolor{darkblue}{\textbf{Anonymus  }}\hfillplus{228}
\newline Wer in dem Schutz des Höchsten ist  C  
\newline org
\newline \begin{itshape}[f.30v, at left:] Wer in den | schutz des | Höchsten | ist.\end{itshape} 
\newline \textcolor{darkblue}{\ding{\numexpr181 + 01}}  1 parts  
\newline Manuscript copy
\newline 1.1.1  org  C  
\begin{filecontents*}{228-1.code}
@clef:G-2
@keysig:
@timesig:0
@data:2''C4'B''C2DD4CD2E/F4EDC'AB''C2DC://:
\end{filecontents*}
\commandline{ verovio --spacing-non-linear=0.50 -w 1500 --spacing-system=0.5 --adjust-page-height -b 0 228-1.code }
\newline
\includesvg[width=220pt]{228-1}%

\newline RISM-ID: 455010875
\newline Alte Foliierung auf f.30v mit "31"
\newline D-Fh  Ms 001
\newline $\rightarrow$ In collection 418 (455008343)

\newline \par \vspace{7pt} \textcolor{darkblue}{\textbf{Anonymus  }}\hfillplus{229}
\newline Wie schön leuchtet der Morgenstern  F  
\newline org
\newline \begin{itshape}[f.26v, at left:] Wie schön leuch= | tet der Mor= | genstern\end{itshape} 
\newline \textcolor{darkblue}{\ding{\numexpr181 + 01}}  1 parts  
\newline Manuscript copy
\newline 1.1.1  org  F  
\begin{filecontents*}{229-1.code}
@clef:G-2
@keysig:bB
@timesig:0
@data:4'F''C'AF''CDD(C)-CDE2FE4DD2C
\end{filecontents*}
\commandline{ verovio --spacing-non-linear=0.50 -w 1500 --spacing-system=0.5 --adjust-page-height -b 0 229-1.code }
\newline
\includesvg[width=220pt]{229-1}%

\newline RISM-ID: 455010863
\newline Alte Foliierung auf f.26v mit "27"
\newline D-Fh  Ms 001
\newline $\rightarrow$ In collection 418 (455008343)

\newline \par \vspace{7pt} \textcolor{darkblue}{\textbf{Anonymus  }}\hfillplus{230}
\newline Wie schön leuchtet der Morgenstern  G  
\newline org
\newline \begin{itshape}[f.26v, at left:] Aliter ex | g\end{itshape} 
\newline \textcolor{darkblue}{\ding{\numexpr181 + 01}}  1 parts  
\newline Manuscript copy
\newline 1.1.1  org  G  
\begin{filecontents*}{230-1.code}
@clef:G-2
@keysig:xF
@timesig:0
@data:4'G''D'BG''DEE(D)-DEF2GF4EE2D
\end{filecontents*}
\commandline{ verovio --spacing-non-linear=0.50 -w 1500 --spacing-system=0.5 --adjust-page-height -b 0 230-1.code }
\newline
\includesvg[width=220pt]{230-1}%

\newline RISM-ID: 455010864
\newline Alte Foliierung auf f.26v mit "27"
\newline D-Fh  Ms 001
\newline $\rightarrow$ In collection 418 (455008343)

\newline \par \vspace{7pt} \textcolor{darkblue}{\textbf{Anonymus  }}\hfillplus{231}
\newline Wie schön leuchtet der Morgenstern  C  
\newline V, org
\newline \begin{itshape}[without title]\end{itshape} 
\newline \textcolor{darkblue}{\ding{\numexpr181 + 01}}  1 parts  
\newline Manuscript copy
\newline 1.1.1  V  C
\newline \begin{footnotesize} Wie schön leuchtet der Morgenstern \end{footnotesize}  
\begin{filecontents*}{231-1.code}
@clef:G-2
@keysig:
@timesig:c/
@data:2'C4GECGAA2G4-EFG{8AG}4G+GxF2G
\end{filecontents*}
\commandline{ verovio --spacing-non-linear=0.50 -w 1500 --spacing-system=0.5 --adjust-page-height -b 0 231-1.code }
\newline
\includesvg[width=220pt]{231-1}%

\newline RISM-ID: 455010906
\newline Vier Textstrophen unter der Notentabulatur aufgeführt
\newline Alte Foliierung auf f.45v mit "44"
\newline D-Fh  Ms 001
\newline $\rightarrow$ In collection 418 (455008343)

\newline \par \vspace{7pt} \textcolor{darkblue}{\textbf{Anonymus  }}\hfillplus{232}
\newline Wie schön leuchtet der Morgenstern  F  
\newline org
\newline \begin{itshape}[f.109v, heading:] Wie schön leuchtet der Morgen=Stern\end{itshape} 
\newline \textcolor{darkblue}{\ding{\numexpr181 + 01}}  1 parts  
\newline Manuscript copy
\newline 1.1.1  org  F  
\begin{filecontents*}{232-1.code}
@clef:C-1
@keysig:bB
@timesig:c
@data:4'CFD,B/'F''C'AF/''C{8.6DD}2C/4-CDE/4.8FEDC/2C4-
\end{filecontents*}
\commandline{ verovio --spacing-non-linear=0.50 -w 1500 --spacing-system=0.5 --adjust-page-height -b 0 232-1.code }
\newline
\includesvg[width=220pt]{232-1}%

\newline RISM-ID: 455011006
\newline Alte Paginierung vorhanden, f.109v: "85" und f.110r: "86"
\newline D-Fh  Ms 001
\newline $\rightarrow$ In collection 418 (455008343)

\newline \par \vspace{7pt} \textcolor{darkblue}{\textbf{Anonymus  }}\hfillplus{233}
\newline Wie's Gott gefällt    
\newline org
\newline \begin{itshape}[f.30r, at left:] Wies Gott | gefällt.\end{itshape} 
\newline \textcolor{darkblue}{\ding{\numexpr181 + 01}}  1 parts  
\newline Manuscript copy
\newline 1.1.1  org  
\begin{filecontents*}{233-1.code}
@clef:G-2
@keysig:
@timesig:
@data:8'A4''C8'G4A8G{FE}D4E{8FFF}4bB8A4GF://:
\end{filecontents*}
\commandline{ verovio --spacing-non-linear=0.50 -w 1500 --spacing-system=0.5 --adjust-page-height -b 0 233-1.code }
\newline
\includesvg[width=220pt]{233-1}%

\newline RISM-ID: 455010873
\newline Nur eine Melodiestimme ist notiert
\newline Alte Foliierung auf den gegenüberliegenden Seiten als Aufschlagfoliierung jeweils "30"
\newline D-Fh  Ms 001
\newline $\rightarrow$ In collection 418 (455008343)

\newline \par \vspace{7pt} \textcolor{darkblue}{\textbf{Anonymus  }}\hfillplus{234}
\newline Wir armen Sünder unser Missetat    
\newline org
\newline \begin{itshape}[f.56v, at left:] Wir armen Sünder, | Unser Mißetat. | Passionslied.\end{itshape} 
\newline \textcolor{darkblue}{\ding{\numexpr181 + 01}}  1 parts  
\newline Manuscript copy
\newline 1.1.1  org  
\begin{filecontents*}{234-1.code}
@clef:G-2
@keysig:
@timesig:c
@data:{8''DDDD}4ED{8E'BBA}2G{8''DDDD}4EC{8EEFE}2D
\end{filecontents*}
\commandline{ verovio --spacing-non-linear=0.50 -w 1500 --spacing-system=0.5 --adjust-page-height -b 0 234-1.code }
\newline
\includesvg[width=220pt]{234-1}%

\newline RISM-ID: 455010921
\newline Alte Foliierung auf f.56v mit "50"
\newline D-Fh  Ms 001
\newline $\rightarrow$ In collection 418 (455008343)

\newline \par \vspace{7pt} \textcolor{darkblue}{\textbf{Anonymus  }}\hfillplus{235}
\newline Wir glauben all' an einen Gott    
\newline org
\newline \begin{itshape}[f.18r, at left:] Wir Gläuben all | an einen Gott.\end{itshape} 
\newline \textcolor{darkblue}{\ding{\numexpr181 + 01}}  1 parts  
\newline Manuscript copy
\newline 1.1.1  org  
\begin{filecontents*}{235-1.code}
@clef:G-2
@keysig:
@timesig:0
@data:2'D4AxGAE2F4EGFEDxC(2D)/
\end{filecontents*}
\commandline{ verovio --spacing-non-linear=0.50 -w 1500 --spacing-system=0.5 --adjust-page-height -b 0 235-1.code }
\newline
\includesvg[width=220pt]{235-1}%

\newline RISM-ID: 455010835
\newline Alte Foliierung vorhanden: "18"
\newline D-Fh  Ms 001
\newline $\rightarrow$ In collection 418 (455008343)

\newline \par \vspace{7pt} \textcolor{darkblue}{\textbf{Anonymus  }}\hfillplus{236}
\newline Wo Gott zum Haus nicht gibt sein' Gunst  F  
\newline V (3), org
\newline \begin{itshape}[without title]\end{itshape} 
\newline \textcolor{darkblue}{\ding{\numexpr181 + 01}}  1 parts  
\newline Manuscript copy
\newline 1.1.1  org  F
\newline \begin{footnotesize} Wo Gott zum Haus nicht gibt sein' Gunst \end{footnotesize}  
\begin{filecontents*}{236-1.code}
@clef:G-2
@keysig:bB
@timesig:c/
@data:4'F{8''CC}4'A-8-F{''CC}{'AGAB}4''C-8-'G4A+AB''C8-C
\end{filecontents*}
\commandline{ verovio --spacing-non-linear=0.50 -w 1500 --spacing-system=0.5 --adjust-page-height -b 0 236-1.code }
\newline
\includesvg[width=220pt]{236-1}%

\newline RISM-ID: 455010914
\newline Unter der Tabulatur ist der Textbeginn in den drei unterschiedlichen Stimmen jeweils angegeben
\newline Alte Foliierung auf den gegenüberliegenden Seiten als Aufschlagfoliierung jeweils "46"
\newline D-Fh  Ms 001
\newline $\rightarrow$ In collection 418 (455008343)

\newline \par \vspace{7pt} \textcolor{darkblue}{\textbf{Anonymus  }}\hfillplus{237}
\newline Wo Gott zum Haus nicht gibt sein' Gunst  C  
\newline org
\newline \begin{itshape}[f.53v, at left:] Wo Gott Zum | Hauß\end{itshape} 
\newline \textcolor{darkblue}{\ding{\numexpr181 + 01}}  1 parts  
\newline Manuscript copy
\newline 1.1.1  org  C  
\begin{filecontents*}{237-1.code}
@clef:G-2
@keysig:
@timesig:c/
@data:2'C4GGEDEF2(G)4GEGFGED(C)
\end{filecontents*}
\commandline{ verovio --spacing-non-linear=0.50 -w 1500 --spacing-system=0.5 --adjust-page-height -b 0 237-1.code }
\newline
\includesvg[width=220pt]{237-1}%

\newline RISM-ID: 455010916
\newline Alte Foliierung auf f.53v mit "47"
\newline D-Fh  Ms 001
\newline $\rightarrow$ In collection 418 (455008343)

\newline \par \vspace{7pt} \textcolor{darkblue}{\textbf{Anonymus  }}\hfillplus{238}
\newline Wo bist du mein Jesu wo bleibst du so lange  G  
\newline org
\newline \begin{itshape}[f.12v, at left:] 19. | Wo bistu mein Jesu | wo bleibstu so lange?\end{itshape} 
\newline \textcolor{darkblue}{\ding{\numexpr181 + 01}}  1 parts  
\newline Manuscript copy
\newline 1.1.1  org  G  
\begin{filecontents*}{238-1.code}
@clef:G-2
@keysig:xF
@timesig:3/4
@data:8--''D{8.G6F8G}/4FDG/4.G8E4F/GGD/
\end{filecontents*}
\commandline{ verovio --spacing-non-linear=0.50 -w 1500 --spacing-system=0.5 --adjust-page-height -b 0 238-1.code }
\newline
\includesvg[width=220pt]{238-1}%

\newline RISM-ID: 455010820
\newline D-Fh  Ms 001
\newline $\rightarrow$ In collection 418 (455008343)

\newline \par \vspace{7pt} \textcolor{darkblue}{\textbf{Anonymus  }}\hfillplus{239}
\newline Wo soll ich fliehen hin  F  
\newline org
\newline \begin{itshape}[f.13v, at left:] Wo soll ich fliehen | hin | à 4\end{itshape} 
\newline \textcolor{darkblue}{\ding{\numexpr181 + 01}}  1 parts  
\newline Manuscript copy
\newline 1.1.1  org  F  
\begin{filecontents*}{239-1.code}
@clef:G-2
@keysig:bB
@timesig:c/
@data:4'F{8A''C}{8.'G6F}8F6-''C/{8FE8.D6C}4C8-C/{8'A''D8.D6xC}{8DD}-C/{6DE8F8.F6E}{8FF}-C/
\end{filecontents*}
\commandline{ verovio --spacing-non-linear=0.50 -w 1500 --spacing-system=0.5 --adjust-page-height -b 0 239-1.code }
\newline
\includesvg[width=220pt]{239-1}%

\newline RISM-ID: 455010823
\newline D-Fh  Ms 001
\newline $\rightarrow$ In collection 418 (455008343)

\newline \par \vspace{7pt} \textcolor{darkblue}{\textbf{Anonymus  }}\hfillplus{240}
\newline Wohlauf mein ganzes Ich  G  
\newline V, org
\newline \begin{itshape}[without title]\end{itshape} 
\newline \textcolor{darkblue}{\ding{\numexpr181 + 01}}  1 parts  
\newline Manuscript copy
\newline 1.1.1  V  G
\newline \begin{footnotesize} Wohlauf mein ganzes Ich \end{footnotesize}  
\begin{filecontents*}{240-1.code}
@clef:C-1
@keysig:xF
@timesig:c
@data:1'D/4GDGA/2(B)4-''D/CC'BB/2A(G)://:1D/4DDED/2xC(D)/
\end{filecontents*}
\commandline{ verovio --spacing-non-linear=0.50 -w 1500 --spacing-system=0.5 --adjust-page-height -b 0 240-1.code }
\newline
\includesvg[width=220pt]{240-1}%

\newline RISM-ID: 455010992
\newline Nur der Textbeginn wie angegeben steht zwischen den beiden Systemen
\newline Alte Zählung auf f.103r vorhanden: "72"
\newline D-Fh  Ms 001
\newline $\rightarrow$ In collection 418 (455008343)

\newline \par \vspace{7pt} \textcolor{darkblue}{\textbf{Bach, Carl Philipp Emanuel  1714-1788}}\hfillplus{241}
\newline Oratorios    
\newline \begin{itshape}Karl Wilhelm Rammlers Auferstehung und Himmelfahrt Jesu\end{itshape} 
\newline \textcolor{darkblue}{\ding{\numexpr181 + 01}}  1787  score  
\newline print
\newline RISM-ID: 00000990003083
\newline Rammler, Karl Wilhelm  (oth)
\newline Breitkopf, Johann Gottlob Immanuel  (pbl)
\newline A-Wgm; A-Wn; A-Wn-h; CH-Gpu; CH-Wk; D-Bhm; D-Dl; D-DT; D-EIb; D-ERik; D-Fh; D-Gs; D-HAu; D-Hs; D-KIl; D-LEm; D-Lr; D-LÜh; D-Mbs; D-MZsch; D-Ngm; D-Rp; D-SWl; D-W; D-WRh; DK-A; DK-Kk; F-Pc; GB-Bu; GB-Lbl; GB-Lcm; GB-T; H-Bn; I-Fc; IRL-Dtc; J-Tn; NL-DHgm; PL-Wu; RUS-Mk; RUS-Mrg; RUS-SPk  Инв. № 28296; S-Skma; S-Uu; US-AAu; US-Bp; US-CA; US-Cn; US-NH; US-R; US-U; US-Wc
\newline \par \vspace{7pt} \textcolor{darkblue}{\textbf{Bachofen, Johann Caspar  1695-1755}}\hfillplus{242}
\newline Sacred songs    
\newline \begin{itshape}... achte und privilegirte Auflag\end{itshape} 
\newline RISM-ID: 00000990003503
\newline Bürckli  (pbl)
\newline CH-BEsu; CH-E; CH-SGv; CH-Zz; D-Fh; D-Mbs; D-WL; NL-DHk; US-Cn; US-Hm
\newline \par \vspace{7pt} \textcolor{darkblue}{\textbf{Blasius, Matthieu-Frédéric  1758-1829}}\hfillplus{243}
\newline Duets (instr.)    
\newline \begin{itshape}...\end{itshape} 
\newline \textcolor{darkblue}{\ding{\numexpr181 + 01}}  part(s)  
\newline print
\newline RISM-ID: 00000991014368
\newline Legouix, Napoleon  (pbl)
\newline D-Fh  Str Duo 0034
\newline \par \vspace{7pt} \textcolor{darkblue}{\textbf{Boccherini, Luigi  1743-1805}}\hfillplus{244}
\newline Quintets (instr.)  C; g; E|b; A; E|b; d; C; A; B|b; B|b; D; E|b; g; F; B|b; C; D; c; f; B|b; F; F; D; c  
\newline 2 vl, a-vla, 2 vlc
\newline \begin{itshape}Douze ([oder:] Vingt-quatre) nouveaux quintetti [C, g, Es - A, Es, d - C, A, B - B, D, Es - g, F, B - C, D, c - f, B, F - F, D, c] pour deux violons, deux violoncelles et alto ... la première partie de violoncelle pourra être remplacée par l'alto-violoncelle ... 1|e (-8|e) livraison, œuvre 37\end{itshape} 
\newline \textcolor{darkblue}{\ding{\numexpr181 + 01}}  part(s)  
\newline print
\newline RISM-ID: 00000990005924
\newline Richomme); (Aubert, I.  (prt)
\newline Pleyel, Ignace (Richomme / I. Aubert)  (pbl)
\newline A-Wgm; A-Wn; B-Ac; B-Br; B-Gc; CH-BEk; CH-BEl; CH-Bu; CH-E; CH-Gc; CH-Gpu; CH-LAc  FA 5 ONS 1; FA 5 BOC 1-3, 10-12, 14-18; CH-N; CZ-Bu; D-B; D-Bhm; D-BNba; D-Dmb; D-F  Mus. pr. Q 82/72 Nr. 1; D-Fh  Str Quint 0009; D-HVh; D-KNmi; D-LEm; D-Mbs; D-Rp; D-Tmi; DK-Kk; E-Mp  Mp/2394|1; Mp/2394|2; F-A; F-AU; F-C; F-Pc; F-Pn; F-Prothschild; F-TLc; GB-Lbl; GB-Mp; GB-SA; GB-Yu; I-Bc  Z.269; I-Fc; I-Li; I-Mc; I-Tco; J-Tk; J-Tma; NL-DHgm; RUS-Mrg; S-Skma; US-CA; US-Cn; US-PHu
\newline \par \vspace{7pt} \textcolor{darkblue}{\textbf{Boccherini, Luigi  1743-1805}}\hfillplus{245}
\newline Trios (instr.)  f; G; E|b; D; C; E  
\newline 2 vl, vlc
\newline \begin{itshape}Sei trio [f, G, Es, D, C, E] per due violini e violoncello ... œuvre XXXV\end{itshape} 
\newline \textcolor{darkblue}{\ding{\numexpr181 + 01}}  part(s)  
\newline print
\newline RISM-ID: 00000990005831
\newline Artaria  (pbl)
\newline A-Gk; A-Wgm; D-Fh  Str Trio 0030; D-HR; D-Mmb; DK-Kk; E-Mn  Mp/2392|6; FIN-A; GB-Lbl; I-AN; I-Bc  DD.325; I-Mc; I-Nc; I-Vc; NL-Uim; RUS-SPsc
\newline \par \vspace{7pt} \textcolor{darkblue}{\textbf{Böddecker, Philipp Friedrich  1607-1683}}\hfillplus{246}
\newline Ballets  G  
\newline pf
\newline \begin{itshape}[f.89v, at left:] Ballet | Bödeckers.\end{itshape} 
\newline \textcolor{darkblue}{\ding{\numexpr181 + 01}}  1 parts  
\newline Manuscript copy
\newline 1.1.1  pf  G  
\begin{filecontents*}{246-1.code}
@clef:G-2
@keysig:xF
@timesig:0
@data:8''D{8.D6D}{EEFF}{GGFE}4D{6EFGA}{FG8E}4.D://:
\end{filecontents*}
\commandline{ verovio --spacing-non-linear=0.50 -w 1500 --spacing-system=0.5 --adjust-page-height -b 0 246-1.code }
\newline
\includesvg[width=220pt]{246-1}%

\newline RISM-ID: 455010970
\newline Ohne alte Foliierung
\newline D-Fh  Ms 001
\newline $\rightarrow$ In collection 418 (455008343)

\newline \par \vspace{7pt} \textcolor{darkblue}{\textbf{Böddecker, Philipp Friedrich  1607-1683}}\hfillplus{247}
\newline Ballets  G  
\newline pf
\newline \begin{itshape}[f.89v, at left:] Ballet.\end{itshape} 
\newline \textcolor{darkblue}{\ding{\numexpr181 + 01}}  1 parts  
\newline Manuscript copy
\newline 1.1.1  pf  G  
\begin{filecontents*}{247-1.code}
@clef:G-2
@keysig:xF
@timesig:0
@data:8''D{8.D6D}{EEDC}{'BGB''D}4D{8.C6D}{C'BAG}{8.A6G}{FDB''C}4.D
\end{filecontents*}
\commandline{ verovio --spacing-non-linear=0.50 -w 1500 --spacing-system=0.5 --adjust-page-height -b 0 247-1.code }
\newline
\includesvg[width=220pt]{247-1}%

\newline RISM-ID: 455010971
\newline Ohne alte Foliierung
\newline D-Fh  Ms 001
\newline $\rightarrow$ In collection 418 (455008343)

\newline \par \vspace{7pt} \textcolor{darkblue}{\textbf{Böddecker, Philipp Friedrich  1607-1683}}\hfillplus{248}
\newline Ballets  G  
\newline pf
\newline \begin{itshape}[f.89v, at left:] Ballet.\end{itshape} 
\newline \textcolor{darkblue}{\ding{\numexpr181 + 01}}  1 parts  
\newline Manuscript copy
\newline 1.1.1  pf  G  
\begin{filecontents*}{248-1.code}
@clef:G-2
@keysig:xF
@timesig:0
@data:8''G{8.G6G}{FGEF}{8.D6'G}{AB''CD}{8.'B6B}{''CDC'B}{8.A6G}{FGEF}{8.D6E}{8F6GA}
\end{filecontents*}
\commandline{ verovio --spacing-non-linear=0.50 -w 1500 --spacing-system=0.5 --adjust-page-height -b 0 248-1.code }
\newline
\includesvg[width=220pt]{248-1}%

\newline RISM-ID: 455010972
\newline Ohne alte Foliierung
\newline D-Fh  Ms 001
\newline $\rightarrow$ In collection 418 (455008343)

\newline \par \vspace{7pt} \textcolor{darkblue}{\textbf{Bruni, Antonio Bartolomeo  1759-1821}}\hfillplus{249}
\newline Duets (instr.)  C; B|b; D; A; g; F  
\newline 2 vl
\newline \begin{itshape}Trois [= Six] duos concertants [C, B, D, A, g, F - in 2 Teilen] pour deux violons ... œuvre 26, liv: 1 (2)\end{itshape} 
\newline \textcolor{darkblue}{\ding{\numexpr181 + 01}}  part(s)  
\newline print
\newline RISM-ID: 00000990007402
\newline André, Johann  (pbl)
\newline D-Fh; D-HEms; D-LÜh
\newline \par \vspace{7pt} \textcolor{darkblue}{\textbf{Bruni, Antonio Bartolomeo  1759-1821}}\hfillplus{250}
\newline Duets (instr.)  C; B|b; D; A; g; F  
\newline 2 vl
\newline \begin{itshape}Trio | DUOS | CONCERTANS | Pour deux Violons | Composés par | M. BRUNI | Ouv. 26 [D-Fh]\end{itshape} 
\newline RISM-ID: 00000990007404
\newline Garbet, Mlle  (prt)
\newline Duhan, Jeanne-Elisabeth (Mlle Garbet)  (pbl)
\newline B-Gc; D-Fh  Str Duo 0060; I-BRc; RUS-Mrg; S-Skma; S-Uu
\newline \par \vspace{7pt} \textcolor{darkblue}{\textbf{Bruni, Antonio Bartolomeo  1759-1821}}\hfillplus{251}
\newline Duets (instr.)  C; F; B|b; C; G; D  
\newline 2 vl
\newline \begin{itshape}Six duos [C, F, B, C, G, D] pour deux violons ... œuvre 34\end{itshape} 
\newline \textcolor{darkblue}{\ding{\numexpr181 + 01}}  score  
\newline print
\newline RISM-ID: 00000990007405
\newline André, Johann  (pbl)
\newline D-Fh  Str Duo 0058; D-KNh  R 934; D-OF; D-Tu; I-OS; S-Uu
\newline \par \vspace{7pt} \textcolor{darkblue}{\textbf{Bunte, Johann Friedrich  19.sc}}\hfillplus{252}
\newline Duets (instr.)  G; F; C; B|b; C; D  
\newline 2 vl
\newline \begin{itshape}Six petits duos [G, F, C, B, C, D] pour deux violons. Cahier 1.\end{itshape} 
\newline \textcolor{darkblue}{\ding{\numexpr181 + 01}}  part(s)  
\newline print
\newline RISM-ID: 00000991014862
\newline F. J. Weygand  (pbl)
\newline D-Fh  Str Duo 0064
\newline \par \vspace{7pt} \textcolor{darkblue}{\textbf{Bunte, Johann Friedrich  19.sc}}\hfillplus{253}
\newline Trios (instr.)  D  
\newline 2 vl, vlc
\newline \begin{itshape}TRIO [D] | Pour Deux | VIOLONS et VIOLONCELLE | composé par | F. Bunte | Oeuvre VI\end{itshape} 
\newline \textcolor{darkblue}{\ding{\numexpr181 + 01}}  [1816c]  parts  
\newline print
\newline RISM-ID: 00000990007535
\newline Hummel, Johann Julius  (pbl)
\newline Hummel, Johann Julius  (pbl)
\newline D-B  DMS 68082; D-Fh  Str Trio 0034; NL-DHgm
\newline \par \vspace{7pt} \textcolor{darkblue}{\textbf{Call, Leonhard von  1767-1815}}\hfillplus{254}
\newline Duets (instr.)  G; D; G  
\newline 2 vl
\newline \begin{itshape}Trois duos concertants [G, D, G] faciles et progressives pour deux violons ... œuvre 1 des duos\end{itshape} 
\newline \textcolor{darkblue}{\ding{\numexpr181 + 01}}  part(s)  
\newline print
\newline RISM-ID: 00000991014952
\newline Schott, Bernhard  (pbl)
\newline D-Fh  Str Duo 0065; D-MZsch; D-RH; D-Sl; D-Tu
\newline \par \vspace{7pt} \textcolor{darkblue}{\textbf{Call, Leonhard von  1767-1815}}\hfillplus{255}
\newline Quartets (instr.)  F  
\newline 2 vl, vla, vlc
\newline \begin{itshape}No. III. Quartett [F] für zwey Violinen, Viola und Violoncello ... 141|t|e|s Werk\end{itshape} 
\newline \textcolor{darkblue}{\ding{\numexpr181 + 01}}  part(s)  
\newline print
\newline RISM-ID: 00000991015126
\newline Haslinger, Tobias  (pbl)
\newline A-Wn; A-Wst; CH-SO  DA I 635; D-Fh  Str Quart 0185
\newline \par \vspace{7pt} \textcolor{darkblue}{\textbf{Cambini, Giuseppe Maria  1746-1825}}\hfillplus{256}
\newline Duets (instr.)  C; F; B|b; D; G; A  
\newline 2 vl
\newline \begin{itshape}Six | Duo | FACILE [C, F, B, D, G, A] | Pour deux Violons | PAR | G. Cambini, | OEUVRE\end{itshape} 
\newline \textcolor{darkblue}{\ding{\numexpr181 + 01}}  [1788c]  part(s)  
\newline print
\newline RISM-ID: 00000991015240
\newline "Pour Paris et la Province port franc par la Poste. | à Paris. | Chez LE DUC, Editeur de Musique et Md. d'Instruments. Rue Neuve | des Petits-Champs, No. 1286, vis à vis la Trésorerie pres la Rue Vivienne] | Et Rue du Roulle à la Croix d'Or No. 290 " zum Teil überklebt
\newline Leduc  (pbl)
\newline D-Fh  Str Duo 0066
\newline \par \vspace{7pt} \textcolor{darkblue}{\textbf{Campagnoli, Bartolomeo  1751-1827}}\hfillplus{257}
\newline Duets (instr.)  E|b; G; E  
\newline 2 vl
\newline \begin{itshape}Trois duos [Es, G, E] concertans pour deux violons ... op. IX\end{itshape} 
\newline \textcolor{darkblue}{\ding{\numexpr181 + 01}}  part(s)  
\newline print
\newline RISM-ID: 00000990008354
\newline Breitkopf  Härtel  (pbl)
\newline A-Wgm; B-Gc; CZ-Pnm; D-B; D-Fh  Str Duo 0067; D-Mbs; I-Nc; I-Nn; J-Tma; S-Skma
\newline \par \vspace{7pt} \textcolor{darkblue}{\textbf{Campagnoli, Bartolomeo  1751-1827}}\hfillplus{258}
\newline Duets (instr.)  D; A; B|b  
\newline 2 vl
\newline \begin{itshape}Trois duos [D, A, B] pour deux violons\end{itshape} 
\newline \textcolor{darkblue}{\ding{\numexpr181 + 01}}  part(s)  
\newline print
\newline RISM-ID: 00000990008366
\newline Breitkopf  Härtel  (pbl)
\newline A-RB; B-Bc; B-Gc; CH-E; D-F  Mus. pr. Q 52/784; D-Fh; D-LEm; D-Mbs; DK-Kk; I-Fc
\newline \par \vspace{7pt} \textcolor{darkblue}{\textbf{Campagnoli, Bartolomeo  1751-1827}}\hfillplus{259}
\newline Duets (instr.)    
\newline RISM-ID: 00000991015254
\newline Peters, Carl Friedrich  (pbl)
\newline CZ-BER; D-Fh  Str Duo 0067
\newline \par \vspace{7pt} \textcolor{darkblue}{\textbf{Campagnoli, Bartolomeo  1751-1827}}\hfillplus{260}
\newline Duets (instr.)    
\newline 2 vl
\newline \begin{itshape}Six progressive duetts for two violins\end{itshape} 
\newline RISM-ID: 00000991015255
\newline Vernon, Charles  (pbl)
\newline D-Fh  Str Duo 0067
\newline \par \vspace{7pt} \textcolor{darkblue}{\textbf{Campagnoli, Bartolomeo  1751-1827}}\hfillplus{261}
\newline Sonatas  D  
\newline vl, vla
\newline \begin{itshape}L'illusion de la viole d'amour. Sonate notturne [D] pour violon ... avec accompagnement de viola ... œuv. 16\end{itshape} 
\newline \textcolor{darkblue}{\ding{\numexpr181 + 01}}  part(s)  
\newline print
\newline RISM-ID: 00000990008363
\newline Breitkopf  Härtel  (pbl)
\newline B-Gc; CZ-Pk; D-B; D-F  Mus. pr. Q 54/331; D-Fh  Str Duo 0067; I-BGi; S-Skma
\newline \par \vspace{7pt} \textcolor{darkblue}{\textbf{Clemsee, Christoph  16/17}}\hfillplus{262}
\newline Fugues  C  
\newline org
\newline \begin{itshape}[f.86v, at left:] Fuga à 4 voc. | Christoph Klemse | ex C.\end{itshape} 
\newline \textcolor{darkblue}{\ding{\numexpr181 + 01}}  1 parts  
\newline Manuscript copy
\newline 1.1.1  org  C  
\begin{filecontents*}{262-1.code}
@clef:G-2
@keysig:
@timesig:0
@data:{8''GGEG}{GEDF}{EEDC}{DE}4DE
\end{filecontents*}
\commandline{ verovio --spacing-non-linear=0.50 -w 1500 --spacing-system=0.5 --adjust-page-height -b 0 262-1.code }
\newline
\includesvg[width=220pt]{262-1}%

\newline RISM-ID: 455010967
\newline Ohne alte Foliierung
\newline D-Fh  Ms 001
\newline $\rightarrow$ In collection 418 (455008343)

\newline \par \vspace{7pt} \textcolor{darkblue}{\textbf{Cupis, Jean-Baptiste  1741*}}\hfillplus{263}
\newline Airs variés    
\newline 2 vlc
\newline \begin{itshape}Recueil de petits airs variés et dialogués pour deux violoncelles ... œuvre 9\end{itshape} 
\newline \textcolor{darkblue}{\ding{\numexpr181 + 01}}  [1803c]  score  
\newline print
\newline RISM-ID: 00000990011925
\newline Pleyel  (pbl)
\newline D-Fh  Str Duo 0070; GB-LVp
\newline \par \vspace{7pt} \textcolor{darkblue}{\textbf{Danzi, Franz  1763-1826}}\hfillplus{264}
\newline Duets (instr.)  D; g; F  
\newline vla, vlc
\newline \begin{itshape}TROIS DUOS [D, g-G, F] | pour | Alto et Violoncelle | Composés | par | François Danzi |  Oeuvre IX\end{itshape} 
\newline \textcolor{darkblue}{\ding{\numexpr181 + 01}}  [1802c]  part(s)  
\newline print
\newline RISM-ID: 00000990012909
\newline Falter, Makarius  (pbl)
\newline A-SF; B-Bc; D-Fh  Str Duo 0301; HR-Zha
\newline \par \vspace{7pt} \textcolor{darkblue}{\textbf{Danzi, Franz  1763-1826}}\hfillplus{265}
\newline Sonatas  A; C; d; G; B|b; f; F  
\newline 2 vlc
\newline \begin{itshape}Trois sonates pour [deux] violoncelle[s] ... opera I, 1|e|r [A, C, d] (2|e [G, B, f-F]) livre\end{itshape} 
\newline \textcolor{darkblue}{\ding{\numexpr181 + 01}}  part(s)  
\newline print
\newline RISM-ID: 00000990012918
\newline Nägeli, Hans Georg  (pbl)
\newline CH-Bu; CH-W; D-B  N.Mus.Nachl. 99,134; D-Bhm; D-Fh  Str Duo 0300; US-PHu
\newline \par \vspace{7pt} \textcolor{darkblue}{\textbf{Effler, Johann  1640c-1711}}\hfillplus{266}
\newline Preludes  a  
\newline org
\newline \begin{itshape}[f.74v, at left:] Praeludium | Johan̅ Effleri | juniorir.\end{itshape} 
\newline \textcolor{darkblue}{\ding{\numexpr181 + 01}}  1 parts  
\newline Manuscript copy
\newline 1.1.1  org  a  
\begin{filecontents*}{266-1.code}
@clef:G-2
@keysig:
@timesig:0
@data:2'A{6AB''CD}4E6-{EDC}4D6-{DC'B}4''C6-{C'BA}4B6-{GAB}4''C
\end{filecontents*}
\commandline{ verovio --spacing-non-linear=0.50 -w 1500 --spacing-system=0.5 --adjust-page-height -b 0 266-1.code }
\newline
\includesvg[width=220pt]{266-1}%

\newline RISM-ID: 455010949
\newline Ohne alte Foliierung
\newline D-Fh  Ms 001
\newline $\rightarrow$ In collection 418 (455008343)

\newline \par \vspace{7pt} \textcolor{darkblue}{\textbf{Effler, Johann  1640c-1711}}\hfillplus{267}
\newline Preludes  D  
\newline org
\newline \begin{itshape}[f.75r, at left:] Praelud. | ex D. | Joan Effl.\end{itshape} 
\newline \textcolor{darkblue}{\ding{\numexpr181 + 01}}  1 parts  
\newline Manuscript copy
\newline 1.1.1  org  D  
\begin{filecontents*}{267-1.code}
@clef:G-2
@keysig:xFC
@timesig:0
@data:{6'ABAG}4A6-{''DC'A}{GBAF}{EAGE}{F3GA}{B''CDE}4F{8FEDC}6-{ED'B}{''C'ABA}4xGA
\end{filecontents*}
\commandline{ verovio --spacing-non-linear=0.50 -w 1500 --spacing-system=0.5 --adjust-page-height -b 0 267-1.code }
\newline
\includesvg[width=220pt]{267-1}%

\newline RISM-ID: 455010950
\newline Ohne alte Foliierung
\newline D-Fh  Ms 001
\newline $\rightarrow$ In collection 418 (455008343)

\newline \par \vspace{7pt} \textcolor{darkblue}{\textbf{Effler, Johann  1640c-1711}}\hfillplus{268}
\newline Toccatas  1t  
\newline org
\newline \begin{itshape}[f.91v, at left:] Toccata | super Magni= | ficat. JE.\end{itshape} 
\newline \textcolor{darkblue}{\ding{\numexpr181 + 01}}  1 parts  
\newline Manuscript copy
\newline 1.1.1  org  1t  
\begin{filecontents*}{268-1.code}
@clef:G-2
@keysig:
@timesig:0
@data:{8'D6EF}{3GFED}8xC{6.D3E}{FEFD}{6.A3B}{''C'B''C'A}4''D{8CE}2'A
\end{filecontents*}
\commandline{ verovio --spacing-non-linear=0.50 -w 1500 --spacing-system=0.5 --adjust-page-height -b 0 268-1.code }
\newline
\includesvg[width=220pt]{268-1}%

\newline 1.2.1  org  1t  
\begin{filecontents*}{268-2.code}
@clef:G-2
@keysig:
@timesig:0
@data:2'A''C'AAAABAGF
\end{filecontents*}
\commandline{ verovio --spacing-non-linear=0.50 -w 1500 --spacing-system=0.5 --adjust-page-height -b 0 268-2.code }
\newline
\includesvg[width=220pt]{268-2}%

\newline RISM-ID: 455010974
\newline Ohne alte Foliierung
\newline D-Fh  Ms 001
\newline $\rightarrow$ In collection 418 (455008343)

\newline \par \vspace{7pt} \textcolor{darkblue}{\textbf{Fastre, J.  ??}}\hfillplus{269}
\newline Duets (instr.)  D; A; F; B|b  
\newline 2 vl
\newline \begin{itshape}Six Duos faciles | pour | Deux Violons | composés | à l'usage des commençans| par | J. FASTRÉ | Oeuvre XIII, [nos.: III – VI; D, A, F, B]\end{itshape} 
\newline \textcolor{darkblue}{\ding{\numexpr181 + 01}}  part(s)  
\newline print
\newline RISM-ID: 00000991018146
\newline Steup, H. C.  (pbl)
\newline D-Fh  Str Duo 0311
\newline \par \vspace{7pt} \textcolor{darkblue}{\textbf{Fodor, Josephus  1751-1828}}\hfillplus{270}
\newline Duets (instr.)  C; F; B|b; E|b; A; B|b  
\newline 2 vl
\newline \begin{itshape}Six duos [C, F, B, Es, A, B] à deux violons ... op. 1\end{itshape} 
\newline \textcolor{darkblue}{\ding{\numexpr181 + 01}}  [1775c]  part(s)  
\newline print
\newline RISM-ID: 00000990018493
\newline Hummel, Johann Julius  (pbl)
\newline Hummel, Johann Julius  (pbl)
\newline D-Fh  Str Duo 0319; NL-DHgm
\newline \par \vspace{7pt} \textcolor{darkblue}{\textbf{Fodor, Josephus  1751-1828}}\hfillplus{271}
\newline Duets (instr.)  D; G; A; C; D; G  
\newline 2 vl
\newline \begin{itshape}SIX | DUOS | A | DEUX VIOLONS [D, G, A, C, D, G] | COMPOSÉS PAR | J. FODOR, | Premier Violon de la | Musique de Monseigneur le | Duc de Montmorency | OEUVRE X.\end{itshape} 
\newline \textcolor{darkblue}{\ding{\numexpr181 + 01}}  part(s)  
\newline print
\newline RISM-ID: 00000990018501
\newline Hummel, Johann Julius  (pbl)
\newline Hummel, Johann Julius  (pbl)
\newline CH-Zjacobi; D-Fh  Str Duo 0318; NL-At; NL-DHgm; NL-Uim; RUS-Mrg
\newline \par \vspace{7pt} \textcolor{darkblue}{\textbf{Fodor, Josephus  1751-1828}}\hfillplus{272}
\newline Duets (instr.)  C; G; a  
\newline 2 vl
\newline \begin{itshape}TROIS DUOS [C, G, A] | Pour Deux Violons | Melés de Airs Variés | Par  M.|r J. FODOR | Oeuvre XXIV\end{itshape} 
\newline RISM-ID: 00000990018523
\newline Hummel, Johann Julius  (pbl)
\newline Hummel, Johann Julius  (pbl)
\newline D-Fh  Str Duo 0320; FIN-A; J-Tma; NL-DHgm; NL-Uim; S-L; S-Skma
\newline \par \vspace{7pt} \textcolor{darkblue}{\textbf{Fodor, Josephus  1751-1828}}\hfillplus{273}
\newline Duets (instr.)    
\newline \begin{itshape}op. 14. Nouvelle édition corrigée ... et divisée en deux livres\end{itshape} 
\newline \textcolor{darkblue}{\ding{\numexpr181 + 01}}  part(s)  
\newline print
\newline RISM-ID: 00000991018516
\newline Charles  (pbl)
\newline D-Fh  Str Duo 0317
\newline \par \vspace{7pt} \textcolor{darkblue}{\textbf{Fodor, Josephus  1751-1828}}\hfillplus{274}
\newline Quartets (instr.)    
\newline 2 vl, a-vla, b
\newline \begin{itshape}Six quatuors concertants pour deux violons, alto et basse ... œuvre XI|e\end{itshape} 
\newline RISM-ID: 00000990018480
\newline Ollivier, Mme  (prt)
\newline Bailleux, Antoine (Mme Ollivier)  (pbl)
\newline B-Bc; D-Fh  Str Quart 0102; D-WEMl; E-Zac; F-Pn; GB-Lbl; I-Fc; I-Gl; I-Tn; I-Vc; US-AAu; US-NYhill; US-Wc
\newline \par \vspace{7pt} \textcolor{darkblue}{\textbf{Franck, Melchior  1579c-1639}}\hfillplus{275}
\newline Erstanden ist der heilig' Christ    
\newline org
\newline \begin{itshape}[f.56v, at left:] Oster lied | Erstand ist d h | Christ pp | Compos. Melch: frank\end{itshape} 
\newline \textcolor{darkblue}{\ding{\numexpr181 + 01}}  1 parts  
\newline Manuscript copy
\newline 1.1.1  org  
\begin{filecontents*}{275-1.code}
@clef:G-2
@keysig:
@timesig:3
@data:4--'F/24FA/''C'G/''C'B/''C'A/4.F8G4A/24''C'B/AG/
\end{filecontents*}
\commandline{ verovio --spacing-non-linear=0.50 -w 1500 --spacing-system=0.5 --adjust-page-height -b 0 275-1.code }
\newline
\includesvg[width=220pt]{275-1}%

\newline RISM-ID: 455010922
\newline Alte Foliierung auf f.56v mit "50"
\newline D-Fh  Ms 001
\newline $\rightarrow$ In collection 418 (455008343)

\newline \par \vspace{7pt} \textcolor{darkblue}{\textbf{Gallus, Iacobus  1550-1591}}\hfillplus{276}
\newline Laus et perennis gloria    HK 246
\newline Coro (2)
\newline \begin{itshape}[f.62v, at left:] 1. Chor | Laus et peren | nis Gloria | â 8. | Glori lob Ehr | U. herrlichk. 2. Cor.\end{itshape} 
\newline \textcolor{darkblue}{\ding{\numexpr181 + 01}}  short score (tabulature): f.62v-65v  
\newline Manuscript copy
\newline 1.1.1  S coro 1
\newline \begin{footnotesize} Laus et perennis gloria, Deo patri \end{footnotesize}  
\begin{filecontents*}{276-1.code}
@clef:G-2
@keysig:
@timesig:c/
@data:4'E{8xGA}{GABA}/4G-1-
\end{filecontents*}
\commandline{ verovio --spacing-non-linear=0.50 -w 1500 --spacing-system=0.5 --adjust-page-height -b 0 276-1.code }
\newline
\includesvg[width=220pt]{276-1}%

\newline 1.1.2  S coro 2
\newline \begin{footnotesize} Glori Lob Ehr' und Herrlichkeit \end{footnotesize}  
\begin{filecontents*}{276-2.code}
@clef:G-2
@keysig:
@timesig:c/
@data:2--4'E{8xGAGA}{BA}2G1-
\end{filecontents*}
\commandline{ verovio --spacing-non-linear=0.50 -w 1500 --spacing-system=0.5 --adjust-page-height -b 0 276-2.code }
\newline
\includesvg[width=220pt]{276-2}%

\newline RISM-ID: 455010934
\newline Alte Foliierung auf f.62v mit "54"
\newline no.57 in "Tertius tomus musici operis"; dort mit doppelten Notenwerten
\newline Literature: BezecnýH 1899  pt.4, p.182
\newline D-Fh  Ms 001
\newline $\rightarrow$ In collection 418 (455008343)

\newline \par \vspace{7pt} \textcolor{darkblue}{\textbf{Gebauer, Michel-Joseph  1763-1812}}\hfillplus{277}
\newline Duets (instr.)  G; D; C  
\newline 2 vl
\newline \begin{itshape}Trois duos pour deux violons à l'usage des commençans avec des notices pour faciliter aux élèves le moyen de retenir et observer les différentes instructions qui caractérisent chaque morceau ... œuvre X, livre II [G, D, C]\end{itshape} 
\newline RISM-ID: 00000990019993
\newline Nolting, W. C., Vve  fils  (pbl)
\newline D-Fh  Str Duo 0323; NL-At
\newline \par \vspace{7pt} \textcolor{darkblue}{\textbf{Gebauer, Michel-Joseph  1763-1812}}\hfillplus{278}
\newline Duets (instr.)  C; F; a  
\newline 2 vl
\newline \begin{itshape}Trois duos pour deux violons à l'usage des commençans avec des notices pour faciliter aux élèves le moyen de retenir et observer les différentes instructions qui caractérisent chaque morceau ... oeuvre X, livre I [C, F, a]\end{itshape} 
\newline RISM-ID: 00000991018852
\newline Nolting, W. C., Vve  fils  (pbl)
\newline D-Fh  Str Duo 0323
\newline \par \vspace{7pt} \textcolor{darkblue}{\textbf{Grasset, Jean-Jacques  1769c-1839}}\hfillplus{279}
\newline Duets (instr.)  D; d; f  
\newline 2 vl
\newline \begin{itshape}Trois duos concertants [D, d, f] pour deux violons ... œuvre 9e (4|e livre de duos)\end{itshape} 
\newline \textcolor{darkblue}{\ding{\numexpr181 + 01}}  part(s)  
\newline print
\newline RISM-ID: 00000990022612
\newline Sieber, Georges-Julien  (pbl)
\newline D-B; D-Cl; D-Fh  Str Duo 0325; D-Mbs; F-Pn
\newline \par \vspace{7pt} \textcolor{darkblue}{\textbf{Gravrand, Joseph  1770-1847c}}\hfillplus{280}
\newline Duets (instr.)    
\newline 2 vl
\newline \begin{itshape}Trois duos concertants pour deux violons ... œuvre 2|m|e\end{itshape} 
\newline \textcolor{darkblue}{\ding{\numexpr181 + 01}}  part(s)  
\newline print
\newline RISM-ID: 00000990022652
\newline Gaveaux, Guillaume  (pbl)
\newline D-Fh  Str Duo 0327; D-LEmi; D-Mbs; US-STu
\newline \par \vspace{7pt} \textcolor{darkblue}{\textbf{Gravrand, Joseph  1770-1847c}}\hfillplus{281}
\newline Duets (instr.)    
\newline 2 vl
\newline \begin{itshape}Trois duos concertans pour deux violons ... 7|e livre de duos, œuvre 8|e\end{itshape} 
\newline \textcolor{darkblue}{\ding{\numexpr181 + 01}}  part(s)  
\newline print
\newline RISM-ID: 00000990022660
\newline Petit, Philippe  (pbl)
\newline D-Fh  Str Duo 0326; D-Mbs; US-STu
\newline \par \vspace{7pt} \textcolor{darkblue}{\textbf{Guénin, Marie-Alexandre  1744-1835}}\hfillplus{282}
\newline Duets (instr.)    
\newline \textcolor{darkblue}{\ding{\numexpr181 + 01}}  part(s)  
\newline print
\newline RISM-ID: 00000991019592
\newline Imbault  (pbl)
\newline D-Fh  Str Duo 0330
\newline \par \vspace{7pt} \textcolor{darkblue}{\textbf{Haydn, Joseph  1732-1809}}\hfillplus{283}
\newline Quartets (instr.)    
\newline \begin{itshape}[Umschlagtitel:] Bibliothèque musicale. [Titel:] Œuvres d'Haydn en partitions, Quatuors, tome (1) - (10). [Inhalt: Tome (1), [Pl. Bez.] a (Hob. III:75-77). - Tome (2), [Pl.-Bez.] b (Hob. III:78-80). - Tome (3), [Pl.-Bez.] c (Hob. III:31-33). - Tome (4), [Pl.-Bez.] d (Hob. III:35, 34, 36). - Tome (5), [Pl. Bez.] e (Hob. III:48, 45, 46). - Tome (6), [Pl.-Bez.] f (Hob. III:49, 47, 44). - Tome (7), [Pl.-Bez.] g (Hob. III:37-39). - Tome (8), [PL.-Bez.] h (Hob. III:40-42). - Tome (9), [Pl.-Bez.] i (Hob. III:69-71). - Tome (10), [Pl.-Bez.] j (Hob. III:72-74)]\end{itshape} 
\newline \textcolor{darkblue}{\ding{\numexpr181 + 01}}  score  
\newline print
\newline RISM-ID: 00000990027515
\newline Lobry); (Sampierdaréna  (prt)
\newline Pleyel (Lobry / Sampierdaréna)  (pbl)
\newline A-Sm; A-Wgm; A-Wn; A-Wn-h; B-Br; D-B; D-Fh; D-KNu; D-Mbs; D-Mh; D-Mmb; D-WRz; DK-Kk; H-Bami; I-Bc; I-Fc; I-Mc; I-Nc; I-Tn; S-Sm; US-NYj
\newline \par \vspace{7pt} \textcolor{darkblue}{\textbf{Haydn, Joseph  1732-1809}}\hfillplus{284}
\newline Quartets (instr.)    
\newline 2 vl, a-vla, b
\newline \begin{itshape}Trois quatuors pour deux violons, alto et basse ... 1|r|e livraison\end{itshape} 
\newline \textcolor{darkblue}{\ding{\numexpr181 + 01}}  part(s)  
\newline print
\newline RISM-ID: 00000990027612
\newline Leduc  (pbl)
\newline D-Fh  Str Quart 0062
\newline \par \vspace{7pt} \textcolor{darkblue}{\textbf{Haydn, Joseph  1732-1809}}\hfillplus{285}
\newline Symphonies    
\newline 2 vl, a-vla, b
\newline \begin{itshape}Trois nouvelles symphonies composés et arrangés en quatuors pour deux violons, alto et basse\end{itshape} 
\newline \textcolor{darkblue}{\ding{\numexpr181 + 01}}  part(s)  
\newline print
\newline RISM-ID: 00000990028292
\newline PN (203) auf vlc, p. 1-6
\newline Artaria  (pbl)
\newline A-HE; A-Wgm; A-Wn; A-Wn-h; A-Wst; CZ-Bm; CZ-K; CZ-Pnm; D-B; D-Fh  Str Quart 0063; D-HAu; D-WWW; GB-Lcm; I-Mc; I-OS; I-Vnm; US-NYp
\newline \par \vspace{7pt} \textcolor{darkblue}{\textbf{Haydn, Joseph  1732-1809}}\hfillplus{286}
\newline Symphonies    
\newline 2 vl, a-vla, b
\newline \begin{itshape}Trois quatuors pour deux violons, alto et basse ... 1|r|e livraison\end{itshape} 
\newline \textcolor{darkblue}{\ding{\numexpr181 + 01}}  part(s)  
\newline print
\newline RISM-ID: 00000990028298
\newline Leduc  (pbl)
\newline D-Fh  Str Quart 0062; D-Mbs; E-Mn
\newline \par \vspace{7pt} \textcolor{darkblue}{\textbf{Haydn, Joseph  1732-1809}}\hfillplus{287}
\newline Trios (instr.)    
\newline 2 vl, vlc
\newline \begin{itshape}Six trios pour deux violons  violoncelle à l'usage des commençans ... Nro: I(I)\end{itshape} 
\newline \textcolor{darkblue}{\ding{\numexpr181 + 01}}  part(s)  
\newline print
\newline RISM-ID: 00000990027969
\newline Simrock, Nicolaus  (pbl)
\newline A-Wgm; B-Bc; D-AAst; D-Fh  Str Trio 0063; D-Mbs; H-Bl; I-Nc; N-Ou; RUS-Mrg; S-Skma; US-BEm; US-Wc
\newline \par \vspace{7pt} \textcolor{darkblue}{\textbf{Hoffmann, Heinrich Anton  1770-1842}}\hfillplus{288}
\newline Duets (instr.)  F; D  
\newline vl, vlc
\newline \begin{itshape}Deux duos [F, D] pour violon et violoncelle ... œuvre 6\end{itshape} 
\newline \textcolor{darkblue}{\ding{\numexpr181 + 01}}  part(s)  
\newline print
\newline RISM-ID: 00000990029826
\newline André, Johann  (pbl)
\newline A-Wgm; A-Wn; B-Br  Mus. 14.098 C; CH-Zz; D-B; D-Bhm; D-Dl; D-Fh  Str Duo 0342; D-OF; HR-Dsmb; I-Li; RUS-Mrg
\newline \par \vspace{7pt} \textcolor{darkblue}{\textbf{Hoffmeister, Franz Anton  1754-1812}}\hfillplus{289}
\newline Duets (instr.)  B|b; G; E  
\newline 2 vl
\newline \begin{itshape}III Duos concertans [B, G, E] pour deux violons ... œuvre 10, des duos de violon\end{itshape} 
\newline \textcolor{darkblue}{\ding{\numexpr181 + 01}}  [1809c]  part(s)  
\newline print
\newline RISM-ID: 00000990030144
\newline Simrock, Nicolaus  (pbl)
\newline D-Fh  Str Duo 0349; HR-Zh
\newline \par \vspace{7pt} \textcolor{darkblue}{\textbf{Hoffmeister, Franz Anton  1754-1812}}\hfillplus{290}
\newline Trios (instr.)  C; G; D  
\newline 2 vl, vlc
\newline \begin{itshape}Trois trios progressives pour deux violons et violoncelle ... liv.: 1 [C, G, D]\end{itshape} 
\newline RISM-ID: 00000990030070
\newline Simrock, Heinrich  (pbl)
\newline Simrock, Nicolaus  (pbl)
\newline D-Dl; D-Fh  Str Trio 0084; US-R
\newline \par \vspace{7pt} \textcolor{darkblue}{\textbf{Hoffmeister, Franz Anton  1754-1812}}\hfillplus{291}
\newline Trios (instr.)  F; B|b; E|b  
\newline 2 vl, vlc
\newline \begin{itshape}Trois trios progressives pour deux violons et violoncelle ... liv.: II [F, B, Es]\end{itshape} 
\newline RISM-ID: 00000990030073
\newline Simrock, Heinrich  (pbl)
\newline Simrock, Nicolaus  (pbl)
\newline D-Dl; D-Fh  Str Trio 0083; US-R
\newline \par \vspace{7pt} \textcolor{darkblue}{\textbf{Käfer, Johann Philipp  1672-1728}}\hfillplus{292}
\newline Airs  C  
\newline org
\newline \begin{itshape}[f.114v, heading:] Aria ex C di Käfer.\end{itshape} 
\newline \textcolor{darkblue}{\ding{\numexpr181 + 01}}  1 parts  
\newline Manuscript copy
\newline 1.1.1  org  C  
\begin{filecontents*}{292-1.code}
@clef:C-1
@keysig:
@timesig:c
@data:8''C/4C-C8-C/4D6-{DGD}4E6-{CGC}/{ACGC}{FCEC}4D6-{'GBG}/{''E'G''D'G}{''C'GBG}
\end{filecontents*}
\commandline{ verovio --spacing-non-linear=0.50 -w 1500 --spacing-system=0.5 --adjust-page-height -b 0 292-1.code }
\newline
\includesvg[width=220pt]{292-1}%

\newline RISM-ID: 455011018
\newline Ohne alte Paginierung
\newline D-Fh  Ms 001
\newline $\rightarrow$ In collection 418 (455008343)

\newline \par \vspace{7pt} \textcolor{darkblue}{\textbf{Keller, Karl  1784-1855}}\hfillplus{293}
\newline Der Guckkasten  C  
\newline V, Coro, guit
\newline \begin{itshape}[heading, f.59v:] Der Guckkasten. | [at right:] von C. Keller.\end{itshape} 
\newline \textcolor{darkblue}{\ding{\numexpr181 + 01}}  1 parts  
\newline Manuscript copy
\newline 1.1.1  V  C
\newline \begin{footnotesize} Herbei ihr Laute kommt zuhauf \end{footnotesize}  
\begin{filecontents*}{293-1.code}
@clef:G-2
@keysig:
@timesig:c
@data:4,G/4.'C8,B{8.6'CD}{EF}/4.G8A4GG/''C'CEC/8.6{GF}GA4GD/4.D8D{8.6DE}{FG}/4.A8bB4AA/
\end{filecontents*}
\commandline{ verovio --spacing-non-linear=0.50 -w 1500 --spacing-system=0.5 --adjust-page-height -b 0 293-1.code }
\newline
\includesvg[width=220pt]{293-1}%

\newline RISM-ID: 455010795
\newline Acht Strophen
\newline Als Instrumente ist ausdrücklich "Guitarre" angegeben, ein "Chor" singt den Refrain "Tra la ral la"
\newline D-Fh  Ms 002
\newline $\rightarrow$ In collection 419 (455008345)

\newline \par \vspace{7pt} \textcolor{darkblue}{\textbf{Kreutzer, Paul  ??}}\hfillplus{294}
\newline Variations    
\newline \begin{itshape}...\end{itshape} 
\newline \textcolor{darkblue}{\ding{\numexpr181 + 01}}  score  
\newline print
\newline RISM-ID: 00000991022028
\newline Breitkopf  Härtel  (pbl)
\newline D-B; D-Fh  Str Duo 0109
\newline \par \vspace{7pt} \textcolor{darkblue}{\textbf{Krommer, Franz  1759-1831}}\hfillplus{295}
\newline Duets (instr.)  g; C; A  
\newline 2 vl
\newline \begin{itshape}Trois grands duos [g, C, A] pour deux violons ... dédiés à Monsieur le Comte Cesar de Castelbarco ... œuvre 94\end{itshape} 
\newline \textcolor{darkblue}{\ding{\numexpr181 + 01}}  part(s)  
\newline print
\newline RISM-ID: 00000990035571
\newline Sieber, Jean-Georges  Georges-Julien Sieber  (pbl)
\newline CH-E; D-Fh  Str Duo 0113; D-Mbs
\newline \par \vspace{7pt} \textcolor{darkblue}{\textbf{Krommer, Franz  1759-1831}}\hfillplus{296}
\newline Quartets (instr.)  F; B|b; G  
\newline 2 vl, a-vla, vlc
\newline \begin{itshape}Trois quatuors [F, B, G] pour deux violons, alte  violoncelle ... et dédiés à Mr. le Comte Michel de Vilehorsky ... œuvre X\end{itshape} 
\newline \textcolor{darkblue}{\ding{\numexpr181 + 01}}  part(s)  
\newline print
\newline RISM-ID: 00000990035398
\newline Gombart, Johann Carl  (pbl)
\newline A-SF; A-Wgm; A-Wmi; A-Wn; B-Bc; CZ-K; D-B; D-Fh  Str Quart 0097; D-LEm; D-Mbs; D-Mmb; D-W; DK-Kk; GB-Lbl; H-SFm; I-Bca; I-BRc; I-Fc; I-PESc; I-Sac; US-Wc
\newline \par \vspace{7pt} \textcolor{darkblue}{\textbf{Krommer, Franz  1759-1831}}\hfillplus{297}
\newline Quartets (instr.)  G; d; B|b  
\newline 2 vl, a-vla, vlc
\newline \begin{itshape}Trois quatuors [G, d, B] pour deux violons, alto et violoncelle ... dédiés à son Altesse Monseigneur le Prince-Régnant de Lobkowitz c. c. ... œuvre 34\end{itshape} 
\newline \textcolor{darkblue}{\ding{\numexpr181 + 01}}  part(s)  
\newline print
\newline RISM-ID: 00000990035428
\newline Artaria  (pbl)
\newline A-Wgm; A-Wn; A-Wst; CH-E; CZ-Pk; CZ-Pm; CZ-Pnm; D-Fh  Str Quart 0096; D-Mbs; H-Bn; H-Gm; H-KE; HR-Zha; I-BGc; I-Fc; I-Mc; I-Nc; I-OS; I-PESc; I-Raf; I-Ria; I-Sac; I-Vc; US-R
\newline \par \vspace{7pt} \textcolor{darkblue}{\textbf{Krommer, Franz  1759-1831}}\hfillplus{298}
\newline Quintets (instr.)  B|b; E|b; G; F; c; D  
\newline 2 vl, 2 a-vla, vlc
\newline \begin{itshape}Trois quintuors [B, Es, G; F, c, D] pour deux violons, deux altos  violoncelle ... dédiés à M|r le Baron de Stroganof [liv. 1] (et le Comte Leopold de Daun [liv. 2]) ... œuvre VIII, liv. 1 (2)\end{itshape} 
\newline \textcolor{darkblue}{\ding{\numexpr181 + 01}}  part(s)  
\newline print
\newline RISM-ID: 00000990035334
\newline Gombart, Johann Carl  (pbl)
\newline A-HE; A-M; A-Sca; A-SF; A-Sm; A-Wgm; A-Wmi; A-Wn; B-Bc; CH-E; CH-EN; CH-SO; CZ-Bm; CZ-Bu; CZ-K; CZ-Pnm; D-As; D-B; D-BDHlebermann; D-DEsa; D-Dl; D-DO; D-Fh  Str Quint 0018; D-HEms; D-HR; D-LÜh; D-Mbs; D-W; H-Bl; H-Bn; H-PH; H-SFm; HR-Zh; HR-Zha; I-Fc; I-Gl; I-Nc; I-OS; I-Raf; J-Tma; S-LI; S-Skma; S-V
\newline \par \vspace{7pt} \textcolor{darkblue}{\textbf{Krommer, Franz  1759-1831}}\hfillplus{299}
\newline Quintets (instr.)    
\newline 2 vl, 2 vla, vlc
\newline RISM-ID: 00000990035339
\newline vgl. Op. 8
\newline André, Johann  (pbl)
\newline B-Bc; CH-E; CZ-Bu; D-Fh  Str Quint 0016; D-OF; NL-DHgm; S-Skma; US-R
\newline \par \vspace{7pt} \textcolor{darkblue}{\textbf{Krommer, Franz  1759-1831}}\hfillplus{300}
\newline Quintets (instr.)  C; F; E|b; A; f; B|b  
\newline 2 vl, 2 a-vla, vlc
\newline \begin{itshape}Trois quintuors [C, F, Es; A, f, B] pour deux violons, deux altos  violoncelle ... œuvre XXV, 1. 1 (2)\end{itshape} 
\newline \textcolor{darkblue}{\ding{\numexpr181 + 01}}  part(s)  
\newline print
\newline RISM-ID: 00000990035340
\newline Kunst- und Industrie-Comptoir  (pbl)
\newline A-M; A-SF; A-Wgm; A-Wn; A-Wst; B-Bc; CH-E; CZ-BER; CZ-Pk; CZ-PLa; CZ-Pnm; D-B; D-Bhm; D-DO; D-Fh  Str Quint 0020; D-HL; D-Mbs; D-SWl; D-W; H-Bn; H-KE; H-SFm; HR-Zh; HR-Zha; I-Fc; I-Mc; I-Nc; I-OS; I-PESc; I-Vc; SK-KE
\newline \par \vspace{7pt} \textcolor{darkblue}{\textbf{Krommer, Franz  1759-1831}}\hfillplus{301}
\newline Quintets (instr.)  C; F; E|b  
\newline 2 vl, 2 a-vla, b
\newline \begin{itshape}Trois quintetti [C, F, Es] pour deux violons, deux altos et basse ... œuvre 25\end{itshape} 
\newline RISM-ID: 00000990035341
\newline André, Johann  (pbl)
\newline A-Wmi; B-Bc; CH-Bu; CZ-Bu; CZ-Pk; CZ-Pu; D-B; D-Dl; D-Fh; D-OF; D-RH; I-AN; I-BGi; I-Mc; I-OS; I-PESc; J-Tma; N-Ou; NL-At; S-Skma; US-Cn; US-R
\newline \par \vspace{7pt} \textcolor{darkblue}{\textbf{Mestrino, Nicola  1748-1789}}\hfillplus{302}
\newline Duets (instr.)    
\newline 2 vl
\newline \begin{itshape}TROIS DUOS | pour deux | VIOLONS. | Composés | par | MR MESTRINO.\end{itshape} 
\newline RISM-ID: 00000990041085
\newline Schmitt, Joseph  (pbl)
\newline D-Fh  Str Duo 0173; NL-At; NL-Uim
\newline \par \vspace{7pt} \textcolor{darkblue}{\textbf{Mozart, Wolfgang Amadeus  1756-1791}}\hfillplus{303}
\newline Concertos  C  
\newline pf, 2 vl, a-vla, b, fl, 2 ob, 2 fag, 2 cor, 2 tr, timp
\newline \begin{itshape}[Umschlag: Œuvres de W. A. Mozart. Concert pour le pianoforte, N|o I]. Concert pour le pianoforte [C], avec accompagnement de 2 violons, alto et basse, flûte, 2 hautbois, 2 bassons, 2 cors, 2 trompettes et timballes, par W. A. Mozart. N|o I\end{itshape} 
\newline \textcolor{darkblue}{\ding{\numexpr181 + 01}}  part(s)  
\newline print
\newline RISM-ID: 00000990045870
\newline Gesamtausgabe
\newline Breitkopf  Härtel  (pbl)
\newline A-Sm; CH-EN; CZ-Bm; CZ-Bu; CZ-K; CZ-SO; D-ALt; D-B; D-Bhm; D-Bs; D-Dl; D-Fh  Orch Mat 0152; D-LEbh; D-LEm; D-LÜh; D-Mbs  2 Mus.pr. 116-1; D-Mmb; D-MZsch; DK-Kk; F-Pc; GB-Lcm; H-Bb; I-Mc; RUS-Mrg; S-Skma; S-Sm; S-STr; US-BEm
\newline \par \vspace{7pt} \textcolor{darkblue}{\textbf{Mozart, Wolfgang Amadeus  1756-1791}}\hfillplus{304}
\newline Don Giovanni. Excerpts. Arr    KV 527/14
\newline V, mandora
\newline \begin{itshape}[without title]\end{itshape} 
\newline \textcolor{darkblue}{\ding{\numexpr181 + 01}}  1 parts  
\newline Manuscript copy
\newline Zink, Joseph Michael
\newline 1.1.1  V  C
\newline \begin{footnotesize} Als ich noch im Flügelkleide \end{footnotesize}  
\begin{filecontents*}{304-1.code}
@clef:G-2
@keysig:
@timesig:3/4
@data:8.6'GG8GGGG/{G''C}4G-/8.6'FF8FFFF/{6GFEF}4E-/8.6xFF8FFFF/{8.6GA}4B-/
\end{filecontents*}
\commandline{ verovio --spacing-non-linear=0.50 -w 1500 --spacing-system=0.5 --adjust-page-height -b 0 304-1.code }
\newline
\includesvg[width=220pt]{304-1}%

\newline RISM-ID: 455010799
\newline Neun Strophen
\newline D-Fh  Ms 002
\newline $\rightarrow$ In collection 419 (455008345)

\newline \par \vspace{7pt} \textcolor{darkblue}{\textbf{Mussini, Natale Nicolo  1765-1837}}\hfillplus{305}
\newline Duets (instr.)    
\newline RISM-ID: 00000990046562
\newline Ribière  (prt)
\newline Le Menu  (pbl)
\newline Naderman  (pbl)
\newline Philippeaux (Ribière)  (pbl)
\newline B-Bc; D-Fh  Str Duo 0181; I-Sac
\newline \par \vspace{7pt} \textcolor{darkblue}{\textbf{Mussini, Natale Nicolo  1765-1837}}\hfillplus{306}
\newline Duets (instr.)  C; B|b; A; G; E|b; D  
\newline 2 vl
\newline \begin{itshape}Six duos [C, B, A, G, Es, D] pour deux violons\end{itshape} 
\newline \textcolor{darkblue}{\ding{\numexpr181 + 01}}  part(s)  
\newline print
\newline RISM-ID: 00000990046565
\newline Lepreux, Mlle  (prt)
\newline Koliker, Jean-Gabriel (Mlle Lepreux)  (pbl)
\newline D-Fh  Str Duo 0182; F-Pn; NL-Uim
\newline \par \vspace{7pt} \textcolor{darkblue}{\textbf{Naumann, Johann Gottlieb  1741-1801}}\hfillplus{307}
\newline Duets (instr.)    
\newline 2 vl
\newline RISM-ID: 00000990046826
\newline Peters, Carl Friedrich  (pbl)
\newline CZ-Bm; D-Fh  Str Duo 0184; FIN-A
\newline \par \vspace{7pt} \textcolor{darkblue}{\textbf{Paër, Ferdinando  1771-1839}}\hfillplus{308}
\newline La Passione di Gesù Cristo    
\newline V (4), Coro, orch
\newline \begin{itshape}[label on cover, stamped:] ORATORIO | LA PASSIONE | DI GESU CHRISTO | [title page:] Partitura | Oratorio | La | Passione di Gesu Christo | Del Sig|r|e Ferdinando Paer.\end{itshape} 
\newline \textcolor{darkblue}{\ding{\numexpr181 + 01}}  1800-1849 (19.1d)  score: 206f.  22,5 x 30 cm
\newline Watermark: [bow with arrow] / AM; AM
\newline Manuscript copy
\newline 1.1.1  vl 1  c  
\begin{filecontents*}{308-1.code}
@clef:G-2
@keysig:bBEA
@timesig:c
@data:4.'C8E{GFD,nB}/i/4'C-2-/2.,B8-6(-)n''E/!4F+{8.6FnE}!f/
\end{filecontents*}
\commandline{ verovio --spacing-non-linear=0.50 -w 1500 --spacing-system=0.5 --adjust-page-height -b 0 308-1.code }
\newline
\includesvg[width=220pt]{308-1}%

\newline 1.1.2  S solo  c
\newline \begin{footnotesize} Sacrimarti ombre adorate \end{footnotesize}  
\begin{filecontents*}{308-2.code}
@clef:C-1
@keysig:bBEA
@timesig:c
@data:=26/2.''G{8FE}/4CC2-/2Cq8'B4.A8G/q8G4FF-8.6FF/4.8GGA''C/4n'BB-8.6BB/
\end{filecontents*}
\commandline{ verovio --spacing-non-linear=0.50 -w 1500 --spacing-system=0.5 --adjust-page-height -b 0 308-2.code }
\newline
\includesvg[width=220pt]{308-2}%

\newline 1.2.1  vla  c  
\begin{filecontents*}{308-3.code}
@clef:C-3
@keysig:bBEA
@timesig:3/4
@data:4.'G{8AGF}/4.E{8FED}/4CEF/FE-/
\end{filecontents*}
\commandline{ verovio --spacing-non-linear=0.50 -w 1500 --spacing-system=0.5 --adjust-page-height -b 0 308-3.code }
\newline
\includesvg[width=220pt]{308-3}%

\newline 1.2.2  S 2 coro  c
\newline \begin{footnotesize} Pallido esangue \end{footnotesize}  
\begin{filecontents*}{308-4.code}
@clef:C-1
@keysig:bBEA
@timesig:3/4
@data:=6/4''CCC/{8Cn'B}4B8-A/4AG-/
\end{filecontents*}
\commandline{ verovio --spacing-non-linear=0.50 -w 1500 --spacing-system=0.5 --adjust-page-height -b 0 308-4.code }
\newline
\includesvg[width=220pt]{308-4}%

\newline 1.3.1  S  C
\newline \begin{footnotesize} O duolo \end{footnotesize}  
\begin{filecontents*}{308-5.code}
@clef:C-1
@keysig:
@timesig:c
@data:8''C/qb'BAA4-
\end{filecontents*}
\commandline{ verovio --spacing-non-linear=0.50 -w 1500 --spacing-system=0.5 --adjust-page-height -b 0 308-5.code }
\newline
\includesvg[width=220pt]{308-5}%

\newline 1.3.2  B  C
\newline \begin{footnotesize} O angoscia \end{footnotesize}  
\begin{filecontents*}{308-6.code}
@clef:F-4
@keysig:
@timesig:c
@data:8-/4-'DqC8,BB4-/
\end{filecontents*}
\commandline{ verovio --spacing-non-linear=0.50 -w 1500 --spacing-system=0.5 --adjust-page-height -b 0 308-6.code }
\newline
\includesvg[width=220pt]{308-6}%

\newline 1.3.3  S  C
\newline \begin{footnotesize} Oh lutto della terra e del cielo \end{footnotesize}  
\begin{filecontents*}{308-7.code}
@clef:C-1
@keysig:
@timesig:c
@data:8-/4---''E/8EC-6'B''C8DDDC/'AA-
\end{filecontents*}
\commandline{ verovio --spacing-non-linear=0.50 -w 1500 --spacing-system=0.5 --adjust-page-height -b 0 308-7.code }
\newline
\includesvg[width=220pt]{308-7}%

\newline RISM-ID: 455008344
\newline Auf dem vorderen Einbanddeckel Aufkleber mit der Zahl "909."
\newline Auf dem Vorsatzblatt recto, oben "Gesch. d. H. Mack." und rechts oben Stempel "Dr. Hoch's Conservatorium in Frankfurt A M.", auf der Mitte der Seite mit blauer Kreide: "570." darunter Stempel "Bibliothek der Staatl. Hochschule | für Musik | Frankfurta. Main"
\newline Mack  (fmo)
\newline D-Fh  Ms 003
\newline \par \vspace{7pt} \textcolor{darkblue}{\textbf{Pechmann, Nanette B.  19.sc}}\hfillplus{309}
\newline Ans Liebchen  E|b  
\newline V, mandora
\newline \begin{itshape}[heading, f.40v:] Composée p. N: B: P: | Ans Liebchen\end{itshape} 
\newline \textcolor{darkblue}{\ding{\numexpr181 + 01}}  1 parts  
\newline Manuscript copy
\newline 1.1.1  V  E|b
\newline \begin{footnotesize} Siehst du jene Rose blühen \end{footnotesize}  
\begin{filecontents*}{309-1.code}
@clef:G-2
@keysig:bBEA
@timesig:3/8
@data:!48'B''E/i/{8DFD}/!E{6GE}8'B/f4''E8-/
\end{filecontents*}
\commandline{ verovio --spacing-non-linear=0.50 -w 1500 --spacing-system=0.5 --adjust-page-height -b 0 309-1.code }
\newline
\includesvg[width=220pt]{309-1}%

\newline RISM-ID: 455011115
\newline Zwei Strophen
\newline D-Fh  Ms 002
\newline $\rightarrow$ In collection 419 (455008345)

\newline \par \vspace{7pt} \textcolor{darkblue}{\textbf{Pechmann, Nanette B.  19.sc}}\hfillplus{310}
\newline Die Linde  E|b  
\newline V, mandora
\newline \begin{itshape}[heading, f.37v:] Composée par Nanette B: Pechmann | [at right:] Die Linde\end{itshape} 
\newline \textcolor{darkblue}{\ding{\numexpr181 + 01}}  1 parts  
\newline Autograph manuscript
\newline 1.1.1  V  E|b
\newline \begin{footnotesize} Ach hier vor der Linde die uns lieb oft Schatten gab \end{footnotesize}  
\begin{filecontents*}{310-1.code}
@clef:G-2
@keysig:bBEA
@timesig:3/4
@data:!4''E'GG/2G4G/!4.F8BAF/q6G{8FE}4E-/f{8FBnABbAF}/4G--/
\end{filecontents*}
\commandline{ verovio --spacing-non-linear=0.50 -w 1500 --spacing-system=0.5 --adjust-page-height -b 0 310-1.code }
\newline
\includesvg[width=220pt]{310-1}%

\newline RISM-ID: 455011111
\newline Vier Strophen
\newline D-Fh  Ms 002
\newline $\rightarrow$ In collection 419 (455008345)

\newline \par \vspace{7pt} \textcolor{darkblue}{\textbf{Pechmann, Nanette B.  19.sc}}\hfillplus{311}
\newline Lied eines Schäfers  B|b  
\newline V, mandora
\newline \begin{itshape}[heading, f.38v:] Composée p: N: B: P: | [at right:] lied eines Schäfers\end{itshape} 
\newline \textcolor{darkblue}{\ding{\numexpr181 + 01}}  1 parts  
\newline Autograph manuscript
\newline 1.1.1  V  B|b
\newline \begin{footnotesize} Mich soll die Liebe nicht kränken \end{footnotesize}  
\begin{filecontents*}{311-1.code}
@clef:G-2
@keysig:bBE
@timesig:c
@data:8'DFB/4.''D8C'BAGF/{FE}4E8-EFA/4.''C8'BAGFE/4ED8-FB''D/4.8FDEC/
\end{filecontents*}
\commandline{ verovio --spacing-non-linear=0.50 -w 1500 --spacing-system=0.5 --adjust-page-height -b 0 311-1.code }
\newline
\includesvg[width=220pt]{311-1}%

\newline RISM-ID: 455011112
\newline Sechs Strophen
\newline D-Fh  Ms 002
\newline $\rightarrow$ In collection 419 (455008345)

\newline \par \vspace{7pt} \textcolor{darkblue}{\textbf{Pleyel, Ignace  1757-1831}}\hfillplus{312}
\newline Duets (instr.)    
\newline 2 vl
\newline \begin{itshape}Trois Grands Duos Pour deux Violons ... œuvre 69|m|e\end{itshape} 
\newline \textcolor{darkblue}{\ding{\numexpr181 + 01}}  [1805]  part(s)  
\newline print
\newline RISM-ID: 00000990051527
\newline André, Johann Anton  (pbl)
\newline A-M; A-Sm; B-Br  Mus. 14.280 C; D-Fh  Str Duo 0198; D-KWbeer; D-OF; D-RH; D-W; S-Uu
\newline \par \vspace{7pt} \textcolor{darkblue}{\textbf{Pleyel, Ignace  1757-1831}}\hfillplus{313}
\newline Quintets (instr.)    
\newline 2 vl, 2 a-vla, vlc
\newline \begin{itshape}Quintetto pour deux violons, deux altos et violoncelle ... œuvre 19 ... 2|d|e édition ...\end{itshape} 
\newline RISM-ID: 00000990050482
\newline André, Johann Anton  (pbl)
\newline D-Fh  Str Quint 0037; N-Ou
\newline \par \vspace{7pt} \textcolor{darkblue}{\textbf{Pleyel, Ignace  1757-1831}}\hfillplus{314}
\newline Quintets (instr.)  B|b; G  
\newline 2 vl, 2 a-vla, vlc
\newline \begin{itshape}II Quintetti [B, G] pour deux violons, deux alto et un violoncelle ... œuvre 22|m|e\end{itshape} 
\newline \textcolor{darkblue}{\ding{\numexpr181 + 01}}  [1789]  part(s)  
\newline print
\newline RISM-ID: 00000990050490
\newline André, Johann  (pbl)
\newline CH-AR; CZ-Pk; D-B; D-Bhm; D-Fh  Str Quint 0036; D-HEms; D-LÜh; D-OF; D-RH; S-Skma
\newline \par \vspace{7pt} \textcolor{darkblue}{\textbf{Poessinger, Franz Alexander  1767-1827}}\hfillplus{315}
\newline Divertimentos  E|b  
\newline vl, vla, vlc
\newline \begin{itshape}Serenata [Es] in trio concertante per il violino, viola e violoncello ... op. 10.\end{itshape} 
\newline \textcolor{darkblue}{\ding{\numexpr181 + 01}}  [1805c]  part(s)  
\newline print
\newline RISM-ID: 00000991024986
\newline Artaria  (pbl)
\newline A-Wn; A-Wst; CZ-Bm; D-Fh  Str Trio 0123; H-Bl
\newline \par \vspace{7pt} \textcolor{darkblue}{\textbf{Poessinger, Franz Alexander  1767-1827}}\hfillplus{316}
\newline Trios (instr.)    
\newline 2 vl, vlc
\newline \begin{itshape}TRIOS TRIOS | tres agréables | pour | Deux Violons | et | Violoncelle | composées et dediés | aux | JEUNES AMATEURS | par | Alexandre Poessinger | Lib. I\end{itshape} 
\newline \textcolor{darkblue}{\ding{\numexpr181 + 01}}  part(s)  
\newline print
\newline RISM-ID: 00000991025009
\newline Plattner, Ludwig  (pbl)
\newline D-Fh  Str Trio 0124
\newline \par \vspace{7pt} \textcolor{darkblue}{\textbf{Pujolas, J.  1806†}}\hfillplus{317}
\newline Duets (instr.)    
\newline 2 vl
\newline \begin{itshape}Trois [= 6] duo pour deux violons ... opera X|m|e, ([handschriftlich:] 1-2) livraison.\end{itshape} 
\newline \textcolor{darkblue}{\ding{\numexpr181 + 01}}  part(s)  
\newline print
\newline RISM-ID: 00000990053020
\newline Joannes  (prt)
\newline s.n. (Joannes)  (pbl)
\newline D-Fh  Str Duo 0201; US-Wc
\newline \par \vspace{7pt} \textcolor{darkblue}{\textbf{Raimondi, Ignazio  1735c-1813}}\hfillplus{318}
\newline Trios (instr.)  F; A; B|b; C; E|b; D  
\newline 2 vl, vla, vlc, b
\newline \begin{itshape}Six trios [F, A, B, C, Es, D], trois, à deux violons  basse,  trois, à un violon, taille  violoncello obligés ... œuvre V.\end{itshape} 
\newline \textcolor{darkblue}{\ding{\numexpr181 + 01}}  part(s)  
\newline print
\newline RISM-ID: 00000990053500
\newline Hummel, Johann Julius  (pbl)
\newline Hummel, Johann Julius  (pbl)
\newline CH-Bu; D-B; D-Fh  Str Trio 0127; GB-Lbl; NL-Uim; S-KA; S-L; US-CHua; US-Wc
\newline \par \vspace{7pt} \textcolor{darkblue}{\textbf{Rolla, Alessandro  1757-1841}}\hfillplus{319}
\newline Duets (instr.)  E|b; C; F  
\newline 2 vl
\newline \begin{itshape}Tre duetti [Es, C, F] per due violini ... op. 4. dei duetti a due violini, parte 1|m|a.\end{itshape} 
\newline RISM-ID: 00000990055354
\newline Artaria, Ferdinando  (pbl)
\newline A-Wgm; A-Wmi; D-Fh  Str Duo 0210; I-Fc; I-Mc; I-Nc; I-PAVi; I-Vnm
\newline \par \vspace{7pt} \textcolor{darkblue}{\textbf{Rolla, Alessandro  1757-1841}}\hfillplus{320}
\newline Duets (instr.)  B|b; D; E|b  
\newline 2 vl
\newline \begin{itshape}Tre duetti [B, D, Es] per due violini ... op. 5|t|a dei duetti a due violini.\end{itshape} 
\newline \textcolor{darkblue}{\ding{\numexpr181 + 01}}  [1813-1814]  part(s)  
\newline print
\newline RISM-ID: 00000990055356
\newline Ricordi, Giovanni  (pbl)
\newline A-Wgm; CH-SO; CZ-Pk; D-Fh  Str Duo 0211; I-BGc; I-BGi; I-Fc; I-Mc; I-Nc; I-Nn; I-PAVi; I-Rsc
\newline \par \vspace{7pt} \textcolor{darkblue}{\textbf{Rolla, Alessandro  1757-1841}}\hfillplus{321}
\newline Duets (instr.)    
\newline \begin{itshape}... œuvre 6.\end{itshape} 
\newline RISM-ID: 00000990055358
\newline Gombart, Johann Carl  (pbl)
\newline A-Wgm; D-Fh  Str Duo 0212
\newline \par \vspace{7pt} \textcolor{darkblue}{\textbf{Rolla, Alessandro  1757-1841}}\hfillplus{322}
\newline Exercises (instr.)  D; C; B|b; F; A; C; D; B|b; E|b; C  
\newline 2 vl
\newline \begin{itshape}Étude [D, C, B, F, A, C, D, B, Es, C] pour deux violons ... œuvre X.\end{itshape} 
\newline \textcolor{darkblue}{\ding{\numexpr181 + 01}}  [1808]  part(s)  
\newline print
\newline RISM-ID: 00000990055368
\newline Simrock, Nicolaus  (pbl)
\newline D-Fh  Str Duo 0213
\newline \par \vspace{7pt} \textcolor{darkblue}{\textbf{Romberg, Andreas Jakob  1767-1821}}\hfillplus{323}
\newline Duets (instr.)  F; A; d  
\newline 2 vl
\newline \begin{itshape}Trois duos concertans [F, A, d] pour deux violons ... œuvre 56, III|e suite des duos.\end{itshape} 
\newline \textcolor{darkblue}{\ding{\numexpr181 + 01}}  part(s)  
\newline print
\newline RISM-ID: 00000990055588
\newline Peters, Carl Friedrich  (pbl)
\newline A-Wgm; B-Bc; CZ-Pk; D-B; D-Dl; D-Fh; D-GRu; D-Hs; D-LÜh; D-MÜu; FIN-A; I-Vc; S-F; S-J; S-ÖS; S-Skma; S-V
\newline \par \vspace{7pt} \textcolor{darkblue}{\textbf{Romberg, Bernhard Heinrich  1767-1841}}\hfillplus{324}
\newline Duets (instr.)  G; A; c  
\newline 2 vlc
\newline \begin{itshape}Trois grands duos [G, A, c] pour deux violoncelles ... œuvre 33, Nr. I (II, III).\end{itshape} 
\newline \textcolor{darkblue}{\ding{\numexpr181 + 01}}  part(s)  
\newline print
\newline RISM-ID: 00000990055758
\newline Peters, Carl Friedrich  (pbl)
\newline A-M; A-Sm; A-Wgm; A-Wmi; A-Wn; D-Dl; D-Fh  Str Duo 0215; D-Hmb; D-Hs; D-Mbs; D-MÜu; FIN-A; H-Bb; H-Bl; I-BGi; I-BRc; I-Mc; I-Nc; N-Ou; NL-At; RUS-Mrg; S-Skma
\newline \par \vspace{7pt} \textcolor{darkblue}{\textbf{Romberg, Bernhard Heinrich  1767-1841}}\hfillplus{325}
\newline Trios (instr.)    
\newline vl, a-vla, vlc
\newline \begin{itshape}Grand trio pour violoncelle, violon et alto ... œuvre 8.\end{itshape} 
\newline RISM-ID: 00000990055741
\newline Erard, Mlles  (pbl)
\newline Garnier  (pbl)
\newline A-Wn; D-Bhm; D-Fh  Str Trio 0135; D-Mbs; I-Nc; I-OS; NL-At; US-R
\newline \par \vspace{7pt} \textcolor{darkblue}{\textbf{Scheidt, Samuel  1587-1654}}\hfillplus{326}
\newline Ei du feiner Reiter  C  SSWV 111
\newline org
\newline \begin{itshape}[f.91v, at left:] Eÿ du feiner | Reiter | [at the beginning of the variation:] Variatio | c[on]tra punct | S. Scheid.\end{itshape} 
\newline \textcolor{darkblue}{\ding{\numexpr181 + 01}}  1 parts  
\newline Manuscript copy
\newline 1.1.1  org  C  
\begin{filecontents*}{326-1.code}
@clef:G-2
@keysig:
@timesig:0
@data:8''CCCC4'BA8GGGxF2G://:8GEAA4GFED2C://:
\end{filecontents*}
\commandline{ verovio --spacing-non-linear=0.50 -w 1500 --spacing-system=0.5 --adjust-page-height -b 0 326-1.code }
\newline
\includesvg[width=220pt]{326-1}%

\newline 1.2.1  org  C  
\begin{filecontents*}{326-2.code}
@clef:G-2
@keysig:
@timesig:0
@data:8''CCCC4'BA8GGGxF2G6-{ECD}{EDEF}
\end{filecontents*}
\commandline{ verovio --spacing-non-linear=0.50 -w 1500 --spacing-system=0.5 --adjust-page-height -b 0 326-2.code }
\newline
\includesvg[width=220pt]{326-2}%

\newline RISM-ID: 455010975
\newline Ohne alte Foliierung
\newline D-Fh  Ms 001
\newline $\rightarrow$ In collection 418 (455008343)

\newline \par \vspace{7pt} \textcolor{darkblue}{\textbf{Schenk, Johann Baptist  1753-1836}}\hfillplus{327}
\newline Der Dorfbarbier. Excerpts. Arr  G  TakS 9/3
\newline V, mandora
\newline \begin{itshape}[at left, f.44v:] Andante\end{itshape} 
\newline \textcolor{darkblue}{\ding{\numexpr181 + 01}}  1 parts  
\newline Manuscript copy
\newline 1.1.1  mandora  G  
\begin{filecontents*}{327-1.code}
@clef:G-2
@keysig:xF
@timesig:2/4
@data:q8'E{6D3xCD}/{8GDD}!qA{6G3FG}/{8BGG}!fq''C{6'B3AB}/{8''D'GG''D}/
\end{filecontents*}
\commandline{ verovio --spacing-non-linear=0.50 -w 1500 --spacing-system=0.5 --adjust-page-height -b 0 327-1.code }
\newline
\includesvg[width=220pt]{327-1}%

\newline 1.1.2  V  G
\newline \begin{footnotesize} Einst sprach mein Herr der Bader \end{footnotesize}  
\begin{filecontents*}{327-2.code}
@clef:G-2
@keysig:xF
@timesig:2/4
@data:8-/=9/4-8-q6'E{D3xCD}/{8GDDG}/4B8G-/{ADD''C}/q6C{'BAB''C}{8'BD}/{GDDG}/4B{8GB}/
\end{filecontents*}
\commandline{ verovio --spacing-non-linear=0.50 -w 1500 --spacing-system=0.5 --adjust-page-height -b 0 327-2.code }
\newline
\includesvg[width=220pt]{327-2}%

\newline RISM-ID: 455011120
\newline Zwei Strophen
\newline Weidmann, Paul  (lyr)
\newline D-Fh  Ms 002
\newline $\rightarrow$ In collection 419 (455008345)

\newline \par \vspace{7pt} \textcolor{darkblue}{\textbf{Schönebeck, Karl Sigmund  1758-1800}}\hfillplus{328}
\newline Duets (instr.)  E|b; C; F  
\newline 2 a-vla
\newline \begin{itshape}Trois duos concertans [Es, C, F] pour deux altos-violes ... œuvre XIII.\end{itshape} 
\newline \textcolor{darkblue}{\ding{\numexpr181 + 01}}  part(s)  
\newline print
\newline RISM-ID: 00000990058479
\newline Bureau de Musique  (pbl)
\newline Hoffmeister  Kühnel  (pbl)
\newline B-Lc; D-B; D-Fh  Str Duo 0221; S-Skma
\newline \par \vspace{7pt} \textcolor{darkblue}{\textbf{Schubert, Johann Friedrich  1770-1811}}\hfillplus{329}
\newline Duets (instr.)    
\newline \begin{itshape}Trois | DUOS | pour Deux Violons | COMPOSÉS | par | J: F: Schubert\end{itshape} 
\newline \textcolor{darkblue}{\ding{\numexpr181 + 01}}  part(s)  
\newline print
\newline RISM-ID: 00000991004684
\newline Imbault  (pbl)
\newline D-Fh  Str Duo 0223
\newline \par \vspace{7pt} \textcolor{darkblue}{\textbf{Stamitz, Carl  1745-1801}}\hfillplus{330}
\newline Duets (instr.)  E|b; B|b; F; C; G; D  
\newline 2 vl
\newline \begin{itshape}Six duos [Es, B, F, C, G, D] pour deux violons.\end{itshape} 
\newline \textcolor{darkblue}{\ding{\numexpr181 + 01}}  part(s)  
\newline print
\newline RISM-ID: 00000990060859
\newline Markordt, Siegfried  (pbl)
\newline D-B; D-Fh  Str Duo 0240; E-Mn; NL-DHgm
\newline \par \vspace{7pt} \textcolor{darkblue}{\textbf{Št'astný, František Jan  1764c-1826p}}\hfillplus{331}
\newline Duets (instr.)    
\newline RISM-ID: 00000990062672
\newline Simrock, Nicolaus  (pbl)
\newline Simrock, Nicolaus  (pbl)
\newline A-Wgm; D-Fh  Str Duo 0245; D-Gs; I-Fc; I-Mc
\newline \par \vspace{7pt} \textcolor{darkblue}{\textbf{Št'astný, František Jan  1764c-1826p}}\hfillplus{332}
\newline Instrumental pieces  C; G; D; F; B|b; E|b; D; e; G; F; D; G  
\newline vlc, b
\newline \begin{itshape}XII Pièces [C, G, D, F, B, Es, D, e, G, F, D, G] faciles et progressives pour violoncelle et basse ... œuvre 4.\end{itshape} 
\newline \textcolor{darkblue}{\ding{\numexpr181 + 01}}  part(s)  
\newline print
\newline RISM-ID: 00000990062666
\newline Simrock, Nicolaus  (pbl)
\newline D-B; D-Dl; D-F  Mus. pr. Q 54/1754; D-Fh  Str Duo 0244
\newline \par \vspace{7pt} \textcolor{darkblue}{\textbf{Št'astný, František Jan  1764c-1826p}}\hfillplus{333}
\newline Instrumental pieces  B|b; e; B|b; D; d; e  
\newline vlc, b
\newline \begin{itshape}Six pièces faciles [B, e, B, D, d, e] pour le violoncelle avec accompagnement de basse ... œuvre 5.\end{itshape} 
\newline \textcolor{darkblue}{\ding{\numexpr181 + 01}}  part(s)  
\newline print
\newline RISM-ID: 00000990062668
\newline Simrock, Nicolaus  (pbl)
\newline Simrock, Nicolaus  (pbl)
\newline D-Dl; D-Fh  Str Duo 0246; US-NYp
\newline \par \vspace{7pt} \textcolor{darkblue}{\textbf{Sterkel, Johann Franz Xaver  1750-1817}}\hfillplus{334}
\newline An Minna  B|b  StWV 54/5
\newline V, mandora
\newline \begin{itshape}[without title]\end{itshape} 
\newline \textcolor{darkblue}{\ding{\numexpr181 + 01}}  1 parts  
\newline Manuscript copy
\newline 1.1.1  V  B|b
\newline \begin{footnotesize} Die Sonne sinkt der Abend winkt \end{footnotesize}  
\begin{filecontents*}{334-1.code}
@clef:G-2
@keysig:bBE
@timesig:c/
@data:8'FFF/4B-8-FFF/4''C--8-'F/4.''D8'B{''CD}{EC}/q'B2A8-''CCC/4.C8C{CD}{'GB}/qB4A-8-''CCC/
\end{filecontents*}
\commandline{ verovio --spacing-non-linear=0.50 -w 1500 --spacing-system=0.5 --adjust-page-height -b 0 334-1.code }
\newline
\includesvg[width=220pt]{334-1}%

\newline RISM-ID: 455011118
\newline Vier Strophen
\newline Zuweisung nach RISM A I: S 5804, darin die Nr. 5
\newline D-Fh  Ms 002
\newline $\rightarrow$ In collection 419 (455008345)

\newline \par \vspace{7pt} \textcolor{darkblue}{\textbf{Telemann, Georg Philipp  1681-1767}}\hfillplus{335}
\newline Suites. Excerpts. Arr  D  TWV 55: D 8
\newline org
\newline \begin{itshape}[f.116r, heading:] Aria\end{itshape} 
\newline \textcolor{darkblue}{\ding{\numexpr181 + 01}}  1 parts  
\newline Manuscript copy
\newline 1.1.1  org  D  
\begin{filecontents*}{335-1.code}
@clef:C-1
@keysig:xFC
@timesig:c/
@data:!4'D{8E6FG}4F{866EFG}/!{FED}{B''CD}{8.C6'B}{AGFE}/f{8FB}{8.xG6A}2A://:
\end{filecontents*}
\commandline{ verovio --spacing-non-linear=0.50 -w 1500 --spacing-system=0.5 --adjust-page-height -b 0 335-1.code }
\newline
\includesvg[width=220pt]{335-1}%

\newline RISM-ID: 455011021
\newline Ohne alte Paginierung
\newline D-Fh  Ms 001
\newline $\rightarrow$ In collection 418 (455008343)

\newline \par \vspace{7pt} \textcolor{darkblue}{\textbf{Veltheim, Baron von  19.sc}}\hfillplus{336}
\newline Ständchen  E  
\newline V, mandora
\newline \begin{itshape}[heading, f.41v:] Composée p: Baron de Veltheim\end{itshape} 
\newline \textcolor{darkblue}{\ding{\numexpr181 + 01}}  1 parts  
\newline Manuscript copy
\newline Pechmann, Nanette B.
\newline 1.1.1  V  E
\newline \begin{footnotesize} Komm fein Liebchen komm ans Fenster \end{footnotesize}  
\begin{filecontents*}{336-1.code}
@clef:G-2
@keysig:xFCGD
@timesig:6/8
@data:{8'B''C}D/4E8E{DE}C/4'B8E{BG}E/{BG}E4F{6''FD}/4.('B){8B''C}D/4E8E{DE}C/
\end{filecontents*}
\commandline{ verovio --spacing-non-linear=0.50 -w 1500 --spacing-system=0.5 --adjust-page-height -b 0 336-1.code }
\newline
\includesvg[width=220pt]{336-1}%

\newline RISM-ID: 455011116
\newline Vier Strophen
\newline Kotzebue, August von  (lyr)
\newline D-Fh  Ms 002
\newline $\rightarrow$ In collection 419 (455008345)

\newline \par \vspace{7pt} \textcolor{darkblue}{\textbf{Viotti, Giovanni Battista  1755-1824}}\hfillplus{337}
\newline Duets (instr.)  b; D; B|b  
\newline 2 vl, b
\newline \begin{itshape}[identisch mit den Duos GiaV. 53-55]. Trois trios [h, D, B] pour deux violons et basse ... œuvre 18.\end{itshape} 
\newline \textcolor{darkblue}{\ding{\numexpr181 + 01}}  part(s)  
\newline print
\newline RISM-ID: 00000990066908
\newline Garnier  (pbl)
\newline Magasin de musique dirigé par MM. Chérubini, Méhul, Kreutzer, Rode, Isouard et Boieldieu  (pbl)
\newline A-Wn; B-Bc; D-F  Mus. pr. Q 50/500 Nr. 4; D-Fh; I-Mc; I-Nc; I-PESc; I-Vc; J-Tk; US-Wc
\newline \par \vspace{7pt} \textcolor{darkblue}{\textbf{Viotti, Giovanni Battista  1755-1824}}\hfillplus{338}
\newline Duets (instr.)    
\newline RISM-ID: 00000990066923
\newline Lischke, Ferdinand Samuel  (pbl)
\newline A-Wmi; A-Wn; D-Fh  Str Trio 0156; RUS-Mrg; US-R
\newline \par \vspace{7pt} \textcolor{darkblue}{\textbf{Viotti, Giovanni Battista  1755-1824}}\hfillplus{339}
\newline Trios (instr.)  A  
\newline 2 vl, vlc
\newline \begin{itshape}Trio [A] per due violini e violoncello ... op. 3.\end{itshape} 
\newline \textcolor{darkblue}{\ding{\numexpr181 + 01}}  part(s)  
\newline print
\newline RISM-ID: 00000990066887
\newline Artaria  (pbl)
\newline A-Wgm; A-Wmi; A-Wn; A-Wst; CZ-Pnm; D-BDHlebermann; D-Fh  Str Trio 0159; F-Pn; HR-Zh; HR-Zha; I-Bc; I-Gl; I-Mc; I-PESc; I-Vnm; N-Ou; RUS-Mrg; S-Skma; US-Wc
\newline \par \vspace{7pt} \textcolor{darkblue}{\textbf{Viotti, Giovanni Battista  1755-1824}}\hfillplus{340}
\newline Trios (instr.)  B|b; a; E  
\newline 2 vl, b
\newline \begin{itshape}Trois trios [B, a, E] pour deux violons et basse ... 5|m|e œuvre de trios.\end{itshape} 
\newline \textcolor{darkblue}{\ding{\numexpr181 + 01}}  part(s)  
\newline print
\newline RISM-ID: 00000990066895
\newline André, Johann  (pbl)
\newline D-Bhm; D-Fh; D-LÜh; D-OF; I-Mc; I-Vc; RUS-Mrg; S-Skma
\newline \par \vspace{7pt} \textcolor{darkblue}{\textbf{Winter, Peter von  1754-1825}}\hfillplus{341}
\newline Operas    
\newline 2 vl
\newline \begin{itshape}Douze duos concertants pour deux violons ... arrangés par Stumpf.\end{itshape} 
\newline \textcolor{darkblue}{\ding{\numexpr181 + 01}}  part(s)  
\newline print
\newline RISM-ID: 00000990069007
\newline Schott, Bernhard  (pbl)
\newline D-Fh  Str Duo 0247; D-MZsch; NL-At
\newline \par \vspace{7pt} \textcolor{darkblue}{\textbf{Woldemar, Michel  1750-1815}}\hfillplus{342}
\newline Airs variés  G; B|b; g; a; g; E|b  
\newline 2 vl
\newline \begin{itshape}SIX AIRS | variés [G, B, g, a, g, Es] | pour violon [= 2 violons] | par WOLDEMAR | Elève de Lolli\end{itshape} 
\newline \textcolor{darkblue}{\ding{\numexpr181 + 01}}  part(s)  
\newline print
\newline RISM-ID: 00000990069158
\newline Gaveaux, frères  (pbl)
\newline D-Fh  Str Duo 0291; F-Pc
\newline \par \vspace{7pt} \textcolor{darkblue}{\textbf{Zink, Joseph Michael  1758-1829}}\hfillplus{343}
\newline Allegretto  G  
\newline mandora (2)
\newline \begin{itshape}[f.10r, at left:] Allegretto | [left before the accolade:] 12.\end{itshape} 
\newline \textcolor{darkblue}{\ding{\numexpr181 + 01}}  1 parts  
\newline \begin{small} mandora 2 missing\end{small} 
\newline Autograph manuscript
\newline 1.1.1  mandora 1  G  
\begin{filecontents*}{343-1.code}
@clef:g-2
@keysig:xF
@timesig:2/4
@data:8'D/!,B'C,D'C/C,B,,G'D/!ff
\end{filecontents*}
\commandline{ verovio --spacing-non-linear=0.50 -w 1500 --spacing-system=0.5 --adjust-page-height -b 0 343-1.code }
\newline
\includesvg[width=220pt]{343-1}%

\newline RISM-ID: 455011051
\newline Der Band mit der mandora 2 befindet sich in D Eu mit der Signatur Esl VIII 403
\newline D-Fh  Ms 002
\newline $\rightarrow$ In collection 419 (455008345)

\newline \par \vspace{7pt} \textcolor{darkblue}{\textbf{Zink, Joseph Michael  1758-1829}}\hfillplus{344}
\newline Allegretto  A  
\newline mandora (2)
\newline \begin{itshape}[f.32v, at left:] Allegretto | [left before the accolade:] 62.\end{itshape} 
\newline \textcolor{darkblue}{\ding{\numexpr181 + 01}}  1 parts  
\newline \begin{small} mandora 2 missing\end{small} 
\newline Autograph manuscript
\newline 1.1.1  mandora 1  A  
\begin{filecontents*}{344-1.code}
@clef:g-2
@keysig:xFCG
@timesig:6/8
@data:8,,A/!,E'CEAEC/4.D4,B8B/!8CDEEDC/4.'C4,B8,,A/f
\end{filecontents*}
\commandline{ verovio --spacing-non-linear=0.50 -w 1500 --spacing-system=0.5 --adjust-page-height -b 0 344-1.code }
\newline
\includesvg[width=220pt]{344-1}%

\newline RISM-ID: 455011101
\newline Der Band mit der mandora 2 befindet sich in D Eu mit der Signatur Esl VIII 403
\newline D-Fh  Ms 002
\newline $\rightarrow$ In collection 419 (455008345)

\newline \par \vspace{7pt} \textcolor{darkblue}{\textbf{Zink, Joseph Michael  1758-1829}}\hfillplus{345}
\newline Allegretto  D  
\newline mandora (2)
\newline \begin{itshape}[f.33v, at left:] Allegretto | [left before the accolade:] 64.\end{itshape} 
\newline \textcolor{darkblue}{\ding{\numexpr181 + 01}}  1 parts  
\newline \begin{small} mandora 2 missing\end{small} 
\newline Autograph manuscript
\newline 1.1.1  mandora 1  D  
\begin{filecontents*}{345-1.code}
@clef:g-2
@keysig:xFC
@timesig:c/
@data:!{8,D'F,,A'F}!f/!{,E'G,,A'G}!f/!{,D'G,,A'F}!f/!{,,A'E,C'E}!f/
\end{filecontents*}
\commandline{ verovio --spacing-non-linear=0.50 -w 1500 --spacing-system=0.5 --adjust-page-height -b 0 345-1.code }
\newline
\includesvg[width=220pt]{345-1}%

\newline RISM-ID: 455011103
\newline Der Band mit der mandora 2 befindet sich in D Eu mit der Signatur Esl VIII 403
\newline D-Fh  Ms 002
\newline $\rightarrow$ In collection 419 (455008345)

\newline \par \vspace{7pt} \textcolor{darkblue}{\textbf{Zink, Joseph Michael  1758-1829}}\hfillplus{346}
\newline Allegro  G  
\newline mandora (2)
\newline \begin{itshape}[f.11r, at left:] Allgro | [left before the accolade:] 14.\end{itshape} 
\newline \textcolor{darkblue}{\ding{\numexpr181 + 01}}  1 parts  
\newline \begin{small} mandora 2 missing\end{small} 
\newline Autograph manuscript
\newline 1.1.1  mandora 1  G  
\begin{filecontents*}{346-1.code}
@clef:g-2
@keysig:xF
@timesig:2/4
@data:8'G/!FGFE/{8.6DC}8,B'D/!{C,A}'DC/{8.6,BA}8G'G/f
\end{filecontents*}
\commandline{ verovio --spacing-non-linear=0.50 -w 1500 --spacing-system=0.5 --adjust-page-height -b 0 346-1.code }
\newline
\includesvg[width=220pt]{346-1}%

\newline RISM-ID: 455011053
\newline Der Band mit der mandora 2 befindet sich in D Eu mit der Signatur Esl VIII 403
\newline D-Fh  Ms 002
\newline $\rightarrow$ In collection 419 (455008345)

\newline \par \vspace{7pt} \textcolor{darkblue}{\textbf{Zink, Joseph Michael  1758-1829}}\hfillplus{347}
\newline Allegro  A  
\newline mandora (2)
\newline \begin{itshape}[f.30v, at left:] Allegro | [left before the accolade:] 58.\end{itshape} 
\newline \textcolor{darkblue}{\ding{\numexpr181 + 01}}  1 parts  
\newline \begin{small} mandora 2 missing\end{small} 
\newline Autograph manuscript
\newline 1.1.1  mandora 1  A  
\begin{filecontents*}{347-1.code}
@clef:g-2
@keysig:xFCG
@timesig:c/
@data:4'AAA-/EEE-/CC8,BAB'C/4,A'C,A-/BB'CC/8EDCD4,B-/
\end{filecontents*}
\commandline{ verovio --spacing-non-linear=0.50 -w 1500 --spacing-system=0.5 --adjust-page-height -b 0 347-1.code }
\newline
\includesvg[width=220pt]{347-1}%

\newline RISM-ID: 455011097
\newline Der Band mit der mandora 2 befindet sich in D Eu mit der Signatur Esl VIII 403
\newline D-Fh  Ms 002
\newline $\rightarrow$ In collection 419 (455008345)

\newline \par \vspace{7pt} \textcolor{darkblue}{\textbf{Zink, Joseph Michael  1758-1829}}\hfillplus{348}
\newline Allegro polacca  D  
\newline mandora (2)
\newline \begin{itshape}[f.11v, at left:] Allo | Polaca | [left before the accolade:] 15.\end{itshape} 
\newline \textcolor{darkblue}{\ding{\numexpr181 + 01}}  1 parts  
\newline \begin{small} mandora 2 missing\end{small} 
\newline Autograph manuscript
\newline 1.1.1  mandora 1  D  
\begin{filecontents*}{348-1.code}
@clef:g-2
@keysig:xFC
@timesig:2/4
@data:!8'DDCC/!6DEFD{8,AA}/f6'DEFG4A/8DDEE/FFGG/
\end{filecontents*}
\commandline{ verovio --spacing-non-linear=0.50 -w 1500 --spacing-system=0.5 --adjust-page-height -b 0 348-1.code }
\newline
\includesvg[width=220pt]{348-1}%

\newline RISM-ID: 455011054
\newline Der Band mit der mandora 2 befindet sich in D Eu mit der Signatur Esl VIII 403
\newline D-Fh  Ms 002
\newline $\rightarrow$ In collection 419 (455008345)

\newline \par \vspace{7pt} \textcolor{darkblue}{\textbf{Zink, Joseph Michael  1758-1829}}\hfillplus{349}
\newline Allegro tedesco  G  
\newline mandora (2)
\newline \begin{itshape}[f.8v, at left:] Allo | Tedesco | [left before the accolade:] 6.\end{itshape} 
\newline \textcolor{darkblue}{\ding{\numexpr181 + 01}}  1 parts  
\newline \begin{small} mandora 2 missing\end{small} 
\newline Autograph manuscript
\newline 1.1.1  mandora 1  G  
\begin{filecontents*}{349-1.code}
@clef:g-2
@keysig:xF
@timesig:3/4
@data:!4,,G,B'C/D,,G'D/!C,D'C/,B,,G,B/f
\end{filecontents*}
\commandline{ verovio --spacing-non-linear=0.50 -w 1500 --spacing-system=0.5 --adjust-page-height -b 0 349-1.code }
\newline
\includesvg[width=220pt]{349-1}%

\newline RISM-ID: 455011045
\newline Der Band mit der mandora 2 befindet sich in D Eu mit der Signatur Esl VIII 403
\newline D-Fh  Ms 002
\newline $\rightarrow$ In collection 419 (455008345)

\newline \par \vspace{7pt} \textcolor{darkblue}{\textbf{Zink, Joseph Michael  1758-1829}}\hfillplus{350}
\newline Allemandes  G  
\newline mandora (2)
\newline \begin{itshape}[f.16v, at left:] Allemande | [left before the accolade:] 25.\end{itshape} 
\newline \textcolor{darkblue}{\ding{\numexpr181 + 01}}  1 parts  
\newline \begin{small} mandora 2 missing\end{small} 
\newline Autograph manuscript
\newline 1.1.1  mandora 1  G  
\begin{filecontents*}{350-1.code}
@clef:g-2
@keysig:xF
@timesig:3/8
@data:{6,GA}/!{B'CD}{xCDC}/4D!8,B/'C,A'F/4G{6,GA}/f8'F/GExC/4D://:
\end{filecontents*}
\commandline{ verovio --spacing-non-linear=0.50 -w 1500 --spacing-system=0.5 --adjust-page-height -b 0 350-1.code }
\newline
\includesvg[width=220pt]{350-1}%

\newline RISM-ID: 455011064
\newline Der Band mit der mandora 2 befindet sich in D Eu mit der Signatur Esl VIII 403
\newline D-Fh  Ms 002
\newline $\rightarrow$ In collection 419 (455008345)

\newline \par \vspace{7pt} \textcolor{darkblue}{\textbf{Zink, Joseph Michael  1758-1829}}\hfillplus{351}
\newline Allemandes  G  
\newline mandora (2)
\newline \begin{itshape}[f.16v, at left:] Allemande | [left before the accolade:] 26.\end{itshape} 
\newline \textcolor{darkblue}{\ding{\numexpr181 + 01}}  1 parts  
\newline \begin{small} mandora 2 missing\end{small} 
\newline Autograph manuscript
\newline 1.1.1  mandora 1  G  
\begin{filecontents*}{351-1.code}
@clef:g-2
@keysig:xF
@timesig:3/8
@data:!8,GBD/A'C,D/!{8'F6FF8F}/4G8D/f{8D6DC8,B}/4B8-://:
\end{filecontents*}
\commandline{ verovio --spacing-non-linear=0.50 -w 1500 --spacing-system=0.5 --adjust-page-height -b 0 351-1.code }
\newline
\includesvg[width=220pt]{351-1}%

\newline RISM-ID: 455011065
\newline Der Band mit der mandora 2 befindet sich in D Eu mit der Signatur Esl VIII 403
\newline D-Fh  Ms 002
\newline $\rightarrow$ In collection 419 (455008345)

\newline \par \vspace{7pt} \textcolor{darkblue}{\textbf{Zink, Joseph Michael  1758-1829}}\hfillplus{352}
\newline Allemandes  G  
\newline mandora (2)
\newline \begin{itshape}[f.17r, at left:] Allemande | [left before the accolade:] 27.\end{itshape} 
\newline \textcolor{darkblue}{\ding{\numexpr181 + 01}}  1 parts  
\newline \begin{small} mandora 2 missing\end{small} 
\newline Autograph manuscript
\newline 1.1.1  mandora 1  G  
\begin{filecontents*}{352-1.code}
@clef:g-2
@keysig:xF
@timesig:3/8
@data:{6,GA}{B'C}DE/8DDD/E{6EF}{GE}/4D8,B/{6'DC}8,AA/{6'D,B}8GG/
\end{filecontents*}
\commandline{ verovio --spacing-non-linear=0.50 -w 1500 --spacing-system=0.5 --adjust-page-height -b 0 352-1.code }
\newline
\includesvg[width=220pt]{352-1}%

\newline RISM-ID: 455011066
\newline Der Band mit der mandora 2 befindet sich in D Eu mit der Signatur Esl VIII 403
\newline D-Fh  Ms 002
\newline $\rightarrow$ In collection 419 (455008345)

\newline \par \vspace{7pt} \textcolor{darkblue}{\textbf{Zink, Joseph Michael  1758-1829}}\hfillplus{353}
\newline Allemandes  G  
\newline mandora (2)
\newline \begin{itshape}[f.17r, at left:] Allemande | [left before the accolade:] 28.\end{itshape} 
\newline \textcolor{darkblue}{\ding{\numexpr181 + 01}}  1 parts  
\newline \begin{small} mandora 2 missing\end{small} 
\newline Autograph manuscript
\newline 1.1.1  mandora 1  G  
\begin{filecontents*}{353-1.code}
@clef:g-2
@keysig:xF
@timesig:3/8
@data:8'GD,B/'EC,A/'C,D'C/,B,,G,B/'GD,B/'EC,A/'F,D'F/4G8-://:
\end{filecontents*}
\commandline{ verovio --spacing-non-linear=0.50 -w 1500 --spacing-system=0.5 --adjust-page-height -b 0 353-1.code }
\newline
\includesvg[width=220pt]{353-1}%

\newline RISM-ID: 455011067
\newline Der Band mit der mandora 2 befindet sich in D Eu mit der Signatur Esl VIII 403
\newline D-Fh  Ms 002
\newline $\rightarrow$ In collection 419 (455008345)

\newline \par \vspace{7pt} \textcolor{darkblue}{\textbf{Zink, Joseph Michael  1758-1829}}\hfillplus{354}
\newline Amoroso  G  
\newline mandora (2)
\newline \begin{itshape}[f.19v, at left:] Amoroso | [left before the accolade:] 33.\end{itshape} 
\newline \textcolor{darkblue}{\ding{\numexpr181 + 01}}  1 parts  
\newline \begin{small} mandora 2 missing\end{small} 
\newline Autograph manuscript
\newline 1.1.1  mandora 1  G  
\begin{filecontents*}{354-1.code}
@clef:g-2
@keysig:xF
@timesig:6/8
@data:{8.6'GA8G}{GAB}/{CDE}4.D/{8DC,A}{'DC,G}/{6'DCDCD,B}4A8-/
\end{filecontents*}
\commandline{ verovio --spacing-non-linear=0.50 -w 1500 --spacing-system=0.5 --adjust-page-height -b 0 354-1.code }
\newline
\includesvg[width=220pt]{354-1}%

\newline RISM-ID: 455011072
\newline Der Band mit der mandora 2 befindet sich in D Eu mit der Signatur Esl VIII 403
\newline D-Fh  Ms 002
\newline $\rightarrow$ In collection 419 (455008345)

\newline \par \vspace{7pt} \textcolor{darkblue}{\textbf{Zink, Joseph Michael  1758-1829}}\hfillplus{355}
\newline Amoroso  A  
\newline mandora (2)
\newline \begin{itshape}[f.35r, at left:] Ouv | [left before the accolade:] 67.\end{itshape} 
\newline \textcolor{darkblue}{\ding{\numexpr181 + 01}}  1 parts  
\newline \begin{small} mandora 2 missing\end{small} 
\newline Autograph manuscript
\newline 1.1.1  mandora 1  A  
\begin{filecontents*}{355-1.code}
@clef:g-2
@keysig:xFCG
@timesig:6/8
@data:48'ECEC/8DCD4C8E/FGAAFG/4.F4E8E/DCD,B'CD/CDE4E8E/
\end{filecontents*}
\commandline{ verovio --spacing-non-linear=0.50 -w 1500 --spacing-system=0.5 --adjust-page-height -b 0 355-1.code }
\newline
\includesvg[width=220pt]{355-1}%

\newline RISM-ID: 455011106
\newline Der Band mit der mandora 2 befindet sich in D Eu mit der Signatur Esl VIII 403
\newline D-Fh  Ms 002
\newline $\rightarrow$ In collection 419 (455008345)

\newline \par \vspace{7pt} \textcolor{darkblue}{\textbf{Zink, Joseph Michael  1758-1829}}\hfillplus{356}
\newline Andante amoroso  G  
\newline mandora (2)
\newline \begin{itshape}[f.12r, at left:] Andante | amoroso | [left before the accolade:] 16.\end{itshape} 
\newline \textcolor{darkblue}{\ding{\numexpr181 + 01}}  1 parts  
\newline \begin{small} mandora 2 missing\end{small} 
\newline Autograph manuscript
\newline 1.1.1  mandora 1  G  
\begin{filecontents*}{356-1.code}
@clef:g-2
@keysig:xF
@timesig:2/4
@data:2,BA/4B-8-GAB/2'C,B/4A-8-A{B'C}/4DD8EDC,B/BAxGA{B'C}DE/
\end{filecontents*}
\commandline{ verovio --spacing-non-linear=0.50 -w 1500 --spacing-system=0.5 --adjust-page-height -b 0 356-1.code }
\newline
\includesvg[width=220pt]{356-1}%

\newline RISM-ID: 455011055
\newline Der Band mit der mandora 2 befindet sich in D Eu mit der Signatur Esl VIII 403
\newline D-Fh  Ms 002
\newline $\rightarrow$ In collection 419 (455008345)

\newline \par \vspace{7pt} \textcolor{darkblue}{\textbf{Zink, Joseph Michael  1758-1829}}\hfillplus{357}
\newline Andante amoroso  A  
\newline mandora (2)
\newline \begin{itshape}[f.22v, at left:] Andante | Amoroso | [left before the accolade:] 38.\end{itshape} 
\newline \textcolor{darkblue}{\ding{\numexpr181 + 01}}  1 parts  
\newline \begin{small} mandora 2 missing\end{small} 
\newline Autograph manuscript
\newline 1.1.1  mandora 1  A  
\begin{filecontents*}{357-1.code}
@clef:g-2
@keysig:xFCG
@timesig:6/8
@data:!48'EECC/8EDC4,B8B/!{'C,B}A'EDC/4.C8,B'CD/f
\end{filecontents*}
\commandline{ verovio --spacing-non-linear=0.50 -w 1500 --spacing-system=0.5 --adjust-page-height -b 0 357-1.code }
\newline
\includesvg[width=220pt]{357-1}%

\newline RISM-ID: 455011077
\newline Der Band mit der mandora 2 befindet sich in D Eu mit der Signatur Esl VIII 403
\newline D-Fh  Ms 002
\newline $\rightarrow$ In collection 419 (455008345)

\newline \par \vspace{7pt} \textcolor{darkblue}{\textbf{Zink, Joseph Michael  1758-1829}}\hfillplus{358}
\newline Andante gratioso  G  
\newline mandora (2)
\newline \begin{itshape}[f.5v, mandora 1, at left:] Andante | gratioso | [left before the accolade:] 2.\end{itshape} 
\newline \textcolor{darkblue}{\ding{\numexpr181 + 01}}  1 parts  
\newline Autograph manuscript
\newline 1.1.1  mandora 2  G  
\begin{filecontents*}{358-1.code}
@clef:g-2
@keysig:xF
@timesig:6/8
@data:8,,G'GGDFF/4.G8GFG/48AAGG/8DFF4D8-/
\end{filecontents*}
\commandline{ verovio --spacing-non-linear=0.50 -w 1500 --spacing-system=0.5 --adjust-page-height -b 0 358-1.code }
\newline
\includesvg[width=220pt]{358-1}%

\newline RISM-ID: 455011041
\newline D-Fh  Ms 002
\newline $\rightarrow$ In collection 419 (455008345)

\newline \par \vspace{7pt} \textcolor{darkblue}{\textbf{Zink, Joseph Michael  1758-1829}}\hfillplus{359}
\newline Andantino gratioso  G  
\newline mandora (2)
\newline \begin{itshape}[f.18v, at left:] Ouv | [left before the accolade:] 31.\end{itshape} 
\newline \textcolor{darkblue}{\ding{\numexpr181 + 01}}  1 parts  
\newline \begin{small} mandora 2 missing\end{small} 
\newline Autograph manuscript
\newline 1.1.1  mandora 1  G  
\begin{filecontents*}{359-1.code}
@clef:g-2
@keysig:xF
@timesig:2/4
@data:!8,GGAA/{8.6B'C}8D-/!{6DC}{,A'C}{C,B}{AB}/8AAA-/f
\end{filecontents*}
\commandline{ verovio --spacing-non-linear=0.50 -w 1500 --spacing-system=0.5 --adjust-page-height -b 0 359-1.code }
\newline
\includesvg[width=220pt]{359-1}%

\newline RISM-ID: 455011070
\newline Der Band mit der mandora 2 befindet sich in D Eu mit der Signatur Esl VIII 403
\newline D-Fh  Ms 002
\newline $\rightarrow$ In collection 419 (455008345)

\newline \par \vspace{7pt} \textcolor{darkblue}{\textbf{Zink, Joseph Michael  1758-1829}}\hfillplus{360}
\newline Burlesques  G  
\newline mandora (2)
\newline \begin{itshape}[f.8r, at left:] Bourlesco | [left before the accolade:] 5.\end{itshape} 
\newline \textcolor{darkblue}{\ding{\numexpr181 + 01}}  1 parts  
\newline \begin{small} mandora 2 missing\end{small} 
\newline Autograph manuscript
\newline 1.1.1  mandora 1  G  
\begin{filecontents*}{360-1.code}
@clef:g-2
@keysig:xF
@timesig:3/8
@data:8,,G,GA/B'CD/EFG/4D8,B/'DC,A/'D,BG/'FEF/4G8-/
\end{filecontents*}
\commandline{ verovio --spacing-non-linear=0.50 -w 1500 --spacing-system=0.5 --adjust-page-height -b 0 360-1.code }
\newline
\includesvg[width=220pt]{360-1}%

\newline RISM-ID: 455011044
\newline Die tiefste Saite ist auf den dritten Bund zu verschieben, so dass die Stimmung ,,G-,,A-,D-,G-'B-'E ist
\newline Der Band mit der mandora 2 befindet sich in D Eu mit der Signatur Esl VIII 403
\newline D-Fh  Ms 002
\newline $\rightarrow$ In collection 419 (455008345)

\newline \par \vspace{7pt} \textcolor{darkblue}{\textbf{Zink, Joseph Michael  1758-1829}}\hfillplus{361}
\newline Burlesques  C  
\newline mandora (2)
\newline \begin{itshape}[f.28r, at left:] Pourlesca | [left before the accolade:] 53.\end{itshape} 
\newline \textcolor{darkblue}{\ding{\numexpr181 + 01}}  1 parts  
\newline \begin{small} mandora 2 missing\end{small} 
\newline Autograph manuscript
\newline 1.1.1  mandora 1  C  
\begin{filecontents*}{361-1.code}
@clef:g-2
@keysig:
@timesig:2/4
@data:!8'C,G'CE/4.F8E/DF{,B'D}/!C,G'CE/f4C-/
\end{filecontents*}
\commandline{ verovio --spacing-non-linear=0.50 -w 1500 --spacing-system=0.5 --adjust-page-height -b 0 361-1.code }
\newline
\includesvg[width=220pt]{361-1}%

\newline RISM-ID: 455011092
\newline Der Band mit der mandora 2 befindet sich in D Eu mit der Signatur Esl VIII 403
\newline D-Fh  Ms 002
\newline $\rightarrow$ In collection 419 (455008345)

\newline \par \vspace{7pt} \textcolor{darkblue}{\textbf{Zink, Joseph Michael  1758-1829}}\hfillplus{362}
\newline Deutsche Tänze  G  
\newline mandora (2)
\newline \begin{itshape}[f.8v, at left:] Tedesco | [left before the accolade:] 7.\end{itshape} 
\newline \textcolor{darkblue}{\ding{\numexpr181 + 01}}  1 parts  
\newline \begin{small} mandora 2 missing\end{small} 
\newline Autograph manuscript
\newline 1.1.1  mandora 1  G  
\begin{filecontents*}{362-1.code}
@clef:g-2
@keysig:xF
@timesig:3/4
@data:8'DCDC4,A/8'DCDC4,G/8'DCDC4,A/'D
\end{filecontents*}
\commandline{ verovio --spacing-non-linear=0.50 -w 1500 --spacing-system=0.5 --adjust-page-height -b 0 362-1.code }
\newline
\includesvg[width=220pt]{362-1}%

\newline RISM-ID: 455011046
\newline Der Band mit der mandora 2 befindet sich in D Eu mit der Signatur Esl VIII 403
\newline D-Fh  Ms 002
\newline $\rightarrow$ In collection 419 (455008345)

\newline \par \vspace{7pt} \textcolor{darkblue}{\textbf{Zink, Joseph Michael  1758-1829}}\hfillplus{363}
\newline Deutsche Tänze    
\newline mandora (2)
\newline \begin{itshape}[f.9v, at left:] Tedesco | [left before the accolade:] 8.\end{itshape} 
\newline \textcolor{darkblue}{\ding{\numexpr181 + 01}}  1 parts  
\newline \begin{small} mandora 2 missing\end{small} 
\newline Autograph manuscript
\newline 1.1.1  mandora 1  G  
\begin{filecontents*}{363-1.code}
@clef:g-2
@keysig:xF
@timesig:3/4
@data:!4'GFE/DC,B/!'CDE/2D4,B/f
\end{filecontents*}
\commandline{ verovio --spacing-non-linear=0.50 -w 1500 --spacing-system=0.5 --adjust-page-height -b 0 363-1.code }
\newline
\includesvg[width=220pt]{363-1}%

\newline RISM-ID: 455011047
\newline Der Band mit der mandora 2 befindet sich in D Eu mit der Signatur Esl VIII 403
\newline D-Fh  Ms 002
\newline $\rightarrow$ In collection 419 (455008345)

\newline \par \vspace{7pt} \textcolor{darkblue}{\textbf{Zink, Joseph Michael  1758-1829}}\hfillplus{364}
\newline Deutsche Tänze  G  
\newline mandora (2)
\newline \begin{itshape}[f.9r, at left:] Tedesco | [left before the accolade:] 9.\end{itshape} 
\newline \textcolor{darkblue}{\ding{\numexpr181 + 01}}  1 parts  
\newline \begin{small} mandora 2 missing\end{small} 
\newline Autograph manuscript
\newline 1.1.1  mandora 1  G  
\begin{filecontents*}{364-1.code}
@clef:g-2
@keysig:xF
@timesig:3/4
@data:4,,G,B,,G/!8'D,B4GB/!D'C,D/B,,G,B/i/f
\end{filecontents*}
\commandline{ verovio --spacing-non-linear=0.50 -w 1500 --spacing-system=0.5 --adjust-page-height -b 0 364-1.code }
\newline
\includesvg[width=220pt]{364-1}%

\newline RISM-ID: 455011048
\newline Der Band mit der mandora 2 befindet sich in D Eu mit der Signatur Esl VIII 403
\newline D-Fh  Ms 002
\newline $\rightarrow$ In collection 419 (455008345)

\newline \par \vspace{7pt} \textcolor{darkblue}{\textbf{Zink, Joseph Michael  1758-1829}}\hfillplus{365}
\newline Deutsche Tänze  G  
\newline mandora (2)
\newline \begin{itshape}[f.9v, at left:] Tedesco | [left before the accolade:] 10.\end{itshape} 
\newline \textcolor{darkblue}{\ding{\numexpr181 + 01}}  1 parts  
\newline \begin{small} mandora 2 missing\end{small} 
\newline Autograph manuscript
\newline 1.1.1  mandora 1  G  
\begin{filecontents*}{365-1.code}
@clef:g-2
@keysig:xF
@timesig:3/4
@data:!{8,GAB'C}4D/{8EDC,B}4A/!D'FF/,,G'GG/f
\end{filecontents*}
\commandline{ verovio --spacing-non-linear=0.50 -w 1500 --spacing-system=0.5 --adjust-page-height -b 0 365-1.code }
\newline
\includesvg[width=220pt]{365-1}%

\newline RISM-ID: 455011049
\newline Der Band mit der mandora 2 befindet sich in D Eu mit der Signatur Esl VIII 403
\newline D-Fh  Ms 002
\newline $\rightarrow$ In collection 419 (455008345)

\newline \par \vspace{7pt} \textcolor{darkblue}{\textbf{Zink, Joseph Michael  1758-1829}}\hfillplus{366}
\newline Deutsche Tänze  G  
\newline mandora (2)
\newline \begin{itshape}[f.9v, at left:] Tedesco | [left before the accolade:] 11.\end{itshape} 
\newline \textcolor{darkblue}{\ding{\numexpr181 + 01}}  1 parts  
\newline \begin{small} mandora 2 missing\end{small} 
\newline Autograph manuscript
\newline 1.1.1  mandora 1  G  
\begin{filecontents*}{366-1.code}
@clef:g-2
@keysig:xF
@timesig:3/4
@data:4,B'CD/2D4C/,AB'C/2D4,B/GAB/2'C4,A/
\end{filecontents*}
\commandline{ verovio --spacing-non-linear=0.50 -w 1500 --spacing-system=0.5 --adjust-page-height -b 0 366-1.code }
\newline
\includesvg[width=220pt]{366-1}%

\newline RISM-ID: 455011050
\newline Der Band mit der mandora 2 befindet sich in D Eu mit der Signatur Esl VIII 403
\newline D-Fh  Ms 002
\newline $\rightarrow$ In collection 419 (455008345)

\newline \par \vspace{7pt} \textcolor{darkblue}{\textbf{Zink, Joseph Michael  1758-1829}}\hfillplus{367}
\newline Deutsche Tänze  A  
\newline mandora (2)
\newline \begin{itshape}[f.25v, at left:] Tedesco | [left before the accolade:] 44.\end{itshape} 
\newline \textcolor{darkblue}{\ding{\numexpr181 + 01}}  1 parts  
\newline \begin{small} mandora 2 missing\end{small} 
\newline Autograph manuscript
\newline 1.1.1  mandora 1  A  
\begin{filecontents*}{367-1.code}
@clef:g-2
@keysig:xFCG
@timesig:3/8
@data:!{8'E6EF}8E/i/!{6EF}{ED}{C,B}/8A'CE/f
\end{filecontents*}
\commandline{ verovio --spacing-non-linear=0.50 -w 1500 --spacing-system=0.5 --adjust-page-height -b 0 367-1.code }
\newline
\includesvg[width=220pt]{367-1}%

\newline RISM-ID: 455011083
\newline Der Band mit der mandora 2 befindet sich in D Eu mit der Signatur Esl VIII 403
\newline D-Fh  Ms 002
\newline $\rightarrow$ In collection 419 (455008345)

\newline \par \vspace{7pt} \textcolor{darkblue}{\textbf{Zink, Joseph Michael  1758-1829}}\hfillplus{368}
\newline Deutsche Tänze  A  
\newline mandora (2)
\newline \begin{itshape}[f.25v, at left:] Tedesco | [left before the accolade:] 45.\end{itshape} 
\newline \textcolor{darkblue}{\ding{\numexpr181 + 01}}  1 parts  
\newline \begin{small} mandora 2 missing\end{small} 
\newline Autograph manuscript
\newline 1.1.1  mandora 1  A  
\begin{filecontents*}{368-1.code}
@clef:g-2
@keysig:xFCG
@timesig:3/8
@data:8'C,B'C/4D8,B/B'CD/4E8C/,A{GA}/4B8B/B'CxD/4E8-://:
\end{filecontents*}
\commandline{ verovio --spacing-non-linear=0.50 -w 1500 --spacing-system=0.5 --adjust-page-height -b 0 368-1.code }
\newline
\includesvg[width=220pt]{368-1}%

\newline RISM-ID: 455011084
\newline Der Band mit der mandora 2 befindet sich in D Eu mit der Signatur Esl VIII 403
\newline D-Fh  Ms 002
\newline $\rightarrow$ In collection 419 (455008345)

\newline \par \vspace{7pt} \textcolor{darkblue}{\textbf{Zink, Joseph Michael  1758-1829}}\hfillplus{369}
\newline Deutsche Tänze  A  
\newline mandora (2)
\newline \begin{itshape}[f.25v, at left:] Tedesco | [left before the accolade:] 46.\end{itshape} 
\newline \textcolor{darkblue}{\ding{\numexpr181 + 01}}  1 parts  
\newline \begin{small} mandora 2 missing\end{small} 
\newline Autograph manuscript
\newline 1.1.1  mandora 1  A  
\begin{filecontents*}{369-1.code}
@clef:g-2
@keysig:xFCG
@timesig:3/8
@data:8,,A'EC/!i/,,A'D,B/!{6'C,B'CD}8C/f4C8-://:
\end{filecontents*}
\commandline{ verovio --spacing-non-linear=0.50 -w 1500 --spacing-system=0.5 --adjust-page-height -b 0 369-1.code }
\newline
\includesvg[width=220pt]{369-1}%

\newline RISM-ID: 455011085
\newline Der Band mit der mandora 2 befindet sich in D Eu mit der Signatur Esl VIII 403
\newline D-Fh  Ms 002
\newline $\rightarrow$ In collection 419 (455008345)

\newline \par \vspace{7pt} \textcolor{darkblue}{\textbf{Zink, Joseph Michael  1758-1829}}\hfillplus{370}
\newline Deutsche Tänze  A  
\newline mandora (2)
\newline \begin{itshape}[f.26r, at left:] Tedesco | [left before the accolade:] 47.\end{itshape} 
\newline \textcolor{darkblue}{\ding{\numexpr181 + 01}}  1 parts  
\newline \begin{small} mandora 2 missing\end{small} 
\newline Autograph manuscript
\newline 1.1.1  mandora 1  A  
\begin{filecontents*}{370-1.code}
@clef:g-2
@keysig:xFCG
@timesig:3/8
@data:8,,A'CD/EAE/FAF/4E8-/D,B'E/C,BA/B'CxD/4E8-://:
\end{filecontents*}
\commandline{ verovio --spacing-non-linear=0.50 -w 1500 --spacing-system=0.5 --adjust-page-height -b 0 370-1.code }
\newline
\includesvg[width=220pt]{370-1}%

\newline RISM-ID: 455011086
\newline Der Band mit der mandora 2 befindet sich in D Eu mit der Signatur Esl VIII 403
\newline D-Fh  Ms 002
\newline $\rightarrow$ In collection 419 (455008345)

\newline \par \vspace{7pt} \textcolor{darkblue}{\textbf{Zink, Joseph Michael  1758-1829}}\hfillplus{371}
\newline Deutsche Tänze  A  
\newline mandora (2)
\newline \begin{itshape}[f.26r, at left:] Tedesco | [left before the accolade:] 48.\end{itshape} 
\newline \textcolor{darkblue}{\ding{\numexpr181 + 01}}  1 parts  
\newline \begin{small} mandora 2 missing\end{small} 
\newline Autograph manuscript
\newline 1.1.1  mandora 1  A  
\begin{filecontents*}{371-1.code}
@clef:g-2
@keysig:xFCG
@timesig:3/8
@data:{8'AEE}/{ECC}/{C,AA}/{AB'C}/{ED,B}/{'EC,A}/{'ED,B}/4A8-://:
\end{filecontents*}
\commandline{ verovio --spacing-non-linear=0.50 -w 1500 --spacing-system=0.5 --adjust-page-height -b 0 371-1.code }
\newline
\includesvg[width=220pt]{371-1}%

\newline RISM-ID: 455011087
\newline Der Band mit der mandora 2 befindet sich in D Eu mit der Signatur Esl VIII 403
\newline D-Fh  Ms 002
\newline $\rightarrow$ In collection 419 (455008345)

\newline \par \vspace{7pt} \textcolor{darkblue}{\textbf{Zink, Joseph Michael  1758-1829}}\hfillplus{372}
\newline Deutsche Tänze  A  
\newline mandora (2)
\newline \begin{itshape}[f.26r, at left:] Tedesco | [left before the accolade:] 49.\end{itshape} 
\newline \textcolor{darkblue}{\ding{\numexpr181 + 01}}  1 parts  
\newline \begin{small} mandora 2 missing\end{small} 
\newline Autograph manuscript
\newline 1.1.1  mandora 1  A  
\begin{filecontents*}{372-1.code}
@clef:g-2
@keysig:xFCG
@timesig:3/8
@data:{6'CDEF}8E/i/8FED/4E8C/{6,B'CDC}8,B/'CEC/
\end{filecontents*}
\commandline{ verovio --spacing-non-linear=0.50 -w 1500 --spacing-system=0.5 --adjust-page-height -b 0 372-1.code }
\newline
\includesvg[width=220pt]{372-1}%

\newline RISM-ID: 455011088
\newline Der Band mit der mandora 2 befindet sich in D Eu mit der Signatur Esl VIII 403
\newline D-Fh  Ms 002
\newline $\rightarrow$ In collection 419 (455008345)

\newline \par \vspace{7pt} \textcolor{darkblue}{\textbf{Zink, Joseph Michael  1758-1829}}\hfillplus{373}
\newline Deutsche Tänze  A  
\newline mandora (2)
\newline \begin{itshape}[f.31v, at left:] Allo | Tedesco | [left before the accolade:] 60.\end{itshape} 
\newline \textcolor{darkblue}{\ding{\numexpr181 + 01}}  1 parts  
\newline \begin{small} mandora 2 missing\end{small} 
\newline Autograph manuscript
\newline 1.1.1  mandora 1  A  
\begin{filecontents*}{373-1.code}
@clef:g-2
@keysig:xFCG
@timesig:3/8
@data:!8,,A'AA/AEC/,,E'D,B/!{6'C,B'CD8C}/f4C8-://:
\end{filecontents*}
\commandline{ verovio --spacing-non-linear=0.50 -w 1500 --spacing-system=0.5 --adjust-page-height -b 0 373-1.code }
\newline
\includesvg[width=220pt]{373-1}%

\newline RISM-ID: 455011099
\newline Der Band mit der mandora 2 befindet sich in D Eu mit der Signatur Esl VIII 403
\newline D-Fh  Ms 002
\newline $\rightarrow$ In collection 419 (455008345)

\newline \par \vspace{7pt} \textcolor{darkblue}{\textbf{Zink, Joseph Michael  1758-1829}}\hfillplus{374}
\newline Dreher  D  
\newline mandora
\newline \begin{itshape}[f.39v, at left:] Dreher: v: Z:\end{itshape} 
\newline \textcolor{darkblue}{\ding{\numexpr181 + 01}}  1 parts  
\newline Autograph manuscript
\newline 1.1.1  mandora  D  
\begin{filecontents*}{374-1.code}
@clef:G-2
@keysig:xFC
@timesig:2/4
@data:4'''DD/''A8-'A/{''Dq6EDC}{8DF}/4'A8-''A/{8.6AGGE}/{GFFD}/{FEEC}/
\end{filecontents*}
\commandline{ verovio --spacing-non-linear=0.50 -w 1500 --spacing-system=0.5 --adjust-page-height -b 0 374-1.code }
\newline
\includesvg[width=220pt]{374-1}%

\newline RISM-ID: 455011114
\newline Der Schreiber ist der selbe wie der 70 Divertimenti in der gleichen Handschrift
\newline D-Fh  Ms 002
\newline $\rightarrow$ In collection 419 (455008345)

\newline \par \vspace{7pt} \textcolor{darkblue}{\textbf{Zink, Joseph Michael  1758-1829}}\hfillplus{375}
\newline Finale  G  
\newline mandora (2)
\newline \begin{itshape}[f.21r, at left:] Finale | [left before the accolade:] 36.\end{itshape} 
\newline \textcolor{darkblue}{\ding{\numexpr181 + 01}}  1 parts  
\newline \begin{small} mandora 2 missing\end{small} 
\newline Autograph manuscript
\newline 1.1.1  mandora 1  G  
\begin{filecontents*}{375-1.code}
@clef:g-2
@keysig:xF
@timesig:2/4
@data:4.8'GD/GD/8GDGD/GD,BG/FGAB/'CDEF/4.8GD/GD/8GDGD/
\end{filecontents*}
\commandline{ verovio --spacing-non-linear=0.50 -w 1500 --spacing-system=0.5 --adjust-page-height -b 0 375-1.code }
\newline
\includesvg[width=220pt]{375-1}%

\newline RISM-ID: 455011075
\newline Der Band mit der mandora 2 befindet sich in D Eu mit der Signatur Esl VIII 403
\newline D-Fh  Ms 002
\newline $\rightarrow$ In collection 419 (455008345)

\newline \par \vspace{7pt} \textcolor{darkblue}{\textbf{Zink, Joseph Michael  1758-1829}}\hfillplus{376}
\newline Gratioso  G  
\newline mandora (2)
\newline \begin{itshape}[f.13r, at left:] Gratioso | [left before the accolade:] 18.\end{itshape} 
\newline \textcolor{darkblue}{\ding{\numexpr181 + 01}}  1 parts  
\newline \begin{small} mandora 2 missing\end{small} 
\newline Autograph manuscript
\newline 1.1.1  mandora 1  G  
\begin{filecontents*}{376-1.code}
@clef:g-2
@keysig:xF
@timesig:6/8
@data:8,B/4'G8GGFG/4.B4G8D/{EF}GGFE/4.E4D8,B/'C,B'C,AB'C/{,B'C}D4D8G/
\end{filecontents*}
\commandline{ verovio --spacing-non-linear=0.50 -w 1500 --spacing-system=0.5 --adjust-page-height -b 0 376-1.code }
\newline
\includesvg[width=220pt]{376-1}%

\newline RISM-ID: 455011057
\newline Der Band mit der mandora 2 befindet sich in D Eu mit der Signatur Esl VIII 403
\newline D-Fh  Ms 002
\newline $\rightarrow$ In collection 419 (455008345)

\newline \par \vspace{7pt} \textcolor{darkblue}{\textbf{Zink, Joseph Michael  1758-1829}}\hfillplus{377}
\newline Harlequino  A  
\newline mandora (2)
\newline \begin{itshape}[f.23r, at left:] Harlequino | [left before the accolade:] 39.\end{itshape} 
\newline \textcolor{darkblue}{\ding{\numexpr181 + 01}}  1 parts  
\newline \begin{small} mandora 2 missing\end{small} 
\newline Autograph manuscript
\newline 1.1.1  mandora 1  A  
\begin{filecontents*}{377-1.code}
@clef:g-2
@keysig:xFCG
@timesig:2/4
@data:8'E/!C,,AA'E/C,,AA'C/!{,B'C}{DC}/,B'E,,E'E/f
\end{filecontents*}
\commandline{ verovio --spacing-non-linear=0.50 -w 1500 --spacing-system=0.5 --adjust-page-height -b 0 377-1.code }
\newline
\includesvg[width=220pt]{377-1}%

\newline RISM-ID: 455011078
\newline Der Band mit der mandora 2 befindet sich in D Eu mit der Signatur Esl VIII 403
\newline D-Fh  Ms 002
\newline $\rightarrow$ In collection 419 (455008345)

\newline \par \vspace{7pt} \textcolor{darkblue}{\textbf{Zink, Joseph Michael  1758-1829}}\hfillplus{378}
\newline Marches  G  
\newline mandora (2)
\newline \begin{itshape}[f.14v, at left:] Marchia | [left before the accolade:] 21.\end{itshape} 
\newline \textcolor{darkblue}{\ding{\numexpr181 + 01}}  1 parts  
\newline \begin{small} mandora 2 missing\end{small} 
\newline Autograph manuscript
\newline 1.1.1  mandora 1  G  
\begin{filecontents*}{378-1.code}
@clef:g-2
@keysig:xF
@timesig:c/
@data:4,G{8.6GG}4GA/{8.6BA}{AG}4G-/4B{8.6BB}4B'C/{8.6DC}{C,B}4B-/
\end{filecontents*}
\commandline{ verovio --spacing-non-linear=0.50 -w 1500 --spacing-system=0.5 --adjust-page-height -b 0 378-1.code }
\newline
\includesvg[width=220pt]{378-1}%

\newline RISM-ID: 455011060
\newline Der Band mit der mandora 2 befindet sich in D Eu mit der Signatur Esl VIII 403
\newline D-Fh  Ms 002
\newline $\rightarrow$ In collection 419 (455008345)

\newline \par \vspace{7pt} \textcolor{darkblue}{\textbf{Zink, Joseph Michael  1758-1829}}\hfillplus{379}
\newline Marches  C  
\newline mandora (2)
\newline \begin{itshape}[f.27v, at left:] Marchia | [left before the accolade:] 52.\end{itshape} 
\newline \textcolor{darkblue}{\ding{\numexpr181 + 01}}  1 parts  
\newline \begin{small} mandora 2 missing\end{small} 
\newline Autograph manuscript
\newline 1.1.1  mandora 1  C  
\begin{filecontents*}{379-1.code}
@clef:g-2
@keysig:
@timesig:c/
@data:48.6'C{CC}D{DD}/{8EDDC}{C,G'CD}/48.6E{EE}F{FF}/{8GFFE}4E-/
\end{filecontents*}
\commandline{ verovio --spacing-non-linear=0.50 -w 1500 --spacing-system=0.5 --adjust-page-height -b 0 379-1.code }
\newline
\includesvg[width=220pt]{379-1}%

\newline RISM-ID: 455011091
\newline Der Band mit der mandora 2 befindet sich in D Eu mit der Signatur Esl VIII 403
\newline D-Fh  Ms 002
\newline $\rightarrow$ In collection 419 (455008345)

\newline \par \vspace{7pt} \textcolor{darkblue}{\textbf{Zink, Joseph Michael  1758-1829}}\hfillplus{380}
\newline Marches  D  
\newline mandora (2)
\newline \begin{itshape}[f.32r, at left:] Marchia | [left before the accolade:] 61.\end{itshape} 
\newline \textcolor{darkblue}{\ding{\numexpr181 + 01}}  1 parts  
\newline \begin{small} mandora 2 missing\end{small} 
\newline Autograph manuscript
\newline 1.1.1  mandora 1  D  
\begin{filecontents*}{380-1.code}
@clef:g-2
@keysig:xFC
@timesig:c/
@data:48.6'D{DD}E{EE}/4FED-/48.6F{FF}G{GG}/4AGF-/8,A'CEEEDC,B/
\end{filecontents*}
\commandline{ verovio --spacing-non-linear=0.50 -w 1500 --spacing-system=0.5 --adjust-page-height -b 0 380-1.code }
\newline
\includesvg[width=220pt]{380-1}%

\newline RISM-ID: 455011100
\newline Der Band mit der mandora 2 befindet sich in D Eu mit der Signatur Esl VIII 403
\newline D-Fh  Ms 002
\newline $\rightarrow$ In collection 419 (455008345)

\newline \par \vspace{7pt} \textcolor{darkblue}{\textbf{Zink, Joseph Michael  1758-1829}}\hfillplus{381}
\newline Marches  C  
\newline mandora (2)
\newline \begin{itshape}[f.33v, at left:] Marche | [left before the accolade:] 65.\end{itshape} 
\newline \textcolor{darkblue}{\ding{\numexpr181 + 01}}  1 parts  
\newline \begin{small} mandora 2 missing\end{small} 
\newline Autograph manuscript
\newline 1.1.1  mandora 1  C  
\begin{filecontents*}{381-1.code}
@clef:g-2
@keysig:
@timesig:c/
@data:4'C{8.6CC}4E{8.6EE}/{8GEGE}{C,GAB}/{'CDEF}{AGFE}/4D{8.6DD}4D-/
\end{filecontents*}
\commandline{ verovio --spacing-non-linear=0.50 -w 1500 --spacing-system=0.5 --adjust-page-height -b 0 381-1.code }
\newline
\includesvg[width=220pt]{381-1}%

\newline RISM-ID: 455011104
\newline Der Band mit der mandora 2 befindet sich in D Eu mit der Signatur Esl VIII 403
\newline D-Fh  Ms 002
\newline $\rightarrow$ In collection 419 (455008345)

\newline \par \vspace{7pt} \textcolor{darkblue}{\textbf{Zink, Joseph Michael  1758-1829}}\hfillplus{382}
\newline Marches  A  
\newline mandora (2)
\newline \begin{itshape}[f.36r, at left:] Marche | [left before the accolade:] 69.\end{itshape} 
\newline \textcolor{darkblue}{\ding{\numexpr181 + 01}}  1 parts  
\newline \begin{small} mandora 2 missing\end{small} 
\newline Autograph manuscript
\newline 1.1.1  mandora 1  A  
\begin{filecontents*}{382-1.code}
@clef:g-2
@keysig:xFCG
@timesig:c/
@data:4,,A/,A{8.6AA}4B{8.6BB}/8'C,BBAAGAB/4'C{8.6CC}4D{8.6DD}/8EDDC4C-/
\end{filecontents*}
\commandline{ verovio --spacing-non-linear=0.50 -w 1500 --spacing-system=0.5 --adjust-page-height -b 0 382-1.code }
\newline
\includesvg[width=220pt]{382-1}%

\newline RISM-ID: 455011108
\newline Der Band mit der mandora 2 befindet sich in D Eu mit der Signatur Esl VIII 403
\newline D-Fh  Ms 002
\newline $\rightarrow$ In collection 419 (455008345)

\newline \par \vspace{7pt} \textcolor{darkblue}{\textbf{Zink, Joseph Michael  1758-1829}}\hfillplus{383}
\newline Minuets  G  
\newline mandora (2)
\newline \begin{itshape}[f.7v, at left:] Menuetto | [left before the accolade:] 4.\end{itshape} 
\newline \textcolor{darkblue}{\ding{\numexpr181 + 01}}  1 parts  
\newline \begin{small} mandora 2 missing\end{small} 
\newline Autograph manuscript
\newline 1.1.1  mandora 1  G  
\begin{filecontents*}{383-1.code}
@clef:g-2
@keysig:xF
@timesig:3/4
@data:4,B8B'C,A'C/4,GG{8AB}/'DCC,ABG/2A4-/
\end{filecontents*}
\commandline{ verovio --spacing-non-linear=0.50 -w 1500 --spacing-system=0.5 --adjust-page-height -b 0 383-1.code }
\newline
\includesvg[width=220pt]{383-1}%

\newline RISM-ID: 455011043
\newline Der Band mit der mandora 2 befindet sich in D Eu mit der Signatur Esl VIII 403
\newline D-Fh  Ms 002
\newline $\rightarrow$ In collection 419 (455008345)

\newline \par \vspace{7pt} \textcolor{darkblue}{\textbf{Zink, Joseph Michael  1758-1829}}\hfillplus{384}
\newline Minuets  G  
\newline mandora (2)
\newline \begin{itshape}[f.10v, at left:] Menuetto | [left before the accolade:] 13.\end{itshape} 
\newline \textcolor{darkblue}{\ding{\numexpr181 + 01}}  1 parts  
\newline \begin{small} mandora 2 missing\end{small} 
\newline Autograph manuscript
\newline 1.1.1  mandora 1  G  
\begin{filecontents*}{384-1.code}
@clef:g-2
@keysig:xF
@timesig:3/4
@data:!{8.6'DD}/4DFA/2D{8FD}/!4E+{8EFGE}/4F-f
\end{filecontents*}
\commandline{ verovio --spacing-non-linear=0.50 -w 1500 --spacing-system=0.5 --adjust-page-height -b 0 384-1.code }
\newline
\includesvg[width=220pt]{384-1}%

\newline RISM-ID: 455011052
\newline Der Band mit der mandora 2 befindet sich in D Eu mit der Signatur Esl VIII 403
\newline D-Fh  Ms 002
\newline $\rightarrow$ In collection 419 (455008345)

\newline \par \vspace{7pt} \textcolor{darkblue}{\textbf{Zink, Joseph Michael  1758-1829}}\hfillplus{385}
\newline Minuets  G  
\newline mandora (2)
\newline \begin{itshape}[f.12v, at left:] Menue | [left before the accolade:] Nro. 17\end{itshape} 
\newline \textcolor{darkblue}{\ding{\numexpr181 + 01}}  1 parts  
\newline \begin{small} mandora 2 missing\end{small} 
\newline Autograph manuscript
\newline 1.1.1  mandora 1  G  
\begin{filecontents*}{385-1.code}
@clef:g-2
@keysig:xF
@timesig:3/4
@data:4,,GA,D/,,G-'G/!G{8FE}4F/!G,,G'G/fG--/
\end{filecontents*}
\commandline{ verovio --spacing-non-linear=0.50 -w 1500 --spacing-system=0.5 --adjust-page-height -b 0 385-1.code }
\newline
\includesvg[width=220pt]{385-1}%

\newline RISM-ID: 455011056
\newline Der Band mit der mandora 2 befindet sich in D Eu mit der Signatur Esl VIII 403
\newline D-Fh  Ms 002
\newline $\rightarrow$ In collection 419 (455008345)

\newline \par \vspace{7pt} \textcolor{darkblue}{\textbf{Zink, Joseph Michael  1758-1829}}\hfillplus{386}
\newline Minuets  G  
\newline mandora (2)
\newline \begin{itshape}[f.15v, at left:] Menuetto | [left before the accolade:] 23.\end{itshape} 
\newline \textcolor{darkblue}{\ding{\numexpr181 + 01}}  1 parts  
\newline \begin{small} mandora 2 missing\end{small} 
\newline Autograph manuscript
\newline 1.1.1  mandora 1  G  
\begin{filecontents*}{386-1.code}
@clef:g-2
@keysig:xF
@timesig:3/4
@data:4,B8BG'C,A/4'DDD/E8EFGE/2D{8,B'D}/{DC}4,AA/8'D,B4GG/
\end{filecontents*}
\commandline{ verovio --spacing-non-linear=0.50 -w 1500 --spacing-system=0.5 --adjust-page-height -b 0 386-1.code }
\newline
\includesvg[width=220pt]{386-1}%

\newline RISM-ID: 455011062
\newline Der Band mit der mandora 2 befindet sich in D Eu mit der Signatur Esl VIII 403
\newline D-Fh  Ms 002
\newline $\rightarrow$ In collection 419 (455008345)

\newline \par \vspace{7pt} \textcolor{darkblue}{\textbf{Zink, Joseph Michael  1758-1829}}\hfillplus{387}
\newline Minuets  G  
\newline mandora (2)
\newline \begin{itshape}[f.19r, at left:] Menue | [left before the accolade:] 32.\end{itshape} 
\newline \textcolor{darkblue}{\ding{\numexpr181 + 01}}  1 parts  
\newline \begin{small} mandora 2 missing\end{small} 
\newline Autograph manuscript
\newline 1.1.1  mandora 1  G  
\begin{filecontents*}{387-1.code}
@clef:g-2
@keysig:xF
@timesig:3/4
@data:!4,GGG/AAA/!8BAGAB'C/2D4,B/f8BG'GEDC/2D4-://:
\end{filecontents*}
\commandline{ verovio --spacing-non-linear=0.50 -w 1500 --spacing-system=0.5 --adjust-page-height -b 0 387-1.code }
\newline
\includesvg[width=220pt]{387-1}%

\newline RISM-ID: 455011071
\newline Der Band mit der mandora 2 befindet sich in D Eu mit der Signatur Esl VIII 403
\newline D-Fh  Ms 002
\newline $\rightarrow$ In collection 419 (455008345)

\newline \par \vspace{7pt} \textcolor{darkblue}{\textbf{Zink, Joseph Michael  1758-1829}}\hfillplus{388}
\newline Minuets  C  
\newline mandora (2)
\newline \begin{itshape}[f.20v, at left:] Menue | [left before the accolade:] 35.\end{itshape} 
\newline \textcolor{darkblue}{\ding{\numexpr181 + 01}}  1 parts  
\newline \begin{small} mandora 2 missing\end{small} 
\newline Autograph manuscript
\newline 1.1.1  mandora 1  C  
\begin{filecontents*}{388-1.code}
@clef:g-2
@keysig:
@timesig:3/4
@data:4'EEE/E8-,G'CE/GFEDC,B/4'E8-CDE/4DDD/8DExFGFE/4DDD/
\end{filecontents*}
\commandline{ verovio --spacing-non-linear=0.50 -w 1500 --spacing-system=0.5 --adjust-page-height -b 0 388-1.code }
\newline
\includesvg[width=220pt]{388-1}%

\newline RISM-ID: 455011074
\newline Der Band mit der mandora 2 befindet sich in D Eu mit der Signatur Esl VIII 403
\newline D-Fh  Ms 002
\newline $\rightarrow$ In collection 419 (455008345)

\newline \par \vspace{7pt} \textcolor{darkblue}{\textbf{Zink, Joseph Michael  1758-1829}}\hfillplus{389}
\newline Minuets  A  
\newline mandora (2)
\newline \begin{itshape}[f.23v, at left:] Menue | [left before the accolade:] 40.\end{itshape} 
\newline \textcolor{darkblue}{\ding{\numexpr181 + 01}}  1 parts  
\newline \begin{small} mandora 2 missing\end{small} 
\newline Autograph manuscript
\newline 1.1.1  mandora 1  A  
\begin{filecontents*}{389-1.code}
@clef:g-2
@keysig:xFCG
@timesig:3/4
@data:!4'CC{8,B'C}/!4D,BB/{8'DC},BA{GA}/4.'E{8DC,B}/f
\end{filecontents*}
\commandline{ verovio --spacing-non-linear=0.50 -w 1500 --spacing-system=0.5 --adjust-page-height -b 0 389-1.code }
\newline
\includesvg[width=220pt]{389-1}%

\newline RISM-ID: 455011079
\newline Der Band mit der mandora 2 befindet sich in D Eu mit der Signatur Esl VIII 403
\newline D-Fh  Ms 002
\newline $\rightarrow$ In collection 419 (455008345)

\newline \par \vspace{7pt} \textcolor{darkblue}{\textbf{Zink, Joseph Michael  1758-1829}}\hfillplus{390}
\newline Minuets  D  
\newline mandora (2)
\newline \begin{itshape}[f.25r, at left:] Menue | [left before the accolade:] 43.\end{itshape} 
\newline \textcolor{darkblue}{\ding{\numexpr181 + 01}}  1 parts  
\newline \begin{small} mandora 2 missing\end{small} 
\newline Autograph manuscript
\newline 1.1.1  mandora 1  D  
\begin{filecontents*}{390-1.code}
@clef:g-2
@keysig:xFC
@timesig:3/4
@data:4,A/'DDD/EEE/FDG/E-{8EF}/{GE}{DC}{,BA}/4'DG,,G/
\end{filecontents*}
\commandline{ verovio --spacing-non-linear=0.50 -w 1500 --spacing-system=0.5 --adjust-page-height -b 0 390-1.code }
\newline
\includesvg[width=220pt]{390-1}%

\newline RISM-ID: 455011082
\newline Der Band mit der mandora 2 befindet sich in D Eu mit der Signatur Esl VIII 403
\newline D-Fh  Ms 002
\newline $\rightarrow$ In collection 419 (455008345)

\newline \par \vspace{7pt} \textcolor{darkblue}{\textbf{Zink, Joseph Michael  1758-1829}}\hfillplus{391}
\newline Minuets  A  
\newline mandora (2)
\newline \begin{itshape}[f.27r, at left:] Menue | [left before the accolade:] 51.\end{itshape} 
\newline \textcolor{darkblue}{\ding{\numexpr181 + 01}}  1 parts  
\newline \begin{small} mandora 2 missing\end{small} 
\newline Autograph manuscript
\newline 1.1.1  mandora 1  A  
\begin{filecontents*}{391-1.code}
@clef:g-2
@keysig:xFCG
@timesig:3/4
@data:!4'CCD/{8FE}4ED/!{8ED}4D,B/{8'C,B}{'CD}4C/f
\end{filecontents*}
\commandline{ verovio --spacing-non-linear=0.50 -w 1500 --spacing-system=0.5 --adjust-page-height -b 0 391-1.code }
\newline
\includesvg[width=220pt]{391-1}%

\newline RISM-ID: 455011090
\newline Der Band mit der mandora 2 befindet sich in D Eu mit der Signatur Esl VIII 403
\newline D-Fh  Ms 002
\newline $\rightarrow$ In collection 419 (455008345)

\newline \par \vspace{7pt} \textcolor{darkblue}{\textbf{Zink, Joseph Michael  1758-1829}}\hfillplus{392}
\newline Minuets  A  
\newline mandora (2)
\newline \begin{itshape}[f.28v, at left:] Menue | [left before the accolade:] 54.\end{itshape} 
\newline \textcolor{darkblue}{\ding{\numexpr181 + 01}}  1 parts  
\newline \begin{small} mandora 2 missing\end{small} 
\newline Autograph manuscript
\newline 1.1.1  mandora 1  A  
\begin{filecontents*}{392-1.code}
@clef:g-2
@keysig:xFCG
@timesig:3/4
@data:!4'E{8EAEC}/{ED}4,BB/!{8'DC}{,BA}{GA}/4.B8'D{C,B}/f
\end{filecontents*}
\commandline{ verovio --spacing-non-linear=0.50 -w 1500 --spacing-system=0.5 --adjust-page-height -b 0 392-1.code }
\newline
\includesvg[width=220pt]{392-1}%

\newline RISM-ID: 455011093
\newline Der Band mit der mandora 2 befindet sich in D Eu mit der Signatur Esl VIII 403
\newline D-Fh  Ms 002
\newline $\rightarrow$ In collection 419 (455008345)

\newline \par \vspace{7pt} \textcolor{darkblue}{\textbf{Zink, Joseph Michael  1758-1829}}\hfillplus{393}
\newline Minuets  A  
\newline mandora (2)
\newline \begin{itshape}[f.30r, at left:] Menue | [left before the accolade:] 57.\end{itshape} 
\newline \textcolor{darkblue}{\ding{\numexpr181 + 01}}  1 parts  
\newline \begin{small} mandora 2 missing\end{small} 
\newline Autograph manuscript
\newline 1.1.1  mandora 1  A  
\begin{filecontents*}{393-1.code}
@clef:g-2
@keysig:xFCG
@timesig:3/4
@data:{8,AB}/4'CCC/2D{8,B'C}/4DDD/2E{8AE}/4CC{8EC}/4,AA{8'C,A}/
\end{filecontents*}
\commandline{ verovio --spacing-non-linear=0.50 -w 1500 --spacing-system=0.5 --adjust-page-height -b 0 393-1.code }
\newline
\includesvg[width=220pt]{393-1}%

\newline RISM-ID: 455011096
\newline Der Band mit der mandora 2 befindet sich in D Eu mit der Signatur Esl VIII 403
\newline D-Fh  Ms 002
\newline $\rightarrow$ In collection 419 (455008345)

\newline \par \vspace{7pt} \textcolor{darkblue}{\textbf{Zink, Joseph Michael  1758-1829}}\hfillplus{394}
\newline Minuets  A  
\newline mandora (2)
\newline \begin{itshape}[f.33r, at left:] Menuetto | [left before the accolade:] 63.\end{itshape} 
\newline \textcolor{darkblue}{\ding{\numexpr181 + 01}}  1 parts  
\newline \begin{small} mandora 2 missing\end{small} 
\newline Autograph manuscript
\newline 1.1.1  mandora 1  A  
\begin{filecontents*}{394-1.code}
@clef:g-2
@keysig:xFCG
@timesig:3/4
@data:{8.6'AA}/4AE{8.6EE}/4EC{8,A'C}/C,BBA{B'D}/4C-{8.6AA}/4AG{8xD,B}/
\end{filecontents*}
\commandline{ verovio --spacing-non-linear=0.50 -w 1500 --spacing-system=0.5 --adjust-page-height -b 0 394-1.code }
\newline
\includesvg[width=220pt]{394-1}%

\newline RISM-ID: 455011102
\newline Der Band mit der mandora 2 befindet sich in D Eu mit der Signatur Esl VIII 403
\newline D-Fh  Ms 002
\newline $\rightarrow$ In collection 419 (455008345)

\newline \par \vspace{7pt} \textcolor{darkblue}{\textbf{Zink, Joseph Michael  1758-1829}}\hfillplus{395}
\newline Minuets  A  
\newline mandora (2)
\newline \begin{itshape}[f.35v, at left:] Menue | [left before the accolade:] 68.\end{itshape} 
\newline \textcolor{darkblue}{\ding{\numexpr181 + 01}}  1 parts  
\newline \begin{small} mandora 2 missing\end{small} 
\newline Autograph manuscript
\newline 1.1.1  mandora 1  A  
\begin{filecontents*}{395-1.code}
@clef:g-2
@keysig:xFCG
@timesig:3/4
@data:{8'AE}/4CC{8EC}/4,AA{8AB}/4'C{8C,B}4'C/2D{8,B'C}/4DD{8DE}/4FF{8D,B}/
\end{filecontents*}
\commandline{ verovio --spacing-non-linear=0.50 -w 1500 --spacing-system=0.5 --adjust-page-height -b 0 395-1.code }
\newline
\includesvg[width=220pt]{395-1}%

\newline RISM-ID: 455011107
\newline Der Band mit der mandora 2 befindet sich in D Eu mit der Signatur Esl VIII 403
\newline D-Fh  Ms 002
\newline $\rightarrow$ In collection 419 (455008345)

\newline \par \vspace{7pt} \textcolor{darkblue}{\textbf{Zink, Joseph Michael  1758-1829}}\hfillplus{396}
\newline Mir ist so wohl in deiner Nähe  G  
\newline V, mandora
\newline \begin{itshape}[at left, f.43v:] V: Zink:\end{itshape} 
\newline \textcolor{darkblue}{\ding{\numexpr181 + 01}}  1 parts  
\newline Autograph manuscript
\newline 1.1.1  V  G
\newline \begin{footnotesize} Mir ist so wohl in deiner Nähe \end{footnotesize}  
\begin{filecontents*}{396-1.code}
@clef:G-2
@keysig:xF
@timesig:6/8
@data:8''D/48D'BB''D/i/{8CD}C{'B''C}'B/4A8''D4D6-'A/4A8A{AB}x''C/48DEF'A/
\end{filecontents*}
\commandline{ verovio --spacing-non-linear=0.50 -w 1500 --spacing-system=0.5 --adjust-page-height -b 0 396-1.code }
\newline
\includesvg[width=220pt]{396-1}%

\newline RISM-ID: 455011119
\newline Drei Strophen
\newline Der Schreiber ist der selbe wie der 70 Divertimenti in der gleichen Handschrift
\newline D-Fh  Ms 002
\newline $\rightarrow$ In collection 419 (455008345)

\newline \par \vspace{7pt} \textcolor{darkblue}{\textbf{Zink, Joseph Michael  1758-1829}}\hfillplus{397}
\newline Mir ist so wohl in deiner Nähe  E  
\newline V, mandora
\newline \begin{itshape}[without title]\end{itshape} 
\newline \textcolor{darkblue}{\ding{\numexpr181 + 01}}  1 parts  
\newline Manuscript copy
\newline Zink, Joseph Michael
\newline 1.1.1  V  E
\newline \begin{footnotesize} Mir ist so wohl in deiner Nähe \end{footnotesize}  
\begin{filecontents*}{397-1.code}
@clef:G-2
@keysig:xFCGD
@timesig:c/
@data:8'B/4.B8B''EDC'B/4BA-{8BA}/4.G{6AB}8''CDEF/qE4D--8-'B/4.B8B''EDCx'B/
\end{filecontents*}
\commandline{ verovio --spacing-non-linear=0.50 -w 1500 --spacing-system=0.5 --adjust-page-height -b 0 397-1.code }
\newline
\includesvg[width=220pt]{397-1}%

\newline RISM-ID: 455011123
\newline Eine Strophe
\newline Der Schreiber ist der selbe wie der 70 Divertimenti in der gleichen Handschrift
\newline D-Fh  Ms 002
\newline $\rightarrow$ In collection 419 (455008345)

\newline \par \vspace{7pt} \textcolor{darkblue}{\textbf{Zink, Joseph Michael  1758-1829}}\hfillplus{398}
\newline Overtures  G  
\newline mandora (2)
\newline \begin{itshape}[f.2v, mandora 1, heading:] Ouverture | [left before the accolade:] Nro 1\end{itshape} 
\newline \textcolor{darkblue}{\ding{\numexpr181 + 01}}  1 parts  
\newline Autograph manuscript
\newline 1.1.1  mandora 1  G  
\begin{filecontents*}{398-1.code}
@clef:g-2
@keysig:xF
@timesig:c
@data:4'GGGG/4,,G-8-{,GAB}/4'CC8DEDC/{C,B}4B8-{B'C}D/4EE{8EG}{FE}/4xCD4.G8,B/
\end{filecontents*}
\commandline{ verovio --spacing-non-linear=0.50 -w 1500 --spacing-system=0.5 --adjust-page-height -b 0 398-1.code }
\newline
\includesvg[width=220pt]{398-1}%

\newline RISM-ID: 455011040
\newline D-Fh  Ms 002
\newline $\rightarrow$ In collection 419 (455008345)

\newline \par \vspace{7pt} \textcolor{darkblue}{\textbf{Zink, Joseph Michael  1758-1829}}\hfillplus{399}
\newline Overtures  A  
\newline mandora (2)
\newline \begin{itshape}[f.21v, at left:] Ouverture | Allegro | [left before the accolade:] 37.\end{itshape} 
\newline \textcolor{darkblue}{\ding{\numexpr181 + 01}}  1 parts  
\newline \begin{small} mandora 2 missing\end{small} 
\newline Autograph manuscript
\newline 1.1.1  mandora 1  A  
\begin{filecontents*}{399-1.code}
@clef:g-2
@keysig:xFCG
@timesig:c
@data:4.8,AEAE/4A'CE-/{8DC}{D,B}{AG}{FE}/4A'CE-/{8DC}{D,B}{AG}{FE}/
\end{filecontents*}
\commandline{ verovio --spacing-non-linear=0.50 -w 1500 --spacing-system=0.5 --adjust-page-height -b 0 399-1.code }
\newline
\includesvg[width=220pt]{399-1}%

\newline RISM-ID: 455011076
\newline Der Band mit der mandora 2 befindet sich in D Eu mit der Signatur Esl VIII 403
\newline D-Fh  Ms 002
\newline $\rightarrow$ In collection 419 (455008345)

\newline \par \vspace{7pt} \textcolor{darkblue}{\textbf{Zink, Joseph Michael  1758-1829}}\hfillplus{400}
\newline Polonaises  G  
\newline mandora (2)
\newline \begin{itshape}[f.13v, at left:] Polonese | [left before the accolade:] 19.\end{itshape} 
\newline \textcolor{darkblue}{\ding{\numexpr181 + 01}}  1 parts  
\newline \begin{small} mandora 2 missing\end{small} 
\newline Autograph manuscript
\newline 1.1.1  mandora 1  G  
\begin{filecontents*}{400-1.code}
@clef:g-2
@keysig:xF
@timesig:2/4
@data:!866{,BBG}{'CC,A}/!8,,B'DD-/866{CC,A}{BBG}/8D'FF-/fDD6DEFG/
\end{filecontents*}
\commandline{ verovio --spacing-non-linear=0.50 -w 1500 --spacing-system=0.5 --adjust-page-height -b 0 400-1.code }
\newline
\includesvg[width=220pt]{400-1}%

\newline RISM-ID: 455011058
\newline Der Band mit der mandora 2 befindet sich in D Eu mit der Signatur Esl VIII 403
\newline D-Fh  Ms 002
\newline $\rightarrow$ In collection 419 (455008345)

\newline \par \vspace{7pt} \textcolor{darkblue}{\textbf{Zink, Joseph Michael  1758-1829}}\hfillplus{401}
\newline Polonaises  G  
\newline mandora (2)
\newline \begin{itshape}[f.18r, at left:] Polonese | [left before the accolade:] 30.\end{itshape} 
\newline \textcolor{darkblue}{\ding{\numexpr181 + 01}}  1 parts  
\newline \begin{small} mandora 2 missing\end{small} 
\newline Autograph manuscript
\newline 1.1.1  mandora 1  G  
\begin{filecontents*}{401-1.code}
@clef:g-2
@keysig:xF
@timesig:2/4
@data:6'DED,B'CDC,A/8BBB-/6'CDC,AB'C,BG/8AAA-/866{BBG}{'CC,A}/8'DD{6DE}{FG}/
\end{filecontents*}
\commandline{ verovio --spacing-non-linear=0.50 -w 1500 --spacing-system=0.5 --adjust-page-height -b 0 401-1.code }
\newline
\includesvg[width=220pt]{401-1}%

\newline RISM-ID: 455011069
\newline Der Band mit der mandora 2 befindet sich in D Eu mit der Signatur Esl VIII 403
\newline D-Fh  Ms 002
\newline $\rightarrow$ In collection 419 (455008345)

\newline \par \vspace{7pt} \textcolor{darkblue}{\textbf{Zink, Joseph Michael  1758-1829}}\hfillplus{402}
\newline Polonaises  G  
\newline mandora (2)
\newline \begin{itshape}[f.20r, at left:] Polonese | [left before the accolade:] 34.\end{itshape} 
\newline \textcolor{darkblue}{\ding{\numexpr181 + 01}}  1 parts  
\newline \begin{small} mandora 2 missing\end{small} 
\newline Autograph manuscript
\newline 1.1.1  mandora 1  G  
\begin{filecontents*}{402-1.code}
@clef:g-2
@keysig:xF
@timesig:3/4
@data:{8,B6B'C}8DD4D/{8,A6AA}{8'CC}4C/{8.6,B'C}8DD6EDC,B/4BA-/
\end{filecontents*}
\commandline{ verovio --spacing-non-linear=0.50 -w 1500 --spacing-system=0.5 --adjust-page-height -b 0 402-1.code }
\newline
\includesvg[width=220pt]{402-1}%

\newline RISM-ID: 455011073
\newline Der Band mit der mandora 2 befindet sich in D Eu mit der Signatur Esl VIII 403
\newline D-Fh  Ms 002
\newline $\rightarrow$ In collection 419 (455008345)

\newline \par \vspace{7pt} \textcolor{darkblue}{\textbf{Zink, Joseph Michael  1758-1829}}\hfillplus{403}
\newline Polonaises  A  
\newline mandora (2)
\newline \begin{itshape}[f.24v, at left:] Polonese | [left before the accolade:] 42.\end{itshape} 
\newline \textcolor{darkblue}{\ding{\numexpr181 + 01}}  1 parts  
\newline \begin{small} mandora 2 missing\end{small} 
\newline Autograph manuscript
\newline 1.1.1  mandora 1  A  
\begin{filecontents*}{403-1.code}
@clef:g-2
@keysig:xFCG
@timesig:3/4
@data:{6,AGAB}{'C,B}{'CD}{8EE}/EAEA4E/{8D6DC}8D,B'C,A/'E
\end{filecontents*}
\commandline{ verovio --spacing-non-linear=0.50 -w 1500 --spacing-system=0.5 --adjust-page-height -b 0 403-1.code }
\newline
\includesvg[width=220pt]{403-1}%

\newline RISM-ID: 455011081
\newline Der Band mit der mandora 2 befindet sich in D Eu mit der Signatur Esl VIII 403
\newline D-Fh  Ms 002
\newline $\rightarrow$ In collection 419 (455008345)

\newline \par \vspace{7pt} \textcolor{darkblue}{\textbf{Zink, Joseph Michael  1758-1829}}\hfillplus{404}
\newline Polonaises  A  
\newline mandora (2)
\newline \begin{itshape}[f.26v, at left:] Polonese | [left before the accolade:] 50.\end{itshape} 
\newline \textcolor{darkblue}{\ding{\numexpr181 + 01}}  1 parts  
\newline \begin{small} mandora 2 missing\end{small} 
\newline Autograph manuscript
\newline 1.1.1  mandora 1  A  
\begin{filecontents*}{404-1.code}
@clef:g-2
@keysig:xFCG
@timesig:3/4
@data:{866'ExDE}{C,B'C}4,A/i/8{B'C}{D,B}{'C,A}/{'CD}{xDE}4,B/
\end{filecontents*}
\commandline{ verovio --spacing-non-linear=0.50 -w 1500 --spacing-system=0.5 --adjust-page-height -b 0 404-1.code }
\newline
\includesvg[width=220pt]{404-1}%

\newline RISM-ID: 455011089
\newline Der Band mit der mandora 2 befindet sich in D Eu mit der Signatur Esl VIII 403
\newline D-Fh  Ms 002
\newline $\rightarrow$ In collection 419 (455008345)

\newline \par \vspace{7pt} \textcolor{darkblue}{\textbf{Zink, Joseph Michael  1758-1829}}\hfillplus{405}
\newline Presto  G  
\newline mandora (2)
\newline \begin{itshape}[f.14r, at left:] Presto | [left before the accolade:] 20.\end{itshape} 
\newline \textcolor{darkblue}{\ding{\numexpr181 + 01}}  1 parts  
\newline \begin{small} mandora 2 missing\end{small} 
\newline Autograph manuscript
\newline 1.1.1  mandora 1  G  
\begin{filecontents*}{405-1.code}
@clef:g-2
@keysig:xF
@timesig:6/8
@data:8,D/,,G,B'DD,B'D/,DA'CC,B,,G/{'GFE}{DC,B}/'C,BA4G8-/
\end{filecontents*}
\commandline{ verovio --spacing-non-linear=0.50 -w 1500 --spacing-system=0.5 --adjust-page-height -b 0 405-1.code }
\newline
\includesvg[width=220pt]{405-1}%

\newline RISM-ID: 455011059
\newline Der Band mit der mandora 2 befindet sich in D Eu mit der Signatur Esl VIII 403
\newline D-Fh  Ms 002
\newline $\rightarrow$ In collection 419 (455008345)

\newline \par \vspace{7pt} \textcolor{darkblue}{\textbf{Zink, Joseph Michael  1758-1829}}\hfillplus{406}
\newline Quodlibets    
\newline V, mandora
\newline \begin{itshape}[heading, f.48v:] Quodlibet\end{itshape} 
\newline \textcolor{darkblue}{\ding{\numexpr181 + 01}}  1 parts  
\newline Manuscript copy
\newline Zink, Joseph Michael
\newline 1.1.1  V  C
\newline \begin{footnotesize} Hört liebe Mädchen ich sag' euch geschwind \end{footnotesize}  
\begin{filecontents*}{406-1.code}
@clef:G-2
@keysig:
@timesig:2/4
@data:488'CEF/GEC/AB''C/'G-G/q8G4F8FF/qF488EEE/DEE//
\end{filecontents*}
\commandline{ verovio --spacing-non-linear=0.50 -w 1500 --spacing-system=0.5 --adjust-page-height -b 0 406-1.code }
\newline
\includesvg[width=220pt]{406-1}%

\newline RISM-ID: 455011124
\newline Der Schreiber ist der selbe wie der 70 Divertimenti in der gleichen Handschrift
\newline D-Fh  Ms 002
\newline $\rightarrow$ In collection 419 (455008345)

\newline \par \vspace{7pt} \textcolor{darkblue}{\textbf{Zink, Joseph Michael  1758-1829}}\hfillplus{407}
\newline Quodlibets    
\newline V, mandora
\newline \begin{itshape}[heading, f.51v:] Quodlibet\end{itshape} 
\newline \textcolor{darkblue}{\ding{\numexpr181 + 01}}  1 parts  
\newline Manuscript copy
\newline Zink, Joseph Michael
\newline 1.1.1  V  G
\newline \begin{footnotesize} Zu nachts wenn alles schlafen liegt \end{footnotesize}  
\begin{filecontents*}{407-1.code}
@clef:G-2
@keysig:xF
@timesig:2/4
@data:8'D/!GAB''C/DD'A6-''D/8'AA!{A6.3''DC}/8'BAG6-D/f8A6-''D/
\end{filecontents*}
\commandline{ verovio --spacing-non-linear=0.50 -w 1500 --spacing-system=0.5 --adjust-page-height -b 0 407-1.code }
\newline
\includesvg[width=220pt]{407-1}%

\newline RISM-ID: 455011125
\newline Der Schreiber ist der selbe wie der 70 Divertimenti in der gleichen Handschrift
\newline D-Fh  Ms 002
\newline $\rightarrow$ In collection 419 (455008345)

\newline \par \vspace{7pt} \textcolor{darkblue}{\textbf{Zink, Joseph Michael  1758-1829}}\hfillplus{408}
\newline Romances  A  
\newline mandora (2)
\newline \begin{itshape}[f.29v, at left:] Romance | Amoroso | [left before the accolade:] 56.\end{itshape} 
\newline \textcolor{darkblue}{\ding{\numexpr181 + 01}}  1 parts  
\newline \begin{small} mandora 2 missing\end{small} 
\newline Autograph manuscript
\newline 1.1.1  mandora 1  A  
\begin{filecontents*}{408-1.code}
@clef:g-2
@keysig:xFCG
@timesig:6/8
@data:4'C8CED,B/A'CEEDC/4,B8B'CDE/4,B8BB'CD/4C8CED,B/A'CEAEC/
\end{filecontents*}
\commandline{ verovio --spacing-non-linear=0.50 -w 1500 --spacing-system=0.5 --adjust-page-height -b 0 408-1.code }
\newline
\includesvg[width=220pt]{408-1}%

\newline RISM-ID: 455011095
\newline Der Band mit der mandora 2 befindet sich in D Eu mit der Signatur Esl VIII 403
\newline D-Fh  Ms 002
\newline $\rightarrow$ In collection 419 (455008345)

\newline \par \vspace{7pt} \textcolor{darkblue}{\textbf{Zink, Joseph Michael  1758-1829}}\hfillplus{409}
\newline Scherzando  G  
\newline mandora (2)
\newline \begin{itshape}[f.6v, mandora 1, at left:] Scherzando | [left before the accolade:] 3.\end{itshape} 
\newline \textcolor{darkblue}{\ding{\numexpr181 + 01}}  1 parts  
\newline Autograph manuscript
\newline 1.1.1  mandora 1  G  
\begin{filecontents*}{409-1.code}
@clef:g-2
@keysig:xF
@timesig:2/4
@data:8'D/,B'D,B'D/4D8-,G/AB'CD/4C8,B'D/,B'D,B'D/
\end{filecontents*}
\commandline{ verovio --spacing-non-linear=0.50 -w 1500 --spacing-system=0.5 --adjust-page-height -b 0 409-1.code }
\newline
\includesvg[width=220pt]{409-1}%

\newline RISM-ID: 455011042
\newline D-Fh  Ms 002
\newline $\rightarrow$ In collection 419 (455008345)

\newline \par \vspace{7pt} \textcolor{darkblue}{\textbf{Zink, Joseph Michael  1758-1829}}\hfillplus{410}
\newline Scherzando  G  
\newline mandora (2)
\newline \begin{itshape}[f.15r, at left:] Scherzando | [left before the accolade:] 22.\end{itshape} 
\newline \textcolor{darkblue}{\ding{\numexpr181 + 01}}  1 parts  
\newline \begin{small} mandora 2 missing\end{small} 
\newline Autograph manuscript
\newline 1.1.1  mandora 1  G  
\begin{filecontents*}{410-1.code}
@clef:g-2
@keysig:xF
@timesig:2/4
@data:{6,B'C}/!8D,B'GC/4D!8,B'D/CC,BB/4A8-{6B'C}/f{8GG}/
\end{filecontents*}
\commandline{ verovio --spacing-non-linear=0.50 -w 1500 --spacing-system=0.5 --adjust-page-height -b 0 410-1.code }
\newline
\includesvg[width=220pt]{410-1}%

\newline RISM-ID: 455011061
\newline Der Band mit der mandora 2 befindet sich in D Eu mit der Signatur Esl VIII 403
\newline D-Fh  Ms 002
\newline $\rightarrow$ In collection 419 (455008345)

\newline \par \vspace{7pt} \textcolor{darkblue}{\textbf{Zink, Joseph Michael  1758-1829}}\hfillplus{411}
\newline Scherzando  G  
\newline mandora (2)
\newline \begin{itshape}[f.17v, at left:] Scherzando | [left before the accolade:] 29.\end{itshape} 
\newline \textcolor{darkblue}{\ding{\numexpr181 + 01}}  1 parts  
\newline \begin{small} mandora 2 missing\end{small} 
\newline Autograph manuscript
\newline 1.1.1  mandora 1  G  
\begin{filecontents*}{411-1.code}
@clef:g-2
@keysig:xF
@timesig:6/8
@data:!8,G'CD{D,B}G/!A'CDDC,A/fDA'F4,D8-/fA'CDDC,A/
\end{filecontents*}
\commandline{ verovio --spacing-non-linear=0.50 -w 1500 --spacing-system=0.5 --adjust-page-height -b 0 411-1.code }
\newline
\includesvg[width=220pt]{411-1}%

\newline RISM-ID: 455011068
\newline Der Band mit der mandora 2 befindet sich in D Eu mit der Signatur Esl VIII 403
\newline D-Fh  Ms 002
\newline $\rightarrow$ In collection 419 (455008345)

\newline \par \vspace{7pt} \textcolor{darkblue}{\textbf{Zink, Joseph Michael  1758-1829}}\hfillplus{412}
\newline Scherzando  D  
\newline mandora (2)
\newline \begin{itshape}[f.29r, at left:] Scherzando | [left before the accolade:] 55.\end{itshape} 
\newline \textcolor{darkblue}{\ding{\numexpr181 + 01}}  1 parts  
\newline \begin{small} mandora 2 missing\end{small} 
\newline Autograph manuscript
\newline 1.1.1  mandora 1  D  
\begin{filecontents*}{412-1.code}
@clef:g-2
@keysig:xFC
@timesig:2/4
@data:8,A/!'DFEC/{8.6DF}8AA/DFEC/4D8-!,A/f'E/
\end{filecontents*}
\commandline{ verovio --spacing-non-linear=0.50 -w 1500 --spacing-system=0.5 --adjust-page-height -b 0 412-1.code }
\newline
\includesvg[width=220pt]{412-1}%

\newline RISM-ID: 455011094
\newline Der Band mit der mandora 2 befindet sich in D Eu mit der Signatur Esl VIII 403
\newline D-Fh  Ms 002
\newline $\rightarrow$ In collection 419 (455008345)

\newline \par \vspace{7pt} \textcolor{darkblue}{\textbf{Zink, Joseph Michael  1758-1829}}\hfillplus{413}
\newline Scherzando  D  
\newline mandora (2)
\newline \begin{itshape}[f.31r, at left:] Gratioso | Scherzando | [left before the accolade:] 59.\end{itshape} 
\newline \textcolor{darkblue}{\ding{\numexpr181 + 01}}  1 parts  
\newline \begin{small} mandora 2 missing\end{small} 
\newline Autograph manuscript
\newline 1.1.1  mandora 1  D  
\begin{filecontents*}{413-1.code}
@clef:g-2
@keysig:xFC
@timesig:3/8
@data:8'A/FDA/FDF/{G6GFGE}/4F8A/FDA/FDF/{E6EF8E}/{F6FG8F}/
\end{filecontents*}
\commandline{ verovio --spacing-non-linear=0.50 -w 1500 --spacing-system=0.5 --adjust-page-height -b 0 413-1.code }
\newline
\includesvg[width=220pt]{413-1}%

\newline RISM-ID: 455011098
\newline Der Band mit der mandora 2 befindet sich in D Eu mit der Signatur Esl VIII 403
\newline D-Fh  Ms 002
\newline $\rightarrow$ In collection 419 (455008345)

\newline \par \vspace{7pt} \textcolor{darkblue}{\textbf{Zink, Joseph Michael  1758-1829}}\hfillplus{414}
\newline Scherzando  A  
\newline mandora (2)
\newline \begin{itshape}[f.34v, at left:] Scherzando | [left before the accolade:] 66.\end{itshape} 
\newline \textcolor{darkblue}{\ding{\numexpr181 + 01}}  1 parts  
\newline \begin{small} mandora 2 missing\end{small} 
\newline Autograph manuscript
\newline 1.1.1  mandora 1  A  
\begin{filecontents*}{414-1.code}
@clef:g-2
@keysig:xFCG
@timesig:6/8
@data:8'E/AECD,BG/'CDE4E8C/ED,B'DC,A/,,A'CE4,,A8A/'AECD,B'D/
\end{filecontents*}
\commandline{ verovio --spacing-non-linear=0.50 -w 1500 --spacing-system=0.5 --adjust-page-height -b 0 414-1.code }
\newline
\includesvg[width=220pt]{414-1}%

\newline RISM-ID: 455011105
\newline Der Band mit der mandora 2 befindet sich in D Eu mit der Signatur Esl VIII 403
\newline D-Fh  Ms 002
\newline $\rightarrow$ In collection 419 (455008345)

\newline \par \vspace{7pt} \textcolor{darkblue}{\textbf{Zink, Joseph Michael  1758-1829}}\hfillplus{415}
\newline Scherzando  A  
\newline mandora (2)
\newline \begin{itshape}[f.36v, at left:] Scherzando | [left before the accolade:] 70.\end{itshape} 
\newline \textcolor{darkblue}{\ding{\numexpr181 + 01}}  1 parts  
\newline \begin{small} mandora 2 missing\end{small} 
\newline Autograph manuscript
\newline 1.1.1  mandora 1  A  
\begin{filecontents*}{415-1.code}
@clef:g-2
@keysig:xFCG
@timesig:2/4
@data:8'C/CDDE/4E{8CC}/8DD,BB/{6'C,B'CD}{8CC}/CDDE/4E{8CC}/DD,BB/4'C8-://:
\end{filecontents*}
\commandline{ verovio --spacing-non-linear=0.50 -w 1500 --spacing-system=0.5 --adjust-page-height -b 0 415-1.code }
\newline
\includesvg[width=220pt]{415-1}%

\newline RISM-ID: 455011109
\newline Der Band mit der mandora 2 befindet sich in D Eu mit der Signatur Esl VIII 403
\newline D-Fh  Ms 002
\newline $\rightarrow$ In collection 419 (455008345)

\newline \par \vspace{7pt} \textcolor{darkblue}{\textbf{Zink, Joseph Michael  1758-1829}}\hfillplus{416}
\newline Scherzando allegretto  A  
\newline mandora (2)
\newline \begin{itshape}[f.24r, at left:] Scherzando | Allegretto | [left before the accolade:] 41.\end{itshape} 
\newline \textcolor{darkblue}{\ding{\numexpr181 + 01}}  1 parts  
\newline \begin{small} mandora 2 missing\end{small} 
\newline Autograph manuscript
\newline 1.1.1  mandora 1  A  
\begin{filecontents*}{416-1.code}
@clef:g-2
@keysig:xFCG
@timesig:2/4
@data:!8,,A'CCD/,,A'E,,A'C/!,,E'D,,E,B/{6'C,B'CD}8C-/f
\end{filecontents*}
\commandline{ verovio --spacing-non-linear=0.50 -w 1500 --spacing-system=0.5 --adjust-page-height -b 0 416-1.code }
\newline
\includesvg[width=220pt]{416-1}%

\newline RISM-ID: 455011080
\newline Der Band mit der mandora 2 befindet sich in D Eu mit der Signatur Esl VIII 403
\newline D-Fh  Ms 002
\newline $\rightarrow$ In collection 419 (455008345)

\newline \par \vspace{7pt} \textcolor{darkblue}{\textbf{Zink, Joseph Michael  1758-1829}}\hfillplus{417}
\newline Siciliano  G  
\newline mandora (2)
\newline \begin{itshape}[f.16r, at left:] Siciliano | [left before the accolade:] 24.\end{itshape} 
\newline \textcolor{darkblue}{\ding{\numexpr181 + 01}}  1 parts  
\newline \begin{small} mandora 2 missing\end{small} 
\newline Autograph manuscript
\newline 1.1.1  mandora 1  G  
\begin{filecontents*}{417-1.code}
@clef:g-2
@keysig:xF
@timesig:6/8
@data:8'G/!GDDD,BB/!'CDE4D8G/fC,BA4B8-/
\end{filecontents*}
\commandline{ verovio --spacing-non-linear=0.50 -w 1500 --spacing-system=0.5 --adjust-page-height -b 0 417-1.code }
\newline
\includesvg[width=220pt]{417-1}%

\newline RISM-ID: 455011063
\newline Der Band mit der mandora 2 befindet sich in D Eu mit der Signatur Esl VIII 403
\newline D-Fh  Ms 002
\newline $\rightarrow$ In collection 419 (455008345)

\newline \par \vspace{7pt} \textcolor{darkblue}{\textbf{Collection}}\hfillplus{418}
\newline 239 Pieces    
\newline \begin{itshape}[title page:] Georgius Michael | Oster nd. mag. ind. | in Harraß. | 1684 d. 17. Jun.\end{itshape} 
\newline \textcolor{darkblue}{\ding{\numexpr181 + 01}}  1684-1713 (1684-1713)  score: 127f.  16 x 20 cm
\newline Watermark: Wasserzeichen nicht lesbar
\newline Manuscript copy
\newline Oster, Georg Michael
\newline RISM-ID: 455008343
\newline 93f. größtenteils Klaviertabulaturen, dann 2f. eingeklebte theoretische Anweisungen, danach mit p."64" beginnend 28f. (durchgängig, teilweise paginiert und teilweise foliiert von 64-80) mit Stücken V und begleitender Baßstimme und einigen Klavierstücken danach p."81-86" paginiertes Register
\newline Bis f.12v sind die Stücke größtenteils durchnummeriert, ab f.21r sind zunächst die jeweils gegenüberliegenden Seiten mit der gleichen Blattzahl versehen bis sich die Zählung ab ca. f.40 aufweicht und zur Stückezählung tendiert
\newline Aus einem Briefwechsel vom Bach-Archiv Leipzig, Peter Wollny an Herrn Peter Cahn aus dem Jahr 1995 geht hervor, dass neben dem aus Harras stammenden Oster, Komponisten wie Käfer und Johann Effler enthalten sind
\newline f.74v und 75r: "Praeludium / Johann Effleri"
\newline f.43v "G. M Oster" und f.79v, unten "Georg Michael Oster."
\newline f.86v: "Fuga à 4 voc. / Christoph Klemse / ex C." von Christoph Clemsee
\newline f.91v: "Variatio c[on]tra punct / S. Scheid" über den vorhergehenden Choral "Ei du feiner Reiter"
\newline f.100r: am Ende des Stückes "1713" als Datierung, so dass zumindest die Stücke in dem Umkreis ab f.96r, die auch keine Tabulaturen mehr sind um dieses Datum anzusetzen sind
\newline f.109r mit der Paginierung "85" ist leer
\newline Auf dem Vorsatzblatt mit blauer Kreide. "Ki."
\newline D-Fh  Ms 001
\newline \par \vspace{7pt} \textcolor{darkblue}{\textbf{Collection}}\hfillplus{419}
\newline 95 Instrumental pieces    
\newline \begin{itshape}[title page:] LXXII | Divertimente | a | Due Mandore | Mandora I | [at the tail, at right:] di Zinck\end{itshape} 
\newline \textcolor{darkblue}{\ding{\numexpr181 + 01}}  1700-1749 (18.1d)  score: 85f.  22 x 30,5 cm
\newline Watermark: [KBM 3 D HR 16]
\newline Manuscript copy
\newline RISM-ID: 455008345
\newline Nur die drei ersten Divertimenti sind für zwei Mandoren, alle anderen für die Mandora Prima, daneben noch zahlreiche Liedabschriften, aber teilweise von anderer Hand, die ersten 70 Divertimenti und später noch vereinzelte Stücke stehen in Französischer Lautentabulatur
\newline Die Mandora Seconda befindet sich in Eichstätt unter der Signatur Esl VIII, 403
\newline Der Band ist in Pergament gebunden, auf dessen Innenseite des Vorderdeckels mit Bleistift "Barone de Prodmann / née Barone de Pechmann"
\newline Vorsatzblatt, dann f.1-35 mit den Divertimenti, wobei das Ende der Divertimenti fehlt, da Blätter herausgerissen sind, danach 27f. mit Lieder sowohl mit Klavier als auch Gitarrenbegleitung und danach 20f. unbeschrieben und die folgenden letzten 3f. mit theretischen Aufzeichnungen belegt
\newline Alle Titel, bzw. Nummern der Divertimenti und die Schluß oder Zeilenanfangsstriche sowie Textelemente wie "volti" wurden mit roter Tinte geschrieben
\newline Auf dem Titelblatt recto unten mit blauer Kreide: "Div."
\newline Pechmann, Nanette B.  (fmo)
\newline Zink  (oth)
\newline Literature: MeyerS 1991  deest
\newline D-Fh  Ms 002
    \clearpage  
\chapter*{\centering Index of personal names}
\addcontentsline{toc}{chapter}{Index of personal names}
\fancyhead{}
\fancyhead[C]{\small Répertoire International des Sources Musicales}


\newline 
Alday, F. ..... 1

\newline 
Anonymus ..... 2, 3, 4, 5, 6, 7, 8, 9, 10, 11, 12, 13, 14, 15, 16, 17, 18, 19, 20, 21, 22, 23, 24, 25, 26, 27, 28, 29, 30, 31, 32, 33, 34, 35, 36, 37, 38, 39, 40, 41, 42, 43, 44, 45, 46, 47, 48, 49, 50, 51, 52, 53, 54, 55, 56, 57, 58, 59, 60, 61, 62, 63, 64, 65, 66, 67, 68, 69, 70, 71, 72, 73, 74, 75, 76, 77, 78, 79, 80, 81, 82, 83, 84, 85, 86, 87, 88, 89, 90, 91, 92, 93, 94, 95, 96, 97, 98, 99, 100, 101, 102, 103, 104, 105, 106, 107, 108, 109, 110, 111, 112, 113, 114, 115, 116, 117, 118, 119, 120, 121, 122, 123, 124, 125, 126, 127, 128, 129, 130, 131, 132, 133, 134, 135, 136, 137, 138, 139, 140, 141, 142, 143, 144, 145, 146, 147, 148, 149, 150, 151, 152, 153, 154, 155, 156, 157, 158, 159, 160, 161, 162, 163, 164, 165, 166, 167, 168, 169, 170, 171, 172, 173, 174, 175, 176, 177, 178, 179, 180, 181, 182, 183, 184, 185, 186, 187, 188, 189, 190, 191, 192, 193, 194, 195, 196, 197, 198, 199, 200, 201, 202, 203, 204, 205, 206, 207, 208, 209, 210, 211, 212, 213, 214, 215, 216, 217, 218, 219, 220, 221, 222, 223, 224, 225, 226, 227, 228, 229, 230, 231, 232, 233, 234, 235, 236, 237, 238, 239, 240

\newline 
Bach, Carl Philipp Emanuel ..... 241

\newline 
Bachofen, Johann Caspar ..... 242

\newline 
Blasius, Matthieu-Frédéric ..... 243

\newline 
Boccherini, Luigi ..... 244, 245

\newline 
Böddecker, Philipp Friedrich ..... 246, 247, 248

\newline 
Bretzner, Christoph Friedrich ..... 69

\newline 
Bruni, Antonio Bartolomeo ..... 249, 250, 251

\newline 
Bunte, Johann Friedrich ..... 252, 253

\newline 
Call, Leonhard von ..... 254, 255

\newline 
Cambini, Giuseppe Maria ..... 256

\newline 
Campagnoli, Bartolomeo ..... 257, 258, 259, 260, 261

\newline 
Clemsee, Christoph ..... 262

\newline 
Cupis, Jean-Baptiste ..... 263

\newline 
Danzi, Franz ..... 264, 265

\newline 
Effler, Johann ..... 266, 267, 268

\newline 
Fastre, J. ..... 269

\newline 
Fodor, Josephus ..... 270, 271, 272, 273, 274

\newline 
Franck, Melchior ..... 275

\newline 
Gallus, Iacobus ..... 276

\newline 
Garbet, Mlle ..... 250

\newline 
Gebauer, Michel-Joseph ..... 277, 278

\newline 
Gleim, Johann Wilhelm Ludwig ..... 65

\newline 
Grasset, Jean-Jacques ..... 279

\newline 
Gravrand, Joseph ..... 280, 281

\newline 
Guénin, Marie-Alexandre ..... 282

\newline 
Haydn, Joseph ..... 283, 284, 285, 286, 287

\newline 
Heinrich, Herzog ..... 193

\newline 
Hoffmann, Heinrich Anton ..... 288

\newline 
Hoffmeister, Franz Anton ..... 289, 290, 291

\newline 
Joannes ..... 317

\newline 
Käfer, Johann Philipp ..... 292

\newline 
Keller, Karl ..... 293

\newline 
Kotzebue, August von ..... 336

\newline 
Kreutzer, Paul ..... 294

\newline 
Krommer, Franz ..... 295, 296, 297, 298, 299, 300, 301

\newline 
Lepreux, Mlle ..... 306

\newline 
Lobry); (Sampierdaréna ..... 283

\newline 
Mack ..... 308

\newline 
Mestrino, Nicola ..... 302

\newline 
Mozart, Wolfgang Amadeus ..... 303, 304

\newline 
Mussini, Natale Nicolo ..... 305, 306

\newline 
Naumann, Johann Gottlieb ..... 307

\newline 
Ollivier, Mme ..... 274

\newline 
Oster, Georg Michael ..... 95, 160

\newline 
Paër, Ferdinando ..... 308

\newline 
Pechmann, Nanette B. ..... 309, 310, 311, 419

\newline 
Pleyel, Ignace ..... 312, 313, 314

\newline 
Poessinger, Franz Alexander ..... 315, 316

\newline 
Pujolas, J. ..... 317

\newline 
Raimondi, Ignazio ..... 318

\newline 
Rammler, Karl Wilhelm ..... 241

\newline 
Ribière ..... 305

\newline 
Richomme); (Aubert, I. ..... 244

\newline 
Rolla, Alessandro ..... 319, 320, 321, 322

\newline 
Romberg, Andreas Jakob ..... 323

\newline 
Romberg, Bernhard Heinrich ..... 324, 325

\newline 
Scheidt, Samuel ..... 326

\newline 
Schenk, Johann Baptist ..... 327

\newline 
Schönebeck, Karl Sigmund ..... 328

\newline 
Schubert, Johann Friedrich ..... 329

\newline 
Stamitz, Carl ..... 330

\newline 
Št'astný, František Jan ..... 331, 332, 333

\newline 
Sterkel, Johann Franz Xaver ..... 334

\newline 
Telemann, Georg Philipp ..... 335

\newline 
Veltheim, Baron von ..... 336

\newline 
Viotti, Giovanni Battista ..... 337, 338, 339, 340

\newline 
Weidmann, Paul ..... 327

\newline 
Winter, Peter von ..... 341

\newline 
Woldemar, Michel ..... 342

\newline 
Zink ..... 419

\newline 
Zink, Joseph Michael ..... 343, 344, 345, 346, 347, 348, 349, 350, 351, 352, 353, 354, 355, 356, 357, 358, 359, 360, 361, 362, 363, 364, 365, 366, 367, 368, 369, 370, 371, 372, 373, 374, 375, 376, 377, 378, 379, 380, 381, 382, 383, 384, 385, 386, 387, 388, 389, 390, 391, 392, 393, 394, 395, 396, 397, 398, 399, 400, 401, 402, 403, 404, 405, 406, 407, 408, 409, 410, 411, 412, 413, 414, 415, 416, 417
    \clearpage  
\chapter*{\centering Index of title and text}
\addcontentsline{toc}{chapter}{Index of title and text}
\fancyhead{}
\fancyhead[C]{\small Répertoire International des Sources Musicales}


\newline 
239 Pieces ..... 418

\newline 
95 Instrumental pieces ..... 419

\newline 
Ach Gott tu dich erbarmen ..... 2

\newline 
Ach Gott und Herr wie groß und schwer ..... 3, 4

\newline 
Ach Gott vom Himmel sieh darein ..... 5

\newline 
Ach Herr mich armen Sünder ..... 7, 8

\newline 
Ach Herr mich armen Sünder. Excerpts ..... 6

\newline 
Ach Jesu Christ dich zu uns wend' ..... 9

\newline 
Ach Jesu mein wie große Pein ..... 10

\newline 
Ach bleib bei uns Herr Jesu Christ ..... 11

\newline 
Ach du meines Lebens Lust ..... 12

\newline 
Ach hier vor der Linde die uns lieb oft Schatten gab ..... 310

\newline 
Ach ich liebte war so glücklich ..... 69

\newline 
Ach mein Herr Jesu willst du mich ganz und gar verlassen ..... 13

\newline 
Ach mein Herr Jesu willst du mich ganz und gar verlassen. Arr ..... 13

\newline 
Ach mein herzliebes Jesulein ..... 14

\newline 
[Ach mein herzliebes Jesulein] ..... 14

\newline 
Ach was ist doch unser Leben ..... 15

\newline 
Ach wie sehnlich wart' ich der Zeit ..... 16

\newline 
Ach wie sehnlich wart' ich der Zeit. Excerpts ..... 16

\newline 
Ade du süße Welt ..... 17

\newline 
Ade du süße Welt. Arr ..... 17

\newline 
Airs ..... 19, 20, 21, 22, 23, 24, 292

\newline 
Airs variés ..... 263, 342

\newline 
Airs. Arr ..... 18

\newline 
Alle Menschen müssen sterben ..... 25

\newline 
Allegretto ..... 343, 344, 345

\newline 
Allegro ..... 346, 347

\newline 
Allegro polacca ..... 348

\newline 
Allegro tedesco ..... 349

\newline 
Allein Gott in der Höh sei Ehr ..... 26, 27, 28

\newline 
Allein zu dir Herr Jesu Christ ..... 29

\newline 
Allemandes ..... 30, 31, 350, 351, 352, 353

\newline 
Als ich noch im Flügelkleide ..... 304

\newline 
Amoroso ..... 354, 355

\newline 
An Minna ..... 334

\newline 
An Wasserflüssen Babylon ..... 32

\newline 
An die Geliebte ..... 33

\newline 
Andante amoroso ..... 356, 357

\newline 
Andante gratioso ..... 358

\newline 
Andantino gratioso ..... 359

\newline 
Ännchen von Tharau ..... 34

\newline 
Ännchen von Tharau ist's die mir gefällt ..... 34

\newline 
Ans Liebchen ..... 309

\newline 
Auf meinen lieben Gott ..... 35

\newline 
Aus tiefer Not schrei ich zu dir ..... 36

\newline 
Ballets ..... 37, 38, 39, 40, 246, 247, 248

\newline 
Battaglias. Arr ..... 41

\newline 
Betrübtes Herz sei wohlgemut ..... 42

\newline 
Betrügliche Welt ..... 43

\newline 
Betrügliche Welt. Excerpts. Arr ..... 43

\newline 
Bourrées ..... 44

\newline 
Burlesques ..... 360, 361

\newline 
Chorale arrangements ..... 45

\newline 
Christ der du bist der helle Tag ..... 46

\newline 
[Christ ist erstanden] ..... 47

\newline 
Christ ist erstanden. Excerpts ..... 47

\newline 
Christ lag in Todesbanden ..... 48

\newline 
Christe der du bist Tag und Licht ..... 49

\newline 
Christus der uns selig macht ..... 50

\newline 
Christus ist mein Leben ..... 51

\newline 
Concertos ..... 303

\newline 
Courantes ..... 52, 53, 54

\newline 
Da Jesus an dem Kreuze stund ..... 55, 56

\newline 
Da kam ich auf den Boden ei ei ei ..... 57

\newline 
Dances ..... 58

\newline 
Dank sagen wir alle Gott unserm Herrn ..... 59

\newline 
Dank sagen wir alle Gott unserm Herrn. Arr ..... 59

\newline 
Das Hindu-Mädchen ..... 60

\newline 
Das Jesulein soll doch mein Trost ..... 61

\newline 
Das Jesulein soll doch mein Trost. Arr ..... 61

\newline 
Das Wörtchen Du ..... 62

\newline 
Der Bräut'gam wird bald rufen ..... 63

\newline 
Der Dorfbarbier. Excerpts. Arr ..... 327

\newline 
Der Guckkasten ..... 293

\newline 
Der Herr ist mein getreuer Hirt ..... 64

\newline 
Der Mann ..... 65

\newline 
Der Tag der ist so freudenreich ..... 66

\newline 
Des heil'gen Geistes reiche Gnad' ..... 67

\newline 
Des heil'gen Geistes reiche Gnad'. Excerpts ..... 68

\newline 
Deutsche Tänze ..... 362, 363, 364, 365, 366, 367, 368, 369, 370, 371, 372, 373

\newline 
Die Klage ..... 69

\newline 
Die Linde ..... 310

\newline 
Die Sonne sinkt der Abend winkt ..... 334

\newline 
Die betrunkene Welt ..... 70

\newline 
Divertimentos ..... 315

\newline 
Don Giovanni. Excerpts. Arr ..... 304

\newline 
Dort oben auf der Alm ..... 71

\newline 
Dreher ..... 374

\newline 
Du Friedefürst Herr Jesu Christ ..... 72

\newline 
Du bist aller Dinge Schöne meine Freundin ..... 73

\newline 
Du bist aller Dinge Schöne meine Freundin. Arr ..... 73

\newline 
Du bist der rechte David Herr ..... 74

\newline 
Duets (instr.) ..... 1, 243, 249, 250, 251, 252, 254, 256, 257, 258, 259, 260, 264, 269, 270, 271, 272, 273, 277, 278, 279, 280, 281, 282, 288, 289, 295, 302, 305, 306, 307, 312, 317, 319, 320, 321, 323, 324, 328, 329, 330, 331, 337, 338

\newline 
Durch Adams Fall ist ganz verderbt ..... 75

\newline 
Ei du feiner Reiter ..... 326

\newline 
Ein' feste Burg ist unser Gott ..... 76

\newline 
Einen guten Kampf hab' ich ..... 77

\newline 
Einst sprach mein Herr der Bader ..... 327

\newline 
Erbarm dich mein o Herre Gott ..... 78

\newline 
Erhalt uns Herr bei deinem Wort ..... 79

\newline 
Ermuntert euch ihr müden Seelen ..... 80

\newline 
Ermuntre dich mein schwacher Geist ..... 81

\newline 
Erschienen ist der herrlich' Tag ..... 82, 83

\newline 
Erstanden ist der heilig' Christ ..... 275

\newline 
Es ist das Heil uns kommen her ..... 84

\newline 
Es ist gewisslich an der Zeit ..... 85, 86, 87

\newline 
Es spricht der Unweisen Mund wohl ..... 88, 89

\newline 
Es wollt' und Gott genädig sein ..... 90

\newline 
Exercises (instr.) ..... 322

\newline 
Fanfares. Arr ..... 91

\newline 
Finale ..... 375

\newline 
Frohlockt ..... 92

\newline 
Fugues ..... 93, 94, 95, 96, 97, 98, 99, 262

\newline 
Galliards ..... 100

\newline 
Gelobet seist du Jesu Christ ..... 101

\newline 
Gewonnen der Satanas lieget ..... 102

\newline 
Gewonnen der Satanas lieget. Arr ..... 102

\newline 
Gigues ..... 103

\newline 
Glori Lob Ehr' und Herrlichkeit ..... 276

\newline 
Gott Vater der du deine Sonn' ..... 104

\newline 
Gott der Vater wohn uns bei ..... 105, 106

\newline 
Gott geb der Braut und Bräutigam ..... 107

\newline 
Gott geb der Braut und Bräutigam. Arr ..... 107

\newline 
Gott hat das Evangelium ..... 108

\newline 
Gott ist mein Heil mein' Hilf' und Trost ..... 109

\newline 
Gott lobet noch ..... 110

\newline 
Gott sei gelobet und gebenedeiet ..... 111, 112

\newline 
Gottes Sohn ist kommen ..... 113

\newline 
Gratioso ..... 376

\newline 
Harlequino ..... 377

\newline 
Helft mir Gottes Güte preisen ..... 114

\newline 
Herbei ihr Laute kommt zuhauf ..... 293

\newline 
Herr Gott dich loben alle wir ..... 115, 116

\newline 
Herr Gott dich loben wir ..... 117, 118

\newline 
Herr Gott dich loben wir. Arr ..... 118

\newline 
Herr Gott du bist von Ewigkeit ..... 119

\newline 
Herr Gott nun schleuß den Himmel auf ..... 120

\newline 
Herr Jesu Christ wahr' Mensch und Gott ..... 121

\newline 
Herr Jesu Lebenssonne ..... 122

\newline 
Herr Jesu Lebenssonne. Excerpts. Arr ..... 122

\newline 
Herr hader' mit den Had'rern mein ..... 123

\newline 
Herr vergib' uns unsre Sünde ..... 124

\newline 
Herr von uns nimme deinen Zorn und Grimme ..... 125

\newline 
Herr von uns nimme deinen Zorn und Grimme. Arr ..... 125

\newline 
Hilf Gott daß mir's gelinge ..... 126

\newline 
Hilf Jesu daß wir allzumal ..... 127

\newline 
Hört liebe Mädchen ich sag' euch geschwind ..... 406

\newline 
Ich hab' mein' Sach' Gott heimgestellt ..... 128

\newline 
Ich steh' an deiner Krippen hier ..... 129

\newline 
Ich steh' an deiner Krippen hier. Arr ..... 129

\newline 
Ich weiß dass mein Erlöser lebt ..... 130

\newline 
Ihr hohen Berg ihr lehret mich ..... 131

\newline 
Ihr hohen Berg' ihr lehret mich. Arr ..... 131

\newline 
In dich hab' ich gehoffet Herr ..... 132, 133, 134

\newline 
In dich hab' ich gehoffet Herr. Excerpts. Arr ..... 133

\newline 
Instrumental pieces ..... 332, 333

\newline 
Ist denn Liebe ein Verbrechen ..... 33

\newline 
Jesu der du meine Seele ..... 135, 136

\newline 
Jesu du mein liebstes Leben ..... 137

\newline 
Jesu du mein liebstes Leben. Arr ..... 137

\newline 
Jesu komm doch selbst zu mir ..... 138

\newline 
Jesu wo soll ich dich finden ..... 139

\newline 
Jesulein du bist mein ..... 140

\newline 
Jesus Christus unser Heiland ..... 141

\newline 
Jesus Christus unser Heiland der von uns ..... 142

\newline 
Keinen hat Gott verlassen ..... 143

\newline 
Keyboard pieces ..... 144, 145

\newline 
Komm Gott Schöpfer heiliger Geist ..... 146

\newline 
Komm Heiliger Geist Herre Gott ..... 147

\newline 
Komm Heiliger Geist Herre Gott. Excerpts. Arr ..... 147

\newline 
Komm fein Liebchen komm ans Fenster ..... 336

\newline 
Komm ich soeben zum Wirtshaus heraus ..... 70

\newline 
Kommt her zu mir ..... 148

\newline 
Kyrie Gott Vater in Ewigkeit ..... 149

\newline 
La Passione di Gesù Cristo ..... 308

\newline 
Laus et perennis gloria ..... 276

\newline 
Laus et perennis gloria, Deo patri ..... 276

\newline 
Lebt jemand so wie ich ..... 150

\newline 
Liebeserklärung ..... 151

\newline 
Lied eines Schäfers ..... 311

\newline 
Lob sei dem allmächtigen Gott ..... 152, 153

\newline 
Lobet den Herren alle ..... 154

\newline 
Macht auf die Tor der Gerechtigkeit ..... 155, 156

\newline 
Macht auf die Tor der Gerechtigkeit. Arr ..... 156

\newline 
Mädchen sind wie der Wind ..... 157

\newline 
Mag ich Unglück nicht widerstehn ..... 158

\newline 
Marches ..... 378, 379, 380, 381, 382

\newline 
Masses. Excerpts ..... 159

\newline 
Mein Freund komme in seinen Garten ..... 160

\newline 
Mein Freund komme in seinen Garten. Arr ..... 160

\newline 
Mein Herz ruht und ist stille ..... 161

\newline 
Mein Herz ruht und ist stille. Excerpts. Arr ..... 161

\newline 
Mein' Seel' sich freu' und lustig sei ..... 162

\newline 
Mein' Wallfahrt ich vollendet hab' ..... 163

\newline 
Mich soll die Liebe nicht kränken ..... 311

\newline 
Minuets ..... 164, 165, 166, 383, 384, 385, 386, 387, 388, 389, 390, 391, 392, 393, 394, 395

\newline 
Mir ist ein geistlich Kirchelein ..... 167

\newline 
Mir ist so wohl in deiner Nähe ..... 396, 397

\newline 
Mit Fried und Freud ich fahr dahin ..... 168

\newline 
Mitten wir im Leben sind ..... 169

\newline 
Nice se più non m'ami ..... 170

\newline 
Nun Gott Lob es ist vollbracht ..... 171

\newline 
Nun Gott Lob es ist vollbracht. Arr ..... 171

\newline 
Nun bitten wir den Heiligen Geist ..... 172

\newline 
Nun freut euch Gottes Kinder all ..... 173

\newline 
Nun giebet der Höchste ..... 174

\newline 
Nun komm der Heiden Heiland ..... 175

\newline 
Nun laßt uns Gott den Herren ..... 176, 177

\newline 
Nun laßt uns Gott den Herrn ..... 178

\newline 
Nun lob mein' Seel' den Herrn ..... 179

\newline 
Nun sich der Tag geendet hat ..... 180

\newline 
Nun sich der Tag geendet hat. Arr ..... 180

\newline 
O Angst und Leid o Traurigkeit ..... 181

\newline 
O Ewigkeit du Donnerwort ..... 182

\newline 
O Ewigkeit du Donnerwort. Arr ..... 182

\newline 
O Gott du höchster Gnadenhort ..... 183, 184

\newline 
O Gott du höchster Gnadenhort. Arr ..... 183

\newline 
O Heiliger Geist o heiliger Gott ..... 185

\newline 
O Jesu du edle Gabe ..... 186

\newline 
O Jesu du edle Gabe. Arr ..... 186

\newline 
O Jesu du getreuer Hirt ..... 187

\newline 
O Lamm Gottes unschuldig ..... 188, 189

\newline 
O Lamm Gottes unschuldig. Arr ..... 189

\newline 
O Vater allmächtiger Gott ..... 190

\newline 
O angoscia ..... 308

\newline 
O duolo ..... 308

\newline 
O wie selig seid ihr doch ihr Frommen ..... 191

\newline 
O wie selig seid ihr doch ihr Frommen. Arr ..... 191

\newline 
Oh lutto della terra e del cielo ..... 308

\newline 
Operas ..... 341

\newline 
Oratorios ..... 241

\newline 
Overtures ..... 398, 399

\newline 
Pallido esangue ..... 308

\newline 
Polonaises ..... 400, 401, 402, 403, 404

\newline 
Preludes ..... 192, 193, 194, 195, 196, 197, 198, 199, 266, 267

\newline 
Presto ..... 405

\newline 
Quartets (instr.) ..... 255, 274, 283, 284, 296, 297

\newline 
Quintets (instr.) ..... 244, 298, 299, 300, 301, 313, 314

\newline 
Quodlibets ..... 406, 407

\newline 
Reich mit des Orients Segen beladen ..... 60

\newline 
Romances ..... 408

\newline 
Sacred songs ..... 242

\newline 
Sacrimarti ombre adorate ..... 308

\newline 
Sag was hilft alle Welt ..... 200, 201, 202

\newline 
Sag was hilft alle Welt. Arr ..... 202

\newline 
Sarabandes ..... 204, 205

\newline 
Sarabandes. Arr ..... 203

\newline 
Scherzando ..... 409, 410, 411, 412, 413, 414, 415

\newline 
Scherzando allegretto ..... 416

\newline 
Schmücket das Fest mit Maien ..... 206

\newline 
Selig sind die in Christo sterben ..... 207

\newline 
Siciliano ..... 417

\newline 
Sie ging zum Sonntagstanze ..... 151

\newline 
Siehst du jene Rose blühen ..... 309

\newline 
Singen wir aus Herzensgrund ..... 208

\newline 
So wünsch' ich dir du Eitelkeit ..... 209

\newline 
Sollt' es gleich bisweilen scheinen ..... 210, 211

\newline 
Sollt' es gleich bisweilen scheinen. Arr ..... 211

\newline 
Sonatas ..... 261, 265

\newline 
Ständchen ..... 336

\newline 
Suites. Excerpts. Arr ..... 335

\newline 
Symphonies ..... 285, 286

\newline 
Toccatas ..... 212, 268

\newline 
Trios (instr.) ..... 245, 253, 287, 290, 291, 316, 318, 325, 339, 340

\newline 
Valet will ich dir geben ..... 213

\newline 
Variations ..... 294

\newline 
Vater unser im Himmelreich ..... 214, 215

\newline 
Vom Himmel hoch da komm' ich her ..... 216

\newline 
Von Gott will ich nicht lassen ..... 217, 218

\newline 
Von Herzen Grund ergeben ..... 219

\newline 
Von Herzen Grund ergeben. Arr ..... 219

\newline 
Wär' Gott nicht mit uns diese Zeit ..... 220

\newline 
Was Gott tut das ist wohlgetan ..... 221

\newline 
Was fürchst du Feind Herodes sehr ..... 222

\newline 
Welt packe dich ..... 223

\newline 
Weltlich' Ehr' und zeitlich' Gut ..... 224

\newline 
Wenn dich Unglück tut greifen an ..... 225

\newline 
Wenn mein Stündlein vorhanden ist ..... 226

\newline 
Wer Gott das Herze giebet ..... 227

\newline 
Wer in dem Schutz des Höchsten ist ..... 228

\newline 
Wie kommt es daß in Liebessachen das Wörtchen Du so süße klingt ..... 62

\newline 
Wie schön leuchtet der Morgenstern ..... 229, 230, 231, 232

\newline 
Wie's Gott gefällt ..... 233

\newline 
Wir armen Sünder unser Missetat ..... 234

\newline 
Wir glauben all' an einen Gott ..... 235

\newline 
Wo Gott zum Haus nicht gibt sein' Gunst ..... 236, 237

\newline 
Wo bist du mein Jesu wo bleibst du so lange ..... 238

\newline 
Wo soll ich fliehen hin ..... 239

\newline 
Wohl dem Manne dessen Herz ..... 65

\newline 
Wohlauf mein ganzes Ich ..... 240

\newline 
Zu nachts wenn alles schlafen liegt ..... 407
\end{document}