
\documentclass[a4paper, twocolumn, 11pt]{book}
%\usepackage[papersize={21.0cm, 29.7cm},left=3.5cm,right=2.75cm,top=3.3cm,bottom=2.5cm]{geometry}
\usepackage{fontspec}
\usepackage[utf8]{luainputenc}
\usepackage{newunicodechar}
\setmainfont[Ligatures=TeX]{Linux Libertine O}
\usepackage{ifluatex}
\ifluatex
  \usepackage{pdftexcmds}
  \makeatletter
  \let\pdfstrcmp\pdf@strcmp
  \let\pdffilemoddate\pdf@filemoddate
  \makeatother
\fi
\usepackage{textcomp}
\usepackage{xparse}
\ExplSyntaxOn
\NewDocumentCommand{\commandline}{v}
  { \immediate \write 18 { \tl_to_str:n {#1}  }  }
\ExplSyntaxOff
\usepackage{graphicx}
\usepackage[inkscape={/usr/bin/inkscape -z -C }]{svg}
\usepackage{import}
\usepackage{pifont}
\usepackage{filecontents}
\usepackage[greek,ngerman]{babel}
\usepackage{color} 
\usepackage{fancyhdr}
\usepackage[defaultlines=4, all]{nowidow}
\definecolor{darkblue}{rgb}{0, 0,.4}
\usepackage[colorlinks=true, urlcolor=darkblue, linkcolor=blue]{hyperref}
\newcommand*\chancery{\fontfamily{pzc}\selectfont}
\urlstyle{same}
\parindent0cm
\setlength{\parskip}{0em}
\setlength{\columnsep}{30 pt}
\clubpenalty = 10000
\widowpenalty = 10000
\displaywidowpenalty = 10000
\tolerance=500
\let\mypdfximage\pdfximage
\protected\def\pdfximage{\immediate\mypdfximage}
\pagestyle{fancy}
\begin{titlepage}
\title{RISM Musikquellen \\ 
\vspace{10 mm} \large Kunstsammlungen der Veste Coburg
}
\author{\copyright \ RISM}
\date{\today}
\end{titlepage}
\begin{document}
\maketitle
\thispagestyle{empty}
\twocolumn[
\begin{@twocolumnfalse}
  \vspace*{250px}
  \begin{center}
    This document is licensed under the Creative Commons Attribution - ShareAlike 3.0 License. \\
    To view a copy of this license visit: \\
    \url{http://creativecommons.org/licenses/by-sa/3.0/legalcode}.
  \end{center}
\end{@twocolumnfalse}]
\renewcommand*\contentsname{\hfill Inhaltsverzeichnis \hfill}
\tableofcontents
\thispagestyle{empty}
\newcommand\hfillplus[1]{{\unskip\nobreak\hfill\penalty50\
  \mbox{}\nobreak\hfill#1}}

\chapter*{\centering Katalog der Musikquellen}
\addcontentsline{toc}{chapter}{Katalog der Musikquellen}
\fancyhead{}
\fancyhead[C]{\small RISM -\ Kunstsammlungen der Veste Coburg}
\setlength{\columnseprule}{0.5pt}

\par \vspace{16pt} \textcolor{darkblue}{\textbf{Albert, Eugen d'  1864-1932}}\hfillplus{[1]}\newline Klavierstück. Auswahl - d-Moll\newline pf
\par \begin{itshape}[caption title:] Seiner Excellenz | Herrn Baron von Pawel-Rammingen\end{itshape} 
\par \textcolor{darkblue}{\ding{\numexpr181 + 01}}  Stimme: pf  (1f.); 12,5 x 20,5 cm\newline \begin{small} Quer gefaltetes Doppelblatt, nur Vorderseite beschrieben\end{small} \newline Autograph  18870902-18890902
\par 1.1.1  pf, \begin{itshape}Tempo di [?]\end{itshape}, d-Moll  
\begin{filecontents*}{1-1.code}
@clef:G-2
@keysig:bB
@timesig:4/4
@data:{8DEFD}/4BBB{8AG}/{FE}4D8{FGAF}/''4DDD{8C'B}/{AG}4F{8B''CD'A}/
\end{filecontents*}
\commandline{ if [ ! -f 1-1.svg ]; then verovio --spacing-non-linear=0.54 -w 1500 --spacing-system=0.5 --adjust-page-height -b 0 1-1.code; fi } 
\newline \includesvg[width=209pt]{1-1}%
\par Das Notenblatt enthält nur 4 Takte (+Auftakt), eine Akkolade mit freihändig gezogenen Notenlinien.
\par Unter den Noten: {\textquotedbl}Zur Erinnerung an | Eugen d'Albert | Coburg. 2 Sept. 1889.{\textquotedbl} (Im Katalog korrigiert in 1887 und Zusatz {\textquotedbl}Kammerherr{\textquotedbl}).\newline Pawel-Rammingen, Baron von  (Widmungsträger)
\par RISM-ID: 450109872\newline D-Cv  A.V,1131,(4),1
\par \vspace{16pt} \textcolor{darkblue}{\textbf{Anonymus  }}\hfillplus{[2]}\newline Albumblatt - C-Dur\newline fl
\par \begin{itshape}[heading:] Flauto\end{itshape} 
\par \textcolor{darkblue}{\ding{\numexpr181 + 01}}  Stimme: fl  (1f.); 31 x 24,5 cm\newline Abschrift\newline Schreiber: Aicher, M.
\par 1.1.1  fl, C-Dur  
\begin{filecontents*}{2-1.code}
@clef:G-2
@keysig:
@timesig:3/4
@data:4-8-6''{GA}{bB'''CDE}/{8FC}{6CDC''bB}{8A6AB}/
\end{filecontents*}
\commandline{ if [ ! -f 2-1.svg ]; then verovio --spacing-non-linear=0.54 -w 1500 --spacing-system=0.5 --adjust-page-height -b 0 2-1.code; fi } 
\newline \includesvg[width=209pt]{2-1}%
\par 2 Notenzeilen, darunter: M. Aicher | Prof. der Flöte
\par Jeder Instrumentalist hat auf seinem Albumblatt ein kurzes Motiv für sein jeweiliges Instrument notiert. Es ist nicht klar, ob diese Motive alle aus dem {\textquotedbl}Divertissement{\textquotedbl} von Czerny oder aus der Oper {\textquotedbl}Casilda{\textquotedbl} stammen, oder ob sie von den betreffenden Musikern frei erfunden sind.
\par Comment on scoring: no indication\newline Ernst II., Herzog von Sachsen-Coburg und Gotha  (Widmungsträger)
\par RISM-ID: 450109897\newline D-Cv  A.V.1202,(4)\newline $\rightarrow$ In Sammlung 78 (450109893)
      
\par \vspace{16pt} \textcolor{darkblue}{\textbf{Anonymus  }}\hfillplus{[3]}\newline Albumblatt - F-Dur\newline ob
\par \begin{itshape}[heading:] Oboe\end{itshape} 
\par \textcolor{darkblue}{\ding{\numexpr181 + 01}}  Stimme: ob  (1f.); 31 x 24,5 cm\newline Abschrift\newline Schreiber: Baumberg, J. C.
\par 1.1.1  ob, F-Dur  
\begin{filecontents*}{3-1.code}
@clef:G-2
@keysig:bB
@timesig:c
@data:4-2G4nBtqq{6AB}r/4''C2Cqq{6DC'nB}r{8''CD}/
\end{filecontents*}
\commandline{ if [ ! -f 3-1.svg ]; then verovio --spacing-non-linear=0.54 -w 1500 --spacing-system=0.5 --adjust-page-height -b 0 3-1.code; fi } 
\newline \includesvg[width=209pt]{3-1}%
\par 2 Notenzeilen, darunter: J. Baumberg | Professor d. Oboe.
\par Jeder Instrumentalist hat auf seinem Albumblatt ein kurzes Motiv für sein jeweiliges Instrument notiert. Es ist nicht klar, ob diese Motive alle aus dem {\textquotedbl}Divertissement{\textquotedbl} von Czerny oder aus der Oper {\textquotedbl}Casilda{\textquotedbl} stammen, oder ob sie von den betreffenden Musikern frei erfunden sind.
\par Comment on scoring: no indication\newline Ernst II., Herzog von Sachsen-Coburg und Gotha  (Widmungsträger)
\par RISM-ID: 450109898\newline D-Cv  A.V.1202,(5)\newline $\rightarrow$ In Sammlung 78 (450109893)
      
\par \vspace{16pt} \textcolor{darkblue}{\textbf{Anonymus  }}\hfillplus{[4]}\newline Albumblatt - B-Dur\newline vl
\par \begin{itshape}[heading:] Violino\end{itshape} 
\par \textcolor{darkblue}{\ding{\numexpr181 + 01}}  Stimme: vl  (1f.); 31 x 24,5 cm\newline Abschrift\newline Schreiber: Beneš, Josef
\par 1.1.1  vl, B-Dur  
\begin{filecontents*}{4-1.code}
@clef:G-2
@keysig:bBE
@timesig:6/8
@data:'''8{CDC}gC{''BAB}/4A8FgG{FEF}/
\end{filecontents*}
\commandline{ if [ ! -f 4-1.svg ]; then verovio --spacing-non-linear=0.54 -w 1500 --spacing-system=0.5 --adjust-page-height -b 0 4-1.code; fi } 
\newline \includesvg[width=209pt]{4-1}%
\par 2 Notenzeilen, darunter: Joseph Benesch | Mitglied der k.k. Hofkapelle | Orchester Director des k.k. | Hofburgtheaters und | Professor der Academie | der Tonkunst.
\par Jeder Instrumentalist hat auf seinem Albumblatt ein kurzes Motiv für sein jeweiliges Instrument notiert. Es ist nicht klar, ob diese Motive alle aus dem {\textquotedbl}Divertissement{\textquotedbl} von Czerny oder aus der Oper {\textquotedbl}Casilda{\textquotedbl} stammen, oder ob sie von den betreffenden Musikern frei erfunden sind.
\par Comment on scoring: no indication\newline Ernst II., Herzog von Sachsen-Coburg und Gotha  (Widmungsträger)
\par RISM-ID: 450109899\newline D-Cv  A.V.1202,(6)\newline $\rightarrow$ In Sammlung 78 (450109893)
      
\par \vspace{16pt} \textcolor{darkblue}{\textbf{Anonymus  }}\hfillplus{[5]}\newline Albumblatt - F-Dur\newline pf
\par \begin{itshape}[heading:] Piano Forte. I.\end{itshape} 
\par \textcolor{darkblue}{\ding{\numexpr181 + 01}}  Stimme: pf 1  (1f.); 31 x 24,5 cm\newline Abschrift\newline Schreiber: Bocklet, Carl Maria von
\par 1.1.1  pf 1, F-Dur  
\begin{filecontents*}{5-1.code}
@clef:G-2
@keysig:bB
@timesig:c
@data:'''2F'{8..F3A}{''CFA'''CFA''''C}/8F-4-2-/
\end{filecontents*}
\commandline{ if [ ! -f 5-1.svg ]; then verovio --spacing-non-linear=0.54 -w 1500 --spacing-system=0.5 --adjust-page-height -b 0 5-1.code; fi } 
\newline \includesvg[width=209pt]{5-1}%
\par 1 Akkolade mit 2 Notensystemen (nur 2 Takte), darunter: Karl Edler von | Bocklet | Professor der | Akademie der Tonkunst | in Wien.
\par Jeder Instrumentalist hat auf seinem Albumblatt ein kurzes Motiv für sein jeweiliges Instrument notiert. Es ist nicht klar, ob diese Motive alle aus dem {\textquotedbl}Divertissement{\textquotedbl} von Czerny oder aus der Oper {\textquotedbl}Casilda{\textquotedbl} stammen, oder ob sie von den betreffenden Musikern frei erfunden sind.
\par Comment on scoring: no indication\newline Ernst II., Herzog von Sachsen-Coburg und Gotha  (Widmungsträger)
\par RISM-ID: 450109900\newline D-Cv  A.V.1202,(7)\newline $\rightarrow$ In Sammlung 78 (450109893)
      
\par \vspace{16pt} \textcolor{darkblue}{\textbf{Anonymus  }}\hfillplus{[6]}\newline Albumblatt - C-Dur\newline pf
\par \begin{itshape}[heading:] Piano II.\end{itshape} 
\par \textcolor{darkblue}{\ding{\numexpr181 + 01}}  Stimme: pf 2  (1f.); 31 x 24,5 cm\newline Abschrift\newline Schreiber: Capponi, Anna
\par 1.1.1  pf, C-Dur  
\begin{filecontents*}{6-1.code}
@clef:G-2
@keysig:
@timesig:c
@data:''(8{CDE})/({FFG})({AB'''C})({D''B'''C})({DbEnE}})/
\end{filecontents*}
\commandline{ if [ ! -f 6-1.svg ]; then verovio --spacing-non-linear=0.54 -w 1500 --spacing-system=0.5 --adjust-page-height -b 0 6-1.code; fi } 
\newline \includesvg[width=209pt]{6-1}%
\par 1 Akkolade mit 2 Notensystemen (nur 1 Takt mit Auftakt), darunter: Anna Capponi | Professorin des Pianof: | an der Akademie der | Tonkunst in Wien.
\par Jeder Instrumentalist hat auf seinem Albumblatt ein kurzes Motiv für sein jeweiliges Instrument notiert. Es ist nicht klar, ob diese Motive alle aus dem {\textquotedbl}Divertissement{\textquotedbl} von Czerny oder aus der Oper {\textquotedbl}Casilda{\textquotedbl} stammen, oder ob sie von den betreffenden Musikern frei erfunden sind.
\par Comment on scoring: no indication\newline Ernst II., Herzog von Sachsen-Coburg und Gotha  (Widmungsträger)
\par RISM-ID: 450109901\newline D-Cv  A.V.1202,(8)\newline $\rightarrow$ In Sammlung 78 (450109893)
      
\par \vspace{16pt} \textcolor{darkblue}{\textbf{Anonymus  }}\hfillplus{[7]}\newline Albumblatt - C-Dur\newline cor
\par \begin{itshape}[heading:] Corno.\end{itshape} 
\par \textcolor{darkblue}{\ding{\numexpr181 + 01}}  Stimme: cor  (1f.); 31 x 24,5 cm\newline Abschrift\newline Schreiber: Chott, Franz
\par 1.1.1  cor, C-Dur  
\begin{filecontents*}{7-1.code}
@clef:G-2
@keysig:
@timesig:c
@data:{8G8.A6G}/''4.C'8B{AB''CF}/2F+{8FE'B''C}/
\end{filecontents*}
\commandline{ if [ ! -f 7-1.svg ]; then verovio --spacing-non-linear=0.54 -w 1500 --spacing-system=0.5 --adjust-page-height -b 0 7-1.code; fi } 
\newline \includesvg[width=209pt]{7-1}%
\par 2 Notensysteme, darunter: Franz Chott. | Prof. des Horns.
\par Jeder Instrumentalist hat auf seinem Albumblatt ein kurzes Motiv für sein jeweiliges Instrument notiert. Es ist nicht klar, ob diese Motive alle aus dem {\textquotedbl}Divertissement{\textquotedbl} von Czerny oder aus der Oper {\textquotedbl}Casilda{\textquotedbl} stammen, oder ob sie von den betreffenden Musikern frei erfunden sind.
\par Comment on scoring: no indication\newline Ernst II., Herzog von Sachsen-Coburg und Gotha  (Widmungsträger)
\par RISM-ID: 450109902\newline D-Cv  A.V.1203,(1)\newline $\rightarrow$ In Sammlung 78 (450109893)
      
\par \vspace{16pt} \textcolor{darkblue}{\textbf{Anonymus  }}\hfillplus{[8]}\newline Albumblatt - Es-Dur\newline vlc
\par \begin{itshape}[heading:] Violoncello.\end{itshape} 
\par \textcolor{darkblue}{\ding{\numexpr181 + 01}}  Stimme: vlc  (1f.); 31 x 24,5 cm\newline Abschrift\newline Schreiber: Hartinger, Josef
\par 1.1.1  vlc, Es-Dur  
\begin{filecontents*}{8-1.code}
@clef:F-4
@keysig:bBEA
@timesig:3/4
@data:,{8.G6A}/2B4B/%C-4 2'Etqq{6CD}r{8.G6F}/
\end{filecontents*}
\commandline{ if [ ! -f 8-1.svg ]; then verovio --spacing-non-linear=0.54 -w 1500 --spacing-system=0.5 --adjust-page-height -b 0 8-1.code; fi } 
\newline \includesvg[width=209pt]{8-1}%
\par 2 Notensysteme, darunter: Josef Hartinger. | Rofesor[!] der Akademie | Mitglied der k.k. Hofkapele | und des k.k. Hofopern Theaters
\par Jeder Instrumentalist hat auf seinem Albumblatt ein kurzes Motiv für sein jeweiliges Instrument notiert. Es ist nicht klar, ob diese Motive alle aus dem {\textquotedbl}Divertissement{\textquotedbl} von Czerny oder aus der Oper {\textquotedbl}Casilda{\textquotedbl} stammen, oder ob sie von den betreffenden Musikern frei erfunden sind.
\par Comment on scoring: no indication\newline Ernst II., Herzog von Sachsen-Coburg und Gotha  (Widmungsträger)
\par RISM-ID: 450109903\newline D-Cv  A.V.1203,(2)\newline $\rightarrow$ In Sammlung 78 (450109893)
      
\par \vspace{16pt} \textcolor{darkblue}{\textbf{Anonymus  }}\hfillplus{[9]}\newline Albumblatt - C-Dur\newline timp
\par \begin{itshape}[heading:] Pauke\end{itshape} 
\par \textcolor{darkblue}{\ding{\numexpr181 + 01}}  Stimme: timp  (1f.); 31 x 24,5 cm\newline Abschrift\newline Schreiber: Hudler, Anton
\par 1.1.1  timp, C-Dur  
\begin{filecontents*}{9-1.code}
@clef:F-4
@keysig:
@timesig:6/8
@data:48,C---/C-C-/{8CCC}{CCC}/2.Ct/,,4F8-4-8-/
\end{filecontents*}
\commandline{ if [ ! -f 9-1.svg ]; then verovio --spacing-non-linear=0.54 -w 1500 --spacing-system=0.5 --adjust-page-height -b 0 9-1.code; fi } 
\newline \includesvg[width=209pt]{9-1}%
\par 2 Notensysteme, darunter: Anton Hudler | Profesor der Akademie.
\par Jeder Instrumentalist hat auf seinem Albumblatt ein kurzes Motiv für sein jeweiliges Instrument notiert. Es ist nicht klar, ob diese Motive alle aus dem {\textquotedbl}Divertissement{\textquotedbl} von Czerny oder aus der Oper {\textquotedbl}Casilda{\textquotedbl} stammen, oder ob sie von den betreffenden Musikern frei erfunden sind.
\par Comment on scoring: no indication\newline Ernst II., Herzog von Sachsen-Coburg und Gotha  (Widmungsträger)
\par RISM-ID: 450109904\newline D-Cv  A.V.1203,(3)\newline $\rightarrow$ In Sammlung 78 (450109893)
      
\par \vspace{16pt} \textcolor{darkblue}{\textbf{Anonymus  }}\hfillplus{[10]}\newline Albumblatt - F-Dur\newline cb
\par \begin{itshape}[heading:] Contrabass.\end{itshape} 
\par \textcolor{darkblue}{\ding{\numexpr181 + 01}}  Stimme: cb  (1f.); 31 x 24,5 cm\newline Abschrift\newline Schreiber: Janausch, Adalbert
\par 1.1.1  cb, \begin{itshape}Allegro\end{itshape}, F-Dur  
\begin{filecontents*}{10-1.code}
@clef:F-4
@keysig:bB
@timesig:c
@data:=4/,4F-F-/,,B,DF-/G-,,G-/
\end{filecontents*}
\commandline{ if [ ! -f 10-1.svg ]; then verovio --spacing-non-linear=0.54 -w 1500 --spacing-system=0.5 --adjust-page-height -b 0 10-1.code; fi } 
\newline \includesvg[width=209pt]{10-1}%
\par 2 Notensysteme, darunter: Adalbert Janausch | Professor der | Akademie der Tonkunst.
\par Jeder Instrumentalist hat auf seinem Albumblatt ein kurzes Motiv für sein jeweiliges Instrument notiert. Es ist nicht klar, ob diese Motive alle aus dem {\textquotedbl}Divertissement{\textquotedbl} von Czerny oder aus der Oper {\textquotedbl}Casilda{\textquotedbl} stammen, oder ob sie von den betreffenden Musikern frei erfunden sind.
\par Comment on scoring: no indication\newline Ernst II., Herzog von Sachsen-Coburg und Gotha  (Widmungsträger)
\par RISM-ID: 450109905\newline D-Cv  A.V.1203,(4)\newline $\rightarrow$ In Sammlung 78 (450109893)
      
\par \vspace{16pt} \textcolor{darkblue}{\textbf{Anonymus  }}\hfillplus{[11]}\newline Albumblatt - F-Dur\newline flügelhorn
\par \begin{itshape}[heading:] Flügelhorn in C.\end{itshape} 
\par \textcolor{darkblue}{\ding{\numexpr181 + 01}}  Stimme: flügelhorn  (1f.); 31 x 24,5 cm\newline Abschrift\newline Schreiber: Netrefa, Cölestin
\par 1.1.1  flügelhorn, \begin{itshape}Un poco moderato\end{itshape}, F-Dur  
\begin{filecontents*}{11-1.code}
@clef:G-2
@keysig:bB
@timesig:c
@data:{8.F6F}/4FB''4..D6C/8.6'{BABG}4F8D6-F/
\end{filecontents*}
\commandline{ if [ ! -f 11-1.svg ]; then verovio --spacing-non-linear=0.54 -w 1500 --spacing-system=0.5 --adjust-page-height -b 0 11-1.code; fi } 
\newline \includesvg[width=209pt]{11-1}%
\par 2 Notensysteme, darunter: Cölestin Netrefa | Professor der Akademie | der Tonkunst in Wien.
\par Jeder Instrumentalist hat auf seinem Albumblatt ein kurzes Motiv für sein jeweiliges Instrument notiert. Es ist nicht klar, ob diese Motive alle aus dem {\textquotedbl}Divertissement{\textquotedbl} von Czerny oder aus der Oper {\textquotedbl}Casilda{\textquotedbl} stammen, oder ob sie von den betreffenden Musikern frei erfunden sind.
\par Comment on scoring: no indication\newline Ernst II., Herzog von Sachsen-Coburg und Gotha  (Widmungsträger)
\par RISM-ID: 450109906\newline D-Cv  A.V.1203,(5)\newline $\rightarrow$ In Sammlung 78 (450109893)
      
\par \vspace{16pt} \textcolor{darkblue}{\textbf{Anonymus  }}\hfillplus{[12]}\newline Albumblatt - F-Dur\newline trb
\par \begin{itshape}[heading:] Posaune\end{itshape} 
\par \textcolor{darkblue}{\ding{\numexpr181 + 01}}  Stimme: trb  (1f.); 31 x 24,5 cm\newline Abschrift\newline Schreiber: Seegner, Franz Gregor
\par 1.1.1  trb, F-Dur  
\begin{filecontents*}{12-1.code}
@clef:F-4
@keysig:bB
@timesig:c
@data:,1F+/8F-4-2-/1C+/8C-4-2-/,,1F+/8F-4-2-//
\end{filecontents*}
\commandline{ if [ ! -f 12-1.svg ]; then verovio --spacing-non-linear=0.54 -w 1500 --spacing-system=0.5 --adjust-page-height -b 0 12-1.code; fi } 
\newline \includesvg[width=209pt]{12-1}%
\par 2 Notensysteme, darunter: Franz Gregor Seegner | Professor der Posaune | an der Akademie der | Tonkunst.
\par Jeder Instrumentalist hat auf seinem Albumblatt ein kurzes Motiv für sein jeweiliges Instrument notiert. Es ist nicht klar, ob diese Motive alle aus dem {\textquotedbl}Divertissement{\textquotedbl} von Czerny oder aus der Oper {\textquotedbl}Casilda{\textquotedbl} stammen, oder ob sie von den betreffenden Musikern frei erfunden sind.
\par Comment on scoring: no indication\newline Ernst II., Herzog von Sachsen-Coburg und Gotha  (Widmungsträger)
\par RISM-ID: 450109907\newline D-Cv  A.V.1203,(6)\newline $\rightarrow$ In Sammlung 78 (450109893)
      
\par \vspace{16pt} \textcolor{darkblue}{\textbf{Anonymus  }}\hfillplus{[13]}\newline Albumblatt - F-Dur\newline cl
\par \begin{itshape}[heading:] Clarinetto.\end{itshape} 
\par \textcolor{darkblue}{\ding{\numexpr181 + 01}}  Stimme: cl  (1f.); 31 x 24,5 cm\newline Abschrift\newline Schreiber: Wagner, Wenzel
\par 1.1.1  cl, F-Dur  
\begin{filecontents*}{13-1.code}
@clef:G-2
@keysig:bB
@timesig:3/4
@data:4EG''C/'2E{8.A6B}/2''C4C/{8.F3GF}{8EFAG}/
\end{filecontents*}
\commandline{ if [ ! -f 13-1.svg ]; then verovio --spacing-non-linear=0.54 -w 1500 --spacing-system=0.5 --adjust-page-height -b 0 13-1.code; fi } 
\newline \includesvg[width=209pt]{13-1}%
\par 2 Notensysteme, darunter: Wenzel Wagner | Professor der | Clarinette | in der Akade= | mie der Tonkunst.
\par Jeder Instrumentalist hat auf seinem Albumblatt ein kurzes Motiv für sein jeweiliges Instrument notiert. Es ist nicht klar, ob diese Motive alle aus dem {\textquotedbl}Divertissement{\textquotedbl} von Czerny oder aus der Oper {\textquotedbl}Casilda{\textquotedbl} stammen, oder ob sie von den betreffenden Musikern frei erfunden sind.
\par Comment on scoring: no indication\newline Ernst II., Herzog von Sachsen-Coburg und Gotha  (Widmungsträger)
\par RISM-ID: 450109908\newline D-Cv  A.V.1203,(7)\newline $\rightarrow$ In Sammlung 78 (450109893)
      
\par \vspace{16pt} \textcolor{darkblue}{\textbf{Anonymus  }}\hfillplus{[14]}\newline Albumblatt - F-Dur\newline fag
\par \begin{itshape}[heading:] Fagott.\end{itshape} 
\par \textcolor{darkblue}{\ding{\numexpr181 + 01}}  Stimme: fag  (1f.); 31 x 24,5 cm\newline Abschrift\newline Schreiber: Wittmann, Anton
\par 1.1.1  fag, F-Dur  
\begin{filecontents*}{14-1.code}
@clef:F-4
@keysig:bB
@timesig:c
@data:,,2F,F/4FF2F/4C{8DE}{FGAnB}/'C-C-,F-4-//
\end{filecontents*}
\commandline{ if [ ! -f 14-1.svg ]; then verovio --spacing-non-linear=0.54 -w 1500 --spacing-system=0.5 --adjust-page-height -b 0 14-1.code; fi } 
\newline \includesvg[width=209pt]{14-1}%
\par 2 Notensysteme, darunter: Ant. Wittmann | Prof. der Akademie | der Tonkunst.
\par Jeder Instrumentalist hat auf seinem Albumblatt ein kurzes Motiv für sein jeweiliges Instrument notiert. Es ist nicht klar, ob diese Motive alle aus dem {\textquotedbl}Divertissement{\textquotedbl} von Czerny oder aus der Oper {\textquotedbl}Casilda{\textquotedbl} stammen, oder ob sie von den betreffenden Musikern frei erfunden sind.
\par Comment on scoring: no indication\newline Ernst II., Herzog von Sachsen-Coburg und Gotha  (Widmungsträger)
\par RISM-ID: 450109909\newline D-Cv  A.V.1203,(8)\newline $\rightarrow$ In Sammlung 78 (450109893)
      
\par \vspace{16pt} \textcolor{darkblue}{\textbf{August Emil Leopold, Herzog von Sachsen-Gotha-Altenburg  1772-1822}}\hfillplus{[15]}\newline Beim kindlichen Strahl des erwachenden Phoibos. Arr, J 153 - Es-Dur\newline winds
\par \begin{itshape}[caption title:] Beim Kindlichen Strahl des erwachenden | Phoibos.\end{itshape} 
\par \textcolor{darkblue}{\ding{\numexpr181 + 01}}  Partitur: f.3r-3v\newline Abschrift\newline Schreiber: Weber, Carl Maria von
\par 1.1.1  cor 1, \begin{itshape}Andante con moto\end{itshape}, Es-Dur  
\begin{filecontents*}{15-1.code}
@clef:G-2
@keysig:bBEA
@timesig:6/8
@data:8F/6.3{GFGFGA}4B8F/{6.3GFGFGF}4E8-/
\end{filecontents*}
\commandline{ if [ ! -f 15-1.svg ]; then verovio --spacing-non-linear=0.54 -w 1500 --spacing-system=0.5 --adjust-page-height -b 0 15-1.code; fi } 
\newline \includesvg[width=209pt]{15-1}%
\par 1.1.2  cl 1, Es-Dur  
\begin{filecontents*}{15-2.code}
@clef:G-2
@keysig:bBEA
@timesig:6/8
@data:8-/=1/48--G/8.68{BAG}{''C'BA}/{BAG}''{8C'B}-/
\end{filecontents*}
\commandline{ if [ ! -f 15-2.svg ]; then verovio --spacing-non-linear=0.54 -w 1500 --spacing-system=0.5 --adjust-page-height -b 0 15-2.code; fi } 
\newline \includesvg[width=209pt]{15-2}%\newline Weber, Carl Maria von  (arr)
\par RISM-ID: 450109865\newline D-Cv  A.V,1111,(2),2\newline $\rightarrow$ In Sammlung 80 (450109860)
      
\par \vspace{16pt} \textcolor{darkblue}{\textbf{August Emil Leopold, Herzog von Sachsen-Gotha-Altenburg  1772-1822}}\hfillplus{[16]}\newline Die verliebte Schäferin. Arr, J 152 - B-Dur\newline winds
\par \begin{itshape}[caption title:] Die verliebte Schäferinne.\end{itshape} 
\par \textcolor{darkblue}{\ding{\numexpr181 + 01}}  Partitur: f.3r\newline Abschrift\newline Schreiber: Weber, Carl Maria von
\par 1.1.1  cl 1, \begin{itshape}Andante con moto\end{itshape}, B-Dur  
\begin{filecontents*}{16-1.code}
@clef:G-2
@keysig:bBE
@timesig:3/4
@data:''2(D)4-/'{8BABAB''D}/4C'F-/''{8C'B''C'B''CE}/4D--/
\end{filecontents*}
\commandline{ if [ ! -f 16-1.svg ]; then verovio --spacing-non-linear=0.54 -w 1500 --spacing-system=0.5 --adjust-page-height -b 0 16-1.code; fi } 
\newline \includesvg[width=209pt]{16-1}%
\par 1.1.2  fl, B-Dur  
\begin{filecontents*}{16-2.code}
@clef:G-2
@keysig:bBE
@timesig:3/4
@data:''2(B)4-/=14/2E{8DE}/4F{8ED}4E/{8B'''C''BbA}4G/
\end{filecontents*}
\commandline{ if [ ! -f 16-2.svg ]; then verovio --spacing-non-linear=0.54 -w 1500 --spacing-system=0.5 --adjust-page-height -b 0 16-2.code; fi } 
\newline \includesvg[width=209pt]{16-2}%\newline Weber, Carl Maria von  (arr)
\par RISM-ID: 450109864\newline D-Cv  A.V,1111,(2),2\newline $\rightarrow$ In Sammlung 80 (450109860)
      
\par \vspace{16pt} \textcolor{darkblue}{\textbf{August Emil Leopold, Herzog von Sachsen-Gotha-Altenburg  1772-1822}}\hfillplus{[17]}\newline Ihr kleinen Vögelein. Arr, J 150 - F-Dur\newline winds
\par \begin{itshape}[caption title:] Ihr Kleinen Vögelein\end{itshape} 
\par \textcolor{darkblue}{\ding{\numexpr181 + 01}}  Partitur: f.1v-2r\newline Abschrift\newline Schreiber: Weber, Carl Maria von
\par 1.1.1  cl 1, \begin{itshape}Allegretto\end{itshape}, F-Dur  
\begin{filecontents*}{17-1.code}
@clef:G-2
@keysig:bB
@timesig:2/4
@data:''4(A)8-C/{CCD'nB}/''4CF/{8EDC'B}/4A''F/{8EC'GG}/
\end{filecontents*}
\commandline{ if [ ! -f 17-1.svg ]; then verovio --spacing-non-linear=0.54 -w 1500 --spacing-system=0.5 --adjust-page-height -b 0 17-1.code; fi } 
\newline \includesvg[width=209pt]{17-1}%\newline Weber, Carl Maria von  (arr)
\par RISM-ID: 450109862\newline D-Cv  A.V,1111,(2),2\newline $\rightarrow$ In Sammlung 80 (450109860)
      
\par \vspace{16pt} \textcolor{darkblue}{\textbf{August Emil Leopold, Herzog von Sachsen-Gotha-Altenburg  1772-1822}}\hfillplus{[18]}\newline Lebe wohl mein süßes Leben. Arr, J 151 - Es-Dur\newline winds
\par \begin{itshape}[caption title:] Serenade. Lebe wohl mein süßes Leben.\end{itshape} 
\par \textcolor{darkblue}{\ding{\numexpr181 + 01}}  Partitur: f.2r-2v\newline Abschrift\newline Schreiber: Weber, Carl Maria von
\par 1.1.1  cl 1, \begin{itshape}Andante\end{itshape}, Es-Dur  
\begin{filecontents*}{18-1.code}
@clef:G-2
@keysig:bBEA
@timesig:3/4
@data:{8.G6A}/4.B{8B''CD}/{8.E6'B}4B{8BG}/''4.C'{8BAG}/4GF
\end{filecontents*}
\commandline{ if [ ! -f 18-1.svg ]; then verovio --spacing-non-linear=0.54 -w 1500 --spacing-system=0.5 --adjust-page-height -b 0 18-1.code; fi } 
\newline \includesvg[width=209pt]{18-1}%\newline Weber, Carl Maria von  (arr)
\par RISM-ID: 450109863\newline D-Cv  A.V,1111,(2),2\newline $\rightarrow$ In Sammlung 80 (450109860)
      
\par \vspace{16pt} \textcolor{darkblue}{\textbf{Bach, Johann Sebastian  1685-1750}}\hfillplus{[19]}\newline Herr Gott dich loben alle wir, BWV 130\newline V (4), Coro, orch, bc
\par \begin{itshape}[caption title:] Herr Gott dich loben alle wir\end{itshape} 
\par \textcolor{darkblue}{\ding{\numexpr181 + 01}}  Stimme: clno 3  (2f.); 36 x 21 cm\newline \begin{small} f.2v only blank staves\end{small} \newline Abschrift\newline Wasserzeichen: [crescent with facial profile]  1720-1739\newline Schreiber: Meißner, Christian Gottlob
\par 1.1.1  clno 3, C-Dur  
\begin{filecontents*}{19-1.code}
@clef:G-2
@keysig:
@timesig:c
@data:8'E-G-''C-4-/8'C-''C-C-4-/
\end{filecontents*}
\commandline{ if [ ! -f 19-1.svg ]; then verovio --spacing-non-linear=0.54 -w 1500 --spacing-system=0.5 --adjust-page-height -b 0 19-1.code; fi } 
\newline \includesvg[width=209pt]{19-1}%
\par Bemerkung in der Bach-Quellendatenbank (http://www.bach.gwdg.de/det\_beschr/dcvv03.html): {\textquotedbl}Revision: J. S. Bach{\textquotedbl}. In der neueren Version der Datenbank (http://www.bach-digital.de/receive/BachDigitalSource\_source\_00002707) ist Bach nur noch undifferenziert als 2. Schreiber angegeben.
\par Die Stimme enthält NBA zufolge Zusätze von J.S. Bachs Hand.\newline Bach, Johann Sebastian  (Sonstige)\newline Nägeli, Hans Georg  (Vorbesitzer)\newline Schulz, Hermann  (Vorbesitzer)\newline Literatur: NBA  ser.1, vol.30 (Kritischer Bericht), p.23, 26, 31
\par RISM-ID: 450106269\newline D-Cv  A.V,1109,(1),3
\par \vspace{16pt} \textcolor{darkblue}{\textbf{Bach, Johann Sebastian  1685-1750}}\hfillplus{[20]}\newline Musikalisches Opfer. Fragment, BWV 1079/8 - c-Moll\newline vl, fl, cemb
\par \begin{itshape}[caption title:] Cembalo\end{itshape} 
\par \textcolor{darkblue}{\ding{\numexpr181 + 01}}  1 Stimme: cemb  (1f.)\newline \begin{small} = f.1v of vlne of the first work in this collection, other parts missing\end{small} \newline Abschrift\newline Schreiber: Bach, Johann Christoph Friedrich
\par 1.1.1  cemb, \begin{itshape}Largo\end{itshape}, c-Moll  
\begin{filecontents*}{20-1.code}
@clef:G-2
@keysig:bBEA
@timesig:3/4
@data:{8.G''6E}q8C4.n'Bt''8C/4.C6{'BA}{''C'BAG}/{AGA''C}{'B''bDDC}{C'BB''C}/{C'nB''CE}
\end{filecontents*}
\commandline{ if [ ! -f 20-1.svg ]; then verovio --spacing-non-linear=0.54 -w 1500 --spacing-system=0.5 --adjust-page-height -b 0 20-1.code; fi } 
\newline \includesvg[width=209pt]{20-1}%
\par Die Skizze enthält nur die ersten viereinhalb Takte der Oberstimme. Sie ist auf der Rückseite der vlne-Stimme von BWV 241 notiert (hier das erste Stück der {\textquotedbl}collection{\textquotedbl}: vgl RISM ID no. 450106266).\newline Bach, Carl Philipp Emanuel  (Vorbesitzer)\newline Kittel, Johann Christian  (Vorbesitzer)\newline Poelchau, Georg Johann Daniel  (Vorbesitzer)\newline Literatur: KobayashiC 1988  p.58
\par RISM-ID: 450106267\newline D-Cv  A.V,1109,(1),1b\newline $\rightarrow$ In Sammlung 79 (450106265)
      
\par \vspace{16pt} \textcolor{darkblue}{\textbf{Bach, Johann Sebastian  1685-1750}}\hfillplus{[21]}\newline Süßer Trost mein Jesus kommt, BWV 151\newline V (4), Coro, orch, bc
\par \begin{itshape}[dust cover title:] Feria 3 Nativ. Xsti | Süßer Trost mein Jesus | a | 4 Voci | 1 Trav. | 1 Hautb. d'Amour | 2 Viol. | Viola | e | Contin. | di | J. S. Bach. | [by Pölchau:] No 40. | Partitur u Stimmen von Sebastian Bachs eigener Hand. | (Das Titelblatt von Emanuel Bachs Hand.) | G. Pölchau\newline [caption title:] J. J. Feria 3. Nativitatis Christi Concerto. [later added:] Von Johann Sebastian Bachs [?] Hand geschrieben\end{itshape} 
\par \textcolor{darkblue}{\ding{\numexpr181 + 01}}  Partitur: 6f.; 33,5 x 21 cm\newline Autograph\newline Wasserzeichen: [Sachsen shield]  17200101-17261227
\par \textcolor{darkblue}{\ding{\numexpr181 + 02}}  3 Stimmen: vl 1 and fl, vl 2, {\textquotedbl}Continuo{\textquotedbl} (b.fig)  (4, 1, 2f.); 33,5 x 20,5 (21) cm\newline \begin{small} Schreiber: Violino I: J. A. Kuhnau; Flauto traverso: J. S. Bach; Violino II: A. M. Bach; Continuo: J. H. Bach; Bezifferung: J. S. Bach\end{small} \newline \begin{small} Zur genauen Beschreibung der Stimmen vgl. Alfred Dürr in NBA ser.1, vol.3.1 (Kritischer Bericht), p.155-157\end{small} \newline \begin{small} other parts: D B Mus.ms. Bach St 89\end{small} \newline Partial autograph; [Sachsen shield]; MA  17200101-17261227\newline Schreiber: Kuhnau, Johann Andreas; Bach, Anna Magdalena; Bach, Johann Heinrich
\par 1.1.1  fl, \begin{itshape}Aria. Molto Adagio\end{itshape}, G-Dur  
\begin{filecontents*}{21-1.code}
@clef:G-2
@keysig:xF
@timesig:12/8
@data:2.''D+{3DGFE}4D+{3DC'BA}4G+/{3G''GFE}4D
\end{filecontents*}
\commandline{ if [ ! -f 21-1.svg ]; then verovio --spacing-non-linear=0.54 -w 1500 --spacing-system=0.5 --adjust-page-height -b 0 21-1.code; fi } 
\newline \includesvg[width=209pt]{21-1}%
\par 1.1.2  S, G-Dur\newline \begin{footnotesize} Süßer Trost mein Jesus kommt \end{footnotesize}  
\begin{filecontents*}{21-2.code}
@clef:C-1
@keysig:xF
@timesig:12/8
@data:=11/''2.D+{6DG8F}E4.D+/{6DG8F}E4.D+8D'B''C{DE}C/'4.A
\end{filecontents*}
\commandline{ if [ ! -f 21-2.svg ]; then verovio --spacing-non-linear=0.54 -w 1500 --spacing-system=0.5 --adjust-page-height -b 0 21-2.code; fi } 
\newline \includesvg[width=209pt]{21-2}%
\par 1.2.1  B, \begin{itshape}Recitativo.\end{itshape}, D-Dur\newline \begin{footnotesize} Erfreue dich mein Herz den jetzo weicht der Schmerz \end{footnotesize}  
\begin{filecontents*}{21-3.code}
@clef:F-4
@keysig:xFC
@timesig:c
@data:,4-8-A{6FDEFGAB'C}/8D,AFA4B8-B/GGFE'4C6-
\end{filecontents*}
\commandline{ if [ ! -f 21-3.svg ]; then verovio --spacing-non-linear=0.54 -w 1500 --spacing-system=0.5 --adjust-page-height -b 0 21-3.code; fi } 
\newline \includesvg[width=209pt]{21-3}%
\par 1.3.1  A, \begin{itshape}Aria. Andante\end{itshape}, e-Moll\newline \begin{footnotesize} In Jesu Demut kann ich Trost \end{footnotesize}  
\begin{filecontents*}{21-4.code}
@clef:C-3
@keysig:xF
@timesig:c/
@data:=8/4-8-'B{G''C}{'BA}/{xDA}{GF}{,B'G}{FE}/4''D8-
\end{filecontents*}
\commandline{ if [ ! -f 21-4.svg ]; then verovio --spacing-non-linear=0.54 -w 1500 --spacing-system=0.5 --adjust-page-height -b 0 21-4.code; fi } 
\newline \includesvg[width=209pt]{21-4}%
\par 1.4.1  T, \begin{itshape}Recitativo.\end{itshape}\newline \begin{footnotesize} Du treuer Gottessohn nun hast du mir den Himmel aufgemacht \end{footnotesize}  
\begin{filecontents*}{21-5.code}
@clef:C-4
@keysig:xF
@timesig:c
@data:4-8-,,B,FE8.xCt6D/8D6-D8xC,,B,E6-,,A,EEFG/4F8-
\end{filecontents*}
\commandline{ if [ ! -f 21-5.svg ]; then verovio --spacing-non-linear=0.54 -w 1500 --spacing-system=0.5 --adjust-page-height -b 0 21-5.code; fi } 
\newline \includesvg[width=209pt]{21-5}%
\par 1.5.1  Coro S, \begin{itshape}Choral.\end{itshape}, G-Dur\newline \begin{footnotesize} Heut' schleußt er wieder auf die Tür \end{footnotesize}  
\begin{filecontents*}{21-6.code}
@clef:C-1
@keysig:xF
@timesig:c
@data:'G/''DDDD/E{8DC}'4(B)
\end{filecontents*}
\commandline{ if [ ! -f 21-6.svg ]; then verovio --spacing-non-linear=0.54 -w 1500 --spacing-system=0.5 --adjust-page-height -b 0 21-6.code; fi } 
\newline \includesvg[width=209pt]{21-6}%
\par Spätere Ergänzung, f.1r: {\textquotedbl}Von Johann Sebastian Bach und dessen Hand geschrieben / André.{\textquotedbl}
\par Umschlag von C.P.E. Bach und Pölchau beschriftet; eigene Signatur: A.V,1109,(1),2c\newline André, Johann Anton  (Vorbesitzer)\newline André, Julius  (Vorbesitzer)\newline Bach, Carl Philipp Emanuel  (Sonstige)\newline Bach, Carl Philipp Emanuel  (Vorbesitzer)\newline Gerber, Ernst Ludwig  (Vorbesitzer)\newline Poelchau, Georg Johann Daniel  (Sonstige)\newline Literatur: NBA  ser.1, vol.3.1 (Kritischer Bericht), p.146-178; BachCatalog 1790  p.74, no.10; ZirnbauerB 1950  \newline Olim: 40
\par RISM-ID: 450106270\newline D-Cv  A.V,1109,(1),2a
\par \vspace{16pt} \textcolor{darkblue}{\textbf{Beethoven, Ludwig van  1770-1827}}\hfillplus{[22]}\newline Der Glorreiche Augenblick, op.136\newline V (4), Coro, orch
\par \begin{itshape}[caption title:] Der glorreiche Augenblick: Cantate von L:v: Beethoven.\end{itshape} 
\par \textcolor{darkblue}{\ding{\numexpr181 + 01}}  Particell: 42f.; 24,5 x 33 cm\newline \begin{small} Chorstimmen in Partitur (S und T je 2 Stimmen auf einer Notenzeile) und Klaviersatz; zusätzlich ein auf f.35v aufgenähtes Blatt (Rückseite leer), nur leere Notensysteme: f.6v, 9v, 20v\end{small} \newline Abschrift\newline Wasserzeichen: T.D. WIRTZ [countermark: standing lion with orb]  18141201-18151231\newline Schreiber: Diabelli, Anton
\par 1.1.1  Coro S 1, \begin{itshape}N|r|o 1.. Allegro man non troppo\end{itshape}, A-Dur\newline \begin{footnotesize} Europa steht \end{footnotesize}  
\begin{filecontents*}{22-1.code}
@clef:C-1
@keysig:xFCG
@timesig:c/
@data:2-''E/2.E4E/1E+/E+/4E-2-/
\end{filecontents*}
\commandline{ if [ ! -f 22-1.svg ]; then verovio --spacing-non-linear=0.54 -w 1500 --spacing-system=0.5 --adjust-page-height -b 0 22-1.code; fi } 
\newline \includesvg[width=209pt]{22-1}%
\par 1.2.1  B, \begin{itshape}N|r|o 2. Recitativo. Andante\end{itshape}, D-Dur\newline \begin{footnotesize} O seht sie nah und näher treten \end{footnotesize}  
\begin{filecontents*}{22-2.code}
@clef:F-4
@keysig:xFC
@timesig:c/
@data:=3/,4A8FG8.6AA(B)A/8FF-2-/
\end{filecontents*}
\commandline{ if [ ! -f 22-2.svg ]; then verovio --spacing-non-linear=0.54 -w 1500 --spacing-system=0.5 --adjust-page-height -b 0 22-2.code; fi } 
\newline \includesvg[width=209pt]{22-2}%
\par 1.3.1  S 1, \begin{itshape}N|r|o 3. Recitativo.. Allegro\end{itshape}, B-Dur\newline \begin{footnotesize} O Himmel welch Entzücken \end{footnotesize}  
\begin{filecontents*}{22-3.code}
@clef:C-1
@keysig:bBE
@timesig:c
@data:=3/4-8-''D4FF/2-4.F8F/8DD-
\end{filecontents*}
\commandline{ if [ ! -f 22-3.svg ]; then verovio --spacing-non-linear=0.54 -w 1500 --spacing-system=0.5 --adjust-page-height -b 0 22-3.code; fi } 
\newline \includesvg[width=209pt]{22-3}%
\par 1.4.1  S 2, \begin{itshape}N|r|o 4. Recit|v|o.\end{itshape}, H-Dur\newline \begin{footnotesize} Das Auge schaut in dessen Wimpergleise die Sonnen auf und nieder gehn \end{footnotesize}  
\begin{filecontents*}{22-4.code}
@clef:C-1
@keysig:xFCGDA
@timesig:c
@data:2-8-''F8.F6F/4D6-DDEFFF3FG8.F6F/2G+8GE'BG/E
\end{filecontents*}
\commandline{ if [ ! -f 22-4.svg ]; then verovio --spacing-non-linear=0.54 -w 1500 --spacing-system=0.5 --adjust-page-height -b 0 22-4.code; fi } 
\newline \includesvg[width=209pt]{22-4}%
\par 1.5.1  S 2, \begin{itshape}N|r|o 5. Recitativ. Allegro\end{itshape}, A-Dur\newline \begin{footnotesize} Der den Bund in Sturme festgehalten \end{footnotesize}  
\begin{filecontents*}{22-5.code}
@clef:C-1
@keysig:xFCG
@timesig:c
@data:=1/2-4-''E+/8E'A''CC'8.A6A8AB/''CC4-2-/
\end{filecontents*}
\commandline{ if [ ! -f 22-5.svg ]; then verovio --spacing-non-linear=0.54 -w 1500 --spacing-system=0.5 --adjust-page-height -b 0 22-5.code; fi } 
\newline \includesvg[width=209pt]{22-5}%
\par 1.6.1  Coro S 1, \begin{itshape}N|r|o 6. | Chor der Frauen. Poco Allegro\end{itshape}, C-Dur\newline \begin{footnotesize} Es treten hervor die Scharen der Frauen \end{footnotesize}  
\begin{filecontents*}{22-6.code}
@clef:C-1
@keysig:
@timesig:2/4
@data:=8/4-8-''G/4G8GG/{FE}{DG}/{ED}CD/'4B8GG/
\end{filecontents*}
\commandline{ if [ ! -f 22-6.svg ]; then verovio --spacing-non-linear=0.54 -w 1500 --spacing-system=0.5 --adjust-page-height -b 0 22-6.code; fi } 
\newline \includesvg[width=209pt]{22-6}%
\par Auf der ersten Notenseite unten von Beethovens Hand: {\textquotedbl}die Kantate ist - ebenfalls zum stechen, | geschieht solches nicht recht, so wird es nicht | allein Stiche sondern auch Hiebe | absetzen. - ludwig van | Beethoven.{\textquotedbl} (zunächst abgeschnitten, dann wieder angefügt, vgl. Schmidt-GörgB, pt.7, vol.1, p.336)
\par Spätere Ergänzungen mit Bleistift, z.B. italienische Satztitel durchgestrichen und durch deutsche ersetzt (2.B. {\textquotedbl}Recitativ{\textquotedbl} statt {\textquotedbl}Recitativo{\textquotedbl})
\par Comment on scoring: original von Diabelli {\textquotedbl}Cembalo{\textquotedbl}, in Satz 1 und 2 durchgestrichen mit Bleistift und darüber notiert: {\textquotedbl}Pianoforte{\textquotedbl}\newline Weißenbach, Aloys  (Textdichter)\newline Literatur: Schmidt-GörgB 1961  pt.7, vol.1, p.153-315; Schmidt-GörgB 1961  pt.7, vol.1, p.336
\par RISM-ID: 450106288\newline D-Cv  A.V,1112,(2),3
\par \vspace{16pt} \textcolor{darkblue}{\textbf{Beethoven, Ludwig van  1770-1827}}\hfillplus{[23]}\newline Fidelio. Sketches, op.72/15\newline V (2), orch
\par \begin{itshape}[at bottom, left:] aus Beethovens Fidelio eigenhändig zugeschrieben\end{itshape} 
\par \textcolor{darkblue}{\ding{\numexpr181 + 01}}  Partitur: 1f.; 22,5 x 30 cm\newline \begin{small} in der Mitte früherer Falz erkennbar\end{small} \newline Autograph\newline Wasserzeichen: [without watermark]  1814
\par 1.1.1  S, G-Dur\newline \begin{footnotesize} [O namenlose Freude] \end{footnotesize}  
\begin{filecontents*}{23-1.code}
@clef:G-2
@keysig:xF
@timesig:c
@data:=4/4-''D{8D'B}{B''D}/{DG}4G4.G8F/4AG
\end{filecontents*}
\commandline{ if [ ! -f 23-1.svg ]; then verovio --spacing-non-linear=0.54 -w 1500 --spacing-system=0.5 --adjust-page-height -b 0 23-1.code; fi } 
\newline \includesvg[width=209pt]{23-1}%
\par 1.1.2  i, G-Dur  
\begin{filecontents*}{23-2.code}
@clef:G-2
@keysig:xF
@timesig:c
@data:4G-''{8FA}{A'''C}/2-8{''G'''C}{CE}/1E/D/
\end{filecontents*}
\commandline{ if [ ! -f 23-2.svg ]; then verovio --spacing-non-linear=0.54 -w 1500 --spacing-system=0.5 --adjust-page-height -b 0 23-2.code; fi } 
\newline \includesvg[width=209pt]{23-2}%\newline Sonnleithner, Joseph  (Textdichter)\newline Treitschke, Georg Friedrich  (Textdichter)\newline Literatur: SchmidtB 1969  no.319
\par RISM-ID: 450106659\newline D-Cv  A.V,1112,(1),1
\par \vspace{16pt} \textcolor{darkblue}{\textbf{Beethoven, Ludwig van  1770-1827}}\hfillplus{[24]}\newline Quartett, op.131 - cis-Moll\newline vl (2), vla, vlc
\par \begin{itshape}[without title]\end{itshape} 
\par \textcolor{darkblue}{\ding{\numexpr181 + 01}}  Partitur: 2f.; 24 x 31,5 cm\newline \begin{small} f.2v crossed out\end{small} \newline Autograph\newline Wasserzeichen: JAA [above: lily - countermark: illegible capital letters, partially cut off, possibly {\textquotedbl}W HATMANN{\textquotedbl}[?] / ] N|o 4  1825
\par 1.1.1  vl 1, cis-Moll  
\begin{filecontents*}{24-1.code}
@clef:G-2
@keysig:xFCGD
@timesig:c/
@data:4C-2-/4-{8CE}4G{8AC}/,4xB-2-/
\end{filecontents*}
\commandline{ if [ ! -f 24-1.svg ]; then verovio --spacing-non-linear=0.54 -w 1500 --spacing-system=0.5 --adjust-page-height -b 0 24-1.code; fi } 
\newline \includesvg[width=209pt]{24-1}%
\par 1.2.1  vl 1, cis-Moll  
\begin{filecontents*}{24-2.code}
@clef:G-2
@keysig:xFCGDA
@timesig:c
@data:=2/2-4AF/2'B''E/4DEFG/A-2-/4GFDC/2CF/
\end{filecontents*}
\commandline{ if [ ! -f 24-2.svg ]; then verovio --spacing-non-linear=0.54 -w 1500 --spacing-system=0.5 --adjust-page-height -b 0 24-2.code; fi } 
\newline \includesvg[width=209pt]{24-2}%
\par 1.3.1  vl 1, \begin{itshape}tempo I.\end{itshape}, cis-Moll  
\begin{filecontents*}{24-3.code}
@clef:G-2
@keysig:xFCGD
@timesig:c/
@data:''D'AFA/2''DG/4EG4.A8B/'''4C
\end{filecontents*}
\commandline{ if [ ! -f 24-3.svg ]; then verovio --spacing-non-linear=0.54 -w 1500 --spacing-system=0.5 --adjust-page-height -b 0 24-3.code; fi } 
\newline \includesvg[width=209pt]{24-3}%
\par Nach JohnsonB 1985 Wasserzeichen Typ 34, Welhartiz-Papier.
\par Freundliche Mittelung von Julia Ronge, Bonn, nach JohnsonB 1985: {\textquotedbl}V, 1-96 + 5 unused bars, 141=67 followed by d.c.{\textquotedbl}. Das meint: 5. Satz, Skizzen zu T. 1-96, aber in loser Folge, sowie zu T. 141-167 in fortlaufender Folge, dazwischen ungenutztes Material. Das 14zeilige Papier verwendete Beethoven für frühe Skizzen, je weiter er fortschritt, umso weniger Systeme hatte das Papier. Demnach wäre das Blatt mit 1825 zu datieren.\newline Literatur: SchmidtB 1969  no.318; JohnsonB 1985  p.486
\par RISM-ID: 450106660\newline D-Cv  A.V,1112,(2),2
\par \vspace{16pt} \textcolor{darkblue}{\textbf{Böhner, Johann Ludwig  1787-1860}}\hfillplus{[25]}\newline Louise und Carolo im romantischen Mühltale\newline V (4), Coro, orch
\par \begin{itshape}[title page:] Louise und Carolo | im romantischen Mühlthale; Eine ganz neue | tragik-komische Oper in 1 Act, | mit vollständigem großen Orchester. | Jn Musik gesetzt und dichterisch bearbeitet | von | Johann Louis Böhner. | Seiner Exellenz, | des Herrn Herrn Hofmarschall, Herrn | Grafen etc[?] von Salisch, hochgebohrn Gnaden | am Hofe in Gotha, in Ehrerbietigster | Ehr furchtsvollster Submission, unterthänig | gewidmet und dankbarlichst verehrt | der Componiste JLouis Böhner. | Ende Dezember 1824. | Neues Manuscript.\end{itshape} 
\par \textcolor{darkblue}{\ding{\numexpr181 + 01}}  Partitur: 141p.; 33 x 19,5 cm\newline \begin{small} p.29-38 (38 leer): Eingelegter Faszikel (32 x 19,5 cm) mit Überschrift {\textquotedbl}Beilage zum Quartett zum Schlusse G dur{\textquotedbl}\end{small} \newline Autograph\newline Wasserzeichen: ICK [countermark: heart, crowned]  1824
\par 1.1.1  vl 1, \begin{itshape}N|o 1 Romanze. Andantino\end{itshape}, E-Dur  
\begin{filecontents*}{25-1.code}
@clef:G-2
@keysig:xFCGD
@timesig:6/8
@data:{8GGG}{GGG}/{GGG}{G''C'B}/4B8xA''4D8E/
\end{filecontents*}
\commandline{ if [ ! -f 25-1.svg ]; then verovio --spacing-non-linear=0.54 -w 1500 --spacing-system=0.5 --adjust-page-height -b 0 25-1.code; fi } 
\newline \includesvg[width=209pt]{25-1}%
\par 1.1.2  T, E-Dur\newline \begin{footnotesize} Einsam saß ich in dem Tale an dem schroffen Felsenhang \end{footnotesize}  
\begin{filecontents*}{25-2.code}
@clef:C-4
@keysig:xFCGD
@timesig:6/8
@data:=5/4-8--,BB/48BG'ED/DC{8C,AG}/4F8xF{FBA}/4G8-
\end{filecontents*}
\commandline{ if [ ! -f 25-2.svg ]; then verovio --spacing-non-linear=0.54 -w 1500 --spacing-system=0.5 --adjust-page-height -b 0 25-2.code; fi } 
\newline \includesvg[width=209pt]{25-2}%
\par 1.2.1  S, \begin{itshape}N|o 2. Romanze. Andante\end{itshape}, A-Dur\newline \begin{footnotesize} Es gehen die Töne ins fühlende Herz \end{footnotesize}  
\begin{filecontents*}{25-3.code}
@clef:G-2
@keysig:xFCG
@timesig:2/4
@data:4-8-A/4A8E''C/4C'8A''E/{6ExDnD'B}{''ExDnD'B}/{6B''C}'4A
\end{filecontents*}
\commandline{ if [ ! -f 25-3.svg ]; then verovio --spacing-non-linear=0.54 -w 1500 --spacing-system=0.5 --adjust-page-height -b 0 25-3.code; fi } 
\newline \includesvg[width=209pt]{25-3}%
\par 1.3.1  S, \begin{itshape}N|o 3. Quartett. Poco Allegretto ma piu Andante con moto\end{itshape}, g-Moll\newline \begin{footnotesize} Im ferneren Lande da kämpfen \end{footnotesize}  
\begin{filecontents*}{25-4.code}
@clef:G-2
@keysig:bBE
@timesig:6/8
@data:=1/4-8''DDDD/DC'BBBB/BAGGGG/
\end{filecontents*}
\commandline{ if [ ! -f 25-4.svg ]; then verovio --spacing-non-linear=0.54 -w 1500 --spacing-system=0.5 --adjust-page-height -b 0 25-4.code; fi } 
\newline \includesvg[width=209pt]{25-4}%
\par 1.4.1  B, \begin{itshape}N|o 4. Arie von Michel. Moderato assai\end{itshape}, D-Dur\newline \begin{footnotesize} Die Mädchen sie gleichen bei meiner Treu' \end{footnotesize}  
\begin{filecontents*}{25-5.code}
@clef:F-4
@keysig:xFCG
@timesig:3/4
@data:4--8-,A/{AF}'4.D,8A/{FD}4.A8F/4.B8G'{D,B}/{6ABAG}4F
\end{filecontents*}
\commandline{ if [ ! -f 25-5.svg ]; then verovio --spacing-non-linear=0.54 -w 1500 --spacing-system=0.5 --adjust-page-height -b 0 25-5.code; fi } 
\newline \includesvg[width=209pt]{25-5}%
\par 1.5.1  S, \begin{itshape}N|o 5. Große Scene von Louisen. Largo\end{itshape}, Es-Dur\newline \begin{footnotesize} Carolo schlägt dein Herz noch für Louise \end{footnotesize}  
\begin{filecontents*}{25-6.code}
@clef:G-2
@keysig:bBEA
@timesig:c
@data:=4/{8.B6F}4F''{8.(D)6'B}4B/''4.D8'B2(B)/
\end{filecontents*}
\commandline{ if [ ! -f 25-6.svg ]; then verovio --spacing-non-linear=0.54 -w 1500 --spacing-system=0.5 --adjust-page-height -b 0 25-6.code; fi } 
\newline \includesvg[width=209pt]{25-6}%
\par 1.6.1  B, \begin{itshape}N|o 6. Duetto, Michel und der Müller. Allegro molto\end{itshape}, D-Dur\newline \begin{footnotesize} Alle Donner alle Wetter alle Teufel \end{footnotesize}  
\begin{filecontents*}{25-7.code}
@clef:F-4
@keysig:xFC
@timesig:c
@data:=1/2-4-8,AA/4AF-'8DD/4D,B-8BB/4BG-
\end{filecontents*}
\commandline{ if [ ! -f 25-7.svg ]; then verovio --spacing-non-linear=0.54 -w 1500 --spacing-system=0.5 --adjust-page-height -b 0 25-7.code; fi } 
\newline \includesvg[width=209pt]{25-7}%
\par 1.7.1  Coro S, \begin{itshape}Scene 2. | N|o 7 Marsch. sehr feierlich und langsam\end{itshape}, a-Moll\newline \begin{footnotesize} Er ist da in seiner Gloria \end{footnotesize}  
\begin{filecontents*}{25-8.code}
@clef:G-2
@keysig:
@timesig:c
@data:2AA/2.''C4-/2'AA/2.''C4-/2E4GG/2G4..xFt{3EF}/1G/
\end{filecontents*}
\commandline{ if [ ! -f 25-8.svg ]; then verovio --spacing-non-linear=0.54 -w 1500 --spacing-system=0.5 --adjust-page-height -b 0 25-8.code; fi } 
\newline \includesvg[width=209pt]{25-8}%
\par 1.8.1  Coro S, \begin{itshape}N|o 8. Coro. Poco Allegretto\end{itshape}, C-Dur\newline \begin{footnotesize} Tief aus Tälern aus waldigen Höhen eilen wir freudig heran \end{footnotesize}  
\begin{filecontents*}{25-9.code}
@clef:G-2
@keysig:
@timesig:6/8
@data:=17/4G8B{''CE}'G/{8.G6G}8B''4C8E/'AAA''CCC/4D8-
\end{filecontents*}
\commandline{ if [ ! -f 25-9.svg ]; then verovio --spacing-non-linear=0.54 -w 1500 --spacing-system=0.5 --adjust-page-height -b 0 25-9.code; fi } 
\newline \includesvg[width=209pt]{25-9}%
\par 1.9.1  Coro S, \begin{itshape}N|o 9. Finale und Schluß Chor.\end{itshape}, Es-Dur\newline \begin{footnotesize} Freude füllet alle Herzen \end{footnotesize}  
\begin{filecontents*}{25-10.code}
@clef:G-2
@keysig:bBEA
@timesig:c
@data:=2/2-4BB/4.8''GEDF/4EG'BB/''CDEC/
\end{filecontents*}
\commandline{ if [ ! -f 25-10.svg ]; then verovio --spacing-non-linear=0.54 -w 1500 --spacing-system=0.5 --adjust-page-height -b 0 25-10.code; fi } 
\newline \includesvg[width=209pt]{25-10}%\newline Arnold, Ignaz Ferdinand  (Textdichter)
\par RISM-ID: 450109877\newline D-Cv  A.V,1134,(5),1 (Buch 34)
\par \vspace{16pt} \textcolor{darkblue}{\textbf{Boieldieu, Louis  1815-1883}}\hfillplus{[26]}\newline Les Deux nuits. Auswahl\newline V (2), orch
\par \begin{itshape}[heading:] Duo dans les deux nuits premier intention [on the right:] 2 acte\end{itshape} 
\par \textcolor{darkblue}{\ding{\numexpr181 + 01}}  Partitur: 1f.; 33,5 x 26 cm\newline Autograph  1829
\par 1.1.1  vl 1, \begin{itshape}Andantino\end{itshape}, Es-Dur  
\begin{filecontents*}{26-1.code}
@clef:G-2
@keysig:bBEA
@timesig:6/8
@data:''{6BBBBB'''E}/''{3.5B'''C''BG}8E-'G--/D--''{6BBBBB'''F}/''{3.5B'''C''AF}8D-'D--/
\end{filecontents*}
\commandline{ if [ ! -f 26-1.svg ]; then verovio --spacing-non-linear=0.54 -w 1500 --spacing-system=0.5 --adjust-page-height -b 0 26-1.code; fi } 
\newline \includesvg[width=209pt]{26-1}%
\par 1.1.2  S, Es-Dur\newline \begin{footnotesize} Milord vous avez ma promesse \end{footnotesize}  
\begin{filecontents*}{26-2.code}
@clef:G-2
@keysig:bBEA
@timesig:6/8
@data:=1/8--6-B{8''E'E}-/=1/4-6-3-B''{8F'6A}3AB6''C'3BA/8xFG-
\end{filecontents*}
\commandline{ if [ ! -f 26-2.svg ]; then verovio --spacing-non-linear=0.54 -w 1500 --spacing-system=0.5 --adjust-page-height -b 0 26-2.code; fi } 
\newline \includesvg[width=209pt]{26-2}%
\par Hinweis im Katalog in D-Cv: Erster Entwurf eines Duos im 2\textsuperscript{t}\textsuperscript{e}\textsuperscript{n} Acte der Oper „Les deux nuits“ M.A. | M.P.
\par Unterschrift f.1r, unten: Boieldieu\newline Bouilly, Jean Nicolas  (Textdichter)\newline Scribe, Eugène  (Textdichter)
\par RISM-ID: 450109882\newline D-Cv  A.V,1137,(2),3
\par \vspace{16pt} \textcolor{darkblue}{\textbf{Czerny, Carl  1791-1857}}\hfillplus{[27]}\newline Salvator mundi - B-Dur\newline V, org
\par \begin{itshape}[caption title:] Salvator Mundi. | Graduale für Sopran oder Tenor | Mit Orgel oder fortepiano. [on the right:] Carl Czernÿ | Febr: | 1852\end{itshape} 
\par \textcolor{darkblue}{\ding{\numexpr181 + 01}}  Partitur: 2f.; 25,5 x 20,5 cm\newline \begin{small} f.2r-2v only blank staves\end{small} \newline Abschrift  18520201-18520228
\par 1.1.1  S (T), B-Dur\newline \begin{footnotesize} Salvator mundi, salva nos omnes \end{footnotesize}  
\begin{filecontents*}{27-1.code}
@clef:G-2
@keysig:bBE
@timesig:c
@data:4-FGA/''2C'4B-/2B4''CD/2FE+/4EEDC/
\end{filecontents*}
\commandline{ if [ ! -f 27-1.svg ]; then verovio --spacing-non-linear=0.54 -w 1500 --spacing-system=0.5 --adjust-page-height -b 0 27-1.code; fi } 
\newline \includesvg[width=209pt]{27-1}%
\par 1.1.2  org, B-Dur  
\begin{filecontents*}{27-2.code}
@clef:G-2
@keysig:bBE
@timesig:c
@data:2F4GA/''2C'B/B4''CD/2FE+/4EEDC/
\end{filecontents*}
\commandline{ if [ ! -f 27-2.svg ]; then verovio --spacing-non-linear=0.54 -w 1500 --spacing-system=0.5 --adjust-page-height -b 0 27-2.code; fi } 
\newline \includesvg[width=209pt]{27-2}%
\par Prägestempel {\textquotedbl}EA{\textquotedbl} mit Krone
\par RISM-ID: 450106264\newline D-Cv  A.V,1119,(2),1
\par \vspace{16pt} \textcolor{darkblue}{\textbf{Ernst II., Herzog von Sachsen-Coburg und Gotha  1818-1893}}\hfillplus{[28]}\newline Die Gräberinsel der Fürsten zu Gotha. Arr - Es-Dur\newline pf
\par \begin{itshape}[title page:] Étude melodique für Pianoforte | Composition S. Herzogl. Durchlaucht | des Erbprinzen, von Franz Liszt eigen- | händig corrigirt und ganz umgeschrieben\newline [caption title:] Die Gräberinsel der Fürsten zu Gotha\end{itshape} 
\par \textcolor{darkblue}{\ding{\numexpr181 + 01}}  Stimme: pf  (2f.); 31,5 x 24 cm\newline Abschrift  18421106\newline Schreiber: Liszt, Franz
\par 1.1.1  pf (left hand), \begin{itshape}Andante espressivo\end{itshape}, Es-Dur  
\begin{filecontents*}{28-1.code}
@clef:F-4
@keysig:bBEA
@timesig:c
@data:,,{6EB,EGBGE,,B}{6EB,EGBGE,,B}/{6EB,EGBGE,,B}{6EB,EGBGE,,B}/
\end{filecontents*}
\commandline{ if [ ! -f 28-1.svg ]; then verovio --spacing-non-linear=0.54 -w 1500 --spacing-system=0.5 --adjust-page-height -b 0 28-1.code; fi } 
\newline \includesvg[width=209pt]{28-1}%
\par 1.1.2  pf right hand, Es-Dur  
\begin{filecontents*}{28-2.code}
@clef:G-2
@keysig:bBEA
@timesig:c
@data:=1/2''G4FE+/EDbC'B/''2A4GF+/FEC'B/
\end{filecontents*}
\commandline{ if [ ! -f 28-2.svg ]; then verovio --spacing-non-linear=0.54 -w 1500 --spacing-system=0.5 --adjust-page-height -b 0 28-2.code; fi } 
\newline \includesvg[width=209pt]{28-2}%
\par Kleinformatiges Titelblatt (12,5 x 19,5 cm) vorne auf das Doppel-Notenblatt aufgeklebt.
\par Notiz nach den Noten, f.2v unten: {\textquotedbl}respectueusement offert | à l'auteur par son très humble | et dévoué serviteur | F. Liszt [on the left:] Coburg 6 Novembre 1842.{\textquotedbl}
\par Der Kompostition liegt das gleichnamige Lied Herzog Ernsts II. zugrunde, das sich unter den Beständen der Landebibliothek Coburg (D-Cl) befindet, vgl. RISM-ID.Nr. 450108682.\newline Liszt, Franz  (arr)\newline Literatur: LisztW 1970  ser.2, vol.6, p.36-39; LisztW 1970  ser.2, vol.6, p.XIII, 184
\par RISM-ID: 450109869\newline D-Cv  A.V,1130,(1),4
\par \vspace{16pt} \textcolor{darkblue}{\textbf{Fischhof, Joseph  1804-1857}}\hfillplus{[29]}\newline Das Fräulein - a-Moll\newline S, pf
\par \begin{itshape}[caption title:] Das Fräulein. comp. 13\textsuperscript{t}\textsuperscript{e}\textsuperscript{n} 10\textsuperscript{b}\textsuperscript{e}\textsuperscript{r} [1]835\end{itshape} 
\par \textcolor{darkblue}{\ding{\numexpr181 + 01}}  Partitur: 2f.; 19,5 x 24 cm\newline Autograph\newline Wasserzeichen: [partially cut off:] FR [or FB, ER, EB ...? - countermark indistinct]  18351213
\par 1.1.1  pf, \begin{itshape}nicht schnell, balladenmäßig\end{itshape}, a-Moll  
\begin{filecontents*}{29-1.code}
@clef:G-2
@keysig:
@timesig:c/
@data:4E/AA''{8.D6E}4F/E'A''{8.D6E}4F/E{8.D6C}'4B''E/'2.A4-/
\end{filecontents*}
\commandline{ if [ ! -f 29-1.svg ]; then verovio --spacing-non-linear=0.54 -w 1500 --spacing-system=0.5 --adjust-page-height -b 0 29-1.code; fi } 
\newline \includesvg[width=209pt]{29-1}%
\par 1.1.2  S, a-Moll\newline \begin{footnotesize} Das Fräulein saß und sann \end{footnotesize}  
\begin{filecontents*}{29-2.code}
@clef:G-2
@keysig:
@timesig:c/
@data:4-/=3/2-4-E/AA{8AxG}{AB}/''2.C4-/
\end{filecontents*}
\commandline{ if [ ! -f 29-2.svg ]; then verovio --spacing-non-linear=0.54 -w 1500 --spacing-system=0.5 --adjust-page-height -b 0 29-2.code; fi } 
\newline \includesvg[width=209pt]{29-2}%
\par RISM-ID: 450109875\newline D-Cv  A.V,1133,(5),2
\par \vspace{16pt} \textcolor{darkblue}{\textbf{Fischhof, Joseph  1804-1857}}\hfillplus{[30]}\newline Sehnsucht - Es-Dur\newline S, pf
\par \begin{itshape}[title page:] Sehnsucht | von Carl Baron v. Schweizer | in Musik gesetzt v. | Jos. Fischhof\end{itshape} 
\par \textcolor{darkblue}{\ding{\numexpr181 + 01}}  Partitur: 2f.; 19,5 x 24 cm\newline Autograph\newline Wasserzeichen: [partially cut off:] FR [or FB, ER, EB ...? - countermark indistinct]  18370705
\par 1.1.1  pf, \begin{itshape}Agitato\end{itshape}, Es-Dur  
\begin{filecontents*}{30-1.code}
@clef:G-2
@keysig:bBEA
@timesig:12/8
@data:4.G8{GAB}4.BE/4-8-{''AB'''bC}2.''B/'4.G8{GAB}4.BE/
\end{filecontents*}
\commandline{ if [ ! -f 30-1.svg ]; then verovio --spacing-non-linear=0.54 -w 1500 --spacing-system=0.5 --adjust-page-height -b 0 30-1.code; fi } 
\newline \includesvg[width=209pt]{30-1}%
\par 1.1.2  S, Es-Dur\newline \begin{footnotesize} Wo zieht ihr Wolken so eilig hin \end{footnotesize}  
\begin{filecontents*}{30-2.code}
@clef:G-2
@keysig:bBEA
@timesig:12/8
@data:=5/4.G{8GA}4.B4E8E/''4.E'{8FGA}2.A/
\end{filecontents*}
\commandline{ if [ ! -f 30-2.svg ]; then verovio --spacing-non-linear=0.54 -w 1500 --spacing-system=0.5 --adjust-page-height -b 0 30-2.code; fi } 
\newline \includesvg[width=209pt]{30-2}%
\par Datumsangabe auf f.1v, oben rechts.\newline Schweizer, Carl Friedrich von  (Textdichter)
\par RISM-ID: 450109874\newline D-Cv  A.V,1133,(5),1
\par \vspace{16pt} \textcolor{darkblue}{\textbf{Franz, Robert  1815-1892}}\hfillplus{[31]}\newline Nun die Schatten dunkeln, op.10/1 - Ges-Dur
\par \begin{itshape}[caption title:] Für Musik. Op. 10,1.\end{itshape} 
\par \textcolor{darkblue}{\ding{\numexpr181 + 01}}  Stimme: V  (1f.); 14 x 22 cm\newline \begin{small} pf missing\end{small} \newline Autograph  18920702
\par 1.1.1  V, Ges-Dur\newline \begin{footnotesize} Nun die Schatten dunkeln \end{footnotesize}  
\begin{filecontents*}{31-1.code}
@clef:G-2
@keysig:bBEADGC
@timesig:3/4
@data:8GG4AG/2B4A/
\end{filecontents*}
\commandline{ if [ ! -f 31-1.svg ]; then verovio --spacing-non-linear=0.54 -w 1500 --spacing-system=0.5 --adjust-page-height -b 0 31-1.code; fi } 
\newline \includesvg[width=209pt]{31-1}%
\par Auf dem Blatt (Rückseite leer) sind nur die beiden ersten Takte der Singstimme notiert, rechts daneben {\textquotedbl}etc{\textquotedbl}, darunter: {\textquotedbl}Mit freundlichen Grüssen | Halle d. 2. Jul. 92. | Rob. Franz.{\textquotedbl}\newline Geibel, Emanuel  (Textdichter)
\par RISM-ID: 450109878\newline D-Cv  A.V,1134,(8),2
\par \vspace{16pt} \textcolor{darkblue}{\textbf{Friedrich II., der Große, König von Preußen  1712-1786}}\hfillplus{[32]}\newline 3 Sonaten\newline fl, bc
\par \begin{itshape}[title page:] N. 6. | 30 Stück | [by Harson, later added:] Solo | [probably by Friedrich II.:] per il Traversière | di | Federico.\end{itshape} 
\par \textcolor{darkblue}{\ding{\numexpr181 + 01}}  Partitur: 7f.; 33,9 x 20,6 cm\newline \begin{small} obige Blattgröße bezieht sich auf no.169; no.161: 31,5 x 19,6 cm; no.123: 31,3 x 20,2 cm, Titelblatt: 33,6 x 21,4 cm; starker Tintenfraß bei den Blättern 6 und 7.\end{small} \newline Abschrift\newline Wasserzeichen: IGF [countermark indistinct]  1735-1745\newline Schreiber: Friedrich II., der Große, König von Preußen
\par \textcolor{darkblue}{\ding{\numexpr181 + 02}}  2 text documents; 21,6 x 17,5 cm\newline \begin{small} 2 Expertisen, die die Echtheit des Autographs bestätigen.\end{small} \newline Other  18690601\newline Schreiber: Espagne, Franz; Friedländer, Emil Gottlieb
\par Auf dem Titelblatt, links oben, von Harsons Hand: {\textquotedbl}32 | No. 33. | 34{\textquotedbl} (32 später ergänzt).
\par Auf dem Titelblatt, unten, von Harsons Hand: {\textquotedbl}Diesen kleinen Tietel auf dieser Seite, haben Ihro Majestät der Konig[!] Friederich | von Preußen eigenhändig geschrieben.{\textquotedbl} (Das Wort {\textquotedbl}kleinen{\textquotedbl} ist mit einer Einfügeklammer später ergänzt.)
\par Auf dem Titelblatt, später hinzugefügt: {\textquotedbl}(6 Bl. N\textsuperscript{o} 17){\textquotedbl}
\par Auf folgenden Seiten befinden sich am unteren Rand Notizen mit einem Hinweis, dass {\textquotedbl}diese Noten{\textquotedbl} von König Friedrich eigenhändig geschrieben seien: f.2r, 3v, 4r, 5v und 6r. Diese Notizen stammen, ebenso wie einige Zusätze auf dem Titelblatt von der Hand Johann Samuel Harsons. (Für den Hinweis darauf sei Herrn Peter Wollny, Leipzig, verbindlichst gedankt).
\par Auf f.6v steht dagegen am unteren Rand: {\textquotedbl}diese Noten soll H. Quantz geschrieben haben.{\textquotedbl} Ein Schriftvergleich ergibt jedoch, dass diese Angabe nicht zutrifft. Tatsächlich wechselt mit dieser Seite die Schreiberhand. Die Sätze 2 und 3 der dritten enthaltenen Sonate (No.123) wurden vermutlich von einem Berliner Kopisten notiert.
\par Die 3 Sonaten sind durchgängig auf 2 Notensystemen, die durch eine geschweifte Klammer verbunden sind, notiert, mit der Flötenstimme im oberen System, darunter die Basslinie. Eine Generalbassbezifferung findet sich nur an wenigen Stellen.
\par Für den Hinweis auf diese Musikhandschrift danken wir Frau Silvia Böcking, Untersiemau.
\par Die Handschrift wurde in einer Versteigerung der Autographenhandlung J.A. Stargardt am 09.11.1869 zusammen mit dem übrigen Nachlass August Wilhelm Bachs versteigert; vgl. {\textquotedbl}Verzeichniss einer werthvollen musikalischen und hymnologischen Sammlung u. Autographen [...]{\textquotedbl}; Berlin 1869: [Überschrift:] {\textquotedbl}I. Manuscripten-Sammlung des verstorbenen Directors des Königl. Instituts für Kirchenmusik in Berlin, A.W. Bach.{\textquotedbl} [folgt als no.1:] {\textquotedbl}Eigenhändiges Manuscript Friedrichs des Grossen (Königs v. Preussen). 3 Flötensolos mit Begleitung. 14 Seiten fol. (Schatz ersten Ranges.) (Angebot 50 th.){\textquotedbl}. (Den Hinweis darauf verdanken wir ebenfalls Herrn Peter Wollny, Leipzig.)\newline Bach, August Wilhelm  (Vorbesitzer)\newline Harson, Johann Samuel  (Vorbesitzer)\newline Quantz, Johann Joachim  (Sonstige)\newline Stargardt, Josef A.  (Sonstige)\newline Literatur: LauterwasserN 2013  \newline Olim: N.6; 17
\par RISM-ID: 450111000\newline D-Cv  A.I,179,(1),17
\par \vspace{16pt} \textcolor{darkblue}{\textbf{Friedrich II., der Große, König von Preußen  1712-1786}}\hfillplus{[33]}\newline Sonate, SpiF 63 - c-Moll\newline fl, bc
\par \begin{itshape}[caption title:] No: 169 Solo per il Flauto Traverso [on the right:] di Freder[ico]\end{itshape} 
\par \textcolor{darkblue}{\ding{\numexpr181 + 01}}  Partitur: f.1r-2v\newline Autograph
\par 1.1.1  fl, \begin{itshape}Un poco Andante\end{itshape}, c-Moll  
\begin{filecontents*}{33-1.code}
@clef:G-2
@keysig:bBE
@timesig:c
@data:2G+GgnB/{8''C6'nB''C}+{CEDC}gxF{8G'G}4-/8{B''G+}{6GbDC'B}{8BtbA}''4F/
\end{filecontents*}
\commandline{ if [ ! -f 33-1.svg ]; then verovio --spacing-non-linear=0.54 -w 1500 --spacing-system=0.5 --adjust-page-height -b 0 33-1.code; fi } 
\newline \includesvg[width=209pt]{33-1}%
\par 1.1.2  b, c-Moll  
\begin{filecontents*}{33-2.code}
@clef:F-4
@keysig:bBE
@timesig:c
@data:'{8CCCC}{,BBBB}/{bAAAA}{GGGG}/{nEEEE}{FFFF}/
\end{filecontents*}
\commandline{ if [ ! -f 33-2.svg ]; then verovio --spacing-non-linear=0.54 -w 1500 --spacing-system=0.5 --adjust-page-height -b 0 33-2.code; fi } 
\newline \includesvg[width=209pt]{33-2}%
\par 1.2.1  fl, \begin{itshape}Allegro\end{itshape}, c-Moll  
\begin{filecontents*}{33-3.code}
@clef:G-2
@keysig:bBE
@timesig:2/4
@data:4''CG/'''4.C''{6BA}/{8G6FE}{8DG}/{ECbAC}/
\end{filecontents*}
\commandline{ if [ ! -f 33-3.svg ]; then verovio --spacing-non-linear=0.54 -w 1500 --spacing-system=0.5 --adjust-page-height -b 0 33-3.code; fi } 
\newline \includesvg[width=209pt]{33-3}%
\par 1.3.1  fl, \begin{itshape}Presto\end{itshape}, c-Moll  
\begin{filecontents*}{33-4.code}
@clef:G-2
@keysig:bBE
@timesig:6/8
@data:''{6EF8G'G}/{AnB''C}/{FEDt}/{6ED8CG}/
\end{filecontents*}
\commandline{ if [ ! -f 33-4.svg ]; then verovio --spacing-non-linear=0.54 -w 1500 --spacing-system=0.5 --adjust-page-height -b 0 33-4.code; fi } 
\newline \includesvg[width=209pt]{33-4}%
\par Comment on scoring: cemb: bc
\par RISM-ID: 450111001\newline D-Cv  A.I,179,(1),17\newline $\rightarrow$ In Sammlung 32 (450111000)
      
\par \vspace{16pt} \textcolor{darkblue}{\textbf{Friedrich II., der Große, König von Preußen  1712-1786}}\hfillplus{[34]}\newline Sonate, SpiF 55 - E-Dur\newline fl, bc
\par \begin{itshape}[caption title:] No 161 Solo per il Flauto Traverso [on the right:] di Frederico\end{itshape} 
\par \textcolor{darkblue}{\ding{\numexpr181 + 01}}  Partitur: f.3r-4v\newline \begin{small} Der Schluss des 3. Satzes steht in den beiden untersten Akkoladen von f.3r\end{small} \newline Autograph
\par 1.1.1  fl, \begin{itshape}Arioso\end{itshape}, E-Dur  
\begin{filecontents*}{34-1.code}
@clef:G-2
@keysig:xFCGD
@timesig:3/8
@data:{6.B3''G}{6G8F3ED}/{6Et3DC}8'B-/{6''CE'B''E'A''F}/'8{AtG}-/
\end{filecontents*}
\commandline{ if [ ! -f 34-1.svg ]; then verovio --spacing-non-linear=0.54 -w 1500 --spacing-system=0.5 --adjust-page-height -b 0 34-1.code; fi } 
\newline \includesvg[width=209pt]{34-1}%
\par 1.1.2  b, E-Dur  
\begin{filecontents*}{34-2.code}
@clef:F-4
@keysig:xFCGD
@timesig:3/8
@data:8,{EEF}/{GG}-/{AGF}/{6.3E,,EGB,E,,B}/
\end{filecontents*}
\commandline{ if [ ! -f 34-2.svg ]; then verovio --spacing-non-linear=0.54 -w 1500 --spacing-system=0.5 --adjust-page-height -b 0 34-2.code; fi } 
\newline \includesvg[width=209pt]{34-2}%
\par 1.2.1  fl, \begin{itshape}Allegro\end{itshape}, E-Dur  
\begin{filecontents*}{34-3.code}
@clef:G-2
@keysig:xFCGD
@timesig:c
@data:''{8B6AG}8{FE}gE4.Dt8E/'{8.A3B''C}'{8BA}{G''GFE}/'{8.A3B''C}'{8BA}{G''BAG}/
\end{filecontents*}
\commandline{ if [ ! -f 34-3.svg ]; then verovio --spacing-non-linear=0.54 -w 1500 --spacing-system=0.5 --adjust-page-height -b 0 34-3.code; fi } 
\newline \includesvg[width=209pt]{34-3}%
\par 1.3.1  fl, \begin{itshape}Allegro assai\end{itshape}, E-Dur  
\begin{filecontents*}{34-4.code}
@clef:G-2
@keysig:xFCGD
@timesig:2/4
@data:8B/q6''F{8E6DE}{'8B''G}/q6G8{F6EF}'8{B''D}/q6F{8E6DE}'8{B''G}/
\end{filecontents*}
\commandline{ if [ ! -f 34-4.svg ]; then verovio --spacing-non-linear=0.54 -w 1500 --spacing-system=0.5 --adjust-page-height -b 0 34-4.code; fi } 
\newline \includesvg[width=209pt]{34-4}%
\par Comment on scoring: cemb: bc
\par RISM-ID: 450111002\newline D-Cv  A.I,179,(1),17\newline $\rightarrow$ In Sammlung 32 (450111000)
      
\par \vspace{16pt} \textcolor{darkblue}{\textbf{Friedrich II., der Große, König von Preußen  1712-1786}}\hfillplus{[35]}\newline Sonate, SpiF 18 - D-Dur\newline fl, bc
\par \begin{itshape}[caption title:] No 123 Solo per il Flauto Traverso [on the right:] di Federico.\end{itshape} 
\par \textcolor{darkblue}{\ding{\numexpr181 + 01}}  Partitur: f.5r-6v\newline Partial autograph
\par 1.1.1  fl, \begin{itshape}Adagio et Cantabile\end{itshape}, D-Dur  
\begin{filecontents*}{35-1.code}
@clef:G-2
@keysig:xFC
@timesig:3/8
@data:''{6.DgE3CgC6.D3'A}{6.B3''G}/'q8A{6.Gt5FG8F}-/
\end{filecontents*}
\commandline{ if [ ! -f 35-1.svg ]; then verovio --spacing-non-linear=0.54 -w 1500 --spacing-system=0.5 --adjust-page-height -b 0 35-1.code; fi } 
\newline \includesvg[width=209pt]{35-1}%
\par 1.1.2  b, D-Dur  
\begin{filecontents*}{35-2.code}
@clef:F-4
@keysig:xFC
@timesig:3/8
@data:8{,DFG}/{CD}-/{FGxG}/{A,,A,F}/
\end{filecontents*}
\commandline{ if [ ! -f 35-2.svg ]; then verovio --spacing-non-linear=0.54 -w 1500 --spacing-system=0.5 --adjust-page-height -b 0 35-2.code; fi } 
\newline \includesvg[width=209pt]{35-2}%
\par 1.2.1  fl, \begin{itshape}Allegro\end{itshape}, D-Dur  
\begin{filecontents*}{35-3.code}
@clef:G-2
@keysig:xFC
@timesig:c
@data:''{6FG}/8A'''4D6{C''B}{8A6GF}{8ED}/{6BGAF}{GEFD}{8CD}4-/
\end{filecontents*}
\commandline{ if [ ! -f 35-3.svg ]; then verovio --spacing-non-linear=0.54 -w 1500 --spacing-system=0.5 --adjust-page-height -b 0 35-3.code; fi } 
\newline \includesvg[width=209pt]{35-3}%
\par 1.3.1  fl, \begin{itshape}Presto\end{itshape}, D-Dur  
\begin{filecontents*}{35-4.code}
@clef:G-2
@keysig:xFC
@timesig:2/4
@data:6''{FG}/8A4B'''8C+/C4D''8D/{G6AB}{8GE}/{CD}4-/
\end{filecontents*}
\commandline{ if [ ! -f 35-4.svg ]; then verovio --spacing-non-linear=0.54 -w 1500 --spacing-system=0.5 --adjust-page-height -b 0 35-4.code; fi } 
\newline \includesvg[width=209pt]{35-4}%
\par Nur Satz 1 auf f.5r ist autograph, die beiden Folgesätze von Kopistenhand, jedoch nicht wie auf f.5v unten angegeben von Quantz: {\textquotedbl}diese Noten soll H. Quantz geschrieben haben.{\textquotedbl}
\par Comment on scoring: cemb: bc\newline Quantz, Johann Joachim  (Sonstige)
\par RISM-ID: 450111003\newline D-Cv  A.I,179,(1),17\newline $\rightarrow$ In Sammlung 32 (450111000)
      
\par \vspace{16pt} \textcolor{darkblue}{\textbf{Gluck, Christoph Willibald  1714-1787}}\hfillplus{[36]}\newline Alceste
\par \begin{itshape}Alceste. Tragedia. Messa in musica dal Cristoforo Gluck. IN VIENNA, NELLA STAMPARIA AULICA DI GIOVANNI TOMASO DE TRATTNERN. MDCCLXIX.\end{itshape} 
\par \textcolor{darkblue}{\ding{\numexpr181 + 01}}  1 Partitur: 233 p.\newline Musikdruck  1769\newline Trattner, Johann Thomas von  (pbl)
\par RISM-ID: 00000990021791\newline A-Sm; A-Wgm; A-Wmi; A-Wn; A-Wn-h; A-Wst; B-Bc; B-Br; CDN-Vu; CH-BEl; CH-Bm; CH-Wk; CZ-K; CZ-Pk; CZ-Pnm; D-B; D-Cv; D-Dl  Mus.3030.F.36.a; D-F  Mus Wf 15; D-Hs; D-LEm; D-LÜh; D-Mbs; D-MZfederhofer; D-Rp; DK-Kk; DK-Kmk; E-Boc; F-Pc; F-Po; F-Sim; GB-Cpl; GB-Er; GB-Lbl; GB-Lcm; GB-Mp; H-Bl; H-Bn; I-Bc; I-Fc; I-Li; I-MOe; I-Nc; I-OS; I-Tn; I-TScon; NL-DHgm; PL-Cb; RUS-Mrg; S-Skma; S-Smf; S-Sr  Näs gårdsarkiv; S-St; US-Bc; US-BEm; US-BUu; US-CA; US-NH; US-NO; US-NYcu; US-NYp; US-NYpm; US-PRu; US-R; US-U; US-Wc
\par \vspace{16pt} \textcolor{darkblue}{\textbf{Gluck, Christoph Willibald  1714-1787}}\hfillplus{[37]}\newline Iphigénie en Tauride. Excerpts. Arr, WotG 46\newline V (4)
\par \begin{itshape}[without title]\end{itshape} 
\par \textcolor{darkblue}{\ding{\numexpr181 + 01}}  Partitur: 1f.; 15 x 35,5 cm\newline \begin{small} Prägestempel EA mit Krone\end{small} \newline Abschrift  1840-1884\newline Schreiber: Costa, Michele
\par 1.1.1  S, G-Dur\newline \begin{footnotesize} [Chaste fille de Latone] \end{footnotesize}  
\begin{filecontents*}{37-1.code}
@clef:G-2
@keysig:xF
@timesig:c
@data:2B''D/CE/nFF/E4D-/
\end{filecontents*}
\commandline{ if [ ! -f 37-1.svg ]; then verovio --spacing-non-linear=0.54 -w 1500 --spacing-system=0.5 --adjust-page-height -b 0 37-1.code; fi } 
\newline \includesvg[width=209pt]{37-1}%
\par Hinweis auf der Mappe: Versetzung der Hymne für Frauenstimmen aus Gluck’s Iphigenia in einen vierstimmigen Satz zum Behufe eines Concertes in Buckingham Palace.
\par Die Bearbeitung ist ein Ausschnitt aus dem Chor {\textquotedbl}Chaste fille de Latone{\textquotedbl} im 4. Akt.\newline Costa, Michele  (arr)\newline Guillard, Nicolas-François  (Textdichter)
\par RISM-ID: 450109881\newline D-Cv  A.V,1135,(2),1
\par \vspace{16pt} \textcolor{darkblue}{\textbf{Haydn, Michael  1737-1806}}\hfillplus{[38]}\newline Domine Deus salutis meae, MH 827 - G-Dur\newline V (4), orch, org
\par \begin{itshape}[caption title:] Offertorium de omni tempore. A 4. Voci in Canone: 2VV\textsuperscript{n}\textsuperscript{i}, Viola, 2. Flauti, 2 Corni, 2 Clarini, Timp. Organo e Bassi. | di G.M. Haydn.\end{itshape} 
\par \textcolor{darkblue}{\ding{\numexpr181 + 01}}  Partitur: 32p.; 23 x 31,5 cm\newline \begin{small} p.31-32 only blank staves\end{small} \newline Autograph\newline Wasserzeichen: VC [countermark: 3 stars on crowned shield]  1803
\par 1.1.1  vl 1, \begin{itshape}Adagio cantabile\end{itshape}, G-Dur  
\begin{filecontents*}{38-1.code}
@clef:G-2
@keysig:xF
@timesig:c/
@data:2.B''{8CD}/2D'4FF/GBA''C/4'B--''E/
\end{filecontents*}
\commandline{ if [ ! -f 38-1.svg ]; then verovio --spacing-non-linear=0.54 -w 1500 --spacing-system=0.5 --adjust-page-height -b 0 38-1.code; fi } 
\newline \includesvg[width=209pt]{38-1}%
\par 1.1.2  S, G-Dur\newline \begin{footnotesize} Domine Deus salutis meae, in die clamavi \end{footnotesize}  
\begin{filecontents*}{38-2.code}
@clef:C-1
@keysig:xF
@timesig:c/
@data:=8/2.B8''CD/2D'4FF/GBA''C/'B--E/
\end{filecontents*}
\commandline{ if [ ! -f 38-2.svg ]; then verovio --spacing-non-linear=0.54 -w 1500 --spacing-system=0.5 --adjust-page-height -b 0 38-2.code; fi } 
\newline \includesvg[width=209pt]{38-2}%
\par Datierung am Ende der Partitur: {\textquotedbl}d: 23. Aug. [1]803.{\textquotedbl}
\par Comment on scoring: org: bc
\par RISM-ID: 450109866\newline D-Cv  A.V,1110,(2),1
\par \vspace{16pt} \textcolor{darkblue}{\textbf{Himmel, Friedrich Heinrich  1765-1814}}\hfillplus{[39]}\newline Quand sur les ailes des plaisirs - C-Dur\newline S, pf
\par \begin{itshape}[caption title:] Romance, p. Himmel\end{itshape} 
\par \textcolor{darkblue}{\ding{\numexpr181 + 01}}  Partitur: 1f.; 19,5 x 33,5 cm\newline \begin{small} Das Blatt ist zusammengeklebt mit einem zweiten leeren Blatt (20,5 x 33,5 cm); Adler-Wassserzeichen nur auf dem angehefteten Blatt\end{small} \newline Possible autograph manuscript\newline Wasserzeichen: [capital letters?]; [eagle with sceptre, crowned, on breast shield:] J / W  1800-1850
\par 1.1.1  pf, \begin{itshape}Moderato\end{itshape}, C-Dur  
\begin{filecontents*}{39-1.code}
@clef:G-2
@keysig:
@timesig:c
@data:''2.C4-/'8-E-E-F-F/-E-E-{EG''C}/EEDC}{'BAGF}/
\end{filecontents*}
\commandline{ if [ ! -f 39-1.svg ]; then verovio --spacing-non-linear=0.54 -w 1500 --spacing-system=0.5 --adjust-page-height -b 0 39-1.code; fi } 
\newline \includesvg[width=209pt]{39-1}%
\par 1.1.2  S, C-Dur\newline \begin{footnotesize} Quand sur les ailes des plaisirs \end{footnotesize}  
\begin{filecontents*}{39-2.code}
@clef:G-2
@keysig:
@timesig:c
@data:1-/4G8.6GGAGAB/''2C4-{8.C6D}/4E8DC8.6'{BA}GF/6{GFEF}4E
\end{filecontents*}
\commandline{ if [ ! -f 39-2.svg ]; then verovio --spacing-non-linear=0.54 -w 1500 --spacing-system=0.5 --adjust-page-height -b 0 39-2.code; fi } 
\newline \includesvg[width=209pt]{39-2}%
\par f.1r leer, nur unten links Bleistiftbemerkung: {\textquotedbl}Eckschläger? Himmel V,1111;3{\textquotedbl}
\par Im Katalog in D-Cv ist das Werk irrtümlich unter dem Namen Carl Maria von Webers eingetragen.
\par Das Werk ist die No.2 aus {\textquotedbl}DEUX ROMANCES ...{\textquotedbl}  (RISM A/I: H 5431) von Himmel. (Für die Identifizierung bedanken wir uns bei Herrn Frank Ziegler, Berlin, von der Carl-Maria-von-Weber-Gesamtausgabe.)\newline Canicoff, Basil von  (Textdichter)\newline Weber, Carl Maria von  (cmp)
\par RISM-ID: 450109859\newline D-Cv  A.V,1111,(2),3
\par \vspace{16pt} \textcolor{darkblue}{\textbf{Kerll, Johann Caspar  1627-1693}}\hfillplus{[40]}\newline Missa superba. Excerpts. Arr, BWV 241 - D-Dur\newline Coro (2), orch, bc
\par \begin{itshape}[title page:] Instrumental | Begleitung zu S. Bachs | Sanctus - in original Handschrift | 2 Violinen I\textsuperscript{m} und II in E dur | 3 Violen obl. | 1 Violoncello | 1 Violon | 1 Organo in D | 2 Cembalo in E | 2 Hautbois d'Amore in E | 1 Bassono oder Fagott. [later added with red ink:]  original Handschrift | all diese Stimmen sind einen Tonhöher (exc. organo) | nemlich (aus D) aus E versetzet geschriben\end{itshape} 
\par \textcolor{darkblue}{\ding{\numexpr181 + 01}}  12 Stimmen: vl 1, 2, vla 1, 2, 3, vlc, vlne, ob d'amore 1, 2, fag, cemb (= b.fig), org (= b.fig)  (f.1r, 1r, 1r, 1r, 1r, 1r, 1r, 1r, 1r, 1r, 1r, 1r, 1r)\newline \begin{small} vlc: {\textquotedbl}Violoncello senza Violono{\textquotedbl}; fag: {\textquotedbl}Bassono{\textquotedbl}\end{small} \newline Abschrift  17470701-17480831\newline Schreiber: Bach, Johann Sebastian
\par \textcolor{darkblue}{\ding{\numexpr181 + 02}}  8 Stimmen: Coro 1: S, A, T, B, Coro 2: S, A, T, B  (1, 1, 1, 1, 1, 1, 1, 1f.); 17 (17,5) x 22 (21) cm\newline \begin{small} Title on material: [title page:] Singstimen | zu S. Bachs: Sanctus | in original Handschrifft | 2 Dsc | 2 Alt | 2 Tenor | 2 Bass [on the right, bracket and:] obligat\end{small} \newline \begin{small} Die Vokalstimmen sind für die beiden Chöre jeweils in einem Heft zusammen gebunden, zur Zeit dieser Titelaufnahme (19.03.2012) ist die Reihenfolge der einzelnen Blätter bei Chor 1: S, T, A, B, bei Chor 2: T, A, S, B\end{small} \newline Abschrift\newline Schreiber: Bach, Johann Sebastian
\par 1.1.1  vla 3, D-Dur  
\begin{filecontents*}{40-1.code}
@clef:C-3
@keysig:xFC
@timesig:c
@data:4-4.D{8EFG}/2A4F-/
\end{filecontents*}
\commandline{ if [ ! -f 40-1.svg ]; then verovio --spacing-non-linear=0.54 -w 1500 --spacing-system=0.5 --adjust-page-height -b 0 40-1.code; fi } 
\newline \includesvg[width=209pt]{40-1}%
\par 1.1.2  vlc, D-Dur  
\begin{filecontents*}{40-2.code}
@clef:F-4
@keysig:xFC
@timesig:c
@data:,4DDDD/DDDD/
\end{filecontents*}
\commandline{ if [ ! -f 40-2.svg ]; then verovio --spacing-non-linear=0.54 -w 1500 --spacing-system=0.5 --adjust-page-height -b 0 40-2.code; fi } 
\newline \includesvg[width=209pt]{40-2}%
\par 1.1.3  Coro 1: A, D-Dur\newline \begin{footnotesize} Sanctus, Dominus Deus Sabaoth \end{footnotesize}  
\begin{filecontents*}{40-3.code}
@clef:C-3
@keysig:xFC
@timesig:c
@data:4-4.D{8EFG}/2A4F-/
\end{filecontents*}
\commandline{ if [ ! -f 40-3.svg ]; then verovio --spacing-non-linear=0.54 -w 1500 --spacing-system=0.5 --adjust-page-height -b 0 40-3.code; fi } 
\newline \includesvg[width=209pt]{40-3}%
\par Die 12 Instrumentalstimmen sind in einem Heft zusammengebunden. Die Vokalstimmen sind in je einem weiteren Heft zusammengebunden (hier als eigene Materialart aufgenommen). Die zugehörige Partitur hat eine eigene Signatur, A.V,1109,(1),1a, und wurde deshalb in dieser Datenbank als eigener Datensatz aufgenommen, vgl. RISM ID no. 450106268.
\par Orgel in D-Dur (Chorton-Notierung), alles übrige in E-Dur.
\par Comment on scoring: cemb, org: bc\newline Bach, Carl Philipp Emanuel  (Vorbesitzer)\newline Bach, Johann Sebastian  (arr)\newline Kittel, Johann Christian  (Vorbesitzer)\newline Poelchau, Georg Johann Daniel  (Vorbesitzer)\newline Literatur: DavidB 1961  p.199-223
\par RISM-ID: 450106266\newline D-Cv  A.V,1109,(1),1b\newline $\rightarrow$ In Sammlung 79 (450106265)
      
\par \vspace{16pt} \textcolor{darkblue}{\textbf{Kerll, Johann Caspar  1627-1693}}\hfillplus{[41]}\newline Missa superba. Excerpts. Arr, BWV 241 - D-Dur\newline Coro (2), orch, bc
\par \begin{itshape}[dust cover title:] N\textsuperscript{o} 638.\textsuperscript{b} [crossed out:] N.150. | Sanctus. | ab 8 Vocibu[us] | [later added:] Aus hn[?] Bachs Auction | 2 Hautb. d'Amore | 2 Violini | 3 Viole | Bassono | Violoncello | Violono | Cembalo | e | l'Organo.\end{itshape} 
\par \textcolor{darkblue}{\ding{\numexpr181 + 01}}  Partitur: 4f.; 33 x 20,5 cm\newline \begin{small} f.4v blank\end{small} \newline \begin{small} State of preservation: stark angegriffen von Tintenfraß und häufiger Benutzung\end{small} \newline Autograph\newline Wasserzeichen: [sachsen-shield]  17470701-17480831\newline Schreiber: Bach, Johann Sebastian
\par 1.1.1  Coro 1: A, D-Dur\newline \begin{footnotesize} Sanctus, Dominus Deus Sabaoth \end{footnotesize}  
\begin{filecontents*}{41-1.code}
@clef:C-3
@keysig:xFC
@timesig:c
@data:4-4.D{8EFG}/2A4F-/
\end{filecontents*}
\commandline{ if [ ! -f 41-1.svg ]; then verovio --spacing-non-linear=0.54 -w 1500 --spacing-system=0.5 --adjust-page-height -b 0 41-1.code; fi } 
\newline \includesvg[width=209pt]{41-1}%
\par 1.1.2  Coro 2: B, D-Dur\newline \begin{footnotesize} Sanctus, Dominus Deus Sabaoth \end{footnotesize}  
\begin{filecontents*}{41-2.code}
@clef:F-4
@keysig:xFC
@timesig:c
@data:1,D+/D/,,A+/A/
\end{filecontents*}
\commandline{ if [ ! -f 41-2.svg ]; then verovio --spacing-non-linear=0.54 -w 1500 --spacing-system=0.5 --adjust-page-height -b 0 41-2.code; fi } 
\newline \includesvg[width=209pt]{41-2}%
\par Die zugehörigen Instrumentalstimmen sind in D-Cv unter der Signatur A.V,1109,(1),1b zusammengefasst, die Singstimmen unter A.V,1109,(1),1c, vgl. RISM ID no. 450106266.\newline Bach, Carl Philipp Emanuel  (Vorbesitzer)\newline Bach, Johann Sebastian  (arr)\newline Kittel, Johann Christian  (Vorbesitzer)\newline Poelchau, Georg Johann Daniel  (Vorbesitzer)\newline Literatur: DavidB 1961  p.199-223
\par RISM-ID: 450106268\newline D-Cv  A.V,1109,(1),1a
\par \vspace{16pt} \textcolor{darkblue}{\textbf{Liszt, Franz  1811-1886}}\hfillplus{[42]}\newline Chorsatz. Fragment - a-Moll\newline V (4)
\par \begin{itshape}[caption title:] Franz Liszt\end{itshape} 
\par \textcolor{darkblue}{\ding{\numexpr181 + 01}}  Partitur: 1f; 7,5 x 8,5 cm\newline Autograph
\par 1.1.1  S 1, a-Moll\newline \begin{footnotesize} [...] wohl nein \end{footnotesize}  
\begin{filecontents*}{42-1.code}
@clef:C-1
@keysig:
@timesig:3/4
@data:''2.C+/C/+4C--/-xC-//
\end{filecontents*}
\commandline{ if [ ! -f 42-1.svg ]; then verovio --spacing-non-linear=0.54 -w 1500 --spacing-system=0.5 --adjust-page-height -b 0 42-1.code; fi } 
\newline \includesvg[width=209pt]{42-1}%
\par Die Besetzung ist nicht eindeutig zu klären. Angenommen wurden je ein C-1-Schlüssel in den beiden oberen Systemen, ein C-4-Schlüssel im dritten System und im unteren System ein Bassschlüssel.
\par Das Wort {\textquotedbl}alle{\textquotedbl} über allen 4 Systemen deutet darauf hin, dass die ersten 3 Takte solistisch auszuführen sind und nur der Schlussakkord des Ausschnittes chorisch.
\par Comment on scoring: Besetzung unsicher
\par RISM-ID: 450109870\newline D-Cv  A.V,1130,(1),5
\par \vspace{16pt} \textcolor{darkblue}{\textbf{Marschner, Heinrich August  1795-1861}}\hfillplus{[43]}\newline Der Goldschmied von Ulm. Excerpts. Arr, op.174 - c-Moll\newline pf
\par \begin{itshape}[title page:] Ouverture | zu dem Volksmärchen | Der Goldschmied von Ulm | v. H.S. Mosenthal | componirt | von | Heinrich Marschner. | op. 174.\end{itshape} 
\par \textcolor{darkblue}{\ding{\numexpr181 + 01}}  Stimme: pf  (12p.); 36,5 x 25 cm\newline \begin{small} p.12 only blank staves\end{small} \newline Autograph
\par 1.1.1  pf, \begin{itshape}Andante grave\end{itshape}, c-Moll  
\begin{filecontents*}{43-1.code}
@clef:F-4
@keysig:bBEA
@timesig:3/4
@data:,2.G/2xC+{8.C6D}/8D-2G/2xC+{8.C6D}/8D-4G{8.D6E}/
\end{filecontents*}
\commandline{ if [ ! -f 43-1.svg ]; then verovio --spacing-non-linear=0.54 -w 1500 --spacing-system=0.5 --adjust-page-height -b 0 43-1.code; fi } 
\newline \includesvg[width=209pt]{43-1}%
\par Gedrucktes Notenpapier mit 14 Notenzeilen {\textquotedbl}Adolph Nagel in Hannover.{\textquotedbl}\newline Mosenthal, Salomon Hermann von  (Textdichter)
\par RISM-ID: 450109867\newline D-Cv  A.V,1122,(1),3
\par \vspace{16pt} \textcolor{darkblue}{\textbf{Meyerbeer, Giacomo  1791-1864}}\hfillplus{[44]}\newline Festlied - B-Dur\newline V (3), Coro, orch
\par \begin{itshape}[label on cover:] Festlied | zum | 2\textsuperscript{t}\textsuperscript{e}\textsuperscript{n} Januar 1840 | componirt | von Giacomo Meÿerbeer. | Man. propr.\end{itshape} 
\par \textcolor{darkblue}{\ding{\numexpr181 + 01}}  Partitur: 10f.; 27 x 36,5 cm\newline \begin{small} f.10v only blank staves\end{small} \newline Autograph  18391201-18400102
\par 1.1.1  vl 1, \begin{itshape}Allegro ben moderato\end{itshape}, B-Dur  
\begin{filecontents*}{44-1.code}
@clef:G-2
@keysig:bBE
@timesig:c
@data:2''B8.6-{DFB}/8A-'({8FFF})F-4-/''2G8.6-{ECF}/8D-'(8{BBB})B-4-/
\end{filecontents*}
\commandline{ if [ ! -f 44-1.svg ]; then verovio --spacing-non-linear=0.54 -w 1500 --spacing-system=0.5 --adjust-page-height -b 0 44-1.code; fi } 
\newline \includesvg[width=209pt]{44-1}%
\par 1.1.2  T, \begin{itshape}Allegretto maestoso\end{itshape}, B-Dur\newline \begin{footnotesize} Es braust und wogt des Volkes Drang \end{footnotesize}  
\begin{filecontents*}{44-2.code}
@clef:C-4
@keysig:bBE
@timesig:c
@data:=17//=2/4-{8.,B6F}'8D-C-/,B-{8.B6F}'8D-C-/,B
\end{filecontents*}
\commandline{ if [ ! -f 44-2.svg ]; then verovio --spacing-non-linear=0.54 -w 1500 --spacing-system=0.5 --adjust-page-height -b 0 44-2.code; fi } 
\newline \includesvg[width=209pt]{44-2}%
\par 1.2.1  S, \begin{itshape}2|t|e Strophe | Un Sopran Solo.\end{itshape}, B-Dur\newline \begin{footnotesize} Hier wo der Kunst ein gastlich Dach \end{footnotesize}  
\begin{filecontents*}{44-3.code}
@clef:C-1
@keysig:bBE
@timesig:c
@data:4-'{8.B'6F}8''D-C-/'B-{8.B6F}8''D-C-/'B
\end{filecontents*}
\commandline{ if [ ! -f 44-3.svg ]; then verovio --spacing-non-linear=0.54 -w 1500 --spacing-system=0.5 --adjust-page-height -b 0 44-3.code; fi } 
\newline \includesvg[width=209pt]{44-3}%
\par 1.3.1  T, \begin{itshape}3|t|e Strophe. Andantino cantabile\end{itshape}, Es-Dur\newline \begin{footnotesize} Mein Oheim, tönt’s am Tajo fern \end{footnotesize}  
\begin{filecontents*}{44-4.code}
@clef:C-4
@keysig:bBEA
@timesig:6/8
@data:=6/8--,E4B8G/4A8'CD+{6DC},{BA}/8G-
\end{filecontents*}
\commandline{ if [ ! -f 44-4.svg ]; then verovio --spacing-non-linear=0.54 -w 1500 --spacing-system=0.5 --adjust-page-height -b 0 44-4.code; fi } 
\newline \includesvg[width=209pt]{44-4}%
\par Bericht von der Uraufführung am 02.01.1840, dem 56. Geburtstag Herzog Ernsts I., in den {\textquotedbl}Vorzimmern Se. Herzoglichen Durchlaucht{\textquotedbl} in der Gothaischen Zeitung vom 04.01.1840; zitiert bei BöckingA 2006, p.92.
\par In D-Cv, Inv.-Nr. V,1126,5, autographer Begleitbrief Meyerbeers an den Gothaer Theaterdichter und Hofrat Johann Heinrich Millenet vom 20.12.1839 aus Baden. Darin u.a.: {\textquotedbl}Leider kann ich, durch den späten Entschluss auf wenige Stunden beschränkt, nur ein kleines Festlied darbieten.{\textquotedbl} Faksimile und Transkription des Briefes bei BöckingA 2006, p.89-91.\newline Ernst I., Herzog von Sachsen-Coburg-Saalfeld  (Widmungsträger)\newline Millenet, Johann Heinrich  (Sonstige)\newline Millenet, Johann Heinrich  (Textdichter)\newline Literatur: BöckingA 2006  p.86-93\newline Olim: M. N|o 72.
\par RISM-ID: 450109868\newline D-Cv  A.V,1126,(2),7
\par \vspace{16pt} \textcolor{darkblue}{\textbf{Mozart, Wolfgang Amadeus  1756-1791}}\hfillplus{[45]}\newline Ah se in ciel benigne stelle, KV 538 - F-Dur\newline S, orch
\par \begin{itshape}[caption title, on the left:] per la Sig\textsuperscript{r}\textsuperscript{a} Lange. [centre:] Aria [on the right:] Vienna li 4 di Marzo. 1788. di Wolfg: | Amd: Mozartmp. [later added, on the left:] 4.März | 1788 N.26 [crossed out and corrected in:] N.9. [centre:] vollständig | S. [on the right:]  Eigne | Handschrift.\end{itshape} 
\par \textcolor{darkblue}{\ding{\numexpr181 + 01}}  Partitur: 14f.; 23 x 32,5 cm\newline \begin{small} f.14 blank\end{small} \newline Autograph\newline Wasserzeichen: VA [crowned, countermark: 3 crescents, decreasing]  17880304
\par 1.1.1  vl 1, \begin{itshape}Allegro\end{itshape}, F-Dur  
\begin{filecontents*}{45-1.code}
@clef:G-2
@keysig:bB
@timesig:c
@data:''8C-4.C{6DC}{8'BA}/''F-4F+{6FEAG}{8FE}/
\end{filecontents*}
\commandline{ if [ ! -f 45-1.svg ]; then verovio --spacing-non-linear=0.54 -w 1500 --spacing-system=0.5 --adjust-page-height -b 0 45-1.code; fi } 
\newline \includesvg[width=209pt]{45-1}%
\par 1.1.2  S, F-Dur\newline \begin{footnotesize} Ah se in ciel benigne stelle \end{footnotesize}  
\begin{filecontents*}{45-2.code}
@clef:C-1
@keysig:bB
@timesig:c
@data:=23/''2.C'8{BA}/''2.F4E/2D+{8DxC}{DnC}/{C'B}4A2-/
\end{filecontents*}
\commandline{ if [ ! -f 45-2.svg ]; then verovio --spacing-non-linear=0.54 -w 1500 --spacing-system=0.5 --adjust-page-height -b 0 45-2.code; fi } 
\newline \includesvg[width=209pt]{45-2}%\newline André, Johann Anton  (Vorbesitzer)\newline André, Julius  (Vorbesitzer)\newline Metastasio, Pietro  (Textdichter)\newline Nissen, Georg Nikolaus von  (Vorbesitzer)\newline Weber, Aloysia  (Widmungsträger)\newline Literatur: NMA  1/33/2, no.95; NMA  2/7/2, p.57-78; NMA  2/7/4 (Kritischer Bericht), p.102-108
\par RISM-ID: 450106275\newline D-Cv  A.V,1109,(2),3
\par \vspace{16pt} \textcolor{darkblue}{\textbf{Mozart, Wolfgang Amadeus  1756-1791}}\hfillplus{[46]}\newline Il Burbero di buon cuore. Inserts, KV 582 - C-Dur\newline S, orch
\par \begin{itshape}[caption title:] N.24. [or {\textquotedbl}14.{\textquotedbl}?] Vollständig. Von Mozart und seiner Handschrift [right, beside stave:] october 1789\end{itshape} 
\par \textcolor{darkblue}{\ding{\numexpr181 + 01}}  Partitur: 6f.; 23,5 x 32 cm\newline Autograph\newline Wasserzeichen: [3 crescents (increasing)] / REAL [countermark: shield with:] W  17891001-17891031
\par 1.1.1  vl 1, \begin{itshape}Andante\end{itshape}, C-Dur  
\begin{filecontents*}{46-1.code}
@clef:G-2
@keysig:
@timesig:c/
@data:6'G/4G{8E6-''C}4C'{8G6-''E}/{8.F6D}'4B2-/
\end{filecontents*}
\commandline{ if [ ! -f 46-1.svg ]; then verovio --spacing-non-linear=0.54 -w 1500 --spacing-system=0.5 --adjust-page-height -b 0 46-1.code; fi } 
\newline \includesvg[width=209pt]{46-1}%
\par 1.1.2  S, C-Dur\newline \begin{footnotesize} Chi sà qual sia \end{footnotesize}  
\begin{filecontents*}{46-2.code}
@clef:C-1
@keysig:
@timesig:c/
@data:6-/=1/2-4-8.-6G/4G8.E''6C4C'8.G''6E/{8.F6D}'4B-
\end{filecontents*}
\commandline{ if [ ! -f 46-2.svg ]; then verovio --spacing-non-linear=0.54 -w 1500 --spacing-system=0.5 --adjust-page-height -b 0 46-2.code; fi } 
\newline \includesvg[width=209pt]{46-2}%
\par f.1r, unten, mit Bleistift: {\textquotedbl}55.{\textquotedbl}
\par Beigelegter Zettel (13 x 21 cm) mit folgender Notiz: C dur 4/4 N\textsuperscript{o} 7. 12 Seiten à 10 Linien | Arie in die Oper Il Burbero mit Orchesterbegleitung. | Diese Sopran-Arie der Madame Lucilla ist von Mozart als | Einlage zu einer 1789 beliebten Oper schrieben und reiht | sich in stil und behandlung den andern Compositionen | als Meister aus jener Periode an. Clarinetten u. 1stes Fagott | sind ziemlig obligat behandelt wie dieß Mozart bei | vielen andern Gelegenheiten gethan hat. | Preis 3 Frd'or. | M. 74.\newline André, Johann Anton  (Vorbesitzer)\newline André, Julius  (Vorbesitzer)\newline Martín y Soler, Vicente  (cmp)\newline Nissen, Georg Nikolaus von  (Vorbesitzer)\newline Literatur: NMA  1/33/2, no.66; NMA  2/7/4, p.105-114; NMA  2/7/4 (Kritischer Bericht), p.119-121
\par RISM-ID: 450106273\newline D-Cv  A.V,1109,(2),1
\par \vspace{16pt} \textcolor{darkblue}{\textbf{Mozart, Wolfgang Amadeus  1756-1791}}\hfillplus{[47]}\newline La Clemenza di Tito. Excerpts. Arr, KV 621\newline V (2), strings
\par \begin{itshape}[caption title, f.1r:] Jn eine Oper [last 2 words crossed out] Clemenza di Tito, welches für die Sänger ihrer Stimmen wegen | hatte abgeändert werden müssen [on the right, later added:] Mozarts Handschrift\newline [caption title, f.3r:] Zur Clemenza di Tito [on the right, later added:] Mozarts Handschrift\newline [caption title, f.4r:] Clemenza di Tito anders componirt als die Arie bekannt ist.  [on the right, later added:] Mozarts Handschrift [with pencil, overwritten:] ganz anders\end{itshape} 
\par \textcolor{darkblue}{\ding{\numexpr181 + 01}}  Partitur: 6f.; 23 x 31 (31,5) cm\newline \begin{small} f.3v, 6v blank\end{small} \newline Autograph\newline Wasserzeichen: CS / C [countermark: 3 crescents decreasing] / REAL; [bow] / MA [countermark: 3 crescents increasing] / REAL  17910701-17910906
\par 1.1.1  T, \begin{itshape}[No 1 Duetto]. Andante\end{itshape}, F-Dur\newline \begin{footnotesize} [Come ti piace imponi] \end{footnotesize}  
\begin{filecontents*}{47-1.code}
@clef:C-4
@keysig:bB
@timesig:c
@data:'{8C,B}4B-4-8-B/4.B8'C{DE}{nFE}/{ED}4D-D/8.6{EC}{FD}4.(G)qq{6AGFE}r{6DC}/q4E2D4-
\end{filecontents*}
\commandline{ if [ ! -f 47-1.svg ]; then verovio --spacing-non-linear=0.54 -w 1500 --spacing-system=0.5 --adjust-page-height -b 0 47-1.code; fi } 
\newline \includesvg[width=209pt]{47-1}%
\par 1.2.1  vl 1, \begin{itshape}[No 23 Rondo]. Larghetto\end{itshape}, F-Dur  
\begin{filecontents*}{47-2.code}
@clef:G-2
@keysig:bB
@timesig:3/8
@data:8''{A'''C''E}/8F4C/{6.A3F8D}{6.B3G}/{8.F6G}6E-/
\end{filecontents*}
\commandline{ if [ ! -f 47-2.svg ]; then verovio --spacing-non-linear=0.54 -w 1500 --spacing-system=0.5 --adjust-page-height -b 0 47-2.code; fi } 
\newline \includesvg[width=209pt]{47-2}%
\par 1.2.2  S, F-Dur\newline \begin{footnotesize} Non più di fiori vaghe catene \end{footnotesize}  
\begin{filecontents*}{47-3.code}
@clef:C-1
@keysig:bB
@timesig:3/8
@data:=4/8'A''C'E/F4C/{6.A3F}8D{6.B3G}/{8.F6G}E-/
\end{filecontents*}
\commandline{ if [ ! -f 47-3.svg ]; then verovio --spacing-non-linear=0.54 -w 1500 --spacing-system=0.5 --adjust-page-height -b 0 47-3.code; fi } 
\newline \includesvg[width=209pt]{47-3}%
\par Titeleinträge von Nissen.
\par Die Entwürfe der beiden Nummern aus der Oper enthalten No.1 (Duetto), Takte 29-73 und No.23 (Rondo), 45 Takte. Die Takte 1-28 von No.1 sind überliefert in S-Uu, Signatur Vok.mus.hs. 133
\par Auf mehreren Seiten am unteren Rand kleiner grüner Stempel mit Wappen und Buchstaben V[este] C[oburg]
\par Am Ende der Noten, f.6r: {\textquotedbl}dal Segno ff 18 Battuta.{\textquotedbl}\newline André, Johann Anton  (Vorbesitzer)\newline Nissen, Georg Nikolaus von  (Sonstige)\newline Literatur: NMA  2/5/20, p.26-31, 265-282, 322; NMA  2/5/20 (Kritischer Bericht), p.34, 37; KonradM 1992  p.198-200; NMA  1/33/2, no.100; NMA  1/33/2, no.91; TysonC 1975  p.223-224
\par RISM-ID: 450106278\newline D-Cv  A.V,1109,(2),8
\par \vspace{16pt} \textcolor{darkblue}{\textbf{Mozart, Wolfgang Amadeus  1756-1791}}\hfillplus{[48]}\newline Männer suchen stets zu naschen, KV 433 - F-Dur\newline B, orch
\par \begin{itshape}[caption title, different hands:] N.31. [pencil:] G.g. [ink:] Original [darker ink:] Breitkopfs 5\textsuperscript{t}\textsuperscript{e}\textsuperscript{s} Heft, aber nur im Clavierauszug. | 1783. | Mozarts | Handschrift\end{itshape} 
\par \textcolor{darkblue}{\ding{\numexpr181 + 01}}  Partitur: 12p.; 23,5 x 32 cm\newline \begin{small} p.10-12 only blank staves\end{small} \newline Autograph\newline Wasserzeichen: [3 crescents (decreasing)] / REAL / A  1783
\par 1.1.1  B, F-Dur\newline \begin{footnotesize} Männer suchen stets zu naschen läßt man sie allein \end{footnotesize}  
\begin{filecontents*}{48-1.code}
@clef:F-4
@keysig:bB
@timesig:2/4
@data:=1/4-8-6'C,B/8AAAA/{8.A6B}8G-/4F+6FGAB/4A8G-/
\end{filecontents*}
\commandline{ if [ ! -f 48-1.svg ]; then verovio --spacing-non-linear=0.54 -w 1500 --spacing-system=0.5 --adjust-page-height -b 0 48-1.code; fi } 
\newline \includesvg[width=209pt]{48-1}%
\par 1.1.2  b, F-Dur  
\begin{filecontents*}{48-2.code}
@clef:F-4
@keysig:bB
@timesig:2/4
@data:4,FF/F-/8-{FAF}/4C+8C-/
\end{filecontents*}
\commandline{ if [ ! -f 48-2.svg ]; then verovio --spacing-non-linear=0.54 -w 1500 --spacing-system=0.5 --adjust-page-height -b 0 48-2.code; fi } 
\newline \includesvg[width=209pt]{48-2}%
\par Die Angabe im Titel (von Nissen) bezieht sich auf Cahier V der bei Breitkopf \& Härtel in Leipzig erschienen {\textquotedbl}Oeuvres complettes{\textquotedbl} (vgl. NMA 10/30/4, p.245).
\par Auf den Recto-Seiten unten: Rundstempel mit Krone {\textquotedbl}V C{\textquotedbl} [Veste Coburg].
\par Weitere Eintragungen auf f.1r: ganz oben links: {\textquotedbl}I.{\textquotedbl} , am rechten Rand: {\textquotedbl}211.{\textquotedbl}, unten links mit roter Tinte {\textquotedbl}21{\textquotedbl}, darunter durchgestrichen, ebenfalls mit roter Tinte: {\textquotedbl}22{\textquotedbl}
\par Die Partitur ist unvollständig. Durchgehend notiert sind nur Vokal- und Instrumentalbass, sowie vereinzelt die Violinstimmen und nur im siebtletzten Takt die Oboen.\newline André, Johann Anton  (Vorbesitzer)\newline Nissen, Georg Nikolaus von  (Vorbesitzer)\newline Literatur: NMA  1/33/2, no.56; NMA  2/7/4, p.152-153; NMA  2/7/4 (Kritischer Bericht), p.134-135; NMA  10/30/4, p.99-103, 245
\par RISM-ID: 450106272\newline D-Cv  A.V,1109,(2),10
\par \vspace{16pt} \textcolor{darkblue}{\textbf{Mozart, Wolfgang Amadeus  1756-1791}}\hfillplus{[49]}\newline Messe. Auswahl, KV 427/3 - c-Moll\newline S, orch, bc
\par \begin{itshape}[caption title:] Zum Laudamus der großen C moll Messe von 1783 /:H 210:/ [on the right, by Nissen's hand:] zu einer Vesper [?, crossed out an corrected:] Messe. | Mozarts | handschrift\end{itshape} 
\par \textcolor{darkblue}{\ding{\numexpr181 + 01}}  Partitur: 2f.; 23 x 32,5 cm\newline Autograph\newline Wasserzeichen: W [countermark: 3 crescents (decreasing)] / REAL / A  17820601-17830831
\par 1.1.1  S, C-Dur\newline \begin{footnotesize} Benedicimus te \end{footnotesize}  
\begin{filecontents*}{49-1.code}
@clef:C-1
@keysig:
@timesig:c
@data:4-2''C4C/q8xC2D+{8D6EF}8ED/q4D2C4-
\end{filecontents*}
\commandline{ if [ ! -f 49-1.svg ]; then verovio --spacing-non-linear=0.54 -w 1500 --spacing-system=0.5 --adjust-page-height -b 0 49-1.code; fi } 
\newline \includesvg[width=209pt]{49-1}%
\par 1.1.2  S, C-Dur\newline \begin{footnotesize} Glorificamus te \end{footnotesize}  
\begin{filecontents*}{49-2.code}
@clef:C-1
@keysig:
@timesig:c
@data:''4C2C{8DE}/1F+/F+/F+/2F'bE/
\end{filecontents*}
\commandline{ if [ ! -f 49-2.svg ]; then verovio --spacing-non-linear=0.54 -w 1500 --spacing-system=0.5 --adjust-page-height -b 0 49-2.code; fi } 
\newline \includesvg[width=209pt]{49-2}%
\par Titeleintrag von Johann Anton André; Zusatz von Nissen.
\par Der Partiturentwurf enthält nur die Takte 71-87 und 123-138, ausgeführt in den Stimmen S, vl 1 und b; in zweieinhalb Takten zusätzlich vl 2 und vla; in einem Takt 2 Varianten zur S-Stimme.\newline André, Johann Anton  (Vorbesitzer)\newline André, Julius  (Vorbesitzer)\newline Nissen, Georg Nikolaus von  (Sonstige)\newline Literatur: NMA  1/1/1/5, p.166-169; NMA  NMA 1/1/1/3-5 (Kritischer Bericht), p.e/7; NMA  10/33/2, no.56\newline Olim: 210
\par RISM-ID: 450106286\newline D-Cv  A.V,1109,(2),11
\par \vspace{16pt} \textcolor{darkblue}{\textbf{Mozart, Wolfgang Amadeus  1756-1791}}\hfillplus{[50]}\newline Müßt ich auch durch tausend Drachen, KV 435 - D-Dur\newline T, orch
\par \begin{itshape}[caption title:] H.h. N.2. [crossed out: N.33] Singstimme und Baß ganz, übrigens nicht ganz instrumentirt. [on the right:] Von Mozart | und | seiner Handschrift. | 1783.\end{itshape} 
\par \textcolor{darkblue}{\ding{\numexpr181 + 01}}  Partitur: 8f.; 23,5 x 32 cm\newline Autograph\newline Wasserzeichen: [3 crescents (decreasing)] / REAL / A  1783
\par 1.1.1  vl 1, \begin{itshape}Allegro con brio\end{itshape}, D-Dur  
\begin{filecontents*}{50-1.code}
@clef:G-2
@keysig:xFC
@timesig:c
@data:'''4D8-(6{DC''B})4A8-{3AGFE}/4D8-(6{DC'B})4A8-{3AGFE}/
\end{filecontents*}
\commandline{ if [ ! -f 50-1.svg ]; then verovio --spacing-non-linear=0.54 -w 1500 --spacing-system=0.5 --adjust-page-height -b 0 50-1.code; fi } 
\newline \includesvg[width=209pt]{50-1}%
\par 1.1.2  T, D-Dur\newline \begin{footnotesize} Müßt ich auch durch tausend Drachen \end{footnotesize}  
\begin{filecontents*}{50-2.code}
@clef:C-4
@keysig:xFC
@timesig:c
@data:=2/'2.4D,A/'FD/GE/4CC2-/
\end{filecontents*}
\commandline{ if [ ! -f 50-2.svg ]; then verovio --spacing-non-linear=0.54 -w 1500 --spacing-system=0.5 --adjust-page-height -b 0 50-2.code; fi } 
\newline \includesvg[width=209pt]{50-2}%
\par Die Signaturen A.V,1109,(2),4-7 waren im handschriftlichen Bandkatalog in D-Cv in einer abweichenden Reihenfolge eingetragen und durchnummeriert. Am 20.03.2012 wurden die Nummerierungen im Bandkatalog angeglichen an die Nummerierungen auf den Quellen, die auch den RISM-Titelaufnahmen zugrunde liegen. Diese Quelle trug im Bandkatalog ursprünglich die Signatur A.V,1109,(2),6.\newline André, Johann Anton  (Vorbesitzer)\newline Nissen, Georg Nikolaus von  (Vorbesitzer)\newline Literatur: NMA  1/33/2, no.56; NMA  2/7/4, p.162-167; NMA  2/7/4 (Kritischer Bericht), p.137-138; NMA  10/30/4, p.91-98, 244
\par RISM-ID: 450106271\newline D-Cv  A.V,1109,(2),4
\par \vspace{16pt} \textcolor{darkblue}{\textbf{Mozart, Wolfgang Amadeus  1756-1791}}\hfillplus{[51]}\newline Ombra felice tornerò a rivederti, KV 255 - F-Dur\newline A, orch
\par \begin{itshape}[dust dover title:] Recit. ed Aria en Rondeau. | (Ombra felice) | per il signore Fortini | Septembre 1776 | von | W. A. Mozart.\newline  [caption title, on the left:] Recit: ed Aria en Rondeau [centre, later added:] N.3. [on the right:] Del Sgr. Caval: Amadeo Wolfg: Mozart. | in Salisb: Settembre 1776. per il Sgr: Fortini. [on the right, beside accolade:] Eigne Handschrift | N\textsuperscript{o} 94.\end{itshape} 
\par \textcolor{darkblue}{\ding{\numexpr181 + 01}}  Partitur: 12f.; 16,5 x 22 cm\newline Autograph\newline Wasserzeichen: FC [countermark: 3 hats] / R  17760901-17760930
\par 1.1.1  vl 1, \begin{itshape}Recitativo. Andante\end{itshape}, F-Dur  
\begin{filecontents*}{51-1.code}
@clef:G-2
@keysig:bB
@timesig:c
@data:{8.C6D}{8CC}{6xCD}4D8D/{6EFGA}{BGFE}8F-4-/
\end{filecontents*}
\commandline{ if [ ! -f 51-1.svg ]; then verovio --spacing-non-linear=0.54 -w 1500 --spacing-system=0.5 --adjust-page-height -b 0 51-1.code; fi } 
\newline \includesvg[width=209pt]{51-1}%
\par 1.1.2  A, F-Dur\newline \begin{footnotesize} Ombra felice tornerò a rivederti \end{footnotesize}  
\begin{filecontents*}{51-2.code}
@clef:C-3
@keysig:bB
@timesig:c
@data:1-/2-'4F8C,A/BB-'6GA8BBB''C/'AA4-2-/
\end{filecontents*}
\commandline{ if [ ! -f 51-2.svg ]; then verovio --spacing-non-linear=0.54 -w 1500 --spacing-system=0.5 --adjust-page-height -b 0 51-2.code; fi } 
\newline \includesvg[width=209pt]{51-2}%
\par 1.2.1  A, \begin{itshape}Aria en Ronadeau. Andante moderato\end{itshape}, F-Dur\newline \begin{footnotesize} Io ti lascio e questo addio \end{footnotesize}  
\begin{filecontents*}{51-3.code}
@clef:C-3
@keysig:bB
@timesig:c/
@data:4'A{8BG}/4FEFG/4.C8D4C-/
\end{filecontents*}
\commandline{ if [ ! -f 51-3.svg ]; then verovio --spacing-non-linear=0.54 -w 1500 --spacing-system=0.5 --adjust-page-height -b 0 51-3.code; fi } 
\newline \includesvg[width=209pt]{51-3}%
\par Der Text ist der Oper Arsace von Mortellari entnommen.\newline De Gamerra, Giovanni  (Textdichter)\newline Fortini, Francesco  (Widmungsträger)\newline Gleissner, Franz  (Vorbesitzer)\newline Mortellari, Michele  (cmp)\newline Nissen, Georg Nikolaus von  (Vorbesitzer)\newline Literatur: NMA  1/33/2, no.35; NMA  2/7/2, p.3-14; NMA  2/7/4 (Kritischer Bericht), p.105-107
\par RISM-ID: 450106274\newline D-Cv  A.V,1109,(2),2
\par \vspace{16pt} \textcolor{darkblue}{\textbf{Mozart, Wolfgang Amadeus  1756-1791}}\hfillplus{[52]}\newline Sinfonie concertantes. Auswahl, KV 364 - C-Dur\newline vl, vla, orch
\par \begin{itshape}[caption title by later hand:] Concertante für Violine u. Bratsche [2\textsuperscript{n}\textsuperscript{d} hand:] mit Orchester oder Pf als op 104 bei Joh André verlegt | ebenso à 4ms\end{itshape} 
\par \textcolor{darkblue}{\ding{\numexpr181 + 01}}  Particell: 2f.\newline Autograph\newline Wasserzeichen: REAL | VG [countermark: 3 stars in cartouche, crowned by a crescent]
\par 1.1.1  vl, \begin{itshape}Cadenza per il Primo Allegro:.\end{itshape}, Es-Dur  
\begin{filecontents*}{52-1.code}
@clef:G-2
@keysig:bBEA
@timesig:c
@data:2'''(E)//,{6G'E,B'G}{EBG''E}{'B''GEB}{AGFE}/{8.A6F}4D2-/
\end{filecontents*}
\commandline{ if [ ! -f 52-1.svg ]; then verovio --spacing-non-linear=0.54 -w 1500 --spacing-system=0.5 --adjust-page-height -b 0 52-1.code; fi } 
\newline \includesvg[width=209pt]{52-1}%
\par 1.2.1  vl, \begin{itshape}Cadenza per L'andante..\end{itshape}, c-Moll  
\begin{filecontents*}{52-2.code}
@clef:G-2
@keysig:bBEA
@timesig:3/4
@data:'''2(C)'4G/''C+{6CDED}gF{8E6DC}/4C'nBG/''D+{6DEFE}gG{8F6ED}/
\end{filecontents*}
\commandline{ if [ ! -f 52-2.svg ]; then verovio --spacing-non-linear=0.54 -w 1500 --spacing-system=0.5 --adjust-page-height -b 0 52-2.code; fi } 
\newline \includesvg[width=209pt]{52-2}%
\par Titeleinträge von fremder Hand, Satzüberschriften (s. Incipits) autograph.
\par vl steht in Satz 1 in E\textsuperscript{b} und die vla in D, Satz 2 vl in c und vla in b.
\par Nach den Noten, f.2v unten: {\textquotedbl}Die Echtheit der Handschrift | von W.A. Mozart | bestätigt | Julius André.{\textquotedbl}; links daneben rotes Sigel mit {\textquotedbl}A{\textquotedbl}.\newline André, Julius  (Sonstige)\newline Literatur: NMA  5/14/2, p.3-56; NMA  5/14/2 (Kritischer Bericht), p.b/16-17; NMA  1/33/2, no.51
\par RISM-ID: 450106280\newline D-Cv  A.V,1109,(2),9
\par \vspace{16pt} \textcolor{darkblue}{\textbf{Mozart, Wolfgang Amadeus  1756-1791}}\hfillplus{[53]}\newline Solfeggio, KV 393/5a - C-Dur\newline S
\par \begin{itshape}[left before stave:] 1\end{itshape} 
\par \textcolor{darkblue}{\ding{\numexpr181 + 01}}  Stimme: S  (f.1r)\newline Autograph
\par 1.1.1  S, C-Dur  
\begin{filecontents*}{53-1.code}
@clef:C-1
@keysig:
@timesig:
@data:{6CEDE}{FGEF}{DFEF}{GAFG}/{EGFG}{AbBGA}{FAGA}{B''C'AB}/
\end{filecontents*}
\commandline{ if [ ! -f 53-1.svg ]; then verovio --spacing-non-linear=0.54 -w 1500 --spacing-system=0.5 --adjust-page-height -b 0 53-1.code; fi } 
\newline \includesvg[width=209pt]{53-1}%
\par RISM-ID: 450106279\newline D-Cv  A.V,1109,(2),7\newline $\rightarrow$ In Sammlung 82 (450106283)
      
\par \vspace{16pt} \textcolor{darkblue}{\textbf{Mozart, Wolfgang Amadeus  1756-1791}}\hfillplus{[54]}\newline Solfeggio, KV 393/5b - C-Dur\newline S
\par \begin{itshape}[left before stave:] 2\end{itshape} 
\par \textcolor{darkblue}{\ding{\numexpr181 + 01}}  Stimme: S  (f.1r)\newline Autograph
\par 1.1.1  S, C-Dur  
\begin{filecontents*}{54-1.code}
@clef:C-1
@keysig:
@timesig:
@data:''{6EGFE}{AEFC}{DFED}{GDE'B}/''{CEDC}{FCD'A}{B''DC'B}{''E'B''C'G}/
\end{filecontents*}
\commandline{ if [ ! -f 54-1.svg ]; then verovio --spacing-non-linear=0.54 -w 1500 --spacing-system=0.5 --adjust-page-height -b 0 54-1.code; fi } 
\newline \includesvg[width=209pt]{54-1}%
\par RISM-ID: 450106281\newline D-Cv  A.V,1109,(2),7\newline $\rightarrow$ In Sammlung 82 (450106283)
      
\par \vspace{16pt} \textcolor{darkblue}{\textbf{Mozart, Wolfgang Amadeus  1756-1791}}\hfillplus{[55]}\newline Solfeggio, KV 393/5c - C-Dur\newline S
\par \begin{itshape}[left before stave:] 3\end{itshape} 
\par \textcolor{darkblue}{\ding{\numexpr181 + 01}}  Stimme: S  (f.1r)\newline Autograph
\par 1.1.1  S, C-Dur  
\begin{filecontents*}{55-1.code}
@clef:C-1
@keysig:
@timesig:
@data:{6EGCD}{ECAG}{FADE}{FDbBA}/{GbBEF}{GE''C'B}{A''C'FG}{AF''DC}/
\end{filecontents*}
\commandline{ if [ ! -f 55-1.svg ]; then verovio --spacing-non-linear=0.54 -w 1500 --spacing-system=0.5 --adjust-page-height -b 0 55-1.code; fi } 
\newline \includesvg[width=209pt]{55-1}%
\par RISM-ID: 450106282\newline D-Cv  A.V,1109,(2),7\newline $\rightarrow$ In Sammlung 82 (450106283)
      
\par \vspace{16pt} \textcolor{darkblue}{\textbf{Mozart, Wolfgang Amadeus  1756-1791}}\hfillplus{[56]}\newline Solfeggio, KV 393/4 - C-Dur\newline S
\par \begin{itshape}[caption title:] Essercizio per il Canto. [later added, on the left:] zu 11. [or: II.?] Original\end{itshape} 
\par \textcolor{darkblue}{\ding{\numexpr181 + 01}}  Stimme: S  (f.1r)\newline Autograph  17820801-17820831
\par 1.1.1  S, C-Dur  
\begin{filecontents*}{56-1.code}
@clef:C-1
@keysig:
@timesig:c
@data:2C{8DCEC}/{FCGC}{ACBC}/
\end{filecontents*}
\commandline{ if [ ! -f 56-1.svg ]; then verovio --spacing-non-linear=0.54 -w 1500 --spacing-system=0.5 --adjust-page-height -b 0 56-1.code; fi } 
\newline \includesvg[width=209pt]{56-1}%
\par Die Signaturen A.V,1109,(2),4-7 waren im handschriftlichen Bandkatalog in D-Cv in einer abweichenden Reihenfolge eingetragen und durchnummeriert. Am 20.03.2012 wurden die Nummerierungen im Bandkatalog angeglichen an die Nummerierungen auf den Quellen, die auch diesen Titelaufnahmen zugrunde liegen. Diese Quelle trug im Bandkatalog ursprünglich die Signatur A.V,1109,(2),5.
\par Bemerkungen mit roter Tinte, rechts oben: {\textquotedbl}Solfeggi.{\textquotedbl}, darunter: {\textquotedbl}N\textsuperscript{o} 161.{\textquotedbl}, weiter unten, zwischen den Systemen: {\textquotedbl}unvollendte Sache{\textquotedbl}\newline André, Johann Anton  (Vorbesitzer)\newline André, Julius  (Vorbesitzer)\newline Weber, Aloysia  (Widmungsträger)
\par RISM-ID: 450106284\newline D-Cv  A.V,1109,(2),7\newline $\rightarrow$ In Sammlung 82 (450106283)
      
\par \vspace{16pt} \textcolor{darkblue}{\textbf{Mozart, Wolfgang Amadeus  1756-1791}}\hfillplus{[57]}\newline Solfeggio, KV 393/5 - B-Dur\newline S
\par \begin{itshape}[caption title:] 1. Solfeggio per la mia cara consorte [later added, on the left:] N.116. [crossed out, corrected in:] N.36.  [centre:] 3.\end{itshape} 
\par \textcolor{darkblue}{\ding{\numexpr181 + 01}}  Stimme: S  (f.2r-2v)\newline Autograph  17820801-17820831
\par 1.1.1  S, \begin{itshape}Andante\end{itshape}, B-Dur  
\begin{filecontents*}{57-1.code}
@clef:C-1
@keysig:bBE
@timesig:2/4
@data:2'B/''C/D/4EF/4.G{3FEDC}/
\end{filecontents*}
\commandline{ if [ ! -f 57-1.svg ]; then verovio --spacing-non-linear=0.54 -w 1500 --spacing-system=0.5 --adjust-page-height -b 0 57-1.code; fi } 
\newline \includesvg[width=209pt]{57-1}%
\par Die Signaturen A.V,1109,(2),4-7 waren im handschriftlichen Bandkatalog in D-Cv in einer abweichenden Reihenfolge eingetragen und durchnummeriert. Am 20.03.2012 wurden die Nummerierungen im Bandkatalog angeglichen an die Nummerierungen auf den Quellen, die auch diesen Titelaufnahmen zugrunde liegen. Diese Quelle trug im Bandkatalog ursprünglich die Signatur A.V,1109,(2),5.\newline André, Johann Anton  (Vorbesitzer)\newline André, Julius  (Vorbesitzer)\newline Weber, Aloysia  (Widmungsträger)
\par RISM-ID: 450106285\newline D-Cv  A.V,1109,(2),7\newline $\rightarrow$ In Sammlung 82 (450106283)
      
\par \vspace{16pt} \textcolor{darkblue}{\textbf{Mozart, Wolfgang Amadeus  1756-1791}}\hfillplus{[58]}\newline Solfeggio, KV 393/5d - C-Dur\newline S
\par \begin{itshape}[left before stave:] 4\end{itshape} 
\par \textcolor{darkblue}{\ding{\numexpr181 + 01}}  Stimme: S  (f.1r)\newline Autograph
\par 1.1.1  S, C-Dur  
\begin{filecontents*}{58-1.code}
@clef:C-1
@keysig:
@timesig:
@data:''{6EDCD}{EFGA}{GAGF}{EDC'B}/{''C'BAB}{''CDEF}{EFED}{C'BAG}/
\end{filecontents*}
\commandline{ if [ ! -f 58-1.svg ]; then verovio --spacing-non-linear=0.54 -w 1500 --spacing-system=0.5 --adjust-page-height -b 0 58-1.code; fi } 
\newline \includesvg[width=209pt]{58-1}%
\par RISM-ID: 450109894\newline D-Cv  A.V,1109,(2),7\newline $\rightarrow$ In Sammlung 82 (450106283)
      
\par \vspace{16pt} \textcolor{darkblue}{\textbf{Mozart, Wolfgang Amadeus  1756-1791}}\hfillplus{[59]}\newline Solfeggio, KV 393/5e - C-Dur\newline S
\par \begin{itshape}[left before stave:] 5\end{itshape} 
\par \textcolor{darkblue}{\ding{\numexpr181 + 01}}  Stimme: S  (f.1r)\newline Autograph
\par 1.1.1  S, C-Dur  
\begin{filecontents*}{59-1.code}
@clef:C-1
@keysig:
@timesig:
@data:6{EGAB}{''CD'B''C}{'ABGA}{FGEF}/{DAB''C}{DECD}{'B''C'AB}{GAFG}/
\end{filecontents*}
\commandline{ if [ ! -f 59-1.svg ]; then verovio --spacing-non-linear=0.54 -w 1500 --spacing-system=0.5 --adjust-page-height -b 0 59-1.code; fi } 
\newline \includesvg[width=209pt]{59-1}%
\par RISM-ID: 450109895\newline D-Cv  A.V,1109,(2),7\newline $\rightarrow$ In Sammlung 82 (450106283)
      
\par \vspace{16pt} \textcolor{darkblue}{\textbf{Mozart, Wolfgang Amadeus  1756-1791}}\hfillplus{[60]}\newline Solfeggio, KV 393/5f - C-Dur\newline S
\par \begin{itshape}[left before stave:] 6\end{itshape} 
\par \textcolor{darkblue}{\ding{\numexpr181 + 01}}  Stimme: S  (f.1r)\newline Autograph
\par 1.1.1  S, C-Dur  
\begin{filecontents*}{60-1.code}
@clef:C-1
@keysig:
@timesig:
@data:6''{EGAG}{AGAG}{DGxFG}{FGFG}/{CEnFE}{FEFE}{'B''ExDE}{DEDE}/
\end{filecontents*}
\commandline{ if [ ! -f 60-1.svg ]; then verovio --spacing-non-linear=0.54 -w 1500 --spacing-system=0.5 --adjust-page-height -b 0 60-1.code; fi } 
\newline \includesvg[width=209pt]{60-1}%
\par RISM-ID: 450109896\newline D-Cv  A.V,1109,(2),7\newline $\rightarrow$ In Sammlung 82 (450106283)
      
\par \vspace{16pt} \textcolor{darkblue}{\textbf{Mozart, Wolfgang Amadeus  1756-1791}}\hfillplus{[61]}\newline Sonate, KV 448 - D-Dur\newline pf (2)
\par \begin{itshape}[caption title, on the right:] di Wolfgango Amadeo | Mozart mpia [later added, on the left:] a) | bey Artaria und André. [on the right:] gestochen [with red ink:] c/1784. | N\textsuperscript{o} 90.\end{itshape} 
\par \textcolor{darkblue}{\ding{\numexpr181 + 01}}  Partitur: 13f.; 23,5 x 32 cm\newline Autograph\newline Wasserzeichen: [3 crescents, decreasing] / REAL / A [countermark:] W  17840101-17840228
\par 1.1.1  pf 1, \begin{itshape}Allegro con spirito\end{itshape}, D-Dur  
\begin{filecontents*}{61-1.code}
@clef:G-2
@keysig:xFC
@timesig:c
@data:2'''D''A/4.Ft{6ED}4A8.-(3{'AB''C})/48.6D'{A''D}E{'A''E}/F{DF}4A-/
\end{filecontents*}
\commandline{ if [ ! -f 61-1.svg ]; then verovio --spacing-non-linear=0.54 -w 1500 --spacing-system=0.5 --adjust-page-height -b 0 61-1.code; fi } 
\newline \includesvg[width=209pt]{61-1}%
\par Die Signaturen A.V,1109,(2),4-7 waren im handschriftlichen Bandkatalog in D-Cv in einer abweichenden Reihenfolge eingetragen und durchnummeriert. Am 20.03.2012 wurden die Nummerierungen im Bandkatalog angeglichen an die Nummerierungen auf den Quellen, die auch diesen Titelaufnahmen zugrunde liegen. Diese Quelle trug im Bandkatalog ursprünglich die Signatur A.V,1109,(2),4.\newline André, Johann Anton  (Vorbesitzer)\newline André, Julius  (Vorbesitzer)\newline Nissen, Georg Nikolaus von  (Vorbesitzer)\newline Literatur: NMA  1/33/2, no.56; NMA  9/24/1-2, p.2-38; NMA  9/24/1-2 (Kritischer Bericht), p.9-31
\par RISM-ID: 450106276\newline D-Cv  A.V,1109,(2),5
\par \vspace{16pt} \textcolor{darkblue}{\textbf{Mozart, Wolfgang Amadeus  1756-1791}}\hfillplus{[62]}\newline Sonate, KV 328 - C-Dur\newline org, strings
\par \begin{itshape}[caption title:] N\textsuperscript{o} 9.[crossed out and corrected:] 11. Sonata [later added:] N\textsuperscript{r}\textsuperscript{o} 2. [on the right, red ink:] 158. | [black ink:] 177- | gut.\end{itshape} 
\par \textcolor{darkblue}{\ding{\numexpr181 + 01}}  Partitur: 6f.; 16,5 x 23,5 cm\newline \begin{small} f.6 only blank staves\end{small} \newline Autograph\newline Wasserzeichen: [illegible, partially cut off]  17790101-17790630
\par 1.1.1  vl 1, \begin{itshape}Allegro\end{itshape}, C-Dur  
\begin{filecontents*}{62-1.code}
@clef:G-2
@keysig:
@timesig:c
@data:8''C-E-G-E/8.6{FEDC}'4B8.-6B/8''C-'xG-A-xF/8.6{GAFG}4E-/
\end{filecontents*}
\commandline{ if [ ! -f 62-1.svg ]; then verovio --spacing-non-linear=0.54 -w 1500 --spacing-system=0.5 --adjust-page-height -b 0 62-1.code; fi } 
\newline \includesvg[width=209pt]{62-1}%
\par Die Signaturen A.V,1109,(2),4-7 waren im handschriftlichen Bandkatalog in D-Cv in einer abweichenden Reihenfolge eingetragen und durchnummeriert. Am 20.03.2012 wurden die Nummerierungen im Bandkatalog angeglichen an die Nummerierungen auf den Quellen, die auch diesen Titelaufnahmen zugrunde liegen. Diese Quelle trug im Bandkatalog ursprünglich die Signatur A.V,1109,(2),7.\newline Literatur: NMA  6/16, p.60-64; NMA  6/16 (Kritischer Bericht), p.i/28-30
\par RISM-ID: 450106277\newline D-Cv  A.V,1109,(2),6
\par \vspace{16pt} \textcolor{darkblue}{\textbf{Nicolai, Otto  1810-1849}}\hfillplus{[63]}\newline Die lustigen Weiber von Windsor. Excerpts. Arr - C-Dur\newline S, pf
\par \begin{itshape}[title page:] Herrn H.S. v. Mosenthal | Romanze | aus der Oper {\textquotedbl}die lustigen | Weiber von Windsor{\textquotedbl} | von | Otto Nicolai.\end{itshape} 
\par \textcolor{darkblue}{\ding{\numexpr181 + 01}}  Partitur: 2f.; 25,5 x 36,5 cm\newline \begin{small} gedrucktes Notenpapier,  9 Systeme, {\textquotedbl}M.T.2.{\textquotedbl}, türkisfarbene Notenlinien\end{small} \newline Autograph  1845
\par 1.1.1  V, \begin{itshape}Andante con moto\end{itshape}, C-Dur\newline \begin{footnotesize} Horch die Lerche singt im Hain \end{footnotesize}  
\begin{filecontents*}{63-1.code}
@clef:G-2
@keysig:
@timesig:6/8
@data:=1/4''C8'GgB6AxG8A''C/2.E+/4.E'B/
\end{filecontents*}
\commandline{ if [ ! -f 63-1.svg ]; then verovio --spacing-non-linear=0.54 -w 1500 --spacing-system=0.5 --adjust-page-height -b 0 63-1.code; fi } 
\newline \includesvg[width=209pt]{63-1}%
\par 1.1.2  pf, C-Dur  
\begin{filecontents*}{63-2.code}
@clef:G-2
@keysig:
@timesig:6/8
@data:'''8Ct--Ct--/'{6CEGECE}{CFAFCF}/{,B'ExGECE}{CEAECE}/
\end{filecontents*}
\commandline{ if [ ! -f 63-2.svg ]; then verovio --spacing-non-linear=0.54 -w 1500 --spacing-system=0.5 --adjust-page-height -b 0 63-2.code; fi } 
\newline \includesvg[width=209pt]{63-2}%\newline Mosenthal, Salomon Hermann von  (Textdichter)
\par RISM-ID: 450109876\newline D-Cv  A.V,1134,(3),1
\par \vspace{16pt} \textcolor{darkblue}{\textbf{Paganini, Nicolò  1782-1840}}\hfillplus{[64]}\newline Sonate - A-Dur\newline vl, guit
\par \begin{itshape}[title page:] Sonata p Chitarra, e Violino | dedicata | a Madamigella Emilia Di Negro | par | Niccolo Paganini | Chitarra\end{itshape} 
\par \textcolor{darkblue}{\ding{\numexpr181 + 01}}  1 Stimme: guit  (4f.); 22 x 29,5 cm\newline \begin{small} vl missing\end{small} \newline Autograph\newline Wasserzeichen: [without watermark]
\par 1.1.1  guit, \begin{itshape}Allegro spiritoso\end{itshape}, A-Dur  
\begin{filecontents*}{64-1.code}
@clef:G-2
@keysig:xFCG
@timesig:c
@data:4''A8-{6FE}{8xDEnD'B}/4A--{8.''F6E}/'4B-''4.C'8B}/4B--{8.B''6C}/
\end{filecontents*}
\commandline{ if [ ! -f 64-1.svg ]; then verovio --spacing-non-linear=0.54 -w 1500 --spacing-system=0.5 --adjust-page-height -b 0 64-1.code; fi } 
\newline \includesvg[width=209pt]{64-1}%
\par 1.2.1  guit, \begin{itshape}Adagio espressivo\end{itshape}, a-Moll  
\begin{filecontents*}{64-2.code}
@clef:G-2
@keysig:
@timesig:3/4
@data:gxG{8AA}gB''{8CC}{xDD}/4E-{8F6.E3D}/
\end{filecontents*}
\commandline{ if [ ! -f 64-2.svg ]; then verovio --spacing-non-linear=0.54 -w 1500 --spacing-system=0.5 --adjust-page-height -b 0 64-2.code; fi } 
\newline \includesvg[width=209pt]{64-2}%
\par 1.3.1  guit, \begin{itshape}Rondeau. con brio\end{itshape}, A-Dur  
\begin{filecontents*}{64-3.code}
@clef:G-2
@keysig:xFCG
@timesig:6/8
@data:{8EE}/''{C'BA}{AGA}/4xA8B-{EE}/''{EFE}{DC'B}/gB''4C'8A-
\end{filecontents*}
\commandline{ if [ ! -f 64-3.svg ]; then verovio --spacing-non-linear=0.54 -w 1500 --spacing-system=0.5 --adjust-page-height -b 0 64-3.code; fi } 
\newline \includesvg[width=209pt]{64-3}%
\par Auf dem Schutzumschlag in D-Cv steht {\textquotedbl}S.1146{\textquotedbl}, ist jedoch im Katalog S.1145, Rückseite.\newline Di Negro, Emilia  (Widmungsträger)
\par RISM-ID: 450109890\newline D-Cv  A.V,1145,(4),1
\par \vspace{16pt} \textcolor{darkblue}{\textbf{Rubinštejn, Anton Grigor'evič  1829-1894}}\hfillplus{[65]}\newline Klavierstück. Auswahl - B-Dur\newline pf
\par \begin{itshape}[without title]\end{itshape} 
\par \textcolor{darkblue}{\ding{\numexpr181 + 01}}  1 Stimme: pf  (1f.); 13,5 x 21 cm\newline Autograph\newline Wasserzeichen: [without watermark]  18770208
\par 1.1.1  pf, \begin{itshape}Adagio\end{itshape}, B-Dur  
\begin{filecontents*}{65-1.code}
@clef:F-4
@keysig:bBE
@timesig:c
@data:'4D8{C,A}xF4E8D/{xCDnC,,A}xF4D8C/2,,,B
\end{filecontents*}
\commandline{ if [ ! -f 65-1.svg ]; then verovio --spacing-non-linear=0.54 -w 1500 --spacing-system=0.5 --adjust-page-height -b 0 65-1.code; fi } 
\newline \includesvg[width=209pt]{65-1}%
\par Untere Zeile mit Datum: {\textquotedbl}den 8. Februar 1877{\textquotedbl}, darüber Unterschrift: {\textquotedbl}Ant. Rubinstein{\textquotedbl}
\par Nur 5 Takte sind notiert auf freihändig gezogenen Notensystemen, beginnend mit 3 Takten im unteren System im Bassschlüssel.
\par RISM-ID: 450109891\newline D-Cv  A.V,1146,(3),1
\par \vspace{16pt} \textcolor{darkblue}{\textbf{Salieri, Antonio  1750-1825}}\hfillplus{[66]}\newline Appel à l'amour, AngS 263 - A-Dur\newline V, pf
\par \begin{itshape}[heading:] Appel'à l'amour [on the right:] Salieri\end{itshape} 
\par \textcolor{darkblue}{\ding{\numexpr181 + 01}}  Partitur: 2f.; 23 x 31 cm\newline \begin{small} f.2v only blank staves\end{small} \newline Autograph\newline Wasserzeichen: [eagle, countermark: 3 cresents]
\par 1.1.1  pf, A-Dur  
\begin{filecontents*}{66-1.code}
@clef:G-2
@keysig:xFCG
@timesig:2/4
@data:''6C/{8.C'6B}{AB''CD}/4.E8E/6{EAEC}{ED'BG}/4B8A-/
\end{filecontents*}
\commandline{ if [ ! -f 66-1.svg ]; then verovio --spacing-non-linear=0.54 -w 1500 --spacing-system=0.5 --adjust-page-height -b 0 66-1.code; fi } 
\newline \includesvg[width=209pt]{66-1}%
\par 1.1.2  S, A-Dur\newline \begin{footnotesize} Reviens plaisir d'amour reviens secher mes larmes \end{footnotesize}  
\begin{filecontents*}{66-2.code}
@clef:C-1
@keysig:xFCG
@timesig:2/4
@data:6-/=3/4-8-''C/8.C6'B{AB}{''CD}/4E8-E/{6FE}{DC}{FE}{DC}/8C'B4-/
\end{filecontents*}
\commandline{ if [ ! -f 66-2.svg ]; then verovio --spacing-non-linear=0.54 -w 1500 --spacing-system=0.5 --adjust-page-height -b 0 66-2.code; fi } 
\newline \includesvg[width=209pt]{66-2}%\newline Literatur: AngermüllerS 1976  p.177, no.250; AngermüllerS 1976  p.185-188
\par RISM-ID: 450109883\newline D-Cv  A.V,1145,(1),2
\par \vspace{16pt} \textcolor{darkblue}{\textbf{Salieri, Antonio  1750-1825}}\hfillplus{[67]}\newline Bella arciera per che fiera, AngS 260 - C-Dur\newline S, pf
\par \begin{itshape}[caption title:] Canzonetta, musica del Salieri, parole del seicento.\end{itshape} 
\par \textcolor{darkblue}{\ding{\numexpr181 + 01}}  Partitur: f.2v-3r\newline Autograph
\par 1.1.1, C-Dur  
\begin{filecontents*}{67-1.code}
@clef:G-2
@keysig:
@timesig:6/8
@data:''8{EF}/q6A{8GxF}4G{8AB}/'''q6D{8C''B}'''4C''{8BA}/{GEC}{FD'B}/''4.D4C8-/
\end{filecontents*}
\commandline{ if [ ! -f 67-1.svg ]; then verovio --spacing-non-linear=0.54 -w 1500 --spacing-system=0.5 --adjust-page-height -b 0 67-1.code; fi } 
\newline \includesvg[width=209pt]{67-1}%
\par 1.1.2  S, C-Dur\newline \begin{footnotesize} Bella arciera perche fiera \end{footnotesize}  
\begin{filecontents*}{67-2.code}
@clef:C-1
@keysig:
@timesig:6/8
@data:8--/=3/4-8--''ED/q8D{8C'B}4''C8FE/q6E{8DxC}4D8GF/{ED}Cq6D{8C'B}''C/
\end{filecontents*}
\commandline{ if [ ! -f 67-2.svg ]; then verovio --spacing-non-linear=0.54 -w 1500 --spacing-system=0.5 --adjust-page-height -b 0 67-2.code; fi } 
\newline \includesvg[width=209pt]{67-2}%
\par Die 3 Strophen sind jede auf einer eigenen Notenzeile mit Noten und unterlegtem Text notiert. Insgesamt ergeben sich dadurch eine Partitur-Anordnung mit 5-zeiligen Akkoladen, 3 für die Gesangsstrophen und 2 für die Klavierbegleitung.\newline Literatur: AngermüllerS 1976  p.168, no.29
\par RISM-ID: 450109886\newline D-Cv  A.V,1145,(1),1\newline $\rightarrow$ In Sammlung 81 (450109884)
      
\par \vspace{16pt} \textcolor{darkblue}{\textbf{Salieri, Antonio  1750-1825}}\hfillplus{[68]}\newline Moderata durant, AngS 222 - B-Dur\newline V (3)
\par \begin{itshape}[caption title:] Canone a 3 voci\end{itshape} 
\par \textcolor{darkblue}{\ding{\numexpr181 + 01}}  Partitur: f.4v\newline Autograph  18190922
\par 1.1.1  V, \begin{itshape}Andante\end{itshape}, B-Dur\newline \begin{footnotesize} Moderata durant \end{footnotesize}  
\begin{filecontents*}{68-1.code}
@clef:C-1
@keysig:bBE
@timesig:c/
@data:4BBBB/1B/2.A4-/''DDDD/1D/2.C4-/
\end{filecontents*}
\commandline{ if [ ! -f 68-1.svg ]; then verovio --spacing-non-linear=0.54 -w 1500 --spacing-system=0.5 --adjust-page-height -b 0 68-1.code; fi } 
\newline \includesvg[width=209pt]{68-1}%
\par Nur eine Stimme ist notiert. Sie ist in 3 Zeilen so untereinander geschrieben, dass die Zusammenklänge erkennbar sind.
\par Bemerkung über der ersten Notenzeile: {\textquotedbl}Il canone, che segue fu composto dal Salieri | ritornando con alcuni amici a piedi fra li monti | da Heiligen craize a Vienna il giorno 22 sett[embre] 1819{\textquotedbl}\newline Literatur: AngermüllerS 1976  p.174, no.169; AngermüllerS 1976  p.184
\par RISM-ID: 450109889\newline D-Cv  A.V,1145,(1),1\newline $\rightarrow$ In Sammlung 81 (450109884)
      
\par \vspace{16pt} \textcolor{darkblue}{\textbf{Salieri, Antonio  1750-1825}}\hfillplus{[69]}\newline Prima la musica e poi le parole. Excerpts. Arr, BPI 33.I.2 - Es-Dur\newline S, pf
\par \begin{itshape}[caption title, f.1r:] Cavatina del Salieri composta per il famoso Marchesi nell'opera | intitulata Giulio Sabino. [later added, on the left:] Weidlingau 1814. Salieri m.p. [on the right:] S.W.E.\end{itshape} 
\par \textcolor{darkblue}{\ding{\numexpr181 + 01}}  Partitur: f.1r-2r\newline Autograph  1814
\par 1.1.1  pf, \begin{itshape}Larghetto\end{itshape}, Es-Dur  
\begin{filecontents*}{69-1.code}
@clef:G-2
@keysig:bBEA
@timesig:c
@data:E/2B+{8B''C}{D6EC}/'q4B2A4-A/''2F+{8FD'BA}/{G''B}
\end{filecontents*}
\commandline{ if [ ! -f 69-1.svg ]; then verovio --spacing-non-linear=0.54 -w 1500 --spacing-system=0.5 --adjust-page-height -b 0 69-1.code; fi } 
\newline \includesvg[width=209pt]{69-1}%
\par 1.1.2  S, Es-Dur\newline \begin{footnotesize} Pensieri funesti ah no non tornate \end{footnotesize}  
\begin{filecontents*}{69-2.code}
@clef:C-1
@keysig:bBEA
@timesig:c
@data:4-/=6/2-4-8-B/2B+{8B''C}D{6EC}/'8B4.A4-8-A/''2F+{8FD}'BA/4GG-
\end{filecontents*}
\commandline{ if [ ! -f 69-2.svg ]; then verovio --spacing-non-linear=0.54 -w 1500 --spacing-system=0.5 --adjust-page-height -b 0 69-2.code; fi } 
\newline \includesvg[width=209pt]{69-2}%
\par Vermutlich erfolgte die Bearbeitung des Satzes bei einem dokumentierten Aufenthalt in Weidlingau am 10.06.1814, bei dem 2 weitere Gelegenheitskompositionen entstanden sind; vgl. AngermüllerS 2000, p.69.\newline Casti, Giovanni Battista  (Textdichter)\newline Literatur: AngermüllerS 2000  p.69
\par RISM-ID: 450109885\newline D-Cv  A.V,1145,(1),1\newline $\rightarrow$ In Sammlung 81 (450109884)
      
\par \vspace{16pt} \textcolor{darkblue}{\textbf{Salieri, Antonio  1750-1825}}\hfillplus{[70]}\newline Tutte le volte che in villa trovomi, AngS 237 - F-Dur\newline V (3)
\par \begin{itshape}[caption title:] A Rapolskirken il giorno 10 may: 1813\end{itshape} 
\par \textcolor{darkblue}{\ding{\numexpr181 + 01}}  Partitur: f.3v\newline Autograph  18130510
\par 1.1.1  S 1, \begin{itshape}Andante\end{itshape}, F-Dur\newline \begin{footnotesize} Tutte le volte che in villa trovomi \end{footnotesize}  
\begin{filecontents*}{70-1.code}
@clef:C-1
@keysig:bB
@timesig:c/
@data:4A8AB/''4CCD8DD/8.C'6B4A8-''FFF/8.F'6nB4.B''8FFF/
\end{filecontents*}
\commandline{ if [ ! -f 70-1.svg ]; then verovio --spacing-non-linear=0.54 -w 1500 --spacing-system=0.5 --adjust-page-height -b 0 70-1.code; fi } 
\newline \includesvg[width=209pt]{70-1}%
\par Datierung bei AngermüllerS 2000: 09.06.1813; das könnte möglicherweise das Datum auf dem zweiten Autograph des Werke in A-Wgm sein.\newline Literatur: AngermüllerS 1976  p.179, no.296; AngermüllerS 2000  p.57
\par RISM-ID: 450109887\newline D-Cv  A.V,1145,(1),1\newline $\rightarrow$ In Sammlung 81 (450109884)
      
\par \vspace{16pt} \textcolor{darkblue}{\textbf{Salieri, Antonio  1750-1825}}\hfillplus{[71]}\newline Zum che diavolo è mai, AngS 238 - Es-Dur\newline V (3)
\par \begin{itshape}[caption title:] Dello stresso, il giorno 23 luglio, a Schönkirken, vicino un Ruscello verso serza.\end{itshape} 
\par \textcolor{darkblue}{\ding{\numexpr181 + 01}}  Partitur: f.4r-4v\newline Autograph  18220723
\par 1.1.1  S 1, \begin{itshape}Allegro\end{itshape}, Es-Dur\newline \begin{footnotesize} Zum che diavolo è mai \end{footnotesize}  
\begin{filecontents*}{71-1.code}
@clef:C-1
@keysig:bBEA
@timesig:c/
@data:2-B/BB/4.B8BBBBB/''4E8EEEEE'nA/4BB
\end{filecontents*}
\commandline{ if [ ! -f 71-1.svg ]; then verovio --spacing-non-linear=0.54 -w 1500 --spacing-system=0.5 --adjust-page-height -b 0 71-1.code; fi } 
\newline \includesvg[width=209pt]{71-1}%
\par Datierung übernommen von AngermüllerS 2000.\newline Literatur: AngermüllerS 1976  p.181, no.326; AngermüllerS 2000  p.248-249
\par RISM-ID: 450109888\newline D-Cv  A.V,1145,(1),1\newline $\rightarrow$ In Sammlung 81 (450109884)
      
\par \vspace{16pt} \textcolor{darkblue}{\textbf{Schmidt, Gustav  1816-1882}}\hfillplus{[72]}\newline Prinz Eugen der edle Ritter. Auswahl\newline V (X), Coro, orch
\par \begin{itshape}[caption title:] zu N\textsuperscript{o} 1. Pag. 20 fehlt die Stimme der Engelliese\end{itshape} 
\par \textcolor{darkblue}{\ding{\numexpr181 + 01}}  Partitur: 1f.; 9,5 x 24,5 cm\newline Possible autograph manuscript  1848-1880
\par 1.1.1  Coro S, C-Dur\newline \begin{footnotesize} [...] gut sei wieder \end{footnotesize}  
\begin{filecontents*}{72-1.code}
@clef:G-2
@keysig:
@timesig:2/4
@data:4B8-B/4''ED/'B
\end{filecontents*}
\commandline{ if [ ! -f 72-1.svg ]; then verovio --spacing-non-linear=0.54 -w 1500 --spacing-system=0.5 --adjust-page-height -b 0 72-1.code; fi } 
\newline \includesvg[width=209pt]{72-1}%
\par 1.1.2  S, C-Dur\newline \begin{footnotesize} Wenn ihr hinfort euch brav aufführt \end{footnotesize}  
\begin{filecontents*}{72-2.code}
@clef:G-2
@keysig:
@timesig:2/4
@data:=2/4-8-G/4.G8A/4.G8''D/4CbE/'G
\end{filecontents*}
\commandline{ if [ ! -f 72-2.svg ]; then verovio --spacing-non-linear=0.54 -w 1500 --spacing-system=0.5 --adjust-page-height -b 0 72-2.code; fi } 
\newline \includesvg[width=209pt]{72-2}%
\par f.1v leere Notensysteme und Bleistifteintrag {\textquotedbl}Gustav Schmidt aus Prinz Eugen.{\textquotedbl}
\par Der Ausschnitt umfasst 11 Takte auf einer Akkolade mit 7 Notensystemen.
\par RISM-ID: 450109873\newline D-Cv  A.V,1133,(4),1
\par \vspace{16pt} \textcolor{darkblue}{\textbf{Schulhoff, Julius  1825-1898}}\hfillplus{[73]}\newline Berceuse. Sketches - As-Dur\newline pf
\par \begin{itshape}[caption title:] Berceuse\end{itshape} 
\par \textcolor{darkblue}{\ding{\numexpr181 + 01}}  Stimme: pf  (1f.); 8,5 x 20,5 cm\newline \begin{small} Das Blatt ist aufgeklebt auf eine Pappe (20 x 27 cm)\end{small} \newline Autograph  18500321
\par 1.1.1  pf, \begin{itshape}Andantino\end{itshape}, As-Dur  
\begin{filecontents*}{73-1.code}
@clef:G-2
@keysig:bBEAD
@timesig:6/8
@data:4-8-4-6''{EE}/8A-'Aqq{6B''C}r{'8B6-A8B}/''4.C'8A-''{EE}/8A-'Aqq{6B''C}r{8'B6-A8B}/4A8-4-
\end{filecontents*}
\commandline{ if [ ! -f 73-1.svg ]; then verovio --spacing-non-linear=0.54 -w 1500 --spacing-system=0.5 --adjust-page-height -b 0 73-1.code; fi } 
\newline \includesvg[width=209pt]{73-1}%
\par Nur eine Akkolade mit viereinhalb Takten beschrieben, darunter {\textquotedbl}Wien 21/3/50. J. Schulhoff{\textquotedbl}
\par Auf der Rückseite der Pappe ist ein Portrait des Komponisten aufgeklebt.
\par RISM-ID: 450109871\newline D-Cv  A.V,1130,(5),1
\par \vspace{16pt} \textcolor{darkblue}{\textbf{Stalder, Joseph Franz Xaver Dominik  1725-1765}}\hfillplus{[74]}\newline Trioonatas, op.2 - C-Dur\newline vl (2), b
\par \begin{itshape}[title page, each part:] Sonata a trè\end{itshape} 
\par \textcolor{darkblue}{\ding{\numexpr181 + 01}}  3 Stimmen: vl 1, 2, b  (2, 2, 2f.); 23,5 x 30,5 cm\newline \begin{small} all parts f.1r = title page, f.2v only blank staves; watermark partially cut off\end{small} \newline Abschrift\newline Wasserzeichen: DANNONAY / 1771 / M [lily] JOHANNOT / [coat of arms with grapes (crowned)]  1772-1777\newline Schreiber: Rousseau, Jean-Jacques
\par 1.1.1  vl 1, \begin{itshape}Andante\end{itshape}, C-Dur  
\begin{filecontents*}{74-1.code}
@clef:G-2
@keysig:
@timesig:2/4
@data:''8C4C{6.G3E}/8D4D{6.A3F}/{8EG}{6GEGE}/{8DF}{6FDFD}/
\end{filecontents*}
\commandline{ if [ ! -f 74-1.svg ]; then verovio --spacing-non-linear=0.54 -w 1500 --spacing-system=0.5 --adjust-page-height -b 0 74-1.code; fi } 
\newline \includesvg[width=209pt]{74-1}%
\par 1.2.1  vl 1, \begin{itshape}Minuetto.\end{itshape}, C-Dur  
\begin{filecontents*}{74-2.code}
@clef:G-2
@keysig:
@timesig:3/4
@data:2''G(8{EDC})/2D'({8BAG})/''{8C'B}''4D8-D/{DC}4E8-E/
\end{filecontents*}
\commandline{ if [ ! -f 74-2.svg ]; then verovio --spacing-non-linear=0.54 -w 1500 --spacing-system=0.5 --adjust-page-height -b 0 74-2.code; fi } 
\newline \includesvg[width=209pt]{74-2}%
\par Für den Hinweis auf diese Musikhandschrift danken wir Frau Silvia Böcking, Untersiemau.
\par Kopistenvermerk vl 1, f.2r, unten rechts: {\textquotedbl}E.3. JJR. cop.{\textquotedbl} ({\textquotedbl}J{\textquotedbl} und {\textquotedbl}R{\textquotedbl} verschlungen). Die Chiffre {\textquotedbl}E.3.{\textquotedbl} bezieht sich auf Rousseaus Register der von ihm zwischen April 1772 und August 1777 kopierten Kompositionen. Danach fasst er jeweils 100 Kompositionen unter einem Buchstaben zusammen; vgl. JansenR 1884.
\par Prägestempel EA mit Krone
\par Wahrscheinlich gehört das Trio zu der als op.2 veröffentlichten, heute aber verschollenen Sammlung {\textquotedbl}Sei Trio concertati a due Violini et Basso, dedicati all'Illustr. ma. Sig.ra Baronezza di Du Harde alle sua nobilissima Assambla del concerto forestiere{\textquotedbl} (SaladinL 1948, p.98)\newline Literatur: SaladinL 1948  p.98; JansenR 1884  p.474-479
\par RISM-ID: 450109892\newline D-Cv  A.IV,788,1
\par \vspace{16pt} \textcolor{darkblue}{\textbf{Strauss, Richard  1864-1949}}\hfillplus{[75]}\newline Frau ohne Schatten. Auswahl, op.65
\par \begin{itshape}[caption title:] Frau ohne Schatten I. Akt.\end{itshape} 
\par \textcolor{darkblue}{\ding{\numexpr181 + 01}}  Partitur: 1f.; 25 x 34,5 cm\newline Autograph  1918
\par 1.1.1  B\newline \begin{footnotesize} wehe dir weh uns allen \end{footnotesize}  
\begin{filecontents*}{75-1.code}
@clef:F-4
@keysig:bBEAD
@timesig:3/4
@data:'n8.D6D,4nA'bC+/@3/2 8C,E4A+8AbC4-/
\end{filecontents*}
\commandline{ if [ ! -f 75-1.svg ]; then verovio --spacing-non-linear=0.54 -w 1500 --spacing-system=0.5 --adjust-page-height -b 0 75-1.code; fi } 
\newline \includesvg[width=209pt]{75-1}%
\par f.1r, unten: {\textquotedbl}Der Bibliothek Seiner Königlichen Hoheit des Herzogs am 1. November 1918. | Richard Strauss.{\textquotedbl}
\par Auf der Rückseite sind nur 6 Takte (auf 2 Systemen) notiert, danach: {\textquotedbl}weiter siehe Seite 3{\textquotedbl}\newline Hofmannsthal, Hugo von  (Textdichter)\newline Karl Eduard, Herzog von Sachsen-Coburg-Gotha  (Vorbesitzer)\newline Olim: E. 888./18.
\par RISM-ID: 450109879\newline D-Cv  A.V,1134,(15),3
\par \vspace{16pt} \textcolor{darkblue}{\textbf{Strauss, Richard  1864-1949}}\hfillplus{[76]}\newline Till Eulenspiegels lustige Streiche. Auswahl, op.28
\par \begin{itshape}[heading:] 1. November 1918.\end{itshape} 
\par \textcolor{darkblue}{\ding{\numexpr181 + 01}}  Stimme: i  (2f.); 17,5 x 13,5 cm\newline \begin{small} f.1v-2v blank\end{small} \newline Autograph\newline Wasserzeichen: [coat of arms] / SCHOELLERPOST / [monogram]  19181101
\par 1.1.1  i, d-Moll  
\begin{filecontents*}{76-1.code}
@clef:G-2
@keysig:bB
@timesig:6/8
@data:8-{''6AF8'nB}{''CxCF}/2.'xG
\end{filecontents*}
\commandline{ if [ ! -f 76-1.svg ]; then verovio --spacing-non-linear=0.54 -w 1500 --spacing-system=0.5 --adjust-page-height -b 0 76-1.code; fi } 
\newline \includesvg[width=209pt]{76-1}%
\par Eingelegt in das Doppelblatt sind 1 Foto und mehrere Zeitunsausschnitte mit Fotographien von Richard Strauss.
\par Über den Noten: Stempel {\textquotedbl}SCHLOSS EHRENBURG | COBURG{\textquotedbl}
\par Notiz auf der Mappe: „2 Takte Noten mit U[nterschrift]. 1. November 1918, geschr. für die Autogr. Slg. Dr. Th. Krieg.”\newline Krieg, Thilo  (Vorbesitzer)\newline Schloß Ehrenburg  (Vorbesitzer)
\par RISM-ID: 450109880\newline D-Cv  A.V,1134,(15),6
\par \vspace{16pt} \textcolor{darkblue}{\textbf{Weber, Carl Maria von  1786-1826}}\hfillplus{[77]}\newline Maienblümlein. Arr, J 149 - Es-Dur\newline winds
\par \begin{itshape}[caption title:] Maÿenblümlein\end{itshape} 
\par \textcolor{darkblue}{\ding{\numexpr181 + 01}}  Partitur: f.1r-1v\newline Autograph
\par 1.1.1  cl 1, Es-Dur  
\begin{filecontents*}{77-1.code}
@clef:G-2
@keysig:bBEA
@timesig:3/4
@data:''4G/gG4F8E-4G/gG4F8E-{BB}/B-'''C-''nA-/4B8G-4G/
\end{filecontents*}
\commandline{ if [ ! -f 77-1.svg ]; then verovio --spacing-non-linear=0.54 -w 1500 --spacing-system=0.5 --adjust-page-height -b 0 77-1.code; fi } 
\newline \includesvg[width=209pt]{77-1}%
\par 1.1.2  fl, Es-Dur  
\begin{filecontents*}{77-2.code}
@clef:G-2
@keysig:bBEA
@timesig:3/4
@data:4-/=24/''EGnA/B'''{8CD}4E/ECF/{8DECD}''B-/
\end{filecontents*}
\commandline{ if [ ! -f 77-2.svg ]; then verovio --spacing-non-linear=0.54 -w 1500 --spacing-system=0.5 --adjust-page-height -b 0 77-2.code; fi } 
\newline \includesvg[width=209pt]{77-2}%\newline Eckschlager, August  (Textdichter)
\par RISM-ID: 450109861\newline D-Cv  A.V,1111,(2),2\newline $\rightarrow$ In Sammlung 80 (450109860)
      
\par \vspace{16pt} \textcolor{darkblue}{\textbf{Sammlung}}\hfillplus{78}\newline 13 Albumblätter
\par \begin{itshape}[cover title:] Concertantes | Divertissement | über beliebte Motive | aus der Oper | Casilda | für 2 Pianoforte, Flöte, Oboe, | Clarinette, Fagott, Horn, Flü= | gelhorn, Violin, Violoncell, | Contrabass, Posaune u. Pauken | von | Carl Czerny | Aufgeführt | in der am 8. Februar [1]852 | Seiner Königl: Hoheit | dem Herzoge | Ernst | zu Sachsen=Coburg=Gotha | zu Ehren, | veranstalteten Matinée | musicale, von den Instrumental=Professoren | der | Academie der Ton= | Kunst in Wien.\end{itshape} 
\par \textcolor{darkblue}{\ding{\numexpr181 + 01}}  13 Stimmen: vl, vlc, cb, fl, ob, cl, fag, cor, flügelhorn, trb, timp, pf 1, pf 2  (); 31 x 24,5 cm\newline \begin{small} each part: 1f. and f.1v blank\end{small} \newline Abschrift  18520208-18520305\newline Schreiber: Glöggl, Franz
\par Der Schutzumschlag mit Titel (= f.1r) und der Widmung von Glöggl (f.2r; f.1v und 2v leer) ist ebenso wie die 13 eingelegten Albumblätter auf dickem Papier mit Zierrahmen notiert: {\textquotedbl}Lith. Anst. v.G. Wegelein, Wien.{\textquotedbl}
\par Widmung auf dem Schutzumschlag, f.2r von der Hand Glöggls: Casilda | Dem unerschöpflichen Reich= | thum an Melodien in dieser | Oper gab mir die Veranlassung | zur Veranstaltung eines Diver= | tissements in welchem 13 P[r]ofessoren | ihr Solo fanden, und viele Motive | daraus werden noch zur Zierde | der Verlagshandlung werden. | Wien den 15 März 1852. | Franz Glöggl | Gründer der Akademie | der Tonkunst in Wien | Kunst u Musik Verleger.
\par Die Albumblätter sind oben mit römischen Ziffern durchnummeriert
\par Jeder Instrumentalist hat auf seinem Albumblatt ein kurzes Motiv für sein jeweiliges Instrument notiert. Es ist nicht klar, ob diese Motive alle aus dem {\textquotedbl}Divertissement{\textquotedbl} von Czerny oder aus der Oper {\textquotedbl}Casilda{\textquotedbl} stammen, oder ob sie von den betreffenden Musikern frei erfunden sind.
\par Für den Hinweis auf diese Ammlung danken wir Frau Silvia Böcking, Untersiemau.\newline Czerny, Carl  (cmp)\newline Ernst II., Herzog von Sachsen-Coburg und Gotha  (Widmungsträger)\newline Ernst II., Herzog von Sachsen-Coburg und Gotha  (cmp)
\par RISM-ID: 450109893\newline D-Cv  A.V.1202,(4)-(8) und 1203(1)-(9)
\par \vspace{16pt} \textcolor{darkblue}{\textbf{Sammlung}}\hfillplus{79}\newline 2 Musikstücke
\par \begin{itshape}[dust cover title see score, RISM ID no. 450106268]\end{itshape} 
\par \textcolor{darkblue}{\ding{\numexpr181 + 01}}  12 Stimmen: vl 1, 2, vla 1, 2, 3, vlc, vlne, ob d'amore 1, 2, fag, cemb, org  (1, 1, 1, 1, 1, 1, 1, 1, 1, 1, 1, 1f.); 16,5 x 19,5 cm\newline \begin{small} all parts except vlne f.1v blank\end{small} \newline Abschrift  17470701-17480831\newline Schreiber: Bach, Johann Sebastian
\par Auf der Rückseite von vlne: das zweite Werk, BWV 1079/8, nur die ersten Takte der Oberstimme, vgl. RISM ID no. 450106267.\newline Bach, Carl Philipp Emanuel  (Vorbesitzer)\newline Kittel, Johann Christian  (Vorbesitzer)\newline Poelchau, Georg Johann Daniel  (Vorbesitzer)\newline Literatur: DavidB 1961  p.199-223
\par RISM-ID: 450106265\newline D-Cv  A.V,1109,(1),1b
\par \vspace{16pt} \textcolor{darkblue}{\textbf{Sammlung}}\hfillplus{80}\newline 5 Musikstücke für Bläser
\par \begin{itshape}[cover title, printed in gold:] Autograph | von | Carl Maria von Weber. [title page, by later hand:] Autograph | von | Carl Maria von Weber. | I. | Maienblümlein. | Lied gedichtet von August Eckschläger, componirt | von | Carl Maria von Weber, | und zu einem Walzer umgeschmolzen mit neu hinzu= | componirtem Trio. | II. | Vier Lieder, | gedichtet und componirt | von | Sr. Hoheit dem Herzoge Emil Leopold August von | Sachsen=Coburg=Altenburg: | 1.) {\textquotedbl}Ihr kleinen Vögelein{\textquotedbl} | 2.) Serenade. {\textquotedbl}Lebe wohl, mein süßes Leben{\textquotedbl} | 3. Die verliebte Schäferin. | 4.) {\textquotedbl}Beim kindlichen Strahl des erwachenden Phoibos{\textquotedbl} | Sämmtliche Stücke | arrangirt für Harmonie=Musik von C.M. von Weber | zum Geburtsfeste Sr. Hoheit des Herzogs Emil Leopold August von Sachsen=C.=A., | dem 23. Nov. 1812, am 17. Nov. 1812.\end{itshape} 
\par \textcolor{darkblue}{\ding{\numexpr181 + 01}}  Partitur: 3f.; 27,5 x 32 cm\newline Autograph\newline Wasserzeichen: [no watermark]  1812\newline Schreiber: Weber, Carl Maria von
\par Bemerkung auf dem Titelblatt, unten: Näheres über Composition und Autograph findet sich in {\textquotedbl}'Carl Maria von Weber in seinen Werken pp{\textquotedbl} von Fr. Wilh. Jähns, unter No.150-153, pag.167, 168 und 169.
\par Bemerkung f.1r, unten: {\textquotedbl}Original=Handschrift von Carl Maria von Weber. Geschenk der Witwe Weber's an mich. F.W. Jähns. K. Musik=Direktor in Berlin{\textquotedbl}; nach {\textquotedbl}K.{\textquotedbl} mit Einfügungsklammer: {\textquotedbl}Professor und{\textquotedbl}\newline August Emil Leopold, Herzog von Sachsen-Gotha-Altenburg  (Widmungsträger)\newline Brandt-Weber, Caroline von  (Vorbesitzer)\newline Jähns, Friedrich Wilhelm  (Vorbesitzer)
\par RISM-ID: 450109860\newline D-Cv  A.V,1111,(2),2
\par \vspace{16pt} \textcolor{darkblue}{\textbf{Sammlung}}\hfillplus{81}\newline 5 Vokalstücke
\par \begin{itshape}[without title]\end{itshape} 
\par \textcolor{darkblue}{\ding{\numexpr181 + 01}}  Partitur: 4f.; 25,5 x 37 cm\newline \begin{small} war mit 3 rosa Bändchen zusammen gebunden, Bindung defekt\end{small} \newline Abschrift  1813-1823\newline Schreiber: Salieri, Antonio
\par RISM-ID: 450109884\newline D-Cv  A.V,1145,(1),1
\par \vspace{16pt} \textcolor{darkblue}{\textbf{Sammlung}}\hfillplus{82}\newline 8 Solfeggien
\par \begin{itshape}[without title]\end{itshape} 
\par \textcolor{darkblue}{\ding{\numexpr181 + 01}}  Stimme: S  (2f.)\newline \begin{small} f.1v only blank staves\end{small} \newline Abschrift\newline Wasserzeichen: A / HF [countermark: 3 stars in a cartouche (crowned)]; [3 crescents (decreasing)] / REAL [countermark:] GF [crowned]  17820801-17820831\newline Schreiber: Mozart, Wolfgang Amadeus
\par Im Katalog der Veste Coburg: {\textquotedbl}(für Frau Lange){\textquotedbl}.
\par Die Signaturen A.V,1109,(2),4-7 waren im handschriftlichen Bandkatalog in D-Cv in einer abweichenden Reihenfolge eingetragen und durchnummeriert. Am 20.03.2012 wurden die Nummerierungen im Bandkatalog angeglichen an die Nummerierungen auf den Quellen, die auch diesen Titelaufnahmen zugrunde liegen. Diese Quelle trug im Bandkatalog ursprünglich die Signatur A.V,1109,(2),5.
\par Unter der Signatur sind 2 Einzelblätter zusammengefasst.\newline André, Johann Anton  (Vorbesitzer)\newline André, Julius  (Vorbesitzer)\newline Weber, Aloysia  (Widmungsträger)\newline Literatur: NMA  10/31/4; NMA  10/31/4 (Kritischer Bericht); NMA  1/33/2, no.58, 60
\par RISM-ID: 450106283\newline D-Cv  A.V,1109,(2),7
    \clearpage  
\chapter*{\centering Register der Personennamen}
\addcontentsline{toc}{chapter}{Register der Personennamen}
\fancyhead{}
\fancyhead[C]{\small RISM -\ Kunstsammlungen der Veste Coburg}


\newline 
Albert, Eugen d' ..... [1]

\newline 
André, Johann Anton ..... [21], [45], [46], [47], [48], [49], [50], [56], [57], [61], 82

\newline 
André, Julius ..... [21], [45], [46], [49], [52], [56], [57], [61], 82

\newline 
Anonymus ..... [2], [3], [4], [5], [6], [7], [8], [9], [10], [11], [12], [13], [14]

\newline 
Arnold, Ignaz Ferdinand ..... [25]

\newline 
August Emil Leopold, Herzog von Sachsen-Gotha-Altenburg ..... [15], [16], [17], [18], 80

\newline 
Bach, August Wilhelm ..... [32]

\newline 
Bach, Carl Philipp Emanuel ..... [20], [21], [40], [41], 79

\newline 
Bach, Johann Sebastian ..... [19], [20], [21], [40], [41]

\newline 
Beethoven, Ludwig van ..... [22], [23], [24]

\newline 
Böhner, Johann Ludwig ..... [25]

\newline 
Boieldieu, Louis ..... [26]

\newline 
Bouilly, Jean Nicolas ..... [26]

\newline 
Brandt-Weber, Caroline von ..... 80

\newline 
Canicoff, Basil von ..... [39]

\newline 
Casti, Giovanni Battista ..... [69]

\newline 
Costa, Michele ..... [37]

\newline 
Czerny, Carl ..... [27], 78

\newline 
De Gamerra, Giovanni ..... [51]

\newline 
Di Negro, Emilia ..... [64]

\newline 
Eckschlager, August ..... [77]

\newline 
Ernst I., Herzog von Sachsen-Coburg-Saalfeld ..... [44]

\newline 
Ernst II., Herzog von Sachsen-Coburg und Gotha ..... [2], [3], [4], [5], [6], [7], [8], [9], [10], [11], [12], [13], [14], [28], 78

\newline 
Fischhof, Joseph ..... [29], [30]

\newline 
Fortini, Francesco ..... [51]

\newline 
Franz, Robert ..... [31]

\newline 
Friedrich II., der Große, König von Preußen ..... [32], [33], [34], [35]

\newline 
Geibel, Emanuel ..... [31]

\newline 
Gerber, Ernst Ludwig ..... [21]

\newline 
Gleissner, Franz ..... [51]

\newline 
Gluck, Christoph Willibald ..... [36], [37]

\newline 
Guillard, Nicolas-François ..... [37]

\newline 
Harson, Johann Samuel ..... [32]

\newline 
Haydn, Michael ..... [38]

\newline 
Himmel, Friedrich Heinrich ..... [39]

\newline 
Hofmannsthal, Hugo von ..... [75]

\newline 
Jähns, Friedrich Wilhelm ..... 80

\newline 
Karl Eduard, Herzog von Sachsen-Coburg-Gotha ..... [75]

\newline 
Kerll, Johann Caspar ..... [40], [41]

\newline 
Kittel, Johann Christian ..... [20], [40], [41], 79

\newline 
Krieg, Thilo ..... [76]

\newline 
Liszt, Franz ..... [28], [42]

\newline 
Marschner, Heinrich August ..... [43]

\newline 
Martín y Soler, Vicente ..... [46]

\newline 
Metastasio, Pietro ..... [45]

\newline 
Meyerbeer, Giacomo ..... [44]

\newline 
Millenet, Johann Heinrich ..... [44]

\newline 
Mortellari, Michele ..... [51]

\newline 
Mosenthal, Salomon Hermann von ..... [43], [63]

\newline 
Mozart, Wolfgang Amadeus ..... [45], [46], [47], [48], [49], [50], [51], [52], [53], [54], [55], [56], [57], [58], [59], [60], [61], [62]

\newline 
Nägeli, Hans Georg ..... [19]

\newline 
Nicolai, Otto ..... [63]

\newline 
Nissen, Georg Nikolaus von ..... [45], [46], [47], [48], [49], [50], [51], [61]

\newline 
Paganini, Nicolò ..... [64]

\newline 
Pawel-Rammingen, Baron von ..... [1]

\newline 
Poelchau, Georg Johann Daniel ..... [20], [21], [40], [41], 79

\newline 
Quantz, Johann Joachim ..... [32], [35]

\newline 
Rubinštejn, Anton Grigor'evič ..... [65]

\newline 
Salieri, Antonio ..... [66], [67], [68], [69], [70], [71]

\newline 
Schmidt, Gustav ..... [72]

\newline 
Schulhoff, Julius ..... [73]

\newline 
Schulz, Hermann ..... [19]

\newline 
Schweizer, Carl Friedrich von ..... [30]

\newline 
Scribe, Eugène ..... [26]

\newline 
Sonnleithner, Joseph ..... [23]

\newline 
Stalder, Joseph Franz Xaver Dominik ..... [74]

\newline 
Stargardt, Josef A. ..... [32]

\newline 
Strauss, Richard ..... [75], [76]

\newline 
Treitschke, Georg Friedrich ..... [23]

\newline 
Weber, Aloysia ..... [45], [56], [57], 82

\newline 
Weber, Carl Maria von ..... [15], [16], [17], [18], [39], [77]

\newline 
Weißenbach, Aloys ..... [22]
    \clearpage  
    \chapter*{\centering Register der Titel und Texte}
\addcontentsline{toc}{chapter}{Register der Titel und Texte}
\fancyhead{}
\fancyhead[C]{\small RISM -\ Kunstsammlungen der Veste Coburg}


\newline 
[...] gut sei wieder ..... [72]

\newline 
[...] wohl nein ..... [42]

\newline 
13 Albumblätter ..... 78

\newline 
2 Musikstücke ..... 79

\newline 
3 Sonaten ..... [32]

\newline 
5 Musikstücke für Bläser ..... 80

\newline 
5 Vokalstücke ..... 81

\newline 
8 Solfeggien ..... 82

\newline 
Ah se in ciel benigne stelle ..... [45]

\newline 
Albumblatt ..... [2], [3], [4], [5], [6], [7], [8], [9], [10], [11], [12], [13], [14]

\newline 
Alceste ..... [36]

\newline 
Alle Donner alle Wetter alle Teufel ..... [25]

\newline 
Appel à l'amour ..... [66]

\newline 
Beim kindlichen Strahl des erwachenden Phoibos. Arr ..... [15]

\newline 
Bella arciera per che fiera ..... [67]

\newline 
Bella arciera perche fiera ..... [67]

\newline 
Benedicimus te ..... [49]

\newline 
Berceuse. Sketches ..... [73]

\newline 
Carolo schlägt dein Herz noch für Louise ..... [25]

\newline 
[Chaste fille de Latone] ..... [37]

\newline 
Chi sà qual sia ..... [46]

\newline 
Chorsatz. Fragment ..... [42]

\newline 
[Come ti piace imponi] ..... [47]

\newline 
Das Auge schaut in dessen Wimpergleise die Sonnen auf und nieder gehn ..... [22]

\newline 
Das Fräulein ..... [29]

\newline 
Das Fräulein saß und sann ..... [29]

\newline 
Der Glorreiche Augenblick ..... [22]

\newline 
Der Goldschmied von Ulm. Excerpts. Arr ..... [43]

\newline 
Der den Bund in Sturme festgehalten ..... [22]

\newline 
Die Gräberinsel der Fürsten zu Gotha. Arr ..... [28]

\newline 
Die Mädchen sie gleichen bei meiner Treu' ..... [25]

\newline 
Die lustigen Weiber von Windsor. Excerpts. Arr ..... [63]

\newline 
Die verliebte Schäferin. Arr ..... [16]

\newline 
Domine Deus salutis meae ..... [38]

\newline 
Domine Deus salutis meae, in die clamavi ..... [38]

\newline 
Du treuer Gottessohn nun hast du mir den Himmel aufgemacht ..... [21]

\newline 
Einsam saß ich in dem Tale an dem schroffen Felsenhang ..... [25]

\newline 
Er ist da in seiner Gloria ..... [25]

\newline 
Erfreue dich mein Herz den jetzo weicht der Schmerz ..... [21]

\newline 
Es braust und wogt des Volkes Drang ..... [44]

\newline 
Es gehen die Töne ins fühlende Herz ..... [25]

\newline 
Es treten hervor die Scharen der Frauen ..... [22]

\newline 
Europa steht ..... [22]

\newline 
Festlied ..... [44]

\newline 
Fidelio. Sketches ..... [23]

\newline 
Frau ohne Schatten. Auswahl ..... [75]

\newline 
Freude füllet alle Herzen ..... [25]

\newline 
Glorificamus te ..... [49]

\newline 
Herr Gott dich loben alle wir ..... [19]

\newline 
Heut' schleußt er wieder auf die Tür ..... [21]

\newline 
Hier wo der Kunst ein gastlich Dach ..... [44]

\newline 
Horch die Lerche singt im Hain ..... [63]

\newline 
Ihr kleinen Vögelein. Arr ..... [17]

\newline 
Il Burbero di buon cuore. Inserts ..... [46]

\newline 
Im ferneren Lande da kämpfen ..... [25]

\newline 
In Jesu Demut kann ich Trost ..... [21]

\newline 
Io ti lascio e questo addio ..... [51]

\newline 
Iphigénie en Tauride. Excerpts. Arr ..... [37]

\newline 
Klavierstück. Auswahl ..... [1], [65]

\newline 
La Clemenza di Tito. Excerpts. Arr ..... [47]

\newline 
Lebe wohl mein süßes Leben. Arr ..... [18]

\newline 
Les Deux nuits. Auswahl ..... [26]

\newline 
Louise und Carolo im romantischen Mühltale ..... [25]

\newline 
Maienblümlein. Arr ..... [77]

\newline 
Männer suchen stets zu naschen ..... [48]

\newline 
Männer suchen stets zu naschen läßt man sie allein ..... [48]

\newline 
Mein Oheim, tönt’s am Tajo fern ..... [44]

\newline 
Messe. Auswahl ..... [49]

\newline 
Milord vous avez ma promesse ..... [26]

\newline 
Missa superba. Excerpts. Arr ..... [40], [41]

\newline 
Moderata durant ..... [68]

\newline 
Musikalisches Opfer. Fragment ..... [20]

\newline 
Müßt ich auch durch tausend Drachen ..... [50]

\newline 
Non più di fiori vaghe catene ..... [47]

\newline 
Nun die Schatten dunkeln ..... [31]

\newline 
O Himmel welch Entzücken ..... [22]

\newline 
[O namenlose Freude] ..... [23]

\newline 
O seht sie nah und näher treten ..... [22]

\newline 
Ombra felice tornerò a rivederti ..... [51]

\newline 
Pensieri funesti ah no non tornate ..... [69]

\newline 
Prima la musica e poi le parole. Excerpts. Arr ..... [69]

\newline 
Prinz Eugen der edle Ritter. Auswahl ..... [72]

\newline 
Quand sur les ailes des plaisirs ..... [39]

\newline 
Quartett ..... [24]

\newline 
Reviens plaisir d'amour reviens secher mes larmes ..... [66]

\newline 
Salvator mundi ..... [27]

\newline 
Salvator mundi, salva nos omnes ..... [27]

\newline 
Sanctus, Dominus Deus Sabaoth ..... [40], [41]

\newline 
Sehnsucht ..... [30]

\newline 
Sinfonie concertantes. Auswahl ..... [52]

\newline 
Solfeggio ..... [53], [54], [55], [56], [57], [58], [59], [60]

\newline 
Sonate ..... [33], [34], [35], [61], [62], [64]

\newline 
Süßer Trost mein Jesus kommt ..... [21]

\newline 
Tief aus Tälern aus waldigen Höhen eilen wir freudig heran ..... [25]

\newline 
Till Eulenspiegels lustige Streiche. Auswahl ..... [76]

\newline 
Trioonatas ..... [74]

\newline 
Tutte le volte che in villa trovomi ..... [70]

\newline 
Wenn ihr hinfort euch brav aufführt ..... [72]

\newline 
Wo zieht ihr Wolken so eilig hin ..... [30]

\newline 
Zum che diavolo è mai ..... [71]

\newline 
wehe dir weh uns allen ..... [75]
 \clearpage \onecolumn \ \vfill \center {\chancery Finis.}$ \vfill \thispagestyle{empty} \end{document}